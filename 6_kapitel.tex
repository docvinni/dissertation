\documentclass[12pt,a4paper,oneside]{book} % 'oneside' für einseitigen Druck

% Kodierung, Sprache und Schrift
\usepackage[utf8]{inputenc} % Erlaubt die Verwendung von Umlauten
\usepackage[T1]{fontenc} % Bessere Schriftkodierung
\usepackage[ngerman]{babel} % Deutsche Lokalisierung

% Schriftarten
\usepackage{lmodern} % Modernere Schriftart, gut für Skalierbarkeit und Lesbarkeit

% Für Abbildungen
\usepackage{graphicx}
\graphicspath{{bilder/}} % Verzeichnis, in dem Bilder gespeichert sind

% Für Tabellen
\usepackage{booktabs}

% Für Links und PDF-Metadaten
\usepackage[hidelinks]{hyperref}
\hypersetup{
	pdftitle={Titel der Dissertation},
	pdfauthor={Autor},
	pdfsubject={Doktorarbeit in den Sozial- und Rechtswissenschaften},
	pdfkeywords={Schlüsselwörter},
}

\usepackage{array}
% Für Bibliographie - Anpassung für Geisteswissenschaften
%%\usepackage[style=authoryear-icomp,backend=biber]{biblatex}
%%\usepackage[backend=biber, style=authoryear-icomp]{biblatex}
%%\usepackage[backend=biber, style=verbose-trad1]{biblatex}
\usepackage[backend=biber, style=verbose-inote]{biblatex}
%%\usepackage{biblatex}
\addbibresource{literatur.bib} % Name der BibTeX-Datei
\DeclareNameAlias{author}{family-given} % Nachname des Autors zuerst

% Anpassung der Nummerierung mit Punkten
\renewcommand{\thechapter}{\arabic{chapter}.} % Kapitel: 1., 2., 3., ...
\renewcommand{\thesection}{\Alph{section}.} % Abschnitt: A., B., C., ...
\renewcommand{\thesubsection}{\Roman{subsection}.} % Unterabschnitt: I., II., III., ...
\renewcommand{\thesubsubsection}{\arabic{subsubsection}.} % Unterunterabschnitt: 1., 2., 3., ...
\renewcommand{\theparagraph}{\alph{paragraph}.} % Absatz: a., b., c., ...


% Für Fußnoten
%%\usepackage[bottom]{footmisc} % Fußnoten am Seitenende


% Anpassung der Kapitelüberschriften
%\usepackage{titlesec}
%\titleformat{\chapter}[hang]{\Huge\bfseries}{\thechapter\quad}{0pt}{\Huge\bfseries}

% Abstand der Fußnoten
\setlength{\footnotesep}{0.5cm}

% Tiefe der Nummerierung und des Inhaltsverzeichnisses
\setcounter{secnumdepth}{4} % Nummerierungstiefe einstellen
\setcounter{tocdepth}{4} % Inhaltsverzeichnistiefe einstellen

% Abstand zwischen Absätzen und kein Einzug
%\usepackage{parskip}
%\setlength{\parskip}{0.5em}
%\setlength{\parindent}{0pt}

% Für Abkürzungsverzeichnis
\usepackage[printonlyused]{acronym}

% Für Zitate und Theoreme (falls benötigt)
\usepackage{csquotes}

% Für Gesetzestexte, Zitate und andere strukturierte Texte
\usepackage{enumitem}

% Zeilenabstand auf 1.3
\usepackage{setspace}

% Beginn des Dokuments
\begin{document}
	
\chapter{Umsetzbarkeitsstrategien durch Diskussion der Ergebnisse}	

\section{Szenarien}

\section{Handlungsempfehlungen}

Einpreisung des Risikos


Versicherung gegen politische Risiken bei Explorationsprojekten in Regionen mit hohem Risiko

Stärkung der Stabilität der Governance-Rahmenbedingungen für Mineralien.
Unerwartete bzw. unvorhersahbare Änderungen bei Explorations- und Abbauregulierungen, Verarbeitungsauflagen, Schutzstandards, Lizenzbedingungen oder Exportkontrollen sollten vermieden werden -- die hierdurch sichergestellte langfristige Planbarkeit stellt einerseits nicht nur die Attraktivierung für Investionen dar, sondern bietet auch der regulatorischen Seite eine Sicherheit dass sich Wirtschaftsakteure entsprechend langfristig und nachhaltig in den Sektor einbinden und somit auch den Staatszielen nach Versorgungssicherheit dienlich sind.

Ausweitung der Standards auf nicht-europäische Partner bei Wahrung der Augenhöhe
Die zuvor besprochenen Regulierungen sollten bei Projekten wie Rohstoffpartnerschaften ebenfalls Anwendung finden. Hierbei ist es essentiell, dass diese nicht aufgezwungen werden, sondern der Partner intrinsisch den Mehrwert der Regulierungen erkennt: Durch eine langfristige Bindung an das Rohstoffpolicy-Framework der Union wird einerseits die Abnahme sichergestellt, andererseits auch aus strategischer Sicht die Bindungen an einen multilateralen Rahmen und -- im Falle eine Rohstoffagentur --  die Vertretung dieser Werte und Etablierung als Standard auf globaler Ebene. 
	
\end{document}