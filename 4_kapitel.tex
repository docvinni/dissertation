\documentclass[12pt,a4paper,oneside]{book} % 'oneside' für einseitigen Druck

% Kodierung, Sprache und Schrift
\usepackage[utf8]{inputenc} % Erlaubt die Verwendung von Umlauten
\usepackage[T1]{fontenc} % Bessere Schriftkodierung
\usepackage[ngerman]{babel} % Deutsche Lokalisierung

% Schriftarten
\usepackage{lmodern} % Modernere Schriftart, gut für Skalierbarkeit und Lesbarkeit

% Für Abbildungen
\usepackage{graphicx}
\graphicspath{{bilder/}} % Verzeichnis, in dem Bilder gespeichert sind

% Für Tabellen
\usepackage{booktabs}

% Für Links und PDF-Metadaten
\usepackage[hidelinks]{hyperref}
\hypersetup{
	pdftitle={Titel der Dissertation},
	pdfauthor={Autor},
	pdfsubject={Doktorarbeit in den Sozial- und Rechtswissenschaften},
	pdfkeywords={Schlüsselwörter},
}

\usepackage{array}
% Für Bibliographie - Anpassung für Geisteswissenschaften
%%\usepackage[style=authoryear-icomp,backend=biber]{biblatex}
%%\usepackage[backend=biber, style=authoryear-icomp]{biblatex}
%%\usepackage[backend=biber, style=verbose-trad1]{biblatex}
\usepackage[backend=biber, style=verbose-inote]{biblatex}
%%\usepackage{biblatex}
\addbibresource{literatur.bib} % Name der BibTeX-Datei
\DeclareNameAlias{author}{family-given} % Nachname des Autors zuerst

% Anpassung der Nummerierung mit Punkten
\renewcommand{\thechapter}{\arabic{chapter}.} % Kapitel: 1., 2., 3., ...
\renewcommand{\thesection}{\Alph{section}.} % Abschnitt: A., B., C., ...
\renewcommand{\thesubsection}{\Roman{subsection}.} % Unterabschnitt: I., II., III., ...
\renewcommand{\thesubsubsection}{\arabic{subsubsection}.} % Unterunterabschnitt: 1., 2., 3., ...
\renewcommand{\theparagraph}{\alph{paragraph}.} % Absatz: a., b., c., ...


% Für Fußnoten
%%\usepackage[bottom]{footmisc} % Fußnoten am Seitenende


% Anpassung der Kapitelüberschriften
%\usepackage{titlesec}
%\titleformat{\chapter}[hang]{\Huge\bfseries}{\thechapter\quad}{0pt}{\Huge\bfseries}

% Abstand der Fußnoten
\setlength{\footnotesep}{0.5cm}

% Tiefe der Nummerierung und des Inhaltsverzeichnisses
\setcounter{secnumdepth}{4} % Nummerierungstiefe einstellen
\setcounter{tocdepth}{4} % Inhaltsverzeichnistiefe einstellen

% Abstand zwischen Absätzen und kein Einzug
%\usepackage{parskip}
%\setlength{\parskip}{0.5em}
%\setlength{\parindent}{0pt}

% Für Abkürzungsverzeichnis
\usepackage[printonlyused]{acronym}

% Für Zitate und Theoreme (falls benötigt)
\usepackage{csquotes}

% Für Gesetzestexte, Zitate und andere strukturierte Texte
\usepackage{enumitem}

% Zeilenabstand auf 1.3
\usepackage{setspace}
\usepackage{booktabs}



% Beginn des Dokuments
\begin{document}
	% Hier beginnt der eigentliche Inhalt der Arbeit
	% ...

\chapter{Politisch-wirtschaftliche Betrachtung auf internationaler Ebene}

Die folgenden Erwägungen beruhen auf der Voraussetzung bzw. Annahme, dass der Markt der kritischen Rohstoffe aus marktwirtschaftlicher Sicht dysfunktional ist und auch auf absehbare Zeit kein funktionierender Weltmarkt wiederhergestellt wird.

Ein bedeutendes Problem: Der Einfluss des Preises. Denn: Es würden zwar viele Ziele wie im CRMA formuliert, jedoch seien alle Bestrebungen zur nationalen Gewinnung von seltenen Erden eingestellt worden, denn sie ließen sich \glqq hierzulande oder in der EU nicht preiswerter fördern, als aus China oder Russland einzukaufen\grqq.\autocite{VDI Nachrichten: Seltene Erden: Deutschland importiert lieber, als selbst zu fördern}

Die politische Unsicherheit, die sich entsprechend auf die Nachfrage nach kritischen Rohstoffen auswirkt, stellt einen weiteren zurückhaltenden Faktor dar -- insbesondere im Bereich der langen endgültigen Entscheidungs-, Investitions- und Umsetzungshorizonte im Bergbau, an die sich direkt die Frage nach der ausreichend hohen und ausreichend langen Nachfrage nach Rohstoffen anschließt und ob eine Bereitschaft besteht, u. U. teurere Mineralien einzukaufen, die jedoch aus EU-Quellen stammen.

Die chinesische Dominanz auf dem Seltenerd-Markt und der historische Einfluss auf konkurrierende Unternehmen bieten auch wenig Anreiz für finanzielle Investitionen in die Diversifizierung der internationalen Lieferkette.

\section{Koordinationsversagen?}#

Grundsätzlich ist es zunächst Aufgabe eines Unternehmens, betriebswirtschaftliche Risikoabschätzungen durchzuführen, jedoch ist auch einzuwenden dass die \glqq Vermeidung gefährlicher volkswirtschaftlicher Abhängigkeiten die Aufgabe der Bundesregierung [ist] -- trotz aller Bekenntnis zur Ordnungspolitik\grqq.\autocite{Wirtschaft fordert mehr Tempo bei Rohstofffonds}

\section{Politische Leitlinien und Strategieentwicklungen}

\subsection{Deutschland und EU}

Der Bezug strategischer Rohstoffe erfolgt in Deutschland bislang überwiegend über Importe, während ein heimischer Abbau vielfach unterbleibt. Ein prominentes Beispiel ist Lithium, das für die Herstellung von Akkumulatoren in Elektrofahrzeugen eine zentrale Rolle spielt. Im Oberrheingraben befindet sich eines der weltweit größten bekannten Lithiumvorkommen, dessen wirtschaftlicher Wert nach heutigen Marktpreisen auf etwa 1,5 Billionen Euro geschätzt wird. Das Lithium liegt dort bereits gelöst im Thermalwasser vor, was den Abbau prinzipiell erleichtert. Dennoch wird bislang auf eine Förderung verzichtet, da befürchtet wird, dass der Abbau negative Auswirkungen auf die Bausubstanz in der Region haben könnte. Während bei großflächigen Eingriffen wie dem Kohletagebau solche Bedenken historisch kaum eine vergleichbare Rolle gespielt haben, zeigt sich hier eine besondere Sensibilität gegenüber potenziellen Risiken.


Schon 2012 legte das deutsche Bundesofrschungsministerium das Forschungs- und Entwicklungsprogramm \glqq Wirtschaftsstrategische Rohstoffe für den Hightech-Standort Deutschland\grqq auf.
%darauf eingehen

Status der Projekte heute:

\begin{table}[ht]
	\centering
	\begin{tabular}{p{6cm} p{6cm}}
		\toprule
		\textbf{Projekt} & \textbf{Status} \\
		\midrule
		Storkwitz, Deutsche Rohstoff AG & eingestellt, Rückgabe der Lizenz \\
		Ostsee-Sand & eingestellt \\
		Bayern & Platzhalter \\
		Harz, Stöbich Holding & eingestellt \\
		Rhein, Lanthan & Platzhalter \\
		Platzhalter & Platzhalter \\
		Platzhalter & Platzhalter \\
		Platzhalter & Platzhalter \\
		Platzhalter & Platzhalter \\
		Platzhalter & Platzhalter \\
		\bottomrule
	\end{tabular}
	\caption{Übersicht der Projekte und Status}
	\label{tab:projekt_status}
\end{table}

Auch vergleichsweise neuartige Technologien zur Rohstoffgewinnung scheiterten trotz staatlicher Förderungszuage am Rückzug von Investionen der Industrie.

Auch die Dauer der Errichtung von heimischer Produktion muss bedacht werden. Das Mount-Weld-Vorkommen in Australien (betrieben von Lynas) wurde 1966 im Rahmen einer aeromagnetischen Untersuchung entdeckt. Nach umfangreichen Erkundungen und Machbarkeitsstudien begann der industrielle Abbau im Jahr 2011. Die vorläufige Betriebserlaubnis für das vertikal integrierte Chemiewerk in Malaysia, das zur Weiterverarbeitung der geförderten Erze dient, wurde 2012 erteilt. Erst ab diesem Zeitpunkt konnte die Produktion industriefähiger Materialien aufgenommen werden. Aufgrund der radioaktiven Emissionen, die bei der Verarbeitung entstehen, war die Internationale Atomenergie-Organisation (IAEA) als Gutachterin in das Genehmigungsverfahren eingebunden. Vor dem Hintergrund dieser Erfahrungen erscheint es wenig wahrscheinlich, dass ein vergleichbares Projekt – zumal auf Grundlage deutlich weniger ergiebiger Lagerstätten – in Deutschland realisiert werden könnte. Insbesondere die Errichtung einer Chemieanlage zur Verarbeitung von Erzkonzentraten und der damit verbundenen Entstehung radioaktiver Rückstände und Emissionen dürfte in Deutschland auf erheblichen gesellschaftlichen und politischen Widerstand stoßen. Auch das Mountain-Pass-Projekt in den USA verdeutlicht die langen Entwicklungszeiträume, die mit der (Wieder-)Inbetriebnahme von Bergbau- und Aufbereitungsprojekten verbunden sind.

Im Gegensatz dazu ist der Abbau von Zirkon, das gemeinsam mit Titanmineralien aus sogenannten schweren Mineralsanden gewonnen wird, weniger problematisch. Dennoch wird der Import solcher Rohstoffe, etwa aus Mosambik oder Australien, oft als praktikabler angesehen. Für Deutschland kommt allenfalls die Weiterverarbeitung importierter Rohmaterialien in Betracht, wobei selbst diese Option aufgrund der bei der Erzaufbereitung anfallenden Abfälle und Emissionen mit Herausforderungen verbunden ist. Insgesamt gilt, dass dicht besiedelte Länder wie Deutschland aufgrund ökologischer, gesellschaftlicher und raumplanerischer Restriktionen kaum mehr für großflächigen Bergbau geeignet sind.

Zudem gibt es Forderungen, dass der deutsche Staat in den Bergbau einsteigen sollte, da es sich um eine existentielle Bedrohung für Deutschland als Industrie- und Wirtschaftsstandort handele, insbesondere im Bereich der Verbindung von privater und staatlicher Aktivitäten.\autocite{https://www.africa-business-guide.de/de/praxis/erfahrungen/schluesselrolle-afrikas-bedeutung-bei-den-kritischen-rohstoffen--1920084}

Es wird deutlich, dass daher die Maßnahmen zur Rohstoffsicherung gesteigert werden müssen


\subsection{Japan}

Japan Organization for Metals and Energy Security

Japan investiert somit global über eine staatliche Organisation in Abbaustätten, um seine eigene Versrogung sicherstellen zu können.

%Blueprint für eine EU- Rohstofforganisation

\subsection{USA}
Die USA unter Präsident Trump setzt mit ihrer Strategie gleich zu Beginn der Lieferkette an -- durch die Sicherung von Abbaurechten. Die USA betrachten die Abhängigkeit nicht nur aus einer rein industriellen Perspektive, sondern insbesondere auch aus einer verteidigungs- und sicherheitspolitischen Sicht, mit Fokus auf den im Verteidigungsbereich besonders wichtigen Mineralien wie Wolfram oder Gallium.

Die Trump-Regierung verhandelte daher mit der DRC und Ruanda über die Versorgung mit kritischen Rohstoffe im Tausch für Lösungsbeiträge hinsichtlich des Konflikts in der Grenzregion zwischen der DRK und Ruanda -- quasi ein \glqq Sicherheit für Rohstoffe\grqq-Deal ähnlich dem US-Ukraine-Deal. Im Gegensatz dazu lag hier das initiale Moment bei der DRK, jedoch zog sich die Einigung hin.

%Von OA und ggf nicht aktuell: Letztlich erwägt Washington ein Mineralienabkommen nicht nur mit der Demokratischen Republik Kongo, sondern auch mit Ruanda, die beide von einem umfassenden Friedensabkommen zwischen der Demokratischen Republik Kongo und Ruanda sowie der Demokratischen Republik Kongo und der M23-Rebellion abhängen werden. All diese Angebote werden einzeln extrem schwer zu erreichen sein und ein Paket, das alle noch mehr zufriedenstellt. Nichtsdestotrotz würde ein Abkommen der Demokratischen Republik Kongo einen entscheidenden Hebel an die Hand geben, um zu versuchen, eine peinliche und schädliche innenpolitische Krise zu lösen und gleichzeitig ihre Geschäftspartnerschaften zu diversifizieren, ihre Abhängigkeit von China zu verringern, Investitionen anzuziehen und Einkommen zu generieren. Für die Vereinigten Staaten würde es ein wichtiges Standbein in einem rohstoffreichen Land bieten und gleichzeitig die chinesische Dominanz herausfordern.



Unter Präsident Biden wurde die Rohstoffpolitik zudem auch dem grünen Wandel untergeordnet, sodass u. A. im IRA und im Infrastructure Investment and Jobs Act Steuer- und Kreditvorteile für Produzenten (d.h. also Stufe drei und vier) verankert waren und auch Produkte ansprach, in denen kritische Mineralien verarbeitet sind (konkret: Elektrofahrzeuge und Kaufanreize für diese). Der Übergang hin zu nachhaltigeren Technologien stellt jedoch keine Prioriät der Trump-Administration dar, sodass hier insbesondere Unsicherheit in der Rohstoffpolitik entsteht, da eine Fortsetzung oder ein Umschwenken des grünen Wandels nicht zweifelsfrei sichergestellt ist und somit Investitionen hemmt. In dieser Hinsicht kann der EU zumindest im Bereich des Green Deals eine höhere Policykontinuität bescheinigt werden, sodass dieser Effekt als geringer einzustufen ist.



So hat Trump u. A. auch den Vorschlag unterbreitet, Basen des US-Militärs zur Rohstoffverarbeitung zu nutzen; diese Idee wurde letztlich aber nicht weiter verfolgt.

Die sicherheitspolitische Wahrnehmung der Rohstoffabhängigkeit ist auf EU-Ebene noch nicht in diesem Maße eingetreten.


\section{Industriewirtschaftliche Bedeutung der Rohstoffversorgung}


2018 stellte man während der sog. 2. Statuskonferenz für Wirtschaftsstrategische Rohstoffe in Berlin fest, dass Deutschland durchaus Vorkommen an kritischen Rohstoffen habe -- es hätte vorher \glqq einfach niemand danach gesucht\grqq und an entsprechenden finanziellen Investitionen gemangelt.\autocite{VDINachrichten: Rohstoffe: Deutschland ist reich an seltenen Erden} Die Vorkommen in Sachsen waren so bereits zu DDR-Zeiten bekannt. Auch gerieten Vorkommen in Vergessenheit, und Bürger könnten sich vor einer \glqq Renaisance des Bergbaus\grqq wehren. Der Begriff einer Zeitenwende und die Erforderlichkeit einer solchen wurden also bereits hier genannt.





Es ergibt sich also die weitere Erkenntnis, dass die Nichtbeachtung der Rohstoffthematik auch daher rührt, dass die Versorgung mit Rohstoffen (sowohl generell als auch speziell aus nationalen Vorkommen), so simpel es erscheint, wirtschaftlich nicht attraktiv (genug) war.

\glqq Wenn die deutsche Industrie wirklich Interesse an dem Thema hätte, hätte sie sich dort beteiligen können\grqq \autocite{VDI Nachrichten Seltene Erden: Deutschland importiert lieber, als selbst zu fördern}


\subsection{Auswirkungen der Marktpreise für seltene Erden}

Auf der zweiten Frankfurter Regulierungskonferenz stellte ein Teilnehmer fest: Der Marktpreis für seltene Erden preist das Risiko keineswegs ein bzw. ist das Versorgungsrisiko bei kritischen Rohstoffe im Preis nicht reflektiert -- somit also \glqq zu günstig\grqq. Die Industrie setze hingegen auf den freien Markt und kaufe dementsprechend Rohstoffe auch zu diesen niedrigen Preisen ein, ohne aber hierbei Bemühungen oder Investitionen zur Entwicklung alternativer Versorgungsmöglichkeiten zu verfolgen.\autocite{VDI Nachrichten: Seltene Erden: Deutschland importiert lieber, als selbst zu fördern}

Einerseits handelte die europäische Industrie zu lange nach der Devise, dass ein günstiger Beschaffunsgpreis wichtiger sei als eine resiliente Versorgung und nun realisierten dass Produktionsunterbrechungen deutlich höhere Kosten verursachen, zum Anderen basierten die Preise von chinesischen Mineralien auf Staatsvorgaben und nicht auf Fundamentaldaten.\autocite{"Je länger die EU untätig bleibt, umso verwundbarer macht sie sich}

Die Preise für seltene Erden sind also niedrig, auch durch einen kontinuierlichen Abstieg des Preisniveaus nach 2010. Echte Versorgungsengpässe, die durch das Vorkommen bedingt sind, existieren nicht

Nachfrage
%Abbildung

Dieses Preisgefüge sorgt schließlich auch dafür, dass ein Abbau in Deutschland und den EU-Mitgliedsstaaten nicht rentabel ist und sich die Abhängigkeiten verstärken,

Die ökonomischen Auswirkungen einer solchen Abhängigkeit sind hinlänglich bekannt: 


\section{Rohstoffaußenpolitik}
Auch die Erforderlichkeit einer Rohstoffaußenpolitik zur Rohstoffversorgungssicherheit ist keineswegs eine neue Idee: so wurde die Gefahr von \glqq Rohstoff-Nationalismus\grqq, Exportbeschränkungen und Wettbewerbsverzerrungen schon 2010 umfassend erkannt und entsprechende Maßnahmen vorgeschlagen.\autocite{Eine Frage der Außenpolitik, IP November/Dezember 2010}

Mitunter auch als \grqq Raw materials diplomacy\grqq bezeichntet, fallen in eine \glqq Rohstoffaußenpolitik\grqq eine Vielzahl von insbesondere strategischen Aspekten.[siehe zum Begriff der Rohstoffaußenpolitik]\autocite{Carry Müller Schulze: Elemente einer nachhaltigen Rohstoffaußenpolitik, SWP 2023}{Müller Saulich Schöneich Schulze: Von Der Rohstoffkonkurrenz zur nachhaltigen Rohstoffaußenpolitik, SWP 2022}

Erneut ist auch der Ruf nach einer Rohstoffaußenpolitik nicht neu oder überraschend einzustufen--insgesamt acht bereits 2010 vorgestellte \glqq Denkanstöße\grqq haben an Aktualität nicht verloren.\autocite{Mißfelder, Philipp: Eine Frage der Außenpolitik, IP 2010, 102ff.} Auch die deutsche Automobilindustrie fordert eine \glqq aktive\grqq Rohstoffaußenpolitik, nicht nur aus energiepolitischen Aspekten.\autocite{Müller, Hildegard: Wir brauchen eine aktive Rohstoff-Außenpolitik – sonst droht massiver Wohlstandsverlust, Handelsblatt, 06.04.2022}

Zunächst: Der Bedarf des Westens an kritischen Mineralien könnte zu einem starken Anstieg der Investitionen in rohstoffreichen Schwellenländern führen.


Daher ist eine aktivere Rohstoffaußenpolitik erforderlich

\subsection{Der Fall China}

China hat seine strategische Stärke im Bereich der Mineralien über Jahrzehnte aufgebaut: kritische Rohstoffe und Seltende Erden sind keineswegs nur in China zu finden.

Chinas Monopol hat sich auch aus der Bereitschaft herausgeprägt, dass das Land bereit sich in das \glqq oft schmutzige GEschäft des Abbaus und der Verarbeitung\grqq zu begeben und somit bis zu 70 \% des Abbaus und bis zu 90 \% der Verarbeitung kontrolliert.\autocite{Rachman, FT} Dieser Punkt ziegt auch, dass China für die Verarbeitung zunächst selbst von entsprechenden Importe abhängig ist und daher kein \glqq vollufänglicher\grqq dominierender Akteur ist.

Auch China nutzt seine nicht-energetischen Rohstoffvorkommen und -verarbeitungskapazitäten als außenpolitisches Instrument, mitunter auch agressiv-prononciert, insbesondere im afrikanischen Raum.\footnote{Die Auswirkung von Abhängigkeiten und Anfälligkeiten hinsichtlich rohstoffaußenpolitischer Spannungen und die Instrumemtalisierung dieser hat im energetischen Bereich insbesondere die Gasversorgungslage nach dem russischen Überfall auf die Ukraine gezeigt.} Hier hat sich China durch den frühzeitigen Einstieg in Exploration und Erschließung, Abnahmevereinbarungen und Know-How-Vorsprung eine \glqq Pole-Position \grqq gesichert.\autocite{https://www.africa-business-guide.de/de/praxis/erfahrungen/schluesselrolle-afrikas-bedeutung-bei-den-kritischen-rohstoffen--1920084}

China sichert sich insbesondere durch erhebliche Finanzzusagen nicht nur den Zugang, sondern auch die Verarbeitung der extrahierten Mineralien.

Hierbei ist in China nicht die Rentabilität der treibende Faktor für die Rohstoffsicherung, sondern vielmehr die Sicherstellung der Versorgung, was durch günstige Kredite und Energepreise gefördert wurde.\autocite{Oxford Analytica: Critical minerals will be a global faultline for years}

%Beispiele für chinesische 

Dass die EU und ihre Mitgliedsstaaten keine derart geostrategisch motivierte und oder gar hegemoniale\footnote{QUOTATION NEEDED} (Rohstoff-)Außenpolitik verfolgen können, dürfte nicht weiter überraschen. Ferner sind die Auswirkungen chinesischer Exportkontrollen bzw. -beschränkungen im vorangegangenen Kapitel deutlich geworden, die mit zu den außenpolitischen Instrumenten gezählt werden können. Insofern kann der chinesische Weg zwar als effektiv zur Rohstoffsicherung eingestuft werden, kann aber darüber hinaus nicht als Blaupause für zukünftige Aktivität einer Union dienen, die sich den Art. 21, 22 EUV verpflichtet hat.

Bereits vor Jahren hat China indirekt Rohstoffe als Waffe instrumentalisiert -- und die Importabhängigkeit von China ist bedeutend größer als die von russischen Gasimporten.\autocite{BDI. Rohstoffkongress: Deutschland braucht eine strategische Rohstoffpolitik}

China hat somit seine Abnehmer zunächst durch niedrige Preise\footnote{ZUr Erinnerung: Insbesondere durch die Nicht-Einpreisung des Versorgungsrisikos} an sich gebunden und in Abhängigkeiten driften lassen -- und besitzt nun durch den Hebel der Rohstoffversorgung einen mächtigen solchen, kann es doch effektiv jederzeit die industrielle Produktion zumindest teilweise oder ganz zum Erliegen bringen.

Die Lieferkettendominanz Chinas im Bereich des Abbaus und der Verarbeitung wurden bereits beleuchtet. Chinas eigene Importversorgung ist durch Abkommen mit rohstoffreichen Ländern gesichert: so kontrolliert die Volksrepublik schätzungsweise 70 bis 80 \% der Kobaltminen in DRC.

Jedoch ist zu beachten, dass China in der eigentlichen Nachfrage nach kritischen Rohstoffen keine Dominanz aufweist, denn der Verbrauch im Verhältnis zum BIP ist vergleichsweise gering, sodass China (noch) auf den Exporteinnahmen durch die Ausfuhr der raffinierten, aber noch nicht eingesetzen kritischen mineralischen Rohstoffe angewiesen ist -- was eine Schwachstelle darstellen könnte.

Zusammen mit den Erkenntnissen aus dem vorigen Kapitel zu chinesischen Exportrestriktionen lässt sich die chinesische Rohstoffverwaltung als staatlich gelenktes, geopolitisches motiviertes und industriepolitisch eingebettetes Steuerungsregime charakterisieren, das auf langfristige Kontrolle über globale Rohstoffwertschöpfungsketten abzielt. Im Gegensatz zum europäischen Modell, das auf Marktmechanismen, regulatorische Rahmensetzung und sektorübergreifende Koordination setzt, basiert das chinesische Modell auf einer Kombination aus staatlicher Ressourcenallokation, strategischer Außenwirtschaftspolitik und restriktiven Exportinstrumenten. Die jüngsten Exportkontrollen für Graphit, Gallium und Seltene Erden – eingeführt im Oktober 2023 und erweitert im April 2025 – sind Ausdruck einer gezielten Ressourcenpolitik, die nicht primär ökonomisch, sondern machtpolitisch motiviert ist. 

Zentraler Bestandteil der chinesischen Strategie ist die Sicherung ausländischer Rohstoffquellen durch staatlich unterstützte Direktinvestitionen, etwa in Afrika, Lateinamerika und Zentralasien, kombiniert mit der Verlagerung der Verarbeitungskapazitäten nach China. Dadurch entsteht ein asymmetrisches Machtverhältnis: Während die EU auf internationale Märkte angewiesen bleibt, kontrolliert China nicht nur den Zugang zu Primärrohstoffen, sondern auch deren Raffinierung und Weiterverarbeitung. Diese vertikale Integration erlaubt es China, durch administrative Maßnahmen wie Exportgenehmigungen, Quoten oder Umweltauflagen gezielt Einfluss auf globale Lieferketten zu nehmen.
\\
Demgegenüber verfolgt die EU ein Modell der regulierten Offenheit, das auf Diversifizierung, Nachhaltigkeit und strategischer Partnerschaft beruht. Die europäische Rohstoffverwaltung ist fragmentierter, rechtlich pluralistisch und stärker auf Konsensbildung angewiesen. Während dies demokratietheoretisch legitim ist, führt es in der Praxis zu geringerer Reaktionsgeschwindigkeit und eingeschränkter Steuerungstiefe.

Die Unterschiede sind nicht nur strukturell, sondern auch normativ: Die chinesische Rohstoffpolitik ist Teil eines umfassenden techno-nationalistischen Paradigmas, das wirtschaftliche Abhängigkeiten als strategische Hebel begreift. Die europäische Antwort – etwa durch die Kodifizierung eines Rohstoffverwaltungsrechts – ist daher nicht protektionistisch, sondern eine notwendige Reaktion auf systemische Asymmetrien. Die Automobilindustrie steht dabei exemplarisch für die Verwundbarkeit offener, globalisierter Produktionssysteme gegenüber rohstoffpolitischer Machtprojektion.

Bei Betrachtung der Merkmale des sog. Neokolonialismus fällt auf, dass Chinas Rohstoffpolitik im Ausland diese durchaus erfüllt bzw. eine entsprechende Prägung aufweist

Die chinesische Rohstoffpolitik weist also eine Kopplung außenpolitischer Instrumente mit internen Regulierungsmaßnahmen auf. 

Quellen zu chinesischer Rohstoffpolitik

Chinas heutige Vorrangstellung beruht also auf über Jahrzehnte wirkenden Marktdynamiken, die von westlichen Staaten weitgehend akzeptiert wurden. Angesichts der veränderten sicherheitspolitischen Prioritäten erweist es sich nun als äußerst komplex, diese Struktur aufzubrechen.

\subsection{Exkurs zur US-amerikanischen Rohstoffverwaltung}
Die Rohstoffverwaltung der Vereinigten Staaten unterscheidet sich grundlegend von der europäischen Herangehensweise, da die USA ein interventionsgestütztes, national fokussiertes und sicherheitsstrategisch motiviertes Modell. Die USA stützen sich dabei auf Instrumente wie den Defense Production Act, der gezielt zur Förderung der inländischen Produktion kritischer Mineralien eingesetzt wird 1. Exekutivanordnungen ermöglichen beschleunigte Genehmigungsverfahren, direkte Finanzierungsmaßnahmen und die Priorisierung von Projekten auf Bundeslandebene. Die europäische Rohstoffverwaltung hingegen ist stärker auf Binnenmarktkompatibilität, Umweltverträglichkeit und sektorübergreifende Koordination ausgerichtet. Sie operiert primär über regulatorische Rahmenwerke wie den Critical Raw Materials Act und setzt auf strategische Partnerschaften, etwa mit Kanada, Australien und afrikanischen Staaten.

Die USA verfolgen eine ressourcenpolitische Außenstrategie, die auf bilaterale Sicherung von Rohstoffquellen und geopolitische Einflussnahme abzielt. Die jüngste Erweiterung der US-Liste kritischer Mineralien – etwa um Kupfer und Silber – ist Ausdruck einer protektionistischen Tendenz, die mit Zollerhöhungen und Exportrestriktionen flankiert wird.

Parallelen bestehen in der grundsätzlichen Zielsetzung, die strategische Abhängigkeit von China zu reduzieren. Beide Wirtschaftsräume erkennen die Verwundbarkeit ihrer Industrien – insbesondere der Automobil- und Batteriebranche – gegenüber chinesischen Exportkontrollen, etwa bei Seltenen Erden, Gallium und Graphit. Während die USA jedoch auf nationale Reshoring-Strategien und Lagerhaltung setzen, verfolgt die EU eine resiliente, aber offene Rohstoffarchitektur.

\subsection{USA}

Dass die USA eine aktivere Rohstoffaußenpolitk verfolge, zeigte sich auch im Fall von Tungsten West plc, einem britischen Bergbauunternehmen, was im Vereinigten Königreich die Hemerdon-Mine betreibt, in der Wolfram abgebaut wird: US-Investment sicherte hier den Export des Minerals in die USA, und somit nicht nach Europa. Zwar stufte die EU das Projekt auch als strategisch relevant ein,



\section{Rohstoffabkommen}
Im vorangegangenen Kapitel wurde die Gestalt des Instruments der Rohstoffpartnerschaften (und -abkommen) bereits aus instrumentuell-rechtlicher Sicht beleuchtet; eine genauere Betrachtung vor politisch-wirtschaftlichen Hintergründen der einzelnen Abkommen empfiehlt sich, um eine entsprechene zukünftige Entwicklung abschätzen zu können.

Trotz der frühen Erkenntnis der Erforderlichkeit wurden die ersten Partnerschaften erst 2021 von der EU geschlossen.

\glqq Bei der Umsetzung der Projekte ist ein verständnisvoller Pragmatismus oft hilfreicher als eine unrealistische Hypermoral und das noch immer bei vielen Europäern spürbare Überlegenheitsgefühl.\grqq \autocite{https://www.africa-business-guide.de/de/praxis/erfahrungen/schluesselrolle-afrikas-bedeutung-bei-den-kritischen-rohstoffen--1920084}

Ferner sollten Rohstoffabkommen nicht nur auf die Mineralien an sich abzielen, sondern auch auf Zusammenarbeit im Rahmen von Forschung und Innovation.\autocite{ÖAW}

\subsection{Grundsätzliches}

https://single-market-economy.ec.europa.eu/sectors/raw-materials/areas-specific-interest/raw-materials-diplomacy_en

\subsection{Eurasien}

Serbien (JADAR)

Usbekistan

Kasachstan

Norwegen

\subsubsection{Grönland}

Grönland verfügt über bemerkenswerte Vorkommen an mineralischen Ressourcen, darunter Seltene Erden, Molybdän, Nickel, Kupfer und Grpahit, welche allesamt zu als kritische Rohstoffe gelten. Obwohl die geologische Kartierung dieser Vorkommen vergleichsweise weit fortgeschritten ist, bleibt ihre wirtschaftliche Nutzung bislang weitgehend aus.

Die begrenzte Erschließung dieser Ressourcen lässt sich vor allem durch die ökonomischen Rahmenbedingungen erklären. Die Gewinnung und Verarbeitung der Bodenschätze ist mit erheblichen finanziellen Aufwendungen verbunden. Neben den hohen Investitionskosten stellen auch die extremen klimatischen Bedingungen und die mangelnde Infrastruktur – etwa das Fehlen von Verkehrswegen und Umschlagplätzen – erhebliche Hürden dar. Diese Faktoren führen dazu, dass Grönlands Rohstoffsektor bislang kaum mit etablierten Förderregionen konkurrieren kann.

Trotz dieser Herausforderungen zeigen sowohl private Unternehmen als auch politische Akteure ein wachsendes Interesse an der Region. Die grönländische Regierung verfolgt eine Strategie zur wirtschaftlichen Diversifizierung und vergibt aktiv Förderlizenzen. Dabei wird besonderer Wert auf Umweltverträglichkeit und soziale Nachhaltigkeit gelegt, was sich in strengen regulatorischen Vorgaben widerspiegelt. Dennoch sind bislang nur wenige Projekte über die Planungsphase hinausgekommen.

Ein bemerkenswertes Beispiel für die zunehmende internationale Beteiligung ist ein Projekt, bei dem auch die Europäische Union involviert ist. Dieses Vorhaben stellt eine Ausnahme dar und verdeutlicht, dass geopolitische Interessen und die Sicherung von Versorgungsketten zunehmend Einfluss auf die Entwicklung des Rohstoffsektors in der Arktis nehmen.

Ukraine

\subsubsection{Asiatisch-pazifischer Raum}
Australien

\subsubsection{Afrika}
Afrika zählt zu den rohstoffreichsten Regionen weltweit und verfügt über Vorkommen nahezu aller kritischen Mineralien -- insbesondere bei strategischen Mineralien wie Kobalt, Lithium, Graphit und Seltenen Erden birgt der Kontinent ein bislang vielfach ungenutztes Potenzial, das rund 30 Prozent der globalen Reserven umfasst. Grundsätzlich hat der afrikanische Kontinent im Vergleich zu anderen Abbauregionen keinen Standortnachteil hinsichtlich Energie- und Infrastrukturverfügbarkeit und stabiler Rahmenbedingungen. 


Trotz der bekannten Vorkommen spielen Explorationsinvestitionen in Afrika im globalen Vergelich eine untergordnete Rolle (2024: 10,4\% der globalen Explorationsausgaben), obgleich (Subsahara-)Afrika zu den kostengünstigsten Regionen für Mineralexplorationen zählt (Verhältnis Mineralwert zu Explorationsausgabe: 0,8; zum Vergleich: Australien 0,5; Kanada 0,6; Lateinamerika 0,3), sodass eine kritische Diskrepanz zwischen dem Potential der afrikanischen Staaten und den tatsächlichen Investitionen sichtbar wird und daher eine Neubetrachtung der Explorationsprioritisierung zu forcieren.\autocite{Baskaran, G.: Underexplored and Undervalued: Addressing Africa’s Mineral Exploration Gap, CSIS}




DRC

Seit dem Ausbruch des Konflikts im Osten der DRK steht das Abkommen und auch das mit Ruanda unter Druck durch möglichen illegalen Handel.
https://www.euronews.com/my-europe/2025/01/30/dr-congo-conflict-why-is-the-eu-under-pressure-to-reconsider-its-minerals-partnership-with

https://afripoli.org/navigating-critical-mineral-supply-chains-the-eus-partnerships-with-the-drc-and-zambia

Zambia

2015 verschob sich Sambia unter Präsident Lungu hin zu einem Rohstoffnationalismus 

Rwanda

Namibia
Lofdal 






\subsubsection{Amerika}
Rohstoffverwaltung der USA
Auch die USA verfolgen entsprechende Strategien zur Reduzierung 



Chile

Argentinien

\paragraph{Kanada}

Im August 2025 beschloss die deutsche Bundesregierung unter Kanzler Merz eine Absichtserklärung für eine Rohstoffpartnerschaft mit \glqq Schlüsselpartner\grqq Kanada, inklusive entsprechend vorgesehener Investitionen.\autocites{Stratmann, Handelsblatt, Deutschland plant Rohstoff-Deal mit Kanada}{Joint Declaration of Intent between Canada and Germany on Critical Minerals Cooperation} Hierbei geht das Abkommen jedoch auch nicht über reine Absichtserklärungen hinaus.

\subsection{Exkurs zur Rohstoffaußenpolitik der USA unter Trump}

\subsubsection{Grönland}

Das Interesse der Trump-Regierung an Grönland hat angesichts seiner strategischen Bedeutung als Quelle kritischer Mineralien große Aufmerksamkeit erregt.

Eu Reaktionen darauf

\subsubsection{Rohstoffabkommen Ukraine}


\subsubsection{Abkommen als Instrument der Rohstoffverwaltung}


Aufgrund der 

Im Bereich der EU mangelt es jedoch an verbindlichen Inhalten wie Abnahme- und Investmentgarantien, konkreten Mengen oder Fristen zur Umsetzung.

Ferner ist auch fraglich, inwieweit es im Interesse der Union ist, sich an entsprechenden



\section{Geopolitische und geostrategische Erwägungen}

Generell ist zu erwarten, dass der Druck auf Länder mit einer erhöhten Fähigkeit zur Bereitstellung kritischer Rohstoffe druch Partner weiter zunehmen könnte. 

So ist zwar eine Zunahme der strategischen Bedeutung eines Staates durch heimische Förderung zu erwarten, jedoch ergeben sich hierbei auch zwangsläufig neue Sicherheitsrisiken -- mit der Erkenntnis, dass nicht allein die wirtschaftliche Betrachtung der einer Rohstoffverarbeitung herangezogen werden sollte, sondern auch nationale Sicherheitsinteressen.\autocite[Hier am Beispiel Schweden]{UI Report}

\section{Konflikte und Zieldivergenzen}

Jedoch ist auch zu beachten: Die Engpässe bei der Rohstoffversorgung treiben Innovationen voran, um den Bedarf an dieser Ressource zu verringern – ein Risiko für Nationen, die Kapazitäten für bestimmte Mineralien aufbauen.

Ein weiteres Problem ist, dass kritische Rohstoffe nicht außerhalb Chinas verarbeitet werden, da China eine künstliche Preisabsenkung hervorruft, sodass es schlichtweg für neue Produzenten nicht erschwinglich ist, in das GEschäft einzusteigen. Die künstlich niedrigen Preise könnten gekontert werden, was jedoch entsprechenden Government-Support benötigen würde.

Es sind nicht die Umweltherausforderungen, die überwunden werden müssen -- China simply pollutes.



that is classical problem of short term over long term and the low incentive for companies on the long term. Easy fix from the government; buy or grant some billion to the processers and buy their output.


\subsection{NIMBY}

Umweltängste könnten die Bemühungen des Westens, sich weniger auf China zu verlassen, beeinträchtigen, da der Abbau und die Verarbeitung von Mineralien oft umweltschädlich sind.

Somit kann argumentiert werden, dass NIMBY zumindest in einem gewissen Grad zur Abhängigkeitsproblematik beigetragen hat

\section{Zur Finanzierung der Rohstoffverwaltung}

Die heimische Rohstoffgewinnung stellt die Grundlage für Wertschöpfung von Industriezweigen dar sowie die Basis für die mineralische Rohstoffversorgunssicherheit der Union, sodass es sinnvoll erscheint, entsprechend aussichtsreiche Projekte auf nationaler und europäischer Ebene finanziell zu fördern.\autocite{Commodity Top News 73, S. 14}

Über viele Jahre hinweg hat die Industrie ihre Rohstoffbeschaffung vor allem über Börsen und Zwischenhändler organisiert – häufig ohne Interesse daran, woher die Materialien stammen.\autocite{https://www.africa-business-guide.de/de/praxis/erfahrungen/schluesselrolle-afrikas-bedeutung-bei-den-kritischen-rohstoffen--1920084} 

Kein CRMA Finanzinstrument

CRMa Rohstoffond KfW

DGWA Investmentvehikel zur Finanzierung der Explorationsphase bis hin zum Nachweis der wirtschaftlichen Machbarkeit

\section{Auswirkungen auf die dt. Automobilindustrie}



Als direkte der Folge der chinesischen Ausfuhrbeschränkungen (siehe auch \ref{3b_kapitel.tex}) mussten mehrere Automobilhersteller, nicht nur in Europa (in den USA bspw. Ford\autocite{Naughton_Bloomberg}), zeitweise die Produktion einstellen bzw. reduzieren.

Hinsichtlich des Einsteigs bei Bergbauunternehmen gehen die Meinungen auch bei den deutschen Fahrzeugproduzentenn auseinander: So engagieren sich VW (Kanada) oder Mercedes (Serbien), andere Hersteller hingegen haben bislang ein solches Investment nicht getätigt.

Im Gegensatz zu den Halbleitern können Seltenerdmetalle kaum substituiert werden.

Zunehmend werden zwar auch Batterietechnologien entwickelt, die ohne Kobalt oder Nickel auskommen. Jedoch ist zu beachten, dass die Rohstoffe nicht allein für Batterietechnologien benötigt werden, sondern auch für andere in den Fahrzeugen verbaute Technik wie Halbleiter. Auch zur weiteren Technologie-Entwicklung sind Metalle wie Antimon, Beryllium, Gallium, Germanium, Indium, Kadmium, Seltene Erden, Tellur oder Selen von entscheidender Bedeutung.\autocite{Commodity Top News 73, S. 3}

Neben der generellen Abhängigkeitsreduzierung sollte jedoch die weitere Möglichkeit der Reduzierung des Rohstoffbedarfs für die Batterieproduktion ebenfalls bedacht werden, auf andere Batterietypen mit anderen Verhältnissen der eingesetzten Rohstoffe umzuschwenken. So kann beispielsweise bei Nickel-Mangan-Kobalt-Batterien das Verhältnis modifiziert werden.

\section{Abbau und Verarbeitung in der EU / Nearshoring}

Der BDI sieht in der Rückverlagerung der Verarbeitung den \glqq größten Hebel für Diversifizierung\glqq, denn eine Verarbeitung in Deutschland bzw. Europa sei \glqq die ultimative Rückversicherung gegenüber weiteren Exportstopss\glqq.\autocite{Stefan Steinicke, BDI, 06.05.2025}

\section{Realismus-Überlegungen zur Rohstoffverwaltung}

Langfristig wird monopolistisches Verhalten, hier das Monopol Chinas, durch die Interdependenz moderner Märkte strukturell begrenzt. Zwar unterliegen die strategischen Kalküle und Narrative zwischen Großmächten einem raschen Wandel, doch bleiben die grundlegenden Spielregeln des internationalen Wirtschaftssystems bestehen. Je stärker die globale Öffentlichkeit die Seltenen Erden-Industrie als geopolitisches Druckmittel Pekings wahrnimmt, desto größer wird der Handlungsdruck auf die bereits wirksamen marktwirtschaftlichen Gegenkräfte.

Zwei zentrale Mechanismen sind hierbei von Bedeutung: Erstens die Marktöffnung durch neue Anbieter. Die Erhöhung chinesischer Exportzölle sowie die damit einhergehenden Preissteigerungen signalisieren wirtschaftliche Chancen. Staaten wie Kanada, Indien und das Vereinigte Königreich haben jüngst die Errichtung nationaler Raffinerien für Seltene Erden angekündigt, wobei sicherheitspolitische Erwägungen als treibende Kraft fungieren. Bereits geringe Investitionen könnten mittelfristig zu einer signifikanten Verschiebung der Marktstruktur führen. Die EU könnten diesen Prozess durch gezielte Förderung aussichtsreicher Projekte im Ausland beschleunigen und dabei strategische Friendshoring-Partnerschaften mit alternativen Produzenten etablieren.

Zweitens besteht ein innovationsgetriebener Substitutionsdruck. Technologische Notwendigkeit fördert die Entwicklung alternativer Materialien, die Seltene Erden in Endprodukten ersetzen könnten und somit Versorgungsrisiken mindern. Unternehmen wie Toyota und Volkswagen reagieren bereits mit der Neugestaltung ihrer Elektromotoren, um den Einsatz seltener Metalle zu reduzieren oder durch weniger effiziente Alternativen zu ersetzen. Auch US-amerikanische Behörden – insbesondere die Ministerien für Energie, Verteidigung und Handel – verfolgen aktiv die Erforschung von Ersatzstoffen. Ergänzend sollten staatliche Anreizsysteme geschaffen werden, die Unternehmen für innovative Produktdesigns ohne Seltene Erden belohnen, etwa nach dem Vorbild von „Bug-Bounty“-Programmen. Selbst ohne vollständige Substitution stärkt die Etablierung von Alternativen die Resilienz der Lieferketten und schwächt die monopolistische Machtstellung.

Neben diesen marktwirtschaftlichen Hebeln sollte die Europäische Union die Entwicklung ihrer eigenen Wertschöpfungsketten für Seltene Erden – insbesondere in den mittleren Segmenten wie Erzverarbeitung, Mineralraffinierung und Legierungsherstellung – konsequent subventionieren. Je schneller diese Kapazitäten aufgebaut werden, desto kürzer wird das Zeitfenster, in dem Xi Jinping geopolitisch eskalieren könnte, etwa im Kontext eines möglichen Taiwan-Konflikts. Um das globale Raffinierungsmonopol zu brechen, ohne eine geopolitische Konfrontation zu provozieren, ist eine baldige Diversifizierung der Lieferquellen unabdingbar.

Rein ordnungspolitische Instrumente und Forderungen nach Beachtung dieser sind somit überholt

\section{Fazit}

Es ist ersichtlich, dass die Fortentwicklungen und Anpassungen der Rohstoffverwaltung insbesondere durch das FOrtschreiten der Integration und insbesondere durch wirtschaftliche Verflechtung in den Hintergrund getreten ist -- und die Re-Evaluierung setzte viel zu spät ein, sodass die Möglichkeiten für eine Abhängigkeitsreduzierung verringert sind: die Zahl möglicher Erschließungsstätten in Europa ist begrenzt und nimmt zudem viel Zeit in Anspruch. Auch mögliche Partnerländer verfügen keine unerschöpflichen neuen Rohstoffquellen, wenn diese nicht bereits durch andere Akteure zu einem früheren Zeitpunkt gesichert wurden.

Darüber hinaus is festzuhalten, dass Abkommen und Partnerschaften mit entsprechenden Ländern, in denen mitunter niedrigere Standards in jederlei Hinsicht gelten) realisierbar sind und insbesondere auch unverzichtbar für die Sicherstellung der Versorgung sind.

Ferner stellt sich auch die Frage, inwieweit der US-chinesiche Handelskonflikt und potentiell weitere Handelskonflikte, auch zwischen der EU und China, die ROhstoffpolitik und eine effektivere ROhstofverwaltung entsprechend vorantreiben können -- und dies zudem mit der gebotenen Geschwindigkeit.

Es gibt verschiedene Wege der Herangehensweise: Während die USA Mineralienzusagen bevorzugen, die an Friedensabkommen und Rüstungszusagen gekoppelt sind, verfolgt die Union mit dem CRMA einen deutlich institutionalisierteren Ansatz und legt auch konkrete Zielwerte zur Abhängigkeitsreduzierung vor. Nichtsdestotrotz stellt sich auch die Frage der Erweiterung eines globalen Mineraliennetzwerks, auf welches die Union zur SIcherstellung ihrer Versorgung zurückgreifen kann. Hierzu ist jedoch eine entsprechende Rohstoffaußenpolitik erforderlich, bei der auch die Union beachten muss, dass Konditionen nicht mehr bedingungslos eingearbeitet werden können. 

Auch die militärische Relevanz darf nicht unerwähnt bleiben, drängt sich hierbei doch die Frage der militräisch-systemischen Rivalität auf der einen Seite und andererseits auch der Mineralienbedarf des Verteidigungssektors auf.

	


\end{document}

