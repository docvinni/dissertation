\documentclass[12pt,a4paper,oneside]{book} % 'oneside' für einseitigen Druck

% Kodierung, Sprache und Schrift
\usepackage[utf8]{inputenc} % Erlaubt die Verwendung von Umlauten
\usepackage[T1]{fontenc} % Bessere Schriftkodierung
\usepackage[ngerman]{babel} % Deutsche Lokalisierung

% Schriftarten
\usepackage{lmodern} % Modernere Schriftart, gut für Skalierbarkeit und Lesbarkeit

% Für Abbildungen
\usepackage{graphicx}
\graphicspath{{bilder/}} % Verzeichnis, in dem Bilder gespeichert sind

% Für Tabellen
\usepackage{booktabs}

% Für Links und PDF-Metadaten
\usepackage[hidelinks]{hyperref}
\hypersetup{
	pdftitle={Titel der Dissertation},
	pdfauthor={Autor},
	pdfsubject={Doktorarbeit in den Sozial- und Rechtswissenschaften},
	pdfkeywords={Schlüsselwörter},
}

\usepackage{array}
% Für Bibliographie - Anpassung für Geisteswissenschaften
%%\usepackage[style=authoryear-icomp,backend=biber]{biblatex}
%%\usepackage[backend=biber, style=authoryear-icomp]{biblatex}
%%\usepackage[backend=biber, style=verbose-trad1]{biblatex}
\usepackage[backend=biber, style=verbose-inote]{biblatex}
%%\usepackage{biblatex}
\addbibresource{literatur.bib} % Name der BibTeX-Datei
\DeclareNameAlias{author}{family-given} % Nachname des Autors zuerst

% Anpassung der Nummerierung mit Punkten
\renewcommand{\thechapter}{\arabic{chapter}.} % Kapitel: 1., 2., 3., ...
\renewcommand{\thesection}{\Alph{section}.} % Abschnitt: A., B., C., ...
\renewcommand{\thesubsection}{\Roman{subsection}.} % Unterabschnitt: I., II., III., ...
\renewcommand{\thesubsubsection}{\arabic{subsubsection}.} % Unterunterabschnitt: 1., 2., 3., ...
\renewcommand{\theparagraph}{\alph{paragraph}.} % Absatz: a., b., c., ...


% Für Fußnoten
%%\usepackage[bottom]{footmisc} % Fußnoten am Seitenende


% Anpassung der Kapitelüberschriften
%\usepackage{titlesec}
%\titleformat{\chapter}[hang]{\Huge\bfseries}{\thechapter\quad}{0pt}{\Huge\bfseries}

% Abstand der Fußnoten
\setlength{\footnotesep}{0.5cm}

% Tiefe der Nummerierung und des Inhaltsverzeichnisses
\setcounter{secnumdepth}{4} % Nummerierungstiefe einstellen
\setcounter{tocdepth}{4} % Inhaltsverzeichnistiefe einstellen

% Abstand zwischen Absätzen und kein Einzug
%\usepackage{parskip}
%\setlength{\parskip}{0.5em}
%\setlength{\parindent}{0pt}

% Für Abkürzungsverzeichnis
\usepackage[printonlyused]{acronym}

% Für Zitate und Theoreme (falls benötigt)
\usepackage{csquotes}

% Für Gesetzestexte, Zitate und andere strukturierte Texte
\usepackage{enumitem}

% Zeilenabstand auf 1.3
\usepackage{setspace}
\usepackage{booktabs}



% Beginn des Dokuments
\begin{document}
	% Hier beginnt der eigentliche Inhalt der Arbeit
	% ...

\chapter{Politisch-wirtschaftliche Betrachtung auf internationaler Ebene}


Ein bedeutendes Problem: Der Einfluss des Preises. Denn: Es würden zwar viele Ziele wie im CRMA formuliert, jedoch seien alle Bestrebungen zur nationalen Gewinnung von seltenen Erden eingestellt worden, denn sie ließen sich \glqq hierzulande oder in der EU nicht preiswerter fördern, als aus China oder Russland einzukaufen\grqq.\autocite{VDI Nachrichten: Seltene Erden: Deutschland importiert lieber, als selbst zu fördern}

Die politische Unsicherheit, die sich entsprechend auf die Nachfrage nach kritischen Rohstoffen auswirkt, stellt einen weiteren zurückhaltenden Faktor dar -- insbesondere im Bereich der langen endgültigen Entscheidungs-, Investitions- und Umsetzungshorizonte im Bergbau, an die sich direkt die Frage nach der ausreichend hohen und ausreichend langen Nachfrage nach Rohstoffen anschließt und ob eine Bereitschaft besteht, u. U. teurere Mineralien einzukaufen, die jedoch aus EU-Quellen stammen.

Die chinesische Dominanz auf dem Seltenerd-Markt und der historische Einfluss auf konkurrierende Unternehmen bieten auch wenig Anreiz für finanzielle Investitionen in die Diversifizierung der internationalen Lieferkette.

\section{Politische Leitlinien und Strategieentwicklungen}

\subsection{Deutschland und EU}

Der Bezug strategischer Rohstoffe erfolgt in Deutschland bislang überwiegend über Importe, während ein heimischer Abbau vielfach unterbleibt. Ein prominentes Beispiel ist Lithium, das für die Herstellung von Akkumulatoren in Elektrofahrzeugen eine zentrale Rolle spielt. Im Oberrheingraben befindet sich eines der weltweit größten bekannten Lithiumvorkommen, dessen wirtschaftlicher Wert nach heutigen Marktpreisen auf etwa 1,5 Billionen Euro geschätzt wird. Das Lithium liegt dort bereits gelöst im Thermalwasser vor, was den Abbau prinzipiell erleichtert. Dennoch wird bislang auf eine Förderung verzichtet, da befürchtet wird, dass der Abbau negative Auswirkungen auf die Bausubstanz in der Region haben könnte. Während bei großflächigen Eingriffen wie dem Kohletagebau solche Bedenken historisch kaum eine vergleichbare Rolle gespielt haben, zeigt sich hier eine besondere Sensibilität gegenüber potenziellen Risiken.


Schon 2012 legte das deutsche Bundesofrschungsministerium das Forschungs- und Entwicklungsprogramm \glqq Wirtschaftsstrategische Rohstoffe für den Hightech-Standort Deutschland\grqq auf.
%darauf eingehen

Status der Projekte heute:

\begin{table}[ht]
	\centering
	\begin{tabular}{p{6cm} p{6cm}}
		\toprule
		\textbf{Projekt} & \textbf{Status} \\
		\midrule
		Storkwitz, Deutsche Rohstoff AG & eingestellt, Rückgabe der Lizenz \\
		Ostsee-Sand & eingestellt \\
		Bayern & Platzhalter \\
		Harz, Stöbich Holding & eingestellt \\
		Rhein, Lanthan & Platzhalter \\
		Platzhalter & Platzhalter \\
		Platzhalter & Platzhalter \\
		Platzhalter & Platzhalter \\
		Platzhalter & Platzhalter \\
		Platzhalter & Platzhalter \\
		\bottomrule
	\end{tabular}
	\caption{Übersicht der Projekte und Status}
	\label{tab:projekt_status}
\end{table}

Auch vergleichsweise neuartige Technologien zur Rohstoffgewinnung scheiterten trotz staatlicher Förderungszuage am Rückzug von Investionen der Industrie.

Auch die Dauer der Errichtung von heimischer Produktion muss bedacht werden. Das Mount-Weld-Vorkommen in Australien (betrieben von Lynas) wurde 1966 im Rahmen einer aeromagnetischen Untersuchung entdeckt. Nach umfangreichen Erkundungen und Machbarkeitsstudien begann der industrielle Abbau im Jahr 2011. Die vorläufige Betriebserlaubnis für das vertikal integrierte Chemiewerk in Malaysia, das zur Weiterverarbeitung der geförderten Erze dient, wurde 2012 erteilt. Erst ab diesem Zeitpunkt konnte die Produktion industriefähiger Materialien aufgenommen werden. Aufgrund der radioaktiven Emissionen, die bei der Verarbeitung entstehen, war die Internationale Atomenergie-Organisation (IAEA) als Gutachterin in das Genehmigungsverfahren eingebunden. Vor dem Hintergrund dieser Erfahrungen erscheint es wenig wahrscheinlich, dass ein vergleichbares Projekt – zumal auf Grundlage deutlich weniger ergiebiger Lagerstätten – in Deutschland realisiert werden könnte. Insbesondere die Errichtung einer Chemieanlage zur Verarbeitung von Erzkonzentraten und der damit verbundenen Entstehung radioaktiver Rückstände und Emissionen dürfte in Deutschland auf erheblichen gesellschaftlichen und politischen Widerstand stoßen. Auch das Mountain-Pass-Projekt in den USA verdeutlicht die langen Entwicklungszeiträume, die mit der (Wieder-)Inbetriebnahme von Bergbau- und Aufbereitungsprojekten verbunden sind.

Im Gegensatz dazu ist der Abbau von Zirkon, das gemeinsam mit Titanmineralien aus sogenannten schweren Mineralsanden gewonnen wird, weniger problematisch. Dennoch wird der Import solcher Rohstoffe, etwa aus Mosambik oder Australien, oft als praktikabler angesehen. Für Deutschland kommt allenfalls die Weiterverarbeitung importierter Rohmaterialien in Betracht, wobei selbst diese Option aufgrund der bei der Erzaufbereitung anfallenden Abfälle und Emissionen mit Herausforderungen verbunden ist. Insgesamt gilt, dass dicht besiedelte Länder wie Deutschland aufgrund ökologischer, gesellschaftlicher und raumplanerischer Restriktionen kaum mehr für großflächigen Bergbau geeignet sind.



\subsection{Japan}

Japan Organization for Metals and Energy Security

Japan investiert somit global über eine staatliche Organisation in Abbaustätten, um seine eigene Versrogung sicherstellen zu können.

%Blueprint für eine EU- Rohstofforganisation

\subsection{USA}
Die USA unter Präsident Trump setzt mit ihrer Strategie gleich zu Beginn der Lieferkette an -- durch die Sicherung von Abbaurechten. Die USA betrachten die Abhängigkeit nicht nur aus einer rein industriellen Perspektive, sondern insbesondere auch aus einer verteidigungs- und sicherheitspolitischen Sicht, mit Fokus auf den im Verteidigungsbereich besonders wichtigen Mineralien wie Wolfram oder Gallium.

Die Trump-Regierung verhandelte daher mit der DRC und Ruanda über die Versorgung mit kritischen Rohstoffe im Tausch für Lösungsbeiträge hinsichtlich des Konflikts in der Grenzregion zwischen der DRK und Ruanda -- quasi ein \glqq Sicherheit für Rohstoffe\grqq-Deal ähnlich dem US-Ukraine-Deal. Im Gegensatz dazu lag hier das initiale Moment bei der DRK, jedoch zog sich die Einigung hin.

%Von OA und ggf nicht aktuell: Letztlich erwägt Washington ein Mineralienabkommen nicht nur mit der Demokratischen Republik Kongo, sondern auch mit Ruanda, die beide von einem umfassenden Friedensabkommen zwischen der Demokratischen Republik Kongo und Ruanda sowie der Demokratischen Republik Kongo und der M23-Rebellion abhängen werden. All diese Angebote werden einzeln extrem schwer zu erreichen sein und ein Paket, das alle noch mehr zufriedenstellt. Nichtsdestotrotz würde ein Abkommen der Demokratischen Republik Kongo einen entscheidenden Hebel an die Hand geben, um zu versuchen, eine peinliche und schädliche innenpolitische Krise zu lösen und gleichzeitig ihre Geschäftspartnerschaften zu diversifizieren, ihre Abhängigkeit von China zu verringern, Investitionen anzuziehen und Einkommen zu generieren. Für die Vereinigten Staaten würde es ein wichtiges Standbein in einem rohstoffreichen Land bieten und gleichzeitig die chinesische Dominanz herausfordern.



Unter Präsident Biden wurde die Rohstoffpolitik zudem auch dem grünen Wandel untergeordnet, sodass u. A. im IRA und im Infrastructure Investment and Jobs Act Steuer- und Kreditvorteile für Produzenten (d.h. also Stufe drei und vier) verankert waren und auch Produkte ansprach, in denen kritische Mineralien verarbeitet sind (konkret: Elektrofahrzeuge und Kaufanreize für diese). Der Übergang hin zu nachhaltigeren Technologien stellt jedoch keine Prioriät der Trump-Administration dar, sodass hier insbesondere Unsicherheit in der Rohstoffpolitik entsteht, da eine Fortsetzung oder ein Umschwenken des grünen Wandels nicht zweifelsfrei sichergestellt ist und somit Investitionen hemmt. In dieser Hinsicht kann der EU zumindest im Bereich des Green Deals eine höhere Policykontinuität bescheinigt werden, sodass dieser Effekt als geringer einzustufen ist.



So hat Trump u. A. auch den Vorschlag unterbreitet, Basen des US-Militärs zur Rohstoffverarbeitung zu nutzen; diese Idee wurde letztlich aber nicht weiter verfolgt.

Die sicherheitspolitische Wahrnehmung der Rohstoffabhängigkeit ist auf EU-Ebene noch nicht in diesem Maße eingetreten.


\section{Industriewirtschaftliche Bedeutung der Rohstoffversorgung}


2018 stellte man während der sog. 2. Statuskonferenz für Wirtschaftsstrategische Rohstoffe in Berlin fest, dass Deutschland durchaus Vorkommen an kritischen Rohstoffen habe -- es hätte vorher \glqq einfach niemand danach gesucht\grqq und an entsprechenden finanziellen Investitionen gemangelt.\autocite{VDINachrichten: Rohstoffe: Deutschland ist reich an seltenen Erden} Die Vorkommen in Sachsen waren so bereits zu DDR-Zeiten bekannt. Auch gerieten Vorkommen in Vergessenheit, und Bürger könnten sich vor einer \glqq Renaisance des Bergbaus\grqq wehren. Der Begriff einer Zeitenwende und die Erforderlichkeit einer solchen wurden also bereits hier genannt.





Es ergibt sich also die weitere Erkenntnis, dass die Nichtbeachtung der Rohstoffthematik auch daher rührt, dass die Versorgung mit Rohstoffen (sowohl generell als auch speziell aus nationalen Vorkommen), so simpel es erscheint, wirtschaftlich nicht attraktiv (genug) war.

\glqq Wenn die deutsche Industrie wirklich Interesse an dem Thema hätte, hätte sie sich dort beteiligen können\grqq \autocite{VDI Nachrichten Seltene Erden: Deutschland importiert lieber, als selbst zu fördern}


\subsection{Auswirkungen der Marktpreise für seltene Erden}

Auf der zweiten Frankfurter Regulierungskonferenz stellte ein Teilnehmer fest: Der Marktpreis für seltene Erden preist das Risiko keineswegs ein bzw. ist das Versorgungsrisiko bei kritischen Rohstoffe im Preis nicht reflektiert -- somit also \glqq zu günstig\grqq. Die Industrie setze hingegen auf den freien Markt und kaufe dementsprechend Rohstoffe auch zu diesen niedrigen Preisen ein, ohne aber hierbei Bemühungen oder Investitionen zur Entwicklung alternativer Versorgungsmöglichkeiten zu verfolgen.\autocite{VDI Nachrichten: Seltene Erden: Deutschland importiert lieber, als selbst zu fördern}

Die Preise für seltene Erden sind also niedrig, auch durch einen kontinuierlichen Abstieg des Preisniveaus nach 2010. Echte Versorgungsengpässe, die durch das Vorkommen bedingt sind, existieren nicht

Nachfrage
%Abbildung

Dieses Preisgefüge sorgt schließlich auch dafür, dass ein Abbau in Deutschland und den EU-Mitgliedsstaaten nicht rentabel ist und sich die Abhängigkeiten verstärken,

Die ökonomischen Auswirkungen einer solchen Abhängigkeit sind hinlänglich bekannt: 


\section{Rohstoffaußenpolitik}

Mitunter auch als \grqq Raw materials diplomacy\grqq bezeichntet, fallen in eine \glqq Rohstoffaußenpolitik\grqq eine Vielzahl von insbesondere strategischen Aspekten.[siehe zum Begriff der Rohstoffaußenpolitik]\autocite{Carry Müller Schulze: Elemente einer nachhaltigen Rohstoffaußenpolitik, SWP 2023}{Müller Saulich Schöneich Schulze: Von Der Rohstoffkonkurrenz zur nachhaltigen Rohstoffaußenpolitik, SWP 2022}

Erneut ist auch der Ruf nach einer Rohstoffaußenpolitik nicht neu oder überraschend einzustufen--insgesamt acht bereits 2010 vorgestellte \glqq Denkanstöße\grqq haben an Aktualität nicht verloren.\autocite{Mißfelder, Philipp: Eine Frage der Außenpolitik, IP 2010, 102ff.} Auch die deutsche Automobilindustrie fordert eine \glqq aktive\grqq Rohstoffaußenpolitik, nicht nur aus energiepolitischen Aspekten.\autocite{Müller, Hildegard: Wir brauchen eine aktive Rohstoff-Außenpolitik – sonst droht massiver Wohlstandsverlust, Handelsblatt, 06.04.2022}

Zunächst: Der Bedarf des Westens an kritischen Mineralien könnte zu einem starken Anstieg der Investitionen in rohstoffreichen Schwellenländern führen.

\subsection{Der Fall China}


Auch China nutzt seine nicht-energetischen Rohstoffvorkommen und -verarbeitungskapazitäten als außenpolitisches Instrument, mitunter auch agressiv-prononciert, insbesondere im afrikanischen Raum.\footnote{Die Auswirkung von Abhängigkeiten und Anfälligkeiten hinsichtlich rohstoffaußenpolitischer Spannungen und die Instrumemtalisierung dieser hat im energetischen Bereich insbesondere die Gasversorgungslage nach dem russischen Überfall auf die Ukraine gezeigt.} 

China sichert sich insbesondere durch erhebliche Finanzzusagen nicht nur den Zugang, sondern auch die Verarbeitung der extrahierten Mineralien.

Hierbei ist in China nicht die Rentabilität der treibende Faktor für die Rohstoffsicherung, sondern vielmehr die Sicherstellung der Versorgung, was durch günstige Kredite und Energepreise gefördert wurde.\autocite{Oxford Analytica: Critical minerals will be a global faultline for years}

%Beispiele für chinesische 

Dass die EU und ihre Mitgliedsstaaten keine derart geostrategisch motivierte und oder gar hegemoniale\footnote{QUOTATION NEEDED} (Rohstoff-)Außenpolitik verfolgen können, dürfte nicht weiter überraschen. Ferner sind die Auswirkungen chinesischer Exportkontrollen bzw. -beschränkungen im vorangegangenen Kapitel deutlich geworden, die mit zu den außenpolitischen Instrumenten gezählt werden können. Insofern kann der chinesische Weg zwar als effektiv zur Rohstoffsicherung eingestuft werden, kann aber darüber hinaus nicht als Blaupause für zukünftige Aktivität einer Union dienen, die sich den Art. 21, 22 EUV verpflichtet hat.

Bereits vor Jahren hat China indirekt Rohstoffe als Waffe instrumentalisiert -- und die Importabhängigkeit von China ist bedeutend größer als die von russischen Gasimporten.\autocite{BDI. Rohstoffkongress: Deutschland braucht eine strategische Rohstoffpolitik}

China hat somit seine Abnehmer zunächst durch niedrige Preise\footnote{ZUr Erinnerung: Insbesondere durch die Nicht-Einpreisung des Versorgungsrisikos} an sich gebunden und in Abhängigkeiten driften lassen -- und besitzt nun durch den Hebel der Rohstoffversorgung einen mächtigen solchen, kann es doch effektiv jederzeit die industrielle Produktion zumindest teilweise oder ganz zum Erliegen bringen.

Die Lieferkettendominanz Chinas im Bereich des Abbaus und der Verarbeitung wurden bereits beleuchtet. Chinas eigene Importversorgung ist durch Abkommen mit rohstoffreichen Ländern gesichert: so kontrolliert die Volksrepublik schätzungsweise 70 bis 80 \% der Kobaltminen in DRC.

Jedoch ist zu beachten, dass China in der eigentlichen Nachfrage nach kritischen Rohstoffen keine Dominanz aufweist, denn der Verbrauch im Verhältnis zum BIP ist vergleichsweise gering, sodass China (noch) auf den Exporteinnahmen durch die Ausfuhr der raffinierten, aber noch nicht eingesetzen kritischen mineralischen Rohstoffe angewiesen ist -- was eine Schwachstelle darstellen könnte.


\section{Rohstoffabkommen}
Im vorangegangenen Kapitel wurde die Gestalt des Instruments der Rohstoffpartnerschaften (und -abkommen) bereits aus instrumentuell-rechtlicher Sicht beleuchtet; eine genauere Betrachtung vor politisch-wirtschaftlichen Hintergründen der einzelnen Abkommen empfiehlt sich, um eine entsprechene zukünftige Entwicklung abschätzen zu können.

Trotz der frühen Erkenntnis der Erforderlichkeit wurden die ersten Partnerschaften erst 2021 von der EU geschlossen.

\subsection{Grundsätzliches}

https://single-market-economy.ec.europa.eu/sectors/raw-materials/areas-specific-interest/raw-materials-diplomacy_en

\subsection{Eurasien}

Serbien (JADAR)

Usbekistan

Kasachstan

Norwegen

\subsubsection{Grönland}

Grönland verfügt über bemerkenswerte Vorkommen an mineralischen Ressourcen, darunter Seltene Erden, Molybdän, Nickel, Kupfer und Grpahit, welche allesamt zu als kritische Rohstoffe gelten. Obwohl die geologische Kartierung dieser Vorkommen vergleichsweise weit fortgeschritten ist, bleibt ihre wirtschaftliche Nutzung bislang weitgehend aus.

Die begrenzte Erschließung dieser Ressourcen lässt sich vor allem durch die ökonomischen Rahmenbedingungen erklären. Die Gewinnung und Verarbeitung der Bodenschätze ist mit erheblichen finanziellen Aufwendungen verbunden. Neben den hohen Investitionskosten stellen auch die extremen klimatischen Bedingungen und die mangelnde Infrastruktur – etwa das Fehlen von Verkehrswegen und Umschlagplätzen – erhebliche Hürden dar. Diese Faktoren führen dazu, dass Grönlands Rohstoffsektor bislang kaum mit etablierten Förderregionen konkurrieren kann.

Trotz dieser Herausforderungen zeigen sowohl private Unternehmen als auch politische Akteure ein wachsendes Interesse an der Region. Die grönländische Regierung verfolgt eine Strategie zur wirtschaftlichen Diversifizierung und vergibt aktiv Förderlizenzen. Dabei wird besonderer Wert auf Umweltverträglichkeit und soziale Nachhaltigkeit gelegt, was sich in strengen regulatorischen Vorgaben widerspiegelt. Dennoch sind bislang nur wenige Projekte über die Planungsphase hinausgekommen.

Ein bemerkenswertes Beispiel für die zunehmende internationale Beteiligung ist ein Projekt, bei dem auch die Europäische Union involviert ist. Dieses Vorhaben stellt eine Ausnahme dar und verdeutlicht, dass geopolitische Interessen und die Sicherung von Versorgungsketten zunehmend Einfluss auf die Entwicklung des Rohstoffsektors in der Arktis nehmen.

Ukraine

\subsubsection{Asiatisch-pazifischer Raum}
Australien

\subsubsection{Afrika}
Trotz der bekannten Vorkommen spielen Explorationsinvestitionen in Afrika im globalen Vergelich eine untergordnete Rolle (2024: 10,4\% der globalen Explorationsausgaben), obgleich (Subsahara-)Afrika zu den kostengünstigsten Regionen für Mineralexplorationen zählt (Verhältnis Mineralwert zu Explorationsausgabe: 0,8; zum Vergleich: Australien 0,5; Kanada 0,6; Lateinamerika 0,3), sodass eine kritische Diskrepanz zwischen dem Potential der afrikanischen Staaten und den tatsächlichen Investitionen sichtbar wird und daher eine Neubetrachtung der Explorationsprioritisierung zu forcieren.\autocite{Baskaran, G.: Underexplored and Undervalued: Addressing Africa’s Mineral Exploration Gap, CSIS}


DRC

Seit dem Ausbruch des Konflikts im Osten der DRK steht das Abkommen und auch das mit Ruanda unter Druck durch möglichen illegalen Handel.
https://www.euronews.com/my-europe/2025/01/30/dr-congo-conflict-why-is-the-eu-under-pressure-to-reconsider-its-minerals-partnership-with

https://afripoli.org/navigating-critical-mineral-supply-chains-the-eus-partnerships-with-the-drc-and-zambia

Zambia

2015 verschob sich Sambia unter Präsident Lungu hin zu einem Rohstoffnationalismus 

Rwanda

Namibia
Lofdal 






\subsubsection{Amerika}
Chile

Argentinien

Kanada

\subsection{Exkurs zur Rohstoffaußenpolitik der USA unter Trump}

\subsubsection{Grönland}

\subsubsection{Rohstoffabkommen Ukraine}



\section{Konflikte und Zieldivergenzen}

Jedoch ist auch zu beachten: Die Engpässe bei der Rohstoffversorgung treiben Innovationen voran, um den Bedarf an dieser Ressource zu verringern – ein Risiko für Nationen, die Kapazitäten für bestimmte Mineralien aufbauen.

\subsection{NIMBY}

Umweltängste könnten die Bemühungen des Westens, sich weniger auf China zu verlassen, beeinträchtigen, da der Abbau und die Verarbeitung von Mineralien oft umweltschädlich sind.


	


\end{document}

