\documentclass[12pt,a4paper,oneside]{book} % 'oneside' für einseitigen Druck

% Kodierung, Sprache und Schrift
\usepackage[utf8]{inputenc} % Erlaubt die Verwendung von Umlauten
\usepackage[T1]{fontenc} % Bessere Schriftkodierung
\usepackage[ngerman]{babel} % Deutsche Lokalisierung

% Schriftarten
\usepackage{lmodern} % Modernere Schriftart, gut für Skalierbarkeit und Lesbarkeit

% Für Abbildungen
\usepackage{graphicx}
\graphicspath{{bilder/}} % Verzeichnis, in dem Bilder gespeichert sind

% Für Tabellen
\usepackage{booktabs}

% Für Links und PDF-Metadaten
\usepackage[hidelinks]{hyperref}
\hypersetup{
	pdftitle={Titel der Dissertation},
	pdfauthor={Autor},
	pdfsubject={Doktorarbeit in den Sozial- und Rechtswissenschaften},
	pdfkeywords={Schlüsselwörter},
}

\usepackage{array}
% Für Bibliographie - Anpassung für Geisteswissenschaften
%%\usepackage[style=authoryear-icomp,backend=biber]{biblatex}
%%\usepackage[backend=biber, style=authoryear-icomp]{biblatex}
%%\usepackage[backend=biber, style=verbose-trad1]{biblatex}
\usepackage[backend=biber, style=verbose-inote]{biblatex}
%%\usepackage{biblatex}
\addbibresource{literatur.bib} % Name der BibTeX-Datei
\DeclareNameAlias{author}{family-given} % Nachname des Autors zuerst

% Anpassung der Nummerierung mit Punkten
\renewcommand{\thechapter}{\arabic{chapter}.} % Kapitel: 1., 2., 3., ...
\renewcommand{\thesection}{\Alph{section}.} % Abschnitt: A., B., C., ...
\renewcommand{\thesubsection}{\Roman{subsection}.} % Unterabschnitt: I., II., III., ...
\renewcommand{\thesubsubsection}{\arabic{subsubsection}.} % Unterunterabschnitt: 1., 2., 3., ...
\renewcommand{\theparagraph}{\alph{paragraph}.} % Absatz: a., b., c., ...


% Für Fußnoten
%%\usepackage[bottom]{footmisc} % Fußnoten am Seitenende


% Anpassung der Kapitelüberschriften
%\usepackage{titlesec}
%\titleformat{\chapter}[hang]{\Huge\bfseries}{\thechapter\quad}{0pt}{\Huge\bfseries}

% Abstand der Fußnoten
\setlength{\footnotesep}{0.5cm}

% Tiefe der Nummerierung und des Inhaltsverzeichnisses
\setcounter{secnumdepth}{4} % Nummerierungstiefe einstellen
\setcounter{tocdepth}{4} % Inhaltsverzeichnistiefe einstellen

% Abstand zwischen Absätzen und kein Einzug
%\usepackage{parskip}
%\setlength{\parskip}{0.5em}
%\setlength{\parindent}{0pt}

% Für Abkürzungsverzeichnis
\usepackage[printonlyused]{acronym}

% Für Zitate und Theoreme (falls benötigt)
\usepackage{csquotes}

% Für Gesetzestexte, Zitate und andere strukturierte Texte
\usepackage{enumitem}

% Zeilenabstand auf 1.3
\usepackage{setspace}



% Beginn des Dokuments
\begin{document}
	% Hier beginnt der eigentliche Inhalt der Arbeit
	% ...

\chapter{Politisch-wirtschaftliche Betrachtung auf internationaler Ebene}

\section{Politische Leitlinien und Strategieentwicklungen}

\section{Wirtschaftliche Bedeutung der Rohstoffversorgung}

\section{Rohstoffaußenpolitik}

Mitunter auch als \grqq Raw materials diplomacy\grqq bezeichntet, fallen in eine \glqq Rohstoffaußenpolitik\grqq eine Vielzahl von insbesondere strategischen Aspekten.[siehe zum Begriff der Rohstoffaußenpolitik]\autocite{Carry Müller Schulze: Elemente einer nachhaltigen Rohstoffaußenpolitik, SWP 2023}{Müller Saulich Schöneich Schulze: Von Der Rohstoffkonkurrenz zur nachhaltigen Rohstoffaußenpolitik, SWP 2022}

Erneut ist auch der Ruf nach einer Rohstoffaußenpolitik nicht neu oder überraschend einzustufen--insgesamt acht bereits 2010 vorgestellte \glqq Denkanstöße\grqq haben an Aktualität nicht verloren.\autocite{Mißfelder, Philipp: Eine Frage der Außenpolitik, IP 2010, 102ff.} Auch die deutsche Automobilindustrie fordert eine \glqq aktive\grqq Rohstoffaußenpolitik, nicht nur aus energiepolitischen Aspekten.\autocite{Müller, Hildegard: Wir brauchen eine aktive Rohstoff-Außenpolitik – sonst droht massiver Wohlstandsverlust, Handelsblatt, 06.04.2022}

\subsection{Rohstoffabkommen}
Im vorangegangenen Kapitel wurde die Gestalt des Instruments der Rohstoffpartnerschaften (und -abkommen) bereits aus instrumentuell-rechtlicher Sicht beleuchtet; eine genauere Betrachtung vor politisch-wirtschaftlichen Hintergründen der einzelnen Abkommen empfiehlt sich, um eine entsprechene zukünftige Entwicklung abschätzen zu können.

Trotz der frühen Erkenntnis der Erforderlichkeit wurden die ersten Partnerschaften erst 2021 von der EU geschlossen.

\subsubsection{Grundsätzliches}

https://single-market-economy.ec.europa.eu/sectors/raw-materials/areas-specific-interest/raw-materials-diplomacy_en

\subsubsection{Eurasien}

Serbien (JADAR)

Usbekistan

Kasachstan

Norwegen

Grönland

Ukraine

\subsubsection{Asiatisch-pazifischer Raum}
Australien

\subsubsection{Afrika}
DRC

Seit dem Ausbruch des Konflikts im Osten der DRK steht das Abkommen und auch das mit Ruanda unter Druck durch möglichen illegalen Handel.
https://www.euronews.com/my-europe/2025/01/30/dr-congo-conflict-why-is-the-eu-under-pressure-to-reconsider-its-minerals-partnership-with

https://afripoli.org/navigating-critical-mineral-supply-chains-the-eus-partnerships-with-the-drc-and-zambia

Zambia

Rwanda

Namibia
\subsubsection{Amerika}
Chile

Argentinien

Kanada

\subsection{Exkurs zur Rohstoffaußenpolitik der USA unter Trump}

\subsubsection{Grönland}

\subsubsection{Rohstoffabkommen Ukraine}



\section{Konflikte und Zieldivergenzen}

	


\end{document}

