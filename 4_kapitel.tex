\documentclass[12pt,a4paper,oneside]{book} % 'oneside' für einseitigen Druck

% Kodierung, Sprache und Schrift
\usepackage[utf8]{inputenc} % Erlaubt die Verwendung von Umlauten
\usepackage[T1]{fontenc} % Bessere Schriftkodierung
\usepackage[ngerman]{babel} % Deutsche Lokalisierung

% Schriftarten
\usepackage{lmodern} % Modernere Schriftart, gut für Skalierbarkeit und Lesbarkeit

% Für Abbildungen
\usepackage{graphicx}
\graphicspath{{bilder/}} % Verzeichnis, in dem Bilder gespeichert sind

% Für Tabellen
\usepackage{booktabs}

% Für Links und PDF-Metadaten
\usepackage[hidelinks]{hyperref}
\hypersetup{
	pdftitle={Titel der Dissertation},
	pdfauthor={Autor},
	pdfsubject={Doktorarbeit in den Sozial- und Rechtswissenschaften},
	pdfkeywords={Schlüsselwörter},
}

\usepackage{array}
% Für Bibliographie - Anpassung für Geisteswissenschaften
%%\usepackage[style=authoryear-icomp,backend=biber]{biblatex}
%%\usepackage[backend=biber, style=authoryear-icomp]{biblatex}
%%\usepackage[backend=biber, style=verbose-trad1]{biblatex}
\usepackage[backend=biber, style=verbose-inote]{biblatex}
%%\usepackage{biblatex}
\addbibresource{literatur.bib} % Name der BibTeX-Datei
\DeclareNameAlias{author}{family-given} % Nachname des Autors zuerst

% Anpassung der Nummerierung mit Punkten
\renewcommand{\thechapter}{\arabic{chapter}.} % Kapitel: 1., 2., 3., ...
\renewcommand{\thesection}{\Alph{section}.} % Abschnitt: A., B., C., ...
\renewcommand{\thesubsection}{\Roman{subsection}.} % Unterabschnitt: I., II., III., ...
\renewcommand{\thesubsubsection}{\arabic{subsubsection}.} % Unterunterabschnitt: 1., 2., 3., ...
\renewcommand{\theparagraph}{\alph{paragraph}.} % Absatz: a., b., c., ...


% Für Fußnoten
%%\usepackage[bottom]{footmisc} % Fußnoten am Seitenende


% Anpassung der Kapitelüberschriften
%\usepackage{titlesec}
%\titleformat{\chapter}[hang]{\Huge\bfseries}{\thechapter\quad}{0pt}{\Huge\bfseries}

% Abstand der Fußnoten
\setlength{\footnotesep}{0.5cm}

% Tiefe der Nummerierung und des Inhaltsverzeichnisses
\setcounter{secnumdepth}{4} % Nummerierungstiefe einstellen
\setcounter{tocdepth}{4} % Inhaltsverzeichnistiefe einstellen

% Abstand zwischen Absätzen und kein Einzug
%\usepackage{parskip}
%\setlength{\parskip}{0.5em}
%\setlength{\parindent}{0pt}

% Für Abkürzungsverzeichnis
\usepackage[printonlyused]{acronym}

% Für Zitate und Theoreme (falls benötigt)
\usepackage{csquotes}

% Für Gesetzestexte, Zitate und andere strukturierte Texte
\usepackage{enumitem}

% Zeilenabstand auf 1.3
\usepackage{setspace}



% Beginn des Dokuments
\begin{document}
	% Hier beginnt der eigentliche Inhalt der Arbeit
	% ...
	
\chapter{Der Gegenstand des Rohstoffverwaltungsrecht}

Ein direktes „Rohstoffverwaltungsrecht“ ist im Primärrecht der EU nicht ausdrücklich normiert. Allerdings können Regelungen zur Verwaltung und Kontrolle von Rohstoffen als Teil der allgemeinen Verwaltungsaufgaben der EU verstanden werden. Hierzu zählen auch Maßnahmen, die auf Transparenz und Rechenschaftspflicht im Rohstoffsektor abzielen.


Es ist relevant festzustellen, dass die EU auf keine Kompetenz zurückgreifen kann, mit der sie das allgemeine Verwaltungsrecht der Mitgliedsstaaten hinsichtlich des Unionsvollzugs angleichen könnte.\autocite{Sangi/Gärditz in Karpenstein/Kotzur/Vasel 2024, §32. Rn. 3, Callies/Ruffert/Kahl EUV Art. 4 Rn. 127,} 

Ein Vollzug auf Unionsebene ist durchaus möglich -- jedoch ist es hierbei maßgeblich, dass Primärrecht den Organen eine solche Kompetenz zuweist die unmittelbar zur Erreichung des Gegenstandsziel beitragen, wie beispielsweise im umweltrechtlichen Bereich wo Verwaltungsverfahren und -prozesse ausdrücklich geregelt werden.\autocite{Sangi/Gärditz in Karpenstein/Kotzur/Vasel 2024, §32. Rn. 3f.}

\subsection{Anwendung des Art. 41 GrCH}
Grundsätzlich bindet Art. 41 GrCH (\glqq Recht auf eine gute Verwaltung\grqq)  dem Wortlaut nach nur die Organe der Union im Rahmen von Verwaltungsverfahren, sodass solche in Bezug auf mitgliedsstaatliche Verwaltung zunächst nicht erfasst sind bzw. letztere keine Wirkung des Art. 41 GrCh erfahren. Insbesondere bei Betrachtung des Art, 4 III EUV sollte jedoch die Bindungswirkung aus Gründen des Effektivitätsgebots\footnote{C93/12, C-2/06, C-432/05} ersichtlich wirken.

\subsection{Verfahrensautonomie der EU-Mitgliedsstaaten}
Im Rahmen des Mehrebenenverwaltungssystems der Union verwaltet diese nur wenige Rechtsgebiete eigenständig und zudem nicht immer vollständig, denn der Schwerpunkt der Unionsrechtsverwaltung und -vollzugs liegt klar bei den Behörden der Mitgliedsstaaten, die unter Anwendung des nationalen Rechts und unter Beachtung der unionsrechtlichen Prämissen somit den unteren Endpunkt des Unionsverwaltungsrechts bilden.\autocite{Schill/Krenn, Recht der EU, Art 4, Rn 88}
Die Relevanz der Verfahrensautonomie der Mitgliedsstaaten ergibt sich aus Art. 4 III 2 EUV, denn sie \glqq ergreifen alle geeigneten Maßnahmen allgemeiner oder besonderer Art zur Erfüllung der Verpflichtungen, die sich aus den Verträgen oder den Handlungen der Organe der Union ergeben\grqq, etabliert somit also Art Loyalitätsverpflichtung für die Zusammenarbeit, die jedoch Raum für die Ausgestaltung dieser Erüfllung der Verpfichtung lässt, aber somit auf den indirekten Vollzug verweist.\footnote{Ein Beispiel hierfür ist das Urteil in der Rechtssache Rewe-Zentralfinanz eG und Rewe-Zentral AG gegen Landwirtschaftskammer für das Saarland (C-33/76, „Rewe/Comet“), in dem der EuGH bestätigte, dass die Mitgliedstaaten zwar verpflichtet sind, den effektiven Schutz der durch das Unionsrecht gewährleisteten Rechte sicherzustellen, aber die Wahl der Mittel und Verfahren innerhalb ihres nationalen Rechts bleibt. Der Grundsatz der Verfahrensautonomie ist somit auf den EuGH zurückzuführen, siehe auch C-201/2, Rn. 67.} Die Mitgliedsstaaten vollziehen EU-Recht also im Rahmen der eigenen Kompetenz und auf Grundlage der \glqq Ausübung originärer mitgliedsstaatlicher Hoheitsgewalt\grqq und eben nicht durch Delegation von Unionsgewalt an die Mitgliedsstaaten.\autocite{Sangi/Gärditz in Karpenstein/Kotzur/Vasel 2024, §32. Rn. 2}

Die Verfahrensautonomie ist also ein zentraler Grundsatz, denn grundsätzlich obliegt der Vollzug einzelner Titel den Mitgliedsstaaten, mit der Einschränkung durch entsprechende Vorgaben der Union, sodass die Verfahrensautonomie der Mitgliedsstaaten gegenüber der zunehmenden Europäisierung der nationalen (Verwaltungs-)Rechtssysteme als eine \glqq ungeklärte Kardinalsfrage\grqq \autocite{Ludiwgs, NVwZ 2018, 1417} betrachtet werden kann und mit zu den kontroversesten Fragestellungen des Europäischen Verwaltungsrechts zu zählen ist.\autocite[Die Thematik der Verfahrensautonomie wurde in der Literatur bereits ausführlich behandelt, siehe exemplarisch]{Krönke, Verfahrensautonomie der EU-Mitgliedsstaaten}

Auch die Verfahrensautonomie hat Grenzen, spätestens bei der Betrachtung der Art. 4 III EUV und Art. 19 I 2 EUV in Bezug auf die Loyalitätsverpflichtung der Mitgliedsstaaten;

Insbesondere der Effektivitätsgrundsatz erscheint im Kontext dieser Untersuchung, dürfen die mitgliedsstaatlichen Instanzen die Ausführung unionsrechtlich begründeter Ansprüche nicht erschweren oder praktisch unmöglich gestalten.\autocite{Sangi/Gärditz in Karpenstein/Kotzur/Vasel 2024, §32. Rn. 6}

Der EuGH hat sich zwar bereits intensiv mit der Verfahrensautonomie der Mitgliedsstaaten beschäftigt, 


Dies bedeutet, dass die Mitgliedstaaten im Rahmen des Rohstoffverwaltungsrechts grundsätzlich befugt sind, nationale Verfahren zur Verwaltung und Sicherung von Rohstoffressourcen zu gestalten, solange sie die Ziele und Vorgaben etwaiger EU-Verordnungen und -Richtlinien erfüllen, bzw. aktuell den strategischen Kontext der Rechtsakte mittragen.

Die Relevanz der Verfahrensautonomie wird auch in der Rohstoffstrategie der EU deutlich, wie sie in verschiedenen politischen Dokumenten und Mitteilungen der Kommission, wie dem „Rohstoffinitiativ“ (KOM(2008) 699) und dem „Kommissionsbericht über die kritischen Rohstoffe“ (KOM(2020) 474) zum Ausdruck kommt. Diese Dokumente betonen die Notwendigkeit einer koordinierten europäischen Strategie, erkennen jedoch die zentrale Rolle der Mitgliedstaaten bei der Implementierung und Durchführung von Maßnahmen zur Sicherstellung der Versorgung mit kritischen Rohstoffen an.

Der EuGH hat in seiner Rechtsprechung mehrfach klargestellt, dass die Verfahrensautonomie der Mitgliedstaaten nicht zu einer Umgehung der Ziele und Prinzipien des Unionsrechts führen darf. Im Urteil in der Rechtssache Francovich und Bonifaci gegen Italien (C-6/90 und C-9/90) bekräftigte der EuGH, dass Mitgliedstaaten, obwohl sie über eine gewisse Verfahrensautonomie verfügen, sicherstellen müssen, dass die aus dem Unionsrecht resultierenden Rechte effektiv geschützt und umgesetzt werden. Inwieweit dies aber auf den rohstofflichen Bereich zutrifft, bleibt fraglich, denn es fehlt einerseits an einem rohstoffrelevanten Bereich der durch das Unionsrecht geschützt ist, andererseits ist die Autonomie als derart hoch einzutsufen, dass die Autonomie wohl nur selten überreizt wird.

\section{Zum Rechtsschutz und Rechtsmittelverfahren}

Aufgrund mangelnder Praxiserfahrungen kann hier nur auf die generellen Feststellungen zur Weite des verwalungsgerichtlichen Rechtsschutzes auf Unionsebene verwiesen werden.

\end{document}

