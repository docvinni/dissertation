\documentclass[12pt,a4paper,oneside]{book} % 'oneside' für einseitigen Druck

% Kodierung, Sprache und Schrift
\usepackage[utf8]{inputenc} % Erlaubt die Verwendung von Umlauten
\usepackage[T1]{fontenc} % Bessere Schriftkodierung
\usepackage[ngerman]{babel} % Deutsche Lokalisierung

% Schriftarten
\usepackage{lmodern} % Modernere Schriftart, gut für Skalierbarkeit und Lesbarkeit

% Für Abbildungen
\usepackage{graphicx}
\graphicspath{{bilder/}} % Verzeichnis, in dem Bilder gespeichert sind

% Für Tabellen
\usepackage{booktabs}

% Für Links und PDF-Metadaten
\usepackage[hidelinks]{hyperref}
\hypersetup{
	pdftitle={Titel der Dissertation},
	pdfauthor={Autor},
	pdfsubject={Doktorarbeit in den Sozial- und Rechtswissenschaften},
	pdfkeywords={Schlüsselwörter},
}

% Für Bibliographie - Anpassung für Geisteswissenschaften
%%\usepackage[style=authoryear-icomp,backend=biber]{biblatex}
%%\usepackage[backend=biber, style=authoryear-icomp]{biblatex}
%%\usepackage[backend=biber, style=verbose-trad1]{biblatex}
\usepackage[backend=biber, style=verbose-inote]{biblatex}
%%\usepackage{biblatex}
\addbibresource{literatur.bib} % Name der BibTeX-Datei
\DeclareNameAlias{author}{family-given} % Nachname des Autors zuerst

% Anpassung der Nummerierung mit Punkten
\renewcommand{\thechapter}{\arabic{chapter}.} % Kapitel: 1., 2., 3., ...
\renewcommand{\thesection}{\Alph{section}.} % Abschnitt: A., B., C., ...
\renewcommand{\thesubsection}{\Roman{subsection}.} % Unterabschnitt: I., II., III., ...
\renewcommand{\thesubsubsection}{\arabic{subsubsection}.} % Unterunterabschnitt: 1., 2., 3., ...
\renewcommand{\theparagraph}{\alph{paragraph}.} % Absatz: a., b., c., ...


% Für Fußnoten
%%\usepackage[bottom]{footmisc} % Fußnoten am Seitenende


% Anpassung der Kapitelüberschriften
%\usepackage{titlesec}
%\titleformat{\chapter}[hang]{\Huge\bfseries}{\thechapter\quad}{0pt}{\Huge\bfseries}

% Abstand der Fußnoten
\setlength{\footnotesep}{0.5cm}

% Tiefe der Nummerierung und des Inhaltsverzeichnisses
\setcounter{secnumdepth}{4} % Nummerierungstiefe einstellen
\setcounter{tocdepth}{4} % Inhaltsverzeichnistiefe einstellen

% Abstand zwischen Absätzen und kein Einzug
%\usepackage{parskip}
%\setlength{\parskip}{0.5em}
%\setlength{\parindent}{0pt}

% Für Abkürzungsverzeichnis
\usepackage[printonlyused]{acronym}

% Für Zitate und Theoreme (falls benötigt)
\usepackage{csquotes}

% Für Gesetzestexte, Zitate und andere strukturierte Texte
\usepackage{enumitem}

% Zeilenabstand auf 1.3
\usepackage{setspace}

% Beginn des Dokuments
\begin{document}
	% Hier beginnt der eigentliche Inhalt der Arbeit
	% ...
	
	
	\chapter{Rohstoffe als Treibstoff der Industrie}
	
	Das folgende Kapitel dient zunächst der Erfassung der notwendigen Begriffsdefinitionen, der Einschätzung der Relevanz der Rohstoffversorgung und erwarteten -abhängigkeit des deutschen Automobilsektors, und schließlich der abschließenden Beurteilung der Ergebnisse.
	
	\section{Begriffsdefinitionen}
	% Hier auf die unklare Definition von Rohstoff eingehen
	% Auch erwähnen, das ein Rohstoff nicht unbedingt ein Mineral oÄ sein muss, sondern gerade im Automobilsektor auch ein Teil (Stoßstange oÄ) umfassen
	% Unterschied Rohstoff und Ressource
	
	\subsection{Rohstoff}
	
	
 Insbesondere ist die Abgrenzung der hier betrachteteten Rohstoffe zu anderen Rohstoffen zu erwähnen. So fällt insbesondere der Bereich der Agrarrohstoffe bzw. der landwirtschaftlichen Erzeugnisse nicht in den Betrachtungsbereich dieser Arbeit, ebenso wie originäre Energierohstoffe und primäre Energieträger. Für die Agrarrohstoffe existiert eine Fülle an Sonder- und speziell anwendbaren Regelungen, die darüber hinaus für den maßgeblich fokussierten Sektor keinerlei Relevanz haben. Ähnliches gilt auch für die primären Energierohstoffe: zwar sind diese im industriellen Kontext nicht weniger relevant, jedoch gelten auch hier insbesondere aus energierohstoffspezifischer Sicht Sonderregelungen, die hier nicht weiter EInzug halten sollen.
	
	
	Für den weiteren Gang der Arbeit soll festgehalten, aber nicht weiter untersucht werden, dass Rohstoffe (auch bzw. insbesondere im Sinne des Binnenmarktes) als \textit{Waren} zu verstehen sind.\autocite{Schorkopf, Rohstoffverwaltung, Rn. 6ff.}
	
	
	\subsubsection{Das Merkmal der Kritikalität}
	Wie auch beim übergeordneten Begriff des "Rohstoffs" existieren fürderhin auch keine allgemeingültige Definitionen oder generelle Kriterien für einen \textit{kritischen} Rohstoff. Im Bereich der EU-Verordnung zu kritischen Rohstoffen (s. dazu \ref{EU-Verordnung}) wird beispielsweise an das Versorgungsrisiko und ihre wirtschaftliche Bedeutung angeknüpft.
	
	\subsection{Rohstoffwirtschaft}
	
	\subsection{Rohstoffverwaltungsrecht}
	
	\chapter{Der Gegenstand des Rohstoffverwaltungsrecht}
	
	Ein direktes „Rohstoffverwaltungsrecht“ ist im Primärrecht der EU nicht ausdrücklich normiert. Allerdings können Regelungen zur Verwaltung und Kontrolle von Rohstoffen als Teil der allgemeinen Verwaltungsaufgaben der EU verstanden werden. Hierzu zählen auch Maßnahmen, die auf Transparenz und Rechenschaftspflicht im Rohstoffsektor abzielen.
	
	
	Es ist relevant festzustellen, dass die EU auf keine Kompetenz zurückgreifen kann, mit der sie das allgemeine Verwaltungsrecht der Mitgliedsstaaten hinsichtlich des Unionsvollzugs angleichen könnte.\autocite{Sangi/Gärditz in Karpenstein/Kotzur/Vasel 2024, §32. Rn. 3, Callies/Ruffert/Kahl EUV Art. 4 Rn. 127,} 
	
	Ein Vollzug auf Unionsebene ist durchaus möglich -- jedoch ist es hierbei maßgeblich, dass Primärrecht den Organen eine solche Kompetenz zuweist die unmittelbar zur Erreichung des Gegenstandsziel beitragen, wie beispielsweise im umweltrechtlichen Bereich wo Verwaltungsverfahren und -prozesse ausdrücklich geregelt werden.\autocite{Sangi/Gärditz in Karpenstein/Kotzur/Vasel 2024, §32. Rn. 3f.}
	
	\subsection{Anwendung des Art. 41 GrCH}
	Grundsätzlich bindet Art. 41 GrCH (\glqq Recht auf eine gute Verwaltung\grqq)  dem Wortlaut nach nur die Organe der Union im Rahmen von Verwaltungsverfahren, sodass solche in Bezug auf mitgliedsstaatliche Verwaltung zunächst nicht erfasst sind bzw. letztere keine Wirkung des Art. 41 GrCh erfahren. Insbesondere bei Betrachtung des Art, 4 III EUV sollte jedoch die Bindungswirkung aus Gründen des Effektivitätsgebots\footnote{C93/12, C-2/06, C-432/05} ersichtlich wirken.
	
	\subsection{Verfahrensautonomie der EU-Mitgliedsstaaten}
	Im Rahmen des Mehrebenenverwaltungssystems der Union verwaltet diese nur wenige Rechtsgebiete eigenständig und zudem nicht immer vollständig, denn der Schwerpunkt der Unionsrechtsverwaltung und -vollzugs liegt klar bei den Behörden der Mitgliedsstaaten, die unter Anwendung des nationalen Rechts und unter Beachtung der unionsrechtlichen Prämissen somit den unteren Endpunkt des Unionsverwaltungsrechts bilden.\autocite{Schill/Krenn, Recht der EU, Art 4, Rn 88}
	Die Relevanz der Verfahrensautonomie der Mitgliedsstaaten ergibt sich aus Art. 4 III 2 EUV, denn sie \glqq ergreifen alle geeigneten Maßnahmen allgemeiner oder besonderer Art zur Erfüllung der Verpflichtungen, die sich aus den Verträgen oder den Handlungen der Organe der Union ergeben\grqq, etabliert somit also Art Loyalitätsverpflichtung für die Zusammenarbeit, die jedoch Raum für die Ausgestaltung dieser Erüfllung der Verpfichtung lässt, aber somit auf den indirekten Vollzug verweist.\footnote{Ein Beispiel hierfür ist das Urteil in der Rechtssache Rewe-Zentralfinanz eG und Rewe-Zentral AG gegen Landwirtschaftskammer für das Saarland (C-33/76, „Rewe/Comet“), in dem der EuGH bestätigte, dass die Mitgliedstaaten zwar verpflichtet sind, den effektiven Schutz der durch das Unionsrecht gewährleisteten Rechte sicherzustellen, aber die Wahl der Mittel und Verfahren innerhalb ihres nationalen Rechts bleibt. Der Grundsatz der Verfahrensautonomie ist somit auf den EuGH zurückzuführen, siehe auch C-201/2, Rn. 67.} Die Mitgliedsstaaten vollziehen EU-Recht also im Rahmen der eigenen Kompetenz und auf Grundlage der \glqq Ausübung originärer mitgliedsstaatlicher Hoheitsgewalt\grqq und eben nicht durch Delegation von Unionsgewalt an die Mitgliedsstaaten.\autocite{Sangi/Gärditz in Karpenstein/Kotzur/Vasel 2024, §32. Rn. 2}
	
	Die Verfahrensautonomie ist also ein zentraler Grundsatz, denn grundsätzlich obliegt der Vollzug einzelner Titel den Mitgliedsstaaten, mit der Einschränkung durch entsprechende Vorgaben der Union, sodass die Verfahrensautonomie der Mitgliedsstaaten gegenüber der zunehmenden Europäisierung der nationalen (Verwaltungs-)Rechtssysteme als eine \glqq ungeklärte Kardinalsfrage\grqq \autocite{Ludiwgs, NVwZ 2018, 1417} betrachtet werden kann und mit zu den kontroversesten Fragestellungen des Europäischen Verwaltungsrechts zu zählen ist.\autocite[Die Thematik der Verfahrensautonomie wurde in der Literatur bereits ausführlich behandelt, siehe exemplarisch]{Krönke, Verfahrensautonomie der EU-Mitgliedsstaaten}
	
	Auch die Verfahrensautonomie hat Grenzen, spätestens bei der Betrachtung der Art. 4 III EUV und Art. 19 I 2 EUV in Bezug auf die Loyalitätsverpflichtung der Mitgliedsstaaten;
	
	Insbesondere der Effektivitätsgrundsatz erscheint im Kontext dieser Untersuchung, dürfen die mitgliedsstaatlichen Instanzen die Ausführung unionsrechtlich begründeter Ansprüche nicht erschweren oder praktisch unmöglich gestalten.\autocite{Sangi/Gärditz in Karpenstein/Kotzur/Vasel 2024, §32. Rn. 6}
	
	Der EuGH hat sich zwar bereits intensiv mit der Verfahrensautonomie der Mitgliedsstaaten beschäftigt, 
	
	
	Dies bedeutet, dass die Mitgliedstaaten im Rahmen des Rohstoffverwaltungsrechts grundsätzlich befugt sind, nationale Verfahren zur Verwaltung und Sicherung von Rohstoffressourcen zu gestalten, solange sie die Ziele und Vorgaben etwaiger EU-Verordnungen und -Richtlinien erfüllen, bzw. aktuell den strategischen Kontext der Rechtsakte mittragen.
	
	Die Relevanz der Verfahrensautonomie wird auch in der Rohstoffstrategie der EU deutlich, wie sie in verschiedenen politischen Dokumenten und Mitteilungen der Kommission, wie dem „Rohstoffinitiativ“ (KOM(2008) 699) und dem „Kommissionsbericht über die kritischen Rohstoffe“ (KOM(2020) 474) zum Ausdruck kommt. Diese Dokumente betonen die Notwendigkeit einer koordinierten europäischen Strategie, erkennen jedoch die zentrale Rolle der Mitgliedstaaten bei der Implementierung und Durchführung von Maßnahmen zur Sicherstellung der Versorgung mit kritischen Rohstoffen an.
	
	Der EuGH hat in seiner Rechtsprechung mehrfach klargestellt, dass die Verfahrensautonomie der Mitgliedstaaten nicht zu einer Umgehung der Ziele und Prinzipien des Unionsrechts führen darf. Im Urteil in der Rechtssache Francovich und Bonifaci gegen Italien (C-6/90 und C-9/90) bekräftigte der EuGH, dass Mitgliedstaaten, obwohl sie über eine gewisse Verfahrensautonomie verfügen, sicherstellen müssen, dass die aus dem Unionsrecht resultierenden Rechte effektiv geschützt und umgesetzt werden. Inwieweit dies aber auf den rohstofflichen Bereich zutrifft, bleibt fraglich, denn es fehlt einerseits an einem rohstoffrelevanten Bereich der durch das Unionsrecht geschützt ist, andererseits ist die Autonomie als derart hoch einzutsufen, dass die Autonomie wohl nur selten überreizt wird.
	
	\section{Zum Rechtsschutz und Rechtsmittelverfahren}
	
	Aufgrund mangelnder Praxiserfahrungen kann hier nur auf die generellen Feststellungen zur Weite des verwalungsgerichtlichen Rechtsschutzes auf Unionsebene verwiesen werden.
	
	
	\glqq Als europäische Rohstoffverwaltung ist das Regulieren grenzüberschreitender Wirtschaftssachverhalte unter Beteiligung von Trägern hoheitlicher Gewalt auf der Grundlage des rechtlichen Ordnungsrahmens der Europäischen Union zu bezeichnen\grqq \autocite{Schorkopf, Europäische Rohstoffverwaltung, Rn. 2}
	
	\subsection{Slowbalization}
	Das Phänomen „Slowbalization”\footnote{\textit{alternativ Slowbalisation}} trat prominent auf dem Titelbild des „The Economist” zum Jahresbeginn 2019\autocite{economist_slowbalisation_2019} in Erscheinung, und beschreibt die zunehmende Verlangsamung oder Umgestaltung der bisherigen Trends der Globalisierung und des internationalen Handels, sodass die Wirtschaftsleistung schneller steigt als der globale Handel%für Fußnote:The Economist: The steam has gone out of globalisation. The Economist, 24. Januar 2019. https://www.economist.com/leaders/2019/01/24/the-steam-has-gone-out-of-globalisation. .\autocite{bibid} 
	Die Slowbalization stellt eine bedeutende Herausforderung dar und wirkt sich auf verschiedene Wirtschaftszweige aus, darunter die Automobilindustrie in Deutschland. Bei dieser „Globalisierung im Rückwärtsgang“ ist insbesondere das deutsche Export- und Globalisierungsmodell betroffen, sodass eine „neue Wirtschaftspolitik“ erforderlich sei.\autocite{maier_globalisierung_2019} Auch gibt es Tendenzen hin zu einer De-Globalisierung – diese könne aber nicht im Interesse einer Automobilindustrie sein, die auf Diversifikation, Effizienz und Resilienz setze,\autocite{rade_globalisierung_2022} sodass Slowbalization als alternatives Konzept hier hervortritt. Insbesondere unter weltwirtschaftlichen Krisen wie die Covid-19-Pandemie hat die Thematik an Aufmerksamkeit gewonnen, sodass die akademische Auseinandersetzung insbesondere die Abgrenzung zwischen De-Globalisierung und Slowbalisierung fokussiert.\autocites{dalla_longa_urban_2023}{inferrera_globalisation_2021}
	
	
	\subsection{Decoupling}
	Decoupling, also die Entkopplung zweier oder mehrerer Handelsakteure, ist ein denkbares Szenario im Bereich der Handelspolitik – so beispielsweise beschrieben zwischen der EU und China.\autocite{fuest_geopolitische_2022} Decoupling steht in engem Bezug zur Begrifflichkeit Derisking. Insbesondere die Nutzung des Terminus Decoupling wird inzwischen, auch von deutscher Seite aus, vermieden, um stattdessen auf ein „entschärftes” Derisking zurückzugreifen\footnote{\textit{Siehe hierzu auch} Rede U. von der Leyens, Weltwirtschaftsforum Davos, 17. Januar 2023: „EU must seek to de-risk rather than decouple from China”.}, wobei inhaltlich derselbe Prozess gemeint ist. Es ist nicht eindeutig, wo der Begriff erstmalig in der öffentlichen Diskussion auftauchte, lässt sich jedoch auf die EU-Kommission und die deutsche Bundesregierung zurückführen.\autocite{kormbaki_wettstreit_2023} Eine direkte Entkopplung, also ein wortwörtliches Decoupling, scheint derzeit keine gangbare Alternative, wie auch vom Hohen Vertreter der Europäischen Union für Außen- und Sicherheitspolitik, Josep Borrell, beschrieben: Diese Möglichkeit werde nicht in Betracht gezogen, selbst eine gewollte Entkopplung sei nicht möglich.\autocite{europaischer_auswartiger_dienst_foreign_2023} Die Frage nach einem Decoupling stellt sich insbesondere in Bezug auf China. Neben der globalen Schlüsselposition hinsichtlich der eigenen Rohstoffreserven tritt China zunehmend auch als Rohstoffakteur im Ausland auf, verfolgt eine nationalistisch geprägte und vollkommen staatlich gesteuerte Rohstoffpolitik und verhindert privatwirtschaftliche Aktivität, die bereits vor fast zehn Jahren identifiziert wurde.\autocite{sausmikat_chinas_2015} China besitzt zudem die global größten Vorkommen an kritischen Materialien, die auch für die Fahrzeugbatterie- und Halbleiterproduktion vonnöten sind, insbesondere Lithium und andere Seltene Erden. Grundsätzlich sei der Ausweg aus der Importabhängigkeit die Erschließung neuer Lagerstätten und Recycling, zusätzlich verstärkt durch Änderung nationaler und multinationaler Strategien.\autocite{wissenschaftliche_dienste_des_deutschen_bundestages_seltene_2022}

	
	\subsection{Nearshoring}
	Im allgemeinen wirtschaftlichen Verständnis bezeichnet Nearshoring zunächst eine Sonderform des Offshorings, in Form von Verlagerung von Produktion ins nahe Ausland.\autocite{bendel_nearshoring_2022} Der Begriff des Nearshorings ist auch als „Reshoring“ oder „Backshoring“ bekannt, bezeichnet jedoch stets die Rückverlagerung von Produktion aus vielfältigen Gründen in das Heimatland eines Betriebes, fast ausnahmslos bezieht sich dies jedoch auf die Staaten des sog. Globalen Nordens.\autocite{butollo_deglobalisierung_2022} Nearshoring beeinflusst also die Art und Weise, wie auch deutsche Automobilunternehmen Rohstoffe beschaffen und ihre Lieferketten organisieren, u. U. also ebenso eine Rückverlagerung. Im Bereich der Rohstoffverwaltung ist hierfür die Existenz einer tatsächlichen Grundlage erforderlich – die zur Produktion erforderlichen mineralischen Rohstoffe erfordern ein entsprechendes natürliches Vorkommen oder zumindest ein Vorliegen der Rohstoffe aus anderen Quellen wie dem Recycling. Eine besondere Form des Nearshorings kann zudem im Falle der wirtschaftsverwaltungsrechtlichen Ausformung im Sinne von Ausfuhrvorschriften bzw. Exportverboten sichtbar werden – so kann Produktion dazu gezwungen werden, im Land der Rohstoffgewinnung zu verharren, anstatt die der Rohstoffgewinnung nachgelagerten Schritte in ausländische Wertschöpfungsketten zu verlagern. So könnte hier also Produktion zwangsweise verlagert werden, sodass Nearshoring eher aus staatlicher Verwaltungssicht statt aus unternehmerischer Strategie geschieht. China hat bereits solche Ausfuhrkontrollen für die Metalle Gallium und Germanium eingerichtet, die insbesondere für die Halbleiterproduktion von Bedeutung sind – begründet werden diese handelspolitischen Maßnahmen mit „nationaler Sicherheit und nationalem Interesse“. \autocite{lamby-schmitt_china_2023}
	Ein weiterer Effekt ist, dass Produktion erst gar nicht aus dem jeweiligen Land abwandert, sondern rohstoffgewinnende Unternehmen durch Ausfuhrverbote zu lokaler Weiterverarbeitung gedrängt sind und so die nationalen, nachgelagerten Industrien mitaufbauen, wie im Beispiel Indonesien deutlich wird. Die EU hat 2019 gegen solche indonesischen Exportverbote geklagt und die WTO diese Ausfuhrbeschränkungen 2022 für unverhältnismäßig erklärt, wogegen Indonesien Einspruch erhob.footnote{WTO Dispute Settlement DS592, WT/DS592/R.} In Anbetracht der Tatsache, dass der „Appelate Body“ jedoch derzeit nicht arbeitsfähig ist, betrachtet die EU das Arbeitsverfahren des Panels als ausgesetzt\footnote{Ibid.}, sodass mit keiner Entscheidung vor einer Wiederherstellung der Arbeitsfähigkeit der Rechtsinstanzen der WTO zu rechnen ist. Dies eröffnet zudem auch Fragestellungen hinsichtlich eines effektiven Rohstoffwirtschaftsvölkerrecht. Auch hier ist deutlich zu erkennen, dass der Bereich des Rohstoffverwaltungsrecht, besonders im Bereich des Wirtschaftsverwaltungsrechts, tangiert ist und das Untersuchungsinteresse hier anknüpft. Dies bestätigt auch Befürchtungen von Verwaltung in Politik, dass die Rohstoffversorgung durch „Rohstoffnationalismus“ und verstärkter Konkurrenz in rohstoffbesitzenden Schwellenländern erschwert wird, und zudem insbesondere in diesem Bereich zusätzlich mit den Interessen der Rohstoffwirtschaftsnachhaltigkeit und Menschenrechtssicherung in Konflikt geraten.\autocite{feichtner_besonderheit_2016} Schließlich tragen Decoupling und Nearshoring dann zu einer allgemeinen, oben erwähnten De-Globalisierung bei – mit entsprechenden Folgen für die Weltwirtschaft, insbesondere sektoralen Veränderungen der Wertschöpfung für die Industrie sind hier nicht ausgeschlossen, wobei insbesondere bei einem möglichen Handelskrieg zwischen der EU und China als Folge der beschriebenen Phänomene die Automobilindustrie die größten Wertschöpfungsverluste verzeichnet.\autocite[3]{fuest_geopolitische_2022} Nichtsdestotrotz erinnert SCHÄFFER daran, dass trotz der Umorientierung im globalen Außenhandel die nationale Wirtschaft „nicht im Übermaß protegiert“ werden dürfe, „insbesondere nicht zum Kollateralschaden zwischen China und den EU“ trotz der Erforderlichkeit einer EU-Antwort auf die „tiefe Krise“ des Multilateralismus.\autocite[700]{schaffer_ausenwirtschaftsrecht_2023} Auch ökonomische Effekte des Nearshorings nach Europa, insbesondere auf das Realeinkommen, müssen bedacht werden.\autocite{sandkamp_reshoring_2022}
	
	Die Begriffe des De-Risking und Friendshoring finden auch in der unionalen Handelspolitik wieder\footnote{Näheres dazu im folgenden Kapitel.}
	
	
	
	% hier auch auf Reshoring eingehen
	% Regulierung von Nearshoring
	
	Nach einer Studie der Kreditanstalt für Wiederaufbau (KfW) gehen rund ein Drittel der deutschen mittelständischen Unternehmen davon aus, dass eine 
	
	\subsection{Weitere Verortungen}
	% Zeitenwende?
	
	% ggf. umgruppieren 
	
	\subsection{Strategische und kritische Rohstoffe und Mineralien}
	
	\subsection{Rohstoff im engen und weiteren Sinn}
	
	\section{Rohstoffversorgung des deutschen Automobilsektors}
	
	
	Insbesondere bei seltenen Erden ist die deutsche Industrie nahezu vollständig von China abhängig -- es gibt jedoch kaum Bestrebungen, lokale Bezugsquellen zu nutzen und diese Abhängigkeit somit zumindest zu reduzieren.
	
	Kurz gesagt: Sobald China also die Rohstoffausfuhr unterbindet, ergibt sich daraus unmittelbar ein Problem für den deutschen Automobilsektor.
	
	
	\subsection{Aktionsplan für die europäische Autoindustrie}
	Anfang März 2025 kündigte die Europäische Kommission einen Aktionsplan für den europäischen Automobilsektor\footnote{COM(2025) 95 final; IP/25/635} zur Stärkung vor dem Hintergrund des Technologiewandels und veränderten globalen Marktbedingungen vor.
	
	Die vorgestellten und angekündigten Maßnahmen beruhten hierbei auf den Erkenntnissen im Rahmen des strategischen Dialogs der Kommission zur Zukunft der Automobilindustrie.\footnote{IP/25/378} Dieser Dialog beinhaltete neben einer Open Public Consultation (OPC) die Teilnahme von 22 Stakeholdern aus dem Automobilbereich, darunter Verbände, Hersteller, Zulieferer und und Unternehmen aus dem Bereich Elektromobilität.\footnote{ACEA, BEUC, BMW Group, Robert Bosch GmbH, ChargeUp Europe, CLEPA, Daimler Truck, ETF, Forvia, IndustriAll European Trade Union, IVECO Group, MAHLE Group, MILENCE, RECHARGE, Renault Group, Transport \& Environment, Traton Group, Valeo, Volkswagen Group, Volvo Cars, Volvo Group, ZF Group; AC/25/379.}
	
	%Warum diese Auswahl?
	
	Zentral ist der Ausbau der Widerstandsfähigkeit des Sektors gegenüber der globalen Wettbewerber, genauer durch den Einsatz gezielter Handelsschutzmaßnahmen, Abbau bzw. Vereinfachung von regulatorischen Vorschriften sowie durch engere Zusammenarbeit zwischen der Kommission bzw. den Behörden und den Unternehmen. Im Zuge dessen wurden 1,8 Milliarden EUR für die Reduzierung strategischen Abhängigkeiten der Batterie-Lieferkette angekündigt bei zeitgleichem Ausbau der EU-Batterieproduktion. Ferner ist die Schaffung einer \glqq Battery Raw Materials Access Entity\grqq angedacht, die EU-Automobilherstellern bei der Beschaffung von Rohstoffen durch die Zusammenlegung von Zusagen und Investments unterstützen, wobei die Kommission hier unterstützend tätig werden soll. Im Rahmen des CRMA und den dort identifizierten strategischen Projekte soll zudem der Zulassungsprozess für die Verarbeitung von für die Batterieproduktion benötigten Rohstoffen beschleunigt sowie das Recycling ausgebaut werden.\footnote{QANDA/25/636, }
	
	
	
	\section{Interimsfazit}


\end{document}