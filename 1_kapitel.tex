\documentclass[12pt,a4paper,oneside]{book} % 'oneside' für einseitigen Druck

% Kodierung, Sprache und Schrift
\usepackage[utf8]{inputenc} % Erlaubt die Verwendung von Umlauten
\usepackage[T1]{fontenc} % Bessere Schriftkodierung
\usepackage[ngerman]{babel} % Deutsche Lokalisierung

% Schriftarten
\usepackage{lmodern} % Modernere Schriftart, gut für Skalierbarkeit und Lesbarkeit

% Für Abbildungen
\usepackage{graphicx}
\graphicspath{{bilder/}} % Verzeichnis, in dem Bilder gespeichert sind

% Für Tabellen
\usepackage{booktabs}

% Für Links und PDF-Metadaten
\usepackage[hidelinks]{hyperref}
\hypersetup{
	pdftitle={Titel der Dissertation},
	pdfauthor={Autor},
	pdfsubject={Doktorarbeit in den Sozial- und Rechtswissenschaften},
	pdfkeywords={Schlüsselwörter},
}

% Für Bibliographie - Anpassung für Geisteswissenschaften
%%\usepackage[style=authoryear-icomp,backend=biber]{biblatex}
%%\usepackage[backend=biber, style=authoryear-icomp]{biblatex}
%%\usepackage[backend=biber, style=verbose-trad1]{biblatex}
\usepackage[backend=biber, style=verbose-inote]{biblatex}
%%\usepackage{biblatex}
\addbibresource{literatur.bib} % Name der BibTeX-Datei
\DeclareNameAlias{author}{family-given} % Nachname des Autors zuerst

% Anpassung der Nummerierung mit Punkten
\renewcommand{\thechapter}{\arabic{chapter}.} % Kapitel: 1., 2., 3., ...
\renewcommand{\thesection}{\Alph{section}.} % Abschnitt: A., B., C., ...
\renewcommand{\thesubsection}{\Roman{subsection}.} % Unterabschnitt: I., II., III., ...
\renewcommand{\thesubsubsection}{\arabic{subsubsection}.} % Unterunterabschnitt: 1., 2., 3., ...
\renewcommand{\theparagraph}{\alph{paragraph}.} % Absatz: a., b., c., ...


% Für Fußnoten
%%\usepackage[bottom]{footmisc} % Fußnoten am Seitenende


% Anpassung der Kapitelüberschriften
%\usepackage{titlesec}
%\titleformat{\chapter}[hang]{\Huge\bfseries}{\thechapter\quad}{0pt}{\Huge\bfseries}

% Abstand der Fußnoten
\setlength{\footnotesep}{0.5cm}

% Tiefe der Nummerierung und des Inhaltsverzeichnisses
\setcounter{secnumdepth}{4} % Nummerierungstiefe einstellen
\setcounter{tocdepth}{4} % Inhaltsverzeichnistiefe einstellen

% Abstand zwischen Absätzen und kein Einzug
%\usepackage{parskip}
%\setlength{\parskip}{0.5em}
%\setlength{\parindent}{0pt}

% Für Abkürzungsverzeichnis
\usepackage[printonlyused]{acronym}

% Für Zitate und Theoreme (falls benötigt)
\usepackage{csquotes}

% Für Gesetzestexte, Zitate und andere strukturierte Texte
\usepackage{enumitem}

% Zeilenabstand auf 1.3
\usepackage{setspace}

% Beginn des Dokuments
\begin{document}
	% Hier beginnt der eigentliche Inhalt der Arbeit
	% ...
	
	
	\setstretch{1.3}
	
	\tableofcontents
	
	\chapter{Hinführung zur Thematik}
	
	Rohstoffe - insbesondere solche, die den sog. "seltenen Erden" angehören - haben in der jüngeren Vergangenheit eine Steigerung der öffentlichen Aufmerksamkeit in Politik, Wirtschaft und Gesellschaft erfahren: Dies wird deutlich in der (erstmaligen) Veröffentlichung von Rohstoffstrategien von Regierungen, der Suche nach Rohstoffpartnerschaften nicht nur von Global Playern in der Industrie, und 
	
	Der Übergang hin zu \textit{Net Zero}-Emissionen sorgt daher für einen entsprechenden Anstieg der Nachfrage nach Mineralien, die als grundlegender Bestandteil für Technologien zur Umsetzung dieser Bestrebungen erforderlich sind, beispielsweise bei Elektrofahrzeugen.
	
	Das grundsätzliche Problem: Die reine geologische Verfügbarkeit von Mineralien impliziert nicht, dass diese auch der Union zugänglich sind.\footnote{Dies erkannte die Kommission bereits 2008, ebenso dass kurzfristig nicht zwangsläufig Verknappungen zu befürchten sind; KOM(2008) 699, S. 4.} Oftmals sind \glqq seltene Erden\grqq auch nicht selten in ihrem Vorkommen, sondern ungleichmäßig verteilt.
	
	Mehr und mehr scheint eine Art Aufwachen einzusetzen, dass kritische Rohstoffe tatsächlich kritisch werden können -- nicht nur bei direkt betroffenen Akteuren, sondern auch im allgemeinen Bewusstsein, spätestens seit US-Präsident Trump Grönland kaufen möchte, China Exportkontrollen eingeführt und die EU neue Bevorratungsstrategien angekündigt hat. China dominiert zudem derzeit und absehbar weiterhin das weltweite Angebot an seltenen Erden -- mit entsprechenden Folgen für die Versorgungssicherheit. Nicht nur die Abhängigkeit von Staaten im Bereich des Abbaus und der Veredelung ist kritisch zu betrachten, sondern auch die Problematik der Auslagerung dieser Vorgänge in Staaten mit niedrigen Umwelt-, Sicherheits- und Sozialstandards,\autocite[siehe hierzu]{ÖAW Artikel} die in einer Studie des Europäischen Parlaments als zentrale Herausforderung hinsichtlich Nachhaltigsaspekten identifiziert wurde.\autocite[16]{The role of research and innovation in ensuring a safe and sustainable supply of critical raw materials in the EU}
	
	Die Globalisierung ist spätestens seit den 1970er Jahren ein wichtiger Aspekt der Weltwirtschaft, jedoch haben externe Krisen wie die Covid-19-Pandemie, der Halbleitermangel und andere Lieferengpässe die Vulnerabilität nicht nur in der exportorientierten deutschen Automobilindustrie aufgezeigt. Es scheint, dass die Globalisierung einer Slowbalization weichen wird, d. h. einer deutlich verlangsamten Globalisierung. Daher sollten Global Player, nicht nur in der Automobilindustrie, sich auf eine Reihe von Entwicklungen vorbereiten, die bisher in vergleichbarer Form nicht in Erscheinung getreten sind.
	Darüber hinaus stehen der globalisierte Welthandel und das damit verbundene Geschäftsmodell durch ein befürchtetes Decoupling, also das Entkoppeln von Wirtschaftsräumen, zunehmend unter Druck. Dies wird insbesondere durch Akteure wie die USA deutlich, die mit „America First” und zuletzt dem „Inflation Reduction Act” weitere protektionistische, handelspolitische Maßnahmen ergriffen haben – demgegenüber stehen beispielsweise die WTO, dessen Streitbeilegungsverfahren aber seit Dezember 2019 im Wesentlichen seine Aktivität eingestellt hat, die EU mit dem handelsstrategischen Prinzip der „offenen strategischen Autonomie”, oder entsprechende Handelsabkommen. Diese Herausforderungen werden weiter durch den russischen Angriffskrieg auf die Ukraine verschärft, der zudem eine Zeitenwende für die deutsche Wirtschafts- und Energiepolitik darstellte. Dazu kommen weitere geopolitische Herausforderungen und Störfälle, wie die weiterhin bestehende Wahrscheinlichkeit eines China-Taiwan-Krieges und entsprechenden Auswirkungen, sowie generell Kriege und Konflikte im Lichte einer „neuen” Geopolitik. Nearshoring bezeichnet eine Reaktion auf diese Entwicklungen und bezieht sich auf die Neu-Positionierung von Lieferketten, sodass ins Ausland verlagerte Produktion wieder nach Europa und/oder Deutschland zurückverlagert wird. Dies reduziert insbesondere Abhängigkeiten, nicht nur bei Bauteilen wie Halbleiter oder Batterien bei Elektrofahrzeugen, sondern auch bei anderen Elementen und insbesondere kritischen mineralischen Rohstoffen.
	
	Um die Ziele der angestrebten Energiewende und damit den Übergang zur Elektromobilität auch im Bereich der Automobilwirtschaft umzusetzen und zu erreichen, bedarf es eines wirksamen und vor Allem befähigendem rechtlich-regulatorischen Rahmen, sodass bei gleichzeitiger Erreichung dieser Ziele auch die Funktionsfähigkeits des Sektors per se als auch die Versorgungssicherheit mit entsprechenden mineralischen Rohstoffen sichergestellt wird, ohne die die Transformation überhaupt erst gewährleistet werden kann. Daher ist die (zumindest teilweise) Aufrechterhaltung der Versorgung nicht nur während, sondern auch nach der Transformationsphase von entscheidender Bedeutung, da etwaige Unterbrechungen nicht nur den Übergangsprozess unterbrechen oder gänzlich stoppen könnten, sondern insbesondere einerseits wirtschaftliche Risiken für die betroffenen Unternehmen darstellen und andererseits die Policy-Legitmität der jeweilig beteiligten politischen Instituionen gefähreden könnte -- was im Äußersten zur Infragestellung der gesamten Transformation führen könnte (\glqq Carbon Lock-In\grqq).
	
	Hier setzt der Aspekt der rohstoffverwaltungsrechtlichen Perspektive an: Wie können Policy-Akteure in Deutschland und der EU sicherstellen, dass relevante Industrieakteure wie der Automobilsektor einerseits adäquat auf sich verändernde globale Umstände reagieren können, andererseits durch eine gesicherte Rohstoffverfügbarkeit das Fortbestehen und die Weiterentwicklung des Wirtschaftssektors gewährleistet werden kann? Trotz der Aufmerksamkeit für die Rohstoffproblematik ist das Rohstoffverwaltungsrecht zum aktuellen Zeitpunkt ein eher unbeachtetes Forschungsfeld, sodass die Dissertation hier mit einer rechts- und fachübergreifenden Perspektive diese Lücke schließt und zur akademischen Untersuchung des Feldes beiträgt. Ferner wird die Bedeutung der Rechtskomponente im Feld der Rohstoffversorgung weiter beleuchtet und somit gestärkt. Darüber hinaus existieren im Bereich der Rohstoffverwaltung kaum rechtliche Vorgaben, insbesondere im Sekundärrecht und nicht nur auf europäischer Ebene, sodass die Dissertation die Frage beantwortet, ob deutsche und europäische Rahmenrechtsverordnungen auf die beschriebenen Herausforderungen vorbereitet sind. Ein Beispiel hierfür kann im gegenwärtigen EU Critical Raw Materials Act gesehen werden, der eine erste europäische Antwort auf die bekannten Probleme darstellen soll, aber auch auf Vorkommen entsprechender Rohstoffe in der EU. Diese Rohstoffe sind nicht nur im Bereich der voranschreitenden Digitalisierung relevant, die auch im Automobilsektor Einzug hält, sondern auch in Bezug auf Umweltaspekte und die Entwicklung der emissionsfreien Mobilität, die besonders durch die Entscheidung der Europäischen Kommission nur noch emissionsfreien Fahrzeugen ab 2035 die Zulassung zu ermöglichen.
	
	Die Dissertation betrachtet daher zunächst bedeutsame, aktuelle Entwicklungen und Spannungen in Handels- und Geopolitik sowie relevante Policy-Aktivität. Anbindend daran werden mögliche zukünftige Entwicklungen identifiziert und beschrieben; schließlich werden Szenarien präsentiert, wie durch regulatorische Aktivität die Herausforderungen bezüglich der Rohstoffverfügbarkeit adressiert werden können, und welche Strategien sich hierbei für den deutschen Automobilsektor ergeben, insbesondere hinsichtlich gesteigerter Resilienz und verringerter Dependenz. 
	
	Hierbei wird argumentiert, dass eine effektive und vor allem selbstständige, legitimierte und kompetenzrechtlich abgesicherte, europaeinheitliche Rohstoffpolitik als grundlegend für eine effiziente und effektive Rohstoffverwaltung betrachtet wird und vor allem notwendig ist, um die dauerhafte Versorgungssicherheit zu garantieren, welche jedoch zum aktuellen Zeitpunkt mit erheblichen rechtlichen Hürden insbesondere in Form von Regelungslücken, Unklarheiten und Trägheit gekennzeichnet ist.
	
	
	\section{Bestimmung der Thematik/Hintergrund}
	Engpässe bei der Versorgung mit Rohstoffen wie Gallium oder schweren Seltenen Erden verdeutlichen auf schmerzhafte Weise, wie stark der Industriestandort Deutschland von stabilen Lieferketten abhängig ist – und wie gefährdet seine Zukunft ohne diese ist. Dabei kommt insbesondere zum Tragen, dass Rohstoffvorkommen begrenzt sind, der Prozess von Erschließung bis Abbau zeitintensiv, das Bergbaugeschäft ein \glqq hartes\grqq ist und die Rahmenbedingungen insbesondere im Blick auf ausländische Partner nicht (mehr) arbiträr gestaltet werden können.
	
	Die Wechselwirkungen zwischen staatlichem Rohstoffverwaltungsrecht und privatwirtschaftlichen Unternehmen, insbesondere innerhalb der Automobilindustrie, stellen ein facettenreiches Forschungsgebiet dar. Die Thematik geht über den rein rechtlichen Rahmen hinaus und erfordert eine multidimensionale Betrachtung, die sowohl juristische als auch wirtschaftliche, geopolitische und gesellschaftliche Perspektiven integriert. Die Rolle des Rohstoffverwaltungsrechts wird zunehmend kritischer, da es nicht nur den Zugang zu, sondern auch die Nutzung und Verteilung von Rohstoffen reguliert. Dies hat unmittelbare Auswirkungen auf privatwirtschaftliche Unternehmen, insbesondere auf jene, die in der Automobilindustrie tätig sind, und deren Produktionsprozesse und Wettbewerbsfähigkeit stark von der Verfügbarkeit und dem kosteneffizienten Erwerb von Rohstoffen abhängen.
	
	Die Verbindung zwischen Rohstoffrecht und Verwaltung stellt eine entscheidende Schnittstelle dar, die die Rahmenbedingungen für unternehmerische Tätigkeiten maßgeblich beeinflusst. Durch die Untersuchung dieser Zusammenhänge wird nicht nur ein tieferes Verständnis für die rechtlichen Mechanismen geschaffen, sondern auch für die administrativen Prozesse, die die operative Realität von Unternehmen innerhalb der Automobilindustrie prägen. Geopolitische Faktoren gewinnen in diesem Kontext ebenfalls zunehmend an Bedeutung. Die Globalisierung der Rohstoffmärkte und die Abhängigkeit der Automobilindustrie von diversen Ressourcen aus verschiedenen Teilen der Welt bringen komplexe geopolitische Dynamiken mit sich. Die Auswirkungen politischer Entscheidungen, internationaler Konflikte und Handelsbeziehungen auf die Rohstoffbeschaffung werden daher analysiert, um ein umfassendes Bild der externen Einflüsse auf die unternehmerischen Strategien und operativen Abläufe zu zeichnen. Deutschland als eine der führenden Industrienationen ist nicht nur bei Energierohstoffen, sondern auch bei metallischen Ressourcen größtenteils von Importen abhängig. \autocite{dauke_rohstoff-_2011} Rohstoffpolitik kann als ein integraler Bestandteil deutscher Wirtschaftspolitik gesehen werden, tangiert selbstredend weitere Politikbereiche (Außenwirtschaft, Handel, Europa, Umwelt) und ist insbesondere im Hinblick auf die Versorgungssicherheit eine „Querschnittsaufgabe, die effektiv nur im engen Schulterschluss mit der Wirtschaft möglich ist”.\autocite{dauke_rohstoff-_2011}
	
	\subsection{Relevanz des Problems}
	Die Notwendigkeit der Herausbildung eines Rohstoffverwaltungsrechts ergibt sich, wie vom Bundesministerium für Wirtschaft und Klimaschutz (BMWK) beschrieben,\autocite{bundesministerium_fur_wirtschaft_und_klimaschutz_bmwk_industriepolitik_2023} auch aus der folgenden Problemkonstellation: Zwar sind viele mineralische Rohstoffe aus geologischer Sicht in zufriedenstellender Menge vorhanden – dies bedingt aber nicht, dass die Rohstoffmengen rechtzeitig und in den benötigten Mengen zur Verfügung stehen. Insbesondere die teils aufwendigen Prozesse zur Erkundung, Ausbeutung und Aufbereitung von solchen Rohstoffen tragen dazu bei, dass sich eine kurzfristige Ausweitung des Angebots vergleichsweise schwierig gestaltet, sodass der Rohstoffmarkt weiter konzentriert wird; die Internationale Energieagentur (IEA) hat ermittelt, dass die Rohstoffprojektdurchführung zwischen Entdeckung und erster Produktion durchschnittlich 16 Jahre benötigt.\autocite[12]{international_energy_agency_role_2021} Dies erklärt ebenfalls den vergleichsweise langfristigen Horizont der Thematik – es folgt also aus der Erkenntnis, dass für die „Entwicklung und Inbetriebnahme der Rohstoffprojekte (…) lange Zeitperspektiven nötig” sind,\autocite[13]{bundesministerium_fur_wirtschaft_und_klimaschutz_bmwk_industriepolitik_2023} die Schlussfolgerung, dass dies auch für die Verwaltung von rohstoffrechtlichen Aspekten gelten müsse.
	
	Der globale Westen hat seit über einem Jahrzehnt bereits erkannt, dass China die Lieferketten für Mineralien fest im Griff hat -- und dennoch kaum Fortschritte bei der Suche nach alternativen Quellen gemacht.
	
	Wie ein (Subventions-)Rohstoffverwaltungsrecht gestaltet werden kann, zeigt ein thematisches Cluster der European Raw Materials Alliance (ERMA) auf. So sollen gleiche Wettbewerbsbedingungen geschaffen werden, da nicht-europäische Hersteller (auch durch Subventionen) zu geringeren Fertigungskosten als solche in der EU herstellen können. Ferner sollen für europäische Hersteller mögliche Verpflichtungen erwägt werden, kritische Rohstoffe zu einem bestimmten Teil von europäischen Produzenten zu beziehen, sowie Abfallstoffe mit Bezug zu Seltenen Erden in Europa verbleiben. Schließlich wird ebenfalls ein finanzieller Hebel gefordert, sodass durch staatliche Beihilfen „Investitionen in die aufstrebende europäische Wertschöpfungskette“ für Rohstoffe zu ermöglichen.\autocite[7]{gaus_rare_2021}
	
	Spätestens mit den Versorgungsunterbrechungen in den Rohstofflieferketten angesichts chinesischer Exportkontrollen hat auch der Westen erkannt, dass Zeit, Geld und politische Unterstützung vonnöten sind, um bei kritischen Rohstoffen seine Abhängigkeit von China zu reduzieren, welches die Lieferketten für Mineralien dominiert und bereit ist, den Zugang zu ihnen einzuschränken. Auch die USA verfolgen Bemühungen, Abbaurechte in rohstoffreicheren Ländern zu sichern, und auch die EU oder Japan als deutliche Nettoimporteure werden entsprechende Kapazitäten aufbauen müssen bei zeitgleicher Abschätzung des mittel- bis langfristigen Bedarfs dieser kritischen Mineralien.
	
	Die Lieferkette bzw. Problemkaskade umfasst hierbei jedoch nicht nur die Mineralien selbst, sondern auch den Zugang zu Abbaustätten bzw. MIinen
	
	2025 berichtete die IEA, dass die Investitionen in kritische Rohstoffe 2024 um 5\% stiegen, ein deutlich niedrigeres Niveau als das Wachstum 2023 (13\%); zudem hat die geographische Konzentration weiter zugenommen, sodass der durschnittliche Marktanteil der drei größten Verarbeitungsländer auf 86\% (von 82\% im Jahr 2020) stieg.\autocite{IEA Critical Minerals Outlook 2025, S. 68}
	
	Ein weiterer Faktor, der Investitionen limitiert, ist die Tatsache, dass insbesondere nicht-staatliche Bergbauunternehmen auf eine Gewinnerzielung aus dem Abbau von Rohstoffen angewiesen sind. Ein Großteil der Mineralien wird nur in vergleichsweise geringen Mengen abgebaut, oft gar nur als Nebenprodukt beim Abbau anderer Erze; parallel dazu ist die Preislage bei kritischen Rohstoffe teils sehr volatil und nicht transparent. Die Spotmärkte sind in der Regel illiquide, da der Großteil des Transaktionsvolumens auf langfristigen Verträgen beruht. Die Erforderlichkeit bzw. das Bedürfnis nach Profit hat demnach westliche Bergbauunternehmen in ihren Investitionsentscheidungen eingeschränkt, während beispielsweise chinesische Firmen damit nicht konfrontiert waren, sondern vielmehr die Versorungssicherung im Vordergrund stand, wie später noch weiter vedeutlicht wird.
	
	Die europäische Rohstoffpolitik befindet sich in einem tiefgreifenden Wandel. Mit dem Inkrafttreten des CRMA am 23. Mai 2024 wurde erstmals ein verbindlicher Rechtsrahmen geschaffen, der die Versorgungssicherheit mit kritischen und strategischen Rohstoffen systematisch adressiert 1. Ziel ist die Sicherstellung einer krisenfesten, nachhaltigen und geopolitisch resilienten Rohstoffversorgung für Schlüsselindustrien wie die Automobil-, Batterie-, Halbleiter- und Verteidigungsbranche.
	
	Der CRMA definiert quantitative Zielvorgaben bis 2030: mindestens 10\% des EU-Bedarfs an strategischen Rohstoffen sollen aus heimischem Bergbau stammen, 40 \% aus EU-Verarbeitung und 25 \% aus Recycling. Gleichzeitig darf kein Drittstaat mehr als 65 \% eines strategischen Rohstoffs in einer Verarbeitungsstufe liefern. Trotz dieser Fortschritte bestehen strukturelle und rechtliche Lücken, die die Umsetzung gefährden. Zwar sind die Mitgliedstaaten verpflichtet, nationale Explorationsprogramme zu erstellen, doch bleibt offen, wer die Kosten für die Erschließung, Genehmigung und gegebenenfalls Rückbau nicht wirtschaftlich tragfähiger Projekte trägt.
	
	ie EU-Kommission warnt explizit vor \glqq unkoordinierten nationalen Maßnahmen \grqq, die den Binnenmarkt verzerren könnten – etwa durch divergierende Genehmigungsstandards, Förderbedingungen oder Monitoringkapazitäten 3. Dies betrifft insbesondere die Automobilindustrie, deren Wertschöpfungsketten grenzüberschreitend organisiert sind. Analog zur Problematik bei Wasserstoff im Energierecht fehlt bislang ein klarer Rechtsrahmen für die Etablierung von Märkten für neuartige Rohstoffe wie hochreines Silizium, seltene Erden oder metallurgisches Bor. Diese Stoffe sind für die Elektromobilität und Digitalisierung essenziell, können aber technisch oft nicht recycelt werden – was die Kreislaufziele des CRMA konterkariert. Der CRMA enthält keine klaren Mechanismen für den Fall, dass die gesetzten Versorgungsziele bis 2030 nicht erreicht werden. Dies wirft Fragen nach der Verbindlichkeit und Durchsetzbarkeit der Verordnung auf. Die Schnittstellen zu anderen EU-Rechtsakten – etwa der Taxonomie-Verordnung, der Ökodesign-Richtlinie oder dem Konfliktmineralienregime – sind bislang nur rudimentär geregelt. Dies erschwert die rechtssichere Investitionsplanung für Unternehmen und erhöht das Risiko sogenannter \glqq stranded assets\grqq.
	
Trotz des mit dem „Critical Raw Materials Act“ (CRMA) geschaffenen unionsrechtlichen Rahmens zur Sicherung der Versorgung mit kritischen Rohstoffen, divergieren die nationalen Politiken der EU-Mitgliedstaaten in ihrer strategischen Ausgestaltung, Prioritätensetzung und sektoralen Fokussierung erheblich. Diese Divergenzen sind insbesondere in jenen Mitgliedstaaten evident, deren Volkswirtschaften stark von der Automobilproduktion geprägt sind, wie etwa Deutschland, Frankreich oder Italien.

Deutschland verfolgt eine industriepolitisch stark ausgerichtete Rohstoffstrategie, die explizit auf die Bedürfnisse der Automobil- und insbesondere der Batterieindustrie zugeschnitten ist. Projekte wie der Lithiumhydroxid-Konverter von Rock Tech in Guben, das geothermische Lithiumgewinnungsprojekt von Vulcan Energie in der Oberrheinebene sowie die Graphit-Substitutionsforschung von PCC Thorion in Duisburg sind Ausdruck einer nationalen Rohstoffpolitik, die auf vertikale Integration, technologische Souveränität und Standortdiversifizierung abzielt. Ergänzt wird dies durch ein nationales Explorationsprogramm zur Identifikation heimischer Rohstoffpotenziale, das über die unionsrechtlichen Vorgaben hinausgeht.

Frankreich hingegen setzt stärker auf europäische Industriekooperationen und die Diversifizierung internationaler Lieferketten. Die Beteiligung an transnationalen Projekten wie der NGC Battery Materials GmbH, die Graphit aus Namibia für den europäischen Markt aufbereitet, verdeutlicht den französischen Fokus auf strategische Partnerschaften und geopolitische Resilienz. Italien und Spanien zeigen im Vergleich geringere Ambitionen, fokussieren sich jedoch verstärkt auf Recyclinginitiativen und die Entwicklung sekundärer Rohstoffmärkte.

Diese nationalen Unterschiede wirken sich unmittelbar auf die Automobilindustrie aus, die in hohem Maße auf stabile, transparente und nachhaltige Rohstofflieferketten angewiesen ist. Während deutsche OEMs zunehmend vertikal in vorgelagerte Wertschöpfungsstufen investieren, um Versorgungssicherheit zu gewährleisten, sind Hersteller in anderen Mitgliedstaaten stärker auf europäische oder internationale Kooperationsstrukturen angewiesen. Die Folge ist eine fragmentierte Rohstoffgovernance innerhalb der EU, die trotz gemeinsamer Zielvorgaben des CRMA erhebliche Unterschiede in der operativen Umsetzung aufweist.

Für die deutsche Automobilindustrie bedeutet dies eine doppelte Herausforderung: Einerseits muss sie sich in einem zunehmend geopolitisch aufgeladenen Rohstoffumfeld behaupten, andererseits divergieren die regulatorischen und infrastrukturellen Rahmenbedingungen innerhalb der EU, was die Harmonisierung von Lieferketten und Investitionsentscheidungen erschwert. Die nationale Rohstoffpolitik wird damit zu einem entscheidenden Standortfaktor im Kontext von „Slowbalization“, „Decoupling“ und „Nearshoring“.

Auch zeigt sich, dass die Thematik von den EU-Mitgliedsstaaten unterschiedlich stark gewichtet wird: So sind bisland nur 13 von 27 Mitgliedsstaaten mit strategischen Projekten im Rahmen des CRMA vertreten. Insbesondere kleinere Länder haben bislang keine sichtbaren Aktivitäten entfaltet bzw. Strategien vorgelegt. Die Europäische Kommission hat im April 2025 eine Konsultation zur industriellen Zusammenarbeit bei kritischen Rohstoffen gestartet, um Rückmeldungen von Unternehmen einzuholen. Ziel war es, Bewusstsein zu schaffen und Kooperationshemmnisse zu identifizieren – ein Indiz dafür, dass nicht alle Mitgliedstaaten oder Industrien das Thema mit der nötigen Dringlichkeit behandeln.

Die europäische Rohstoffgesetzgebung steht vor einem Dilemma: Einerseits ist mit dem CRMA ein ambitionierter Rechtsrahmen geschaffen worden, der die strategische Bedeutung kritischer Rohstoffe anerkennt und erstmals verbindliche Zielvorgaben formuliert. Andererseits bestehen erhebliche Governance- und Umsetzungsdefizite, die die Effektivität des Rechtsakts gefährden. Die Fragmentierung nationaler Politiken, das Fehlen klarer Zuständigkeitsregelungen und die mangelnde Integration in bestehende Rechtsregime stellen zentrale Herausforderungen dar -- insbesondere für rohstoffintensive Industrien wie die deutsche Automobilwirtschaft.
	
	\subsection{Juristischer Hintergrund}
	Im Anbetracht der Thematik könnte der Anschein eines rechtlichen Randgebietes erweckt werden. Zwar steht zunächst die verwaltungsrechtliche Betrachtung im Mittelpunkt, jedoch dringt die behandelte Thematik unweigerlich in benachbarte und verwandte Rechtsbereiche in einem anzunehmenden unterschiedlichen Maße ein, ähnlich dem internationalen Wirtschaftsrecht, sodass sich auch hier eine rechtswissenschaftliche Verbindung der Bereiche anbietet. \autocite[1, 2]{herdegen_internationales_2020} Unweigerlich ist das Rohstoffverwaltungsrecht eine Auskopplung des Rohstoffrechts und des Verwaltungsrechts, sodass festgehalten werden kann, dass das Rohstoffverwaltungsrecht dem Rohstoffrecht folgend ebenfalls einen „Hybrid“-Charakter aufweist und somit Wurzeln in diversen verwandten Rechtsgebieten vorhanden sind. \autocite[358]{terhechte_konsolidierung_2015}
	
	Des Weiteren sollten auch das Völkerrecht und das Wirtschaftsvölkerrecht hinzugezogen werden; zwar sind die etwaigen Wechselwirkungen hier geringer einzuschätzen als bei direkt verwandten Rechtsgebieten, nichtsdestotrotz bieten diese Regelungen zu internationalen Wirtschaftsbeziehungen im von Außenhandelsaktivität geprägtem Feld verschiedene Zugänge: Insbesondere im Rohstoffbereich sind Regelungen zu den internationalen Wirtschaftsbeziehungen unabdingbar – für den Gegenstand der Arbeit spielt daher der Doppelcharakter des internationalen und öffentlichen Wirtschaftsrechts bestehend aus den rechtsqualitativ zu differenzierenden privaten und öffentlich-rechtlichen Akteuren \autocite[1, 1]{herdegen_internationales_2020} eine Rolle, aber auch die Regelungshierarchien im Sinne des inter- bzw. supranationalen und nationalen Rechts. Folgt man der Darstellung von KRAJEWSKI \autocite[10]{krajewski_wirtschaftsvolkerrecht_2021} werden somit öffentlich-rechtliche Beziehungen des Wirtschaftsvölkerrechts betrachtet, die größtenteils aus förmlichem Recht bestehen – aber sich dennoch auch in den Bereich unverbindlicher Normen erstrecken kann und Schnittpunkte mit dem Zoll- bzw. Außenwirtschaftsrecht und internationalen Standards aufweisen. Schließlich fallen auch Bezüge zum Umwelt- und Energierecht in den Untersuchungsgegenstand. In den Bereich des Verwaltungsrechts fällt die Vergabe von Subventionen an privatwirtschaftliche Akteure, was im Rohstoffbereich entsprechende Bedeutung aufweist – der US-amerikanische IRA hat gezeigt,\autocite[s. insbesondere bzgl. der dt. Automobilindustrie]{book_amerikaner_2023}  wie solche rohstoffbezogenen Subventionen gestaltet werden können. Auch die Untersuchung der Europäischen Kommission zu möglicher Wettbewerbsverzerrung durch Subventionen für chinesische Hersteller und der Import dieser Fahrzeuge in die EU illustriert die Salienz dieses Themas, auch unter Berücksichtigung internationaler Handelsperspektiven: Hinsichtlich des internationalen Rohstoffhandels zeigt bspw. das GATT in Bezug auf Subventionen für die Ausfuhr von Rohstoffen „nur sehr weich formulierte Beschränkungen“, auch da mengenmäßige Ausfuhrbeschränkungen und -verbote temporär erlaubt sind.\autocite[11, 2]{herdegen_internationales_2020}.
	
	Folgend der Feststellung von Frau \autocite[4]{frau2023}, dass ein "öffentliches Rohstoffrecht" in Deutschland trotz der staatlichen Feststellung der Relevanz einer Rohstoffversorgung fehlt, staatliche "Kompetenzen und Instrumente" nicht betrachtet werden und letzlich auch ein Mangel an entsprechender juristischer Auseinandersetzung mit der Thematik \autocite[5]{frau} vorhanden ist, ist bei dieser Arbeit der Lückenschluss als richtungsweisend zu betrachten. Sie untersucht, wie das geltende Rohstoffverwaltungsrecht in Deutschland und der EU mit den Herausforderungen der globalen Rohstoffversorgung umgeht und welche Rolle internationale Abkommen und das Völkerrecht dabei spielen. Konkret soll die Frage beantwortet werden, inwieweit der deutsche Staat und die Europäische Union die Beschaffungsstrategien der Automobilindustrie regulieren und fördern, um eine sichere und nachhaltige Versorgung mit strategisch wichtigen mineralischen Rohstoffen sicherzustellen.
	
	
	
	\subsection{Einordnung der Themenbereiche}
	Die Ausführungen sollen im Rahmen einer interdisziplinären Untersuchung aus einer politischen, einer wirtschaftlichen und einer juristischen Perspektive betrachtet werden, um der Komplexität der Thematik gerecht zu werden und hierbei auch die angenommenen Interessen der betroffenen Akteure adäquat zu reflektieren. Ferner werden die einzelnen Forschungsbereiche so auch nachhaltig miteinander verknüpft.
	
	\subsection{Politisch-wirtschaftliche Hintergründe}
	
	Im Anbetracht des Vorgenannten scheint es daher notwendig, Rohstoffsicherheit und -versorgung nicht nur wirtschaftsrechtlich, sondern auch sicherheits- und geopolitisch zu denken. Es ist ohne Zweifel, dass die Politik als ein wichtiger und notwendiger Faktor im Verwaltungsprozess natürlicher Rohstoffe anerkannt werden kann.\autocite[hierzu ausführlich][239-248]{henning_politics_1968} Darüber hinaus stellt die Sicherstellung der Versorgung mit Rohstoffen eine der maßgeblichen Herausforderungen für die europäischen industriellen Volkswirtschaften dar.\autocite{europaische_kommission_study_2023} Hierbei ergeben sich selbstredend Finanzierungsfragen, die als Ergebnis einer vom BMWK bezeichneten „Wirtschaftssicherheitspolitik“ entstehen und daher durch entsprechende Vorgaben im Sinne eines „Gesetzes für die Rohstoffsicherheit“ Abhängigkeiten von Unternehmen verringern will.\autocite{lohr_gesetz_2022}
	
	Internationale Rohstoffpolitik verfolgt den Ansatz, die „eigene rohstoffverbrauchende Industrie bei der Rohstoffsicherung im Ausland zu unterstützen”, und darüber hinaus die Verbesserung der globalen Voraussetzungen durch die Schaffung „offene[r] und transparente[r] Märkte” in Verbindung mit (hohen) Standards bei der Ausbeutung und Aufbereitung der entsprechenden Rohstoffe.\autocite{acatech__deutsche_akademie_der_technikwissenschaften_rohstoffe_2017} Insbesondere eine „globale Rohstoffverwaltung” wird hier als Instrument aufgezeigt und unterstreicht erneut die Erfordernis der Verbindung der thematischen Einzelbereiche. Aus Sicht des preisgetriebenen Intervenierens in Rohstoffmärkte kann auch das Phänomen der Rohstoffabkommen benannt werden, z. B. Abkommen in der Gestalt einer internationalen Rohstofforganisationen, die entsprechende Marktinterventionen vornimmt – die internationale Organisation verfolgt hierbei also kommerzielle Zwecke wie beispielsweise die Sicherstellung einer bestimmten Preisentwicklung und agiert somit ähnlich einem privatwirtschaftlichen Unternehmen.\autocite[Paragraph 11, Rn. 4]{herdegen_internationales_2020}
	
	Im Rahmen der politikwissenschaftlichen Auseinandersetzung und in Bezug auf internationale Rohstoffpolitik ist daher auch der im Titel präsente Begriff der „Zeitenwende“ relevant – prominent in den öffentlichen Diskurs getragen durch die Rede des Bundeskanzlers Olaf Scholz am 27. Februar 2022 hinsichtlich des russischen Angriffskrieges gegen die Ukraine drei Tage zuvor.\autocite{deutsche_bundesregierung_reden_2022} Der Ausdruck spiegelt in seiner Gesamtheit einerseits die Neuausrichtungen deutscher und europäischer Wirtschafts- und Energiepolitik, inklusive rohstoffpolitischer Aspekte, als auch die Neuevaluierung von bestehenden Beziehungen zu Akteuren im internationalen Raum wider--insbesondere zu China und Russland.\autocite[siehe auch][Rn. 741-746]{moser_zeitenwende_2022}; auch \autocite{[695-701]schaffer_ausenwirtschaftsrecht_2023} Die Zeitenwende ergab hierbei aus handelspolitischer und ökonomischer Sicht eine neue Betrachtung von Abhängigkeiten: So könne die deutsche Wirtschaft als ein „Globalisierungsgewinner“ gesehen werden, sei zugleich dementsprechend aber international verflechtet, sodass sich hier Dependenzen ergeben.\autocite{machnig_zeitenwende_2023} Da ein Zeitalter systemischer Konkurrenz bevorstehe, sind robustere Strukturen für die Wirtschaftsordnung erforderlich,\autocite{europaische_kommission_globale_2015} mit entsprechender Folgen für die Ansprüche an eine effektive Rohstoffverwaltung.

\subsection{Deutscher Automobilsektor}
Die Automobilindustrie, die einen wichtigen Pfeiler der deutschen und der europäischen Wirtschaft darstellt, ist weitgehend auf den Import von Primärrohstoffen angewiesen und verwendet nur eine begrenzte Menge an lokal gewonnenen Rohstoffen: Der Sektor befindet sich durch die Umstellung auf Elektrofahrzeuge in einem massiven Wandel, was zudem bedeutet, dass sich der größte Teil des ökologischen Fußabdrucks der Fahrzeuge von der Nutzungsphase auf die Produktions- und Recyclingphase verlagert und erhebliche Mengen an kritischen Rohstoffen benötigt werden. Zudem kommt es durch die Ausrichtung auf neue Antriebsmöglichkeiten als Alternative zum Verbrennungsmotor zu einer eigenen Zeitenwende im Sektor. Die Automobilindustrie in Deutschland kann als facettenreich eingestuft werden und basiert grundsätzlich auf zwei Säulen: Eine aktive Globalisierung durch die aktive Erschließung von Wachstumsmärkten, und eine „Luxusstrategie“ durch einen hohen globalen Anteil an der Fertigung von Premiumfahrzeugen.\autocite[3ff.]{puls_geschaftsmodell_2021} Daher fokussiert sich die vorliegende Arbeit auf die Rohstoffaktivitäten der Hersteller BMW AG, Mercedes-Benz Group AG, und der Volkswagen Group AG, auch aufgrund der entsprechenden Marktanteile. Diese Auswahl wird in der späteren Arbeit noch genauer dargelegt. 

Die rohstofflichen Liefer- und Wertschöpfungsketten des deutschen Automobilsektors sind global aufgestellt, daher setzt das Resilienzbestreben in erster Linie hier an, und somit auch ein hauptsächlicher Anknüpfpunkt für das Rohstoffverwaltungsrecht. Der deutsche Automobilsektor weist zudem deutlich sichtbare Abhängigkeiten auf: Nicht nur Absatzmärkte konzentrieren sich oftmals auf China, sondern auch der Bezug entsprechender kritischer Rohstoffe. So stammen knapp 98 \% der in die EU importierten Seltenen Erden aus China.\autocite[4]{gaus_rare_2021} Dass diese Marktmacht entsprechend genutzt werden könne, und etwaige Lieferstopps die Industrie bedeutsam träfe, stelle daher auch Auslandsinvestitionen auf die Prüfung.\autocite[64]{becker_vom_2023} Insbesondere im Bereich der Rohstoffe ist daher diese kritische Abhängigkeit, dessen Umfang noch bestimmt werden muss, von Untersuchungsinteresse. Maßgeblich für eine Resilienz der Automobilindustrie ist auch eine erhöhte Transparenz von Lieferketten, besonders aufgrund der international weit vernetzten Wertschöpfungsketten.\autocite[9]{kagermann_resilienz_2021} Der Sektor sieht sich zudem in der Resilienzbereitschaft durch bereits existierende Rechtsvorschriften, wie denen zum Wettbewerb, in seiner Handlungsfähigkeit beeinträchtigt.\autocite[19]{kagermann_resilienz_2021} Es besteht hierbei also ein Spannungsfeld aus wirtschaftlichen Interessen des Industriesektors, insbesondere im Bereich der Rohstoffversorgung und Wettbewerbsfähigkeit, staatlichen Interessen auch in Bezug auf Umweltschutz, sowie politischen Bestrebungen wie das beschriebene Decoupling.

\section{Gang der Untersuchung}
lorem ipsum

Daher wurde eine Charakterisierung und Beurteilung der europäischen Rohstoffverwaltung vorgenommen, die sich von den Anfängen bis hin zu den jüngsten Entwicklungen nach der Veröffentlichcung des CRMA erstreckt. 

Aufmerksamkeit erfuhren der CRMA und nationale Bergbaurechtsrahmen, jedoch nicht im Rahmen einer europäisch-übergreifenden und insbesondere vergleichenden Betrachtung.

Hierbei zeigt die vorliegende Arbeit die Herausforderungen im Rahmen der europöischen Rohstoffverwaltung auf: eine Diskrepanz zwischen Anspruch und Wirklichkeit einer europäischen Rohstoffverwaltung, einer mangelnden Europäisierung und Angleichung dieses Politikbereiches, Zuständigkeitsfragen sowie ein Missverhältnis zwischen rechtssetzenden Punkten der EU und den tatsächlichen Faktoren in der (Automobil-)Industrie auf, sodass eine Entwicklung der Rohstoffverwaltung auf europäischer Ebene sowohl durch politische Ermöglichung als auch industrielle Umsetzung verlangsamt oder, mitunter, nicht ermöglicht wird.

Neben der Ausgestaltung eines Rechtsrahmens zur Rohstoffverwaltung, welcher die über Expertenkonsultationen ermittelten Regelungslücken schließen soll, werden insbesondere mögliche Szenarien aufgezeigt sowie Handlungsempfehlungen an Politik (insbesondere hinsichtlich der weiteren Umsetzung einer europäischen Rohstoffstrategie) und Industrie, ergo an die identifizierten Stakeholder, formuliert. Hierbei wird insbesondere deutlich, dass neben einer Schaffung von Anreizen zum Ausbau einer europäischen Rohstoffgewinnung und -verarbeitung insbesondere die Stärkung von internationalen Partnerschaften und des einheitlichen europäischen Verwaltungsrahmens, ohne aber limitierende Bürokratie zu schaffen, um schließlich die Abhängigkeiten zu reduzieren, Diversifizerung zu stärken und die allgemeine Resilienz insbesondere im Industriesektor auszubauen.

\subsection{Forschungsziel und Fragen}
Die vorliegende Dissertationsabsicht entspringt also der drängenden Notwendigkeit, eine bisher unzureichend behandelte Thematik zu vertiefen, welche eine essenzielle Schnittstelle zwischen Rechtstheorie, politischen Entscheidungen und wirtschaftlichen Unternehmensstrategien bildet. Das Forschungsvorhaben wird begründet durch die Erkenntnis, dass das Zusammenspiel von staatlichem Rohstoffverwaltungsrecht im Generellen und in Bezug zu privatwirtschaftlichen Aktivitäten in der Automobilindustrie im Besonderen bislang eine bedeutende Aufmerksamkeitslücke darstellt. Die vorhandene Diskrepanz zwischen rechtlichen Regelungen, administrativen Vorgängen und unternehmerischer Praxis erfordert eine tiefgreifende Analyse. Die Automobilindustrie, als wesentlicher Pfeiler der deutschen Wirtschaft, ist bereits mit Herausforderungen im Zusammenhang mit der Rohstoffbeschaffung und den damit verbundenen regulatorischen Gegebenheiten konfrontiert. Diese Herausforderungen können nicht isoliert betrachtet werden, sondern erfordern eine umfassende Einordnung in den Kontext des staatlichen Handelns und internationaler geopolitischer Dynamiken.
Es ist offensichtlich, wie sehr staatliche und nichtstaatliche Akteure in der internationalen Wirtschaft zusammenarbeiten, um der internationalen Rohstoffwirtschaft sowohl im Sinne von \textit{soft law} als auch von \textit{hard law} eine rechtliche Struktur zu verleihen. Wie in der Literaturübersicht dargestellt können Entwicklungen in den Akzentuierungen des Rohstoffrechts auf geänderte politische, ökonomische und gesellschaftliche Ausgangsbedingungen zurückgeführt werden; in diesem Sinne erforscht die beabsichtigte Dissertation demnach die Akzentuierung des Rohstoffverwaltungsrecht als Ergebnis dieser Prozesse und trägt hiermit zum Befund des Rechtsbereiches bei. Zudem ist bisher keine nähere Auseinandersetzung mit dem Rechtsgebiet der Rohstoffverwaltung erfolgt.
Ziel des vorgestellten Dissertationsvorhabens ist demnach, die bisherige Entwicklung eines Rohstoffverwaltungsrechts zu analysieren und auf Grundlage der Erkenntnisse hinsichtlich eines rohstoffintensiven Sektors, hier die Automobilindustrie, ein Rohstoffverwaltungsrecht zu konkretisieren das auf die vorliegenden Herausforderungen eingeht. Dies erfolgt unter Erhebung und Diskussion der Anforderungen von und Auswirkungen auf relevante Stakeholder, um sicherzustellen wie ein Rohstoffverwaltungsrecht ein ausreichendes Gleichgewicht zwischen den divergierenden Interessen herstellen kann, und schließlich Auskunft über Status und Entwicklungsszenarien des Rohstoffverwaltungsrecht geben wird. Zusammenfassend wird somit ein Beitrag zur praktischen Entwicklung eines Rechtsrahmens geleistet. Darüber hinaus wird ein interdisziplinärer Dialograhmen geschaffen. Die Integration im Rahmen eines wissenschaftlichen Beitrags von Rechtstheorie, politischen Überlegungen und wirtschaftlichen Erwägungen in dieser Forschung bietet die Möglichkeit, Erkenntnisse zu gewinnen, die nicht nur für die wissenschaftliche Betrachtung von Bedeutung, sondern auch für politische Entscheidungsträger, Unternehmensführungen und andere Interessengruppen relevant sind. Dies wird auch die Kombination theoretischer Überlegungen und praktischer Anwendbarkeit gefördert, sodass schließlich effektiv zu einer verbesserten Rohstoffsicherheit durch nachhaltiges Verwaltungshandeln im Sinne rohstoffabhängiger Sektoren beigetragen wird.
Für die zukünftige Ausrichtung des deutschen Automobilsektors ist eine strategische Gestaltung des Handelns aufgrund geopolitischer Herausforderungen erforderlich – so können die Ergebnisse der Untersuchung als Beitrag zu diesen strategischen Abwägungen durch relevante Entscheidungsträger in Politik und Wirtschaft genutzt werden, insbesondere im Umgang mit rohstoffverwaltungsrechtlichen Fragen, bzw. der Auswirkung von Rechtsnormen auf die Industrie und vice versa. Somit wird ebenfalls untersucht, wie politische und wirtschaftliche Interessen in Bezug auf die Automobilindustrie am Standort Deutschland das Rohstoffverwaltungrechts beeinflussen.
Ferner wird ein Beitrag zur Frage geleistet, ob die Rohstoffversorgung aktuell und zukünftig für deutsche Automobilunternehmen unter sich verändernden rechtlichen Rahmenbedingungen gefährdet ist, da die rohstoffverwaltungsrechtlichen Gegebenheiten eine Gewinnung von Rohstoffen im Inland und die Einführung aus dem Ausland möglicherweise behindern oder nicht ermöglichen. Darüber hinaus sollen dementsprechend Handlungsempfehlungen erstellt werden, die auch vor dem Hintergrund von Rohstoffknappheit und Reduzierungsbemühungen des Rohstoffverbrauchs in eine nationale Rohstoffstrategie einfließen können. Im Fokus der Dissertation liegt demnach die Verwaltung der Rohstoffversorgung. Die Dissertation folgt der grundsätzlichen Annahme, dass Rohstoffe, insbesondere mineralische Primärrohstoffe, weiterhin eine vitale Bedeutung für den deutschen Automobilindustriesektor haben werden. Insbesondere ist auch zu ermitteln, zu welchem Grad die Versorgung mit Rohstoffen in die Verantwortung privatwirtschaftlicher und staatlicher Akteure fällt – und wie dies aus rohstoffverwaltungsrechtlicher Sicht zu beurteilen ist.
Die folgenden spezifischen Forschungsfragen leiten diese Untersuchung:

Forschungsfrage 1: Welche Rolle spielen Rohstoffe, insbesondere strategisch wichtige mineralische Rohstoffe, in der deutschen Automobilindustrie, und wie haben sich die Beschaffungsstrategien unter rechtlichen Aspekten im Laufe der Zeit entwickelt? (deskriptiver Ansatz)

Forschungsfrage 2: Inwieweit besteht ein Rohstoffverwaltungsrecht auf der deutschen und europäischen Ebene, und wie haben sich die rechtlichen Rahmenbedingungen für ein Rohstoffverwaltungsrecht im Laufe der Zeit entwickelt, insbesondere im Kontext des öffentlichen und internationalen Wirtschaftsrechts? (deskriptiver Ansatz)

Forschungsfrage 3: Welche politischen und rechtlichen Herausforderungen und Chancen ergeben sich für die deutsche Automobilindustrie im Zusammenhang mit dem Rohstoffverwaltungsrecht? (theoretischer Ansatz)

Forschungsfrage 4: In welchem Grad beeinflussen die politischen Entscheidungsprozesse das Rohstoffverwaltungsrecht, insbesondere im geopolitischen Sinne, und wie wirken sie sich auf die Automobilindustrie aus? (theoretischer Ansatz)

Forschungsfrage 5: Welche rohstoffverwaltungsrechtlichen Bewältigungs- und Entwicklungsszenarien sind denkbar, und in welchem Verhältnis stehen diese zur (politischen) Realisierbarkeit? (praxeologischer Ansatz)

Die Mitgliedsstaaten der EU wiesen aus geologischer Sicht keine optimalen Bedingungen für den Aufbau einer reinen EU-only-Versorgung mit kritischen Rohstoffen auf, insbesondere was die Vorkommen betrifft, sodass sich die Rohstoffverwaltung zwangsläufig insbesondere auf die letzten Schritte der Rohstofflieferkette (Verarbeitung, Recycling) konzentrieren muss, jedoch auch Innovationen zumindest zur partiellen Abhängigkeitsreduzierung vonnöten sind.



Die fo
\subsection{Theoretischer Rahmen für die Untersuchung von Rohstoffverwaltung}
Grundsätzlich lässt sich ein beziehungsweise das Rohstoffverwaltungsrecht aus verschiedenen, genauer zwei, Perspektiven betrachten: Zum einen kann es zunächst als ein spezifisches Verwaltungsrecht, aber auch als umfassendes rechtliches Konzept verstanden werden, das die allgemeine Verwaltung von Rohstoffen einschießt--losgelöst vom Verwaltungshandeln.

Ein Verwaltungsrecht für Rohstoffe wird dann als ein spezifischer Bereich des Verwaltungsrechts definiert, der sich demnach mit der Regulierung einerseits und mit der Verwaltung von Rohstoffen andererseits befasst. Dies beinhaltet (nicht abschließend) somit die rechtlichen Rahmenbedingungen, die durch staatliche Stellen (man denke hierbei in erster Linie an Behörden, Ministerien, Ämter) erlassen werden, um insbesondere die Exploration, den Abbau, die Verarbeitung, den Handel und letztlich Verarbeitung von Rohstoffen zu steuern, um diese Schritte nur beispielhaft zu nennen. Spezifische Aspekte, die unter diesen Rechtsbereich fallen können sind beispielsweise die Erteilung von Abbaulizenzen, Umwelt- und Sicherheitsauflagen, Festlegung von Steuern und Abgaben auf Förderung und Handel, oder aber die Überwachung und Durchsetzung von Mechanismen der gesetzlichen Bestimmungen und Sanktionierung bei etwaigen Verstößen.

Hierbei ist insbesondere interessant, inwieweit sich nationales und übernationales Recht der Verwaltung auf andere Rechtsbereiche erstrecken kann und besonders im Bereich der Rohstoffe auch die geografische Komponente des Geltungsbereiches abgedeckt ist.

Ein allgemeineres Verständnis des Rohstoffverwaltungsrechts könnte alle rechtlichen Aspekte umfassen, die die Verwaltung und Nutzung von Rohstoffen betreffen. Dies würde nicht nur die spezifischen Verwaltungsakte und -verfahren, sondern auch die allgemeine rechtliche und politische Gestaltung der Rohstoffpolitik einschließen. Hierbei ließe sich der abgedeckte Bereich weiter fassen und das Verständnis in der Hinsicht erweitern, dass z. B. nationale Rohstoffstrategien, internationale Abkommen und Kooperationen, Marktregulierungen und Handelsbestimmungen aber auch Aspekte wie Nachhaltigkeitsförderung in die Betrachtung miteinfließen.



\subsection{Methodik}
Die Basis der Arbeit bildet zunächst eine Auswertung der bestehenden Literatur, um bereits existierende politikwissenschaftliche und wirtschaftliche Theorien sowie Konzepte mit möglichem Bezug zur Thematik zu erfassen und im Zusammenhang mit dem Forschungsgegenstand zu untersuchen. Hauptsächlich herangezogen werden vergangene und aktuelle Strategiedokumente und Rechtsakte der deutschen Bundesregierung sowie der europäischen Verwaltungsorgane, sowie Studien, Presseveröffentlichungen und Stellungnahmen relevanter Industrieverbände und privatwirtschaftlicher Unternehmen. Insbesondere mögliche Strategiewechsel bei Akteuren in der Automobilindustrie hinsichtlich der Rohstoffverfügbarkeit sind hier von Belangen. Ferner ergänzt wird die literarische Betrachtung durch eine detaillierte Analyse hinsichtlich der Beschaffenheit eines deutschen und europäischen Rohstoffverwaltungsrecht, der zeithistorischen Entwicklung und der Rechtsprechung, sowie ein Vergleich des Rechtsrahmens mit etwaigen rohstoffverwaltungsrechtlichen Ausprägungen im internationalen Umfeld. Abschließend erfolgt die Analyse, inwiefern die proklamierte Zeitenwende diese Entwicklung beeinflusst hat und beeinflussen kann. Im Bereich der Rechtsakte werden die klassischen juristischen Auslegungsmethoden verwandt, bei Veröffentlichungen staatlicher Stellen (wie bspw. Strategien oder Sachstände) die Dokumentenanalyse im politikwissenschaftlichen Sinne,\autocite[203]{reh_quellen-_1995} sowie im Bereich der Wirtschaftspolitik eine deskriptive Analyse der Entscheidungsprobleme und Fragestellungen im Bereich der staatlichen Eingriffe.\autocite[6f.]{schmidt_theorie_2019}
Diese Auswertung der Literatur wird dann durch entsprechende Experteninterviews ergänzt, wobei hierbei Interviewpartner ausgewählt werden, die über eine entsprechende Expertise hinsichtlich einer (Weiter)Entwicklung rohstoffverwaltungsrechtlicher Ansätze verfügen bzw. mit der Umsetzung mittelbar und unmittelbar betraut würden, sowie entsprechende Kriterien und Ansprüche für und an diesen Rechtsrahmen formulieren. Die Auswahl der Stakeholder erfolgt somit anhand ihrer Relevanz und ihres Einflusses auf die Branche. Die Interviews sind strukturiert und zielen darauf ab, Meinungen, Erfahrungen und strategische Ansichten im Kontext der politischen und wirtschaftlichen Herausforderungen und des Rohstoffverwaltungsrechts zu ermitteln. Die erhobenen Daten dienen als Grundlage für eine umfassende Analyse, die identifiziert, wie Unternehmen aktuelle und zukünftige rechtliche Herausforderungen bewältigen. Zur Exploration verschiedener Zukunftsszenarien innerhalb der Automobilindustrie wird eine systematische Szenarioanalyse durchgeführt. Diese Analyse basiert auf den gewonnenen Erkenntnissen aus den Stakeholder-Interviews und berücksichtigt verschiedene politische, wirtschaftliche und rechtliche Entwicklungen. Die Szenarien ermöglichen es, potenzielle Verläufe in der Automobilindustrie im Kontext der Slowbalization, des Decouplings/Deriskings und des Nearshorings zu visualisieren und deren Auswirkungen aufzuzeigen. Dies bietet eine Grundlage für die Identifizierung von strategischen Optionen, die Risiken und Chancen abwägen. Die Gesamtheit der Ergebnisse dient abschließend als Grundlage um spezifische Handlungsempfehlungen abzuleiten. Sie dienen dazu, praxisorientierte Lösungen und Strategien für Unternehmen und politische Entscheidungsträger im Umgang mit den Herausforderungen der Zeitenwende zu bieten.
Die Gesamtmethode zielt darauf ab, die komplexen Wechselwirkungen und Herausforderungen, die mit der Slowbalization und dem Rohstoffverwaltungsrecht in der deutschen Automobilindustrie verbunden sind, auf systematische Weise zu erforschen. Sie ermöglicht die Integration von qualitativen und quantitativen Daten sowie die Entwicklung von praxisorientierten Lösungen und Strategien für Unternehmen und politische Entscheidungsträger in dieser dynamischen Umgebung.




\subsection{Abgrenzung des rohstofflichen Untersuchungsbereiches}
Da die vorliegende Arbeit die Auswirkungen und die praktischen Implikationen für die Automobilindustrie betrachtet, werden hauptsächlich solche Rohstoffe in Betracht gezogen, die eine entsprechende Relevanz für den Sektor darstellen – besonders für die Batterieproduktion und die dafür erforderlichen kritischen Rohstoffe, aber auch im Bereich der Halbleiter und anderer kritischer Bauteile im Automobilsektor. Neben den Rohstoffen für die Produktion sind Halbleiter für die zunehmende „automobile Seite der informations- und kommunikationstechnischen Aufrüstung im Verkehr“\autocite[26f.]{schelewsky_vulnerabilitat_2017} notwendig. Die Definition von kritischen mineralischen Rohstoffen variiert: Die USA haben seit der letzten Änderung mit dem Energy Act 2022 aktuell 50 solcher Mineralien als kritisch eingestuft.\autocite{u_s_geological_survey_final_2022} Auf bundesdeutscher Ebene ist eine solche Zusammenstellung zum aktuellen Zeitpunkt nicht vorhanden. Stattdessen hat die Europäische Kommission auf EU-Ebene eine entsprechende Zusammenstellung geschaffen; 2011 wies diese Liste lediglich 14 Mineralien auf, die mittlerweile auf insgesamt 30 solcher Stoffe angewachsen ist und ihre letzte Aktualisierung 2023 erfuhr.\footnote{COM(2023) 160 final} Hierbei existieren jedoch zwei solcher Listen: Einerseits die der \textit{strategischen} Rohstoffe, andererseits die der \textit{kritischen} Rohstoffe.\autocite[vgl. auch]{falke_neue_2023} Zwar überschneiden sich diese Listen teilweise, die der strategischen Rohstoffe zielt jedoch vor Allem auf strategisch bedeutende Sektoren und Technologien ab. 
%Bild Spiegel: Globale Hauptlieferländer der kritischen Rohstoffe
Der 2023 von der EU-Kommission vorgeschlagene „EU Critical Raw Materials Act”\footnote{COM(2023) 160 final} soll daher eine sichere Versorgung Europas mit kritischen Rohstoffen gewährleisten, insbesondere hinsichtlich ihrer Bedeutung für die Wertschöpfungsketten und der Importabhängigkeit von China. Entsprechende Maßnahmen zur Lieferkettenüberwachung und Verpflichtungen für Importeure sind ebenfalls Bestandteil. Abbildung 1 zeigt die Kombination aus ausgewählten kritischen Rohstoffen und deren geographischer Herkunft auf; insbesondere die Monopolstellung Chinas wird verdeutlicht. Ferner kann die Darstellung als Verweis auf sog. „Konfliktrohstoffe“ verstanden werden: Bei Rohstoffen, die aus „Konflikt- und Hochrisikogebieten“ stammen, ist eine besondere Lieferkettensorgfalt erforderlich.\footnote{Verordnung (EU) 2017/821} Im Bereich der Konfliktrohstoffe finden sich ebenfalls verwaltungsrechtliche Anknüpfungspunkte, bspw. durch Steuerungsmechanismen und Steuerungsmodelle sowie die Nutzung von Anreizen und Pflichten.\autocite[ausführlich am Beispiel der USA]{nowrot_rohstoffhandel_2016}

Für aktuelle und kommende Elektro"-fahrzeug"-An"-trieb"-sbatterien sind besonders die Rohstoffe Lithium, Nickel, Graphit, Kupfer, Kobalt und Mangan unerlässlich; zudem besteht die Herausforderung nicht in der absoluten Verfügbarkeit, denn global übersteigen die Vorkommen den prognostizierten Bedarf\autocite{thielmann_batterien_2020}--stattdessen entsteht die Kritikalität durch die ungleiche Verteilung dieser Rohstoffe und die sich daraus ergebende unterschiedliche Verfügbarkeit für Automobilhersteller in verschiedenen Märkten. Nichtsdestotrotz kommt es zu Fluktuationen hinsichtlich momentaner Verfügbarkeit und des Preises.\footnote{\textit{Vgl.} Rohstoffmonitor der Deutsche Rohstoffagentur} Es wird erwartet, dass die Nachfrage nach Lithium bis 2050 auf das 60-Fache der aktuellen Menge anwächst, bei Kobalt um bis zu das 15-Fache.\autocite[2]{europaisches_parlament_securing_2023} Circa 40\% der Wertschöpfung bei der Herstellung eines Elektrofahrzeugs sind in der Batterie vertreten, sodass diese einen entsprechenden Bedeutungsfaktor für die Hersteller darstellt.\autocite{bundesministerium_fur_wirtschaft_und_klimaschutz_bmwk_batterien_2020} Ferner sind insbesondere und zunehmend Halbleiter als kritische Bauteile für die Automobilproduktion zu identifizieren, die sowohl bestimmte (weitere) Seltene Erden zur Produktion, analog zu den Batterien, benötigen, aber auch als Bauteil an sich als kritischer Rohstoff ausgelegt werden können – mit entsprechenden Folgen für den deutschen Automobilsektor bei Nicht-Verfügbarkeit wie geschehen während der Covid-19-Pandemie (vgl. „Halbleiterkrise“).\autocite[77]{frieske_zukunftsfahige_2022} Die komplexe Kostenstruktur der Rohstoffversorgung in der automobilen Batterieproduktion ist zudem eine der zentralen Stellschrauben für mögliche Einsparungen,\autocite{proff_key_2023} sodass sich auch hier rechtliche Rahmen entsprechend auswirken.

Grundsätzlich lässt sich der Verwaltungsaspekt im Bereich der Primärrohstoffe in zwei Phasen einteilen. Eine umfasst hierbei die Gewinnung dieser Rohstoffe, d. h. die Erkundung, Ausbeutung und der Abbau von solchen Rohstoffen; in der Regel beginnen Lieferketten mit der Beschaffung. bzw. Gewinnung von Rohstoffen. Aus verwaltgungsrechtlicher Sicht fallen hier die Rechtsmaterien des Berg- und Abgrabungsrechts wie das Bundesberggesetz (BBergG); hierbei ist zu beachten, dass aber weit mehr als das "reine Recht des Bodenschatzabbaus" umfasst wird \autocite[245]{frau 2023}. Die zweite Phase wiederum umfasst dann die (Weiter-)Verarbeitung, Verteilung und Verwertung dieser Rohstoffe. Auf dieser zweiten Phase liegt der Fokus dieser Arbeit - auch maßgeblich bedingt durch die folgenden Gründe:
1. Insbesondere im Bereich der kritischen und strategischen Rohstoffe wie seltene Erden ist der Rechtsbereichs des Abbaus bzw. der Gewinnung im europäischen und deutschen Rahmen eher von untergeordneter Bedeutung, was der Tatsache geschuldet ist, dass entsprechende Vorkommen schlicht und ergreifend geographisch nicht vorhanden sind
2. Die zweite Phase umfasst hierbei auch den Teilbereich der Rohstoffbeschaffung. Dies bedeutet, dass eine bestimmte Rohstofflieferkette nicht zwangsläufig linear und sukzessiv auf sich selbst aufbauend gestaltet werden muss und immer mit dem Abbau in einem gleichbleibende geographischen Rechtsrahmen beginnt, sondern im Rahmen einer allgmeinen Rohstoffbeschaffung eine Art "Quereinsteig" in den Bereich des Rohstoffverwaltungsrecht möglich ist. Dies wird insbesondere dann relevant, wenn Wirtschaftsakteure Rohstoffe außerhalb des aus geographischer Sicht heimischen Rechtsbereiches beschaffen und dann durch die oben beschriebenen Schritte weiterer Verarbeitung unterziehen.
% ggf. hier noch erwähnen dass zB seltende Erden importiert werden

Ein ganz grundsätzliches Prinzip, insbesondere aus geographischer Sicht, liegt in der räumlichen Trennung der Phasen und ihrer einzelnen Schritte, denn Abbau, Verarbeitung und letztendliche Nutzung des Rohstoffs finden mitunter jeweils an unterschiedlichen Orten unter unterschiedlichen Bedingungen statt, sodass hier verschiedenste regionale, nationale und internationale Regulierungen Anwendung finden.

\subsection{Abgrenzung des verwaltungsrechtlichen Untersuchungsbereiches}

Maßgeblich für die Untersuchung ist daher zum überwiegenden Teil der Rechtsbereichs des Wirtschaftsverwaltunsgrechts.

Hierbei ist zu unterscheiden zwischen zwei möglichen Interpretation des Begriffes Rohstoffverwaltung.

% Definition und Abgrenzung Wirtschaftsverwaltungsrecht und Verwaltungsrecht

\section{Stand der Literatur}
Aufgrund der zahlreichen tangierten rechtlichen Bereiche eines Rohstoffverwaltungsrechts und der Tatsache, dass sowohl politische als auch wirtschaftliche Aspekte in die Evaluierung miteinfließen, soll der nachfolgende Teil zu einem ersten literarischen Überblick verhelfen und Termini des Untersuchungsgebietes erläutern.


Aus Gründen der Vollständigkeit ist letztlich auch die wirtschaftsverwaltungsrechtliche Perspektive des Ressourcenabbaus im Weltraum zu erwähnen, denn auch hier treffen diverse rechtliche Rahmen (Völkerrecht, Weltraumvertrag und nationale Weltraumgesetze) aufeinander, wobei die deutsche Position hier auf einen international abgestimmten Rechtsrahmen drängt. Interessant hierbei ist die Feststellung, dass es sich hier um eine "politische Haltung" handelt.
Grundsätzlich ist festzuhalten, dass die Rohstoffthematik auch in der verwaltungsrechtlichen Literatur zwar keinen rapiden, aber dennoch einen stetigen Aufmerksamkeitsgewinn verzeichnen kann.

\subsection{Überblick über den aktuellen Forschungsstand}

\subsection{Zur Existenz eines Rohstoffverwaltungsrechts}
Das Rohstoff"-verwaltung"-srecht kann bisher nicht als ein eigen"-ständiges Forschungs"-ge"-biet in der Rechts"-wissenschaft betrachtet werden, da in der jüngsten Vergangenheit im Bereich des Rohstoffrecht selbst (noch) kein ausgeprägter literarischer Fokus in Form von Schriften, Lehrbüchern oder akademischen Beiträgen wahrgenommen werden konnte.\autocites{feichtner_besonderheit_2016}{schladebach_zur_2017}
\autocite{terhechte_konsolidierung_2015}
\autocite{terhechte_falle_2012}
%terhechte357-386
\\
Mit der zunehmenden Salienz der Thematik besonders im wirtschaftlichen und politischen Felde sollte sich hierbei aber ein gesteigertes Interesse aus der wissenschaftlichen Perspektive ergeben.
Bezüglich der Schaffung eines Rechtsbereichs des Rohstoffrechts argumentiert KRAJEWSKI, dass die Rohstoffwirtschaft keiner derartige Besonderheit aufweise, sodass es „zweifelhaft“ erscheine, ob sich durch die pure Existenz von Unterschieden zu anderen Wirtschaftssektoren eine Begründung für eine eigene „wirtschaftsvölkerrechtliche Teilordnung“ schaffen lässt.\autocite{krajewski_menschenrechte_2016} Nichtsdestotrotz erscheint KRAJEWSKI eine wissenschaftliche Auseinandersetzung zumindest in Bezug auf die internationale Rohstoffwirtschaft angebracht.\autocite{krajewski_menschenrechte_2016} Das Rohstoffrecht unterliege zudem einer „Akzentverschiebung“, welche auf Veränderungen in den politischen, wirtschaftlichen und gesellschaftlichen Rahmenbedingungen und Vorstellungen im globalen System basieren.\autocite{nowrot_menschenrechtliche_2022} Es ist daher festzuhalten, dass zur Etablierung des Rohstoffrechts die Kreation etwaiger bisher nicht vorhandener Rechtsmittel oder -institutionen nicht genüge, sondern vielmehr um die „Überwindung der Fragmentierung der Rechtswissenschaft und ihrer disziplinären Kompartmentalisierung“.\autocite{feichtner_besonderheit_2016}
Eine erste Behandlung der Thematik, lange vor der aktuell insbesondere von Nachhaltigkeitsaspekten getriebenen Aufmerksamkeit im Bereich der Rohstoffe, erfolgte bereits 1985 in Dissertationsform durch SCHRAVEN;\autocite{schraven_internationale_1982} hier wird ein „brauchbare[r] Überblick” über Rohstoffregulierungsversuche erstellt sowie erste Betrachtungen zur zwischenstaatlichen Rohstoffverwaltung.\autocite{hanisch_besprechung_1984} Der Teil zur europäischen Rohstoffverwaltung erfolgt „weitgehend rechtsdogmatisch – unter Vernachlässigung der Praxis”\autocite{hanisch_besprechung_1984} , was insbesondere in dieser hier vorgestellten Arbeit entsprechend geheilt werden soll. Im Bereich der internationalen Rohstoffverwaltung hat SCHORKOPF die Codierung des Rechtsrahmens, auch in Zusammenhang mit der steigenden Bedeutung von Rohstoffen und dessen Marktfunktion untersucht.\autocite{schorkopf_internationale_2008} Hinsichtlich der bilateralen Rohstoffpartnerschaften stellte NOWROT die praktische Bedeutung von solchen Partnerschaften als klassische „Verwaltungsabkommen in der Form von Regierungsübereinkünften”\autocite{nowrot_bilaterale_2013} dar und zeigt zudem die Schnittpunkte aus unions- und völkerrechtlicher Sicht auf. In ähnlicher Weise haben JAENICKE et al. bereits 1977 eine rohstoffrechtswissenschaftliche Einschätzung zu Rohstofferschließungsvorhaben in Entwicklungsländern durchgeführt.\autocite{Rohstofferschließungsvorhaben in Entwicklungsländern}
\\
Im Bereich von deutschen Verwaltungspartnerschaften und insbesondere Rohstoffpartnerschaften böten diese zudem die Möglichkeit, den „Zugang der deutschen Industrie zu Rohstoffen durch den Abbau von Handelshemmnissen” sicherzustellen, während die entsprechenden Partnerländer eine Unterstützung im „Kapazitätsaufbau” erfahren.\autocite{ruttinger_deutschen_2016} Jedoch schien in der Mitte der vergangenen Dekade das Interesse an solchen Partnerschaften zurückzugehen, auch da die Umsetzungsergebnisse der Partnerschaften vorhandene Erwartungen nicht erfüllen konnten.\autocite{ruttinger_deutschen_2016} In letzter Zeit sind solche Partnerschaften jedoch wieder vermehrt in den Fokus geraten und unterliefen regelrecht einer „Reform”, insbesondere um eine Abhängigkeit von Importen aus China zu reduzieren\autocite{muller_reform_2023} --was wiederum die Decoupling/Derisking-Bemühungen auf deutscher und europäischer Ebene unterstreicht. Um somit die Nutzung der chinesischen Dominanz im Bereich der Rohstoffe als „politische Waffe“ zu minimieren, gibt es Bestrebungen nach einer EU-Rohstoffallianz in Form eines Verbundes der „konzentrischen Kreise“; nach einem Stufenmodell soll hier ein gemeinsamer Markt unter Berücksichtigung von Bedarf, Vorkommen, Zollverzicht und Standards geschaffen werden.\autocite{sauga_klub_2023} Trotz der gesellschaftlichen und wirtschaftlichen Bedeutung von mineralischen Rohstoffen wird kritisiert, dass diese im „öffentlichen Bewusstsein nur wenig präsent” sein würden, was zudem durch eine „mangelnde öffentliche wie politische Akzeptanz ihrer Gewinnung” erweitert würde, sodass mehrere Konfliktparteien in diesem Spannungsfeld aufeinanderträfen, mit möglicherweise sich verschärfenden Zielkonflikten zwischen den Akteuren.\autocite{kuhne_gewinnung_2020} Dies zeigt erneut einerseits die divergierenden Interessen nicht nur wirtschaftlicher Akteure in diesem Feld, sondern andererseits ein weiteres Indiz zur Erforderlichkeit eines Rohstoffverwaltungsrechts.
FRENZ beschreibt die staatliche Aktivität im Rohstoffbereich im europarechtlichen Rahmen, insbesondere hinsichtlich der Versorgungssicherung und betrachtet die Vereinbarkeit mit marktwirtschaftlichen Grundansätzen, den Grundfreiheiten und dem Diskriminierungsverbot sowie unter Aspekten der Subventionierung, mit der Erkenntnis dass für staatliche Eingriffe eine rechtliche Konstruktion zu wählen sei, die die „vielfältigen Anforderungen vor allem aus dem Beihilfenverbot, den Grundfreiheiten und dem Diskriminierungsverbot gerecht wird“.\autocite{frenz_staatliche_2023} Diese Erkenntnis bietet, nicht zuletzt auch aufgrund der Aktualität, einen weitere Ausgangspunkt für die vorliegende Dissertation. Insbesondere staatliche Rohstofffonds und ein auch im Ausland agierendes Rohstoffstaatsunternehmen können nach FRENZ hierfür in Frage kommen, dessen Finanzierung durch Unternehmensabgaben gewährleistet wird.\autocite{frenz_unternehmensabgabe_2023}

Dem Wirtschaftsvölkerrecht folgend hat ZEISBERG eine Abhandlung über ein „Rohstoffvölkerrecht für das 21. Jahrhundert“ geschaffen,\autocite{zeisberg_rohstoffvolkerrecht_2021} und hierbei die gerechte, sichere und nachhaltige Rohstoffverteilung untersucht – mit dem Ergebnis, dass sich der aktuelle Regelungsinhalt größtenteils auf unverbindliche Maßnahmen beschränkt. In diesem Zusammenhang weist GRAMLICH zudem darauf hin, dass es fraglich erscheine, ob das Wirtschaftsvölkerrecht per se überhaupt geeignete Mittel in Bezug auf rohstoffliche Konflikte bietet.\autocite{gramlich_zeisberg_2021} FEICHTNER beschreibt ein „transnationales Rohstoffrecht“ als ein eigenes Rechtsgebiet mit einer hauptsächlich „analytische[n] Perspektive“, umfassend „all jene Normen […] des nationalen, internationalen, öffentlichen oder privaten Rechts, die die politische Ökonomie der Rohstoffwirtschaft konstituieren“.\autocite{feichtner_besonderheit_2016} Die gewählte übergreifende rechtliche Perspektive ermöglicht daher einen weitreichenderen Blick als lediglich auf ein Randgebiet, welche zudem nicht durch die Abgrenzungen wie zwischen Öffentlichen Recht und Privatrecht beschränkt wird. Wenn also transnationales Rohstoffrecht zur Erkenntnis führt, wie Recht Verteilungskonflikte begründet und Verteilungsmechanismen unterstellt,\autocite{feichtner_besonderheit_2016} kann dies aufgrund der festgestellten Nähe der Rechtsbereiche analog angewendet werden. Die Wechselwirkung zwischen transnationalem Rohstoffrecht und Rohstoffverwaltungsrecht entfaltet sich durch die Art und Weise, wie diese Rechtsbereiche Distribution und Konfliktlösung in Bezug auf Ressourcennutzung strukturieren. Das transnationale Rohstoffrecht bietet einen Rahmen, um Einblicke in die Formulierung und Transformation von Verteilungskonflikten zu gewinnen, wobei dies analog auf das Rohstoffverwaltungsrecht anwendbar ist, sich jedoch auf spezifischere, nationale Kontexte fokussiert. Insbesondere auf nationaler Ebene kann das Rohstoffveraltungsrecht auf die interne Verteilung von Ressourcen eingehen, z. B. hinsichtlich der Erteilung von Lizenzen oder entsprechenden Auflagen. Die Erkenntnisse aus dem transnationalen Rohstoffrecht bilden somit einen methodischen Rahmen für die Analyse des Rohstoffverwaltungsrechts. Hierbei fungiert das transnationale Recht als Quelle grundlegender Prinzipien und Überlegungen, während das Rohstoffverwaltungsrecht diese Prinzipien in den spezifischen Kontext von Staaten oder Regionen überträgt. Diese symbiotische Interaktion beider Rechtsbereiche ist von eminenter Bedeutung für die Gestaltung der globalen und nationalen Ressourcenlandschaft sowie ihrer implizierten gesellschaftlichen Auswirkungen. Diese Erkenntnis kann auf das Rohstoffverwaltungsrecht übertragen werden: Es kann also geschlussfolgert werden, dass das Rohstoffverwaltungsrecht die Verwaltungsseite der Rohstoffsicherung darstellt sowie die entsprechenden Rahmenbedingungen für die Ermöglichung offene und transparente Märkte zu schaffen. Dies gilt auch in Bezug auf die Rohstoffpolitik der EU: So gelten hier der ungehinderte Marktzugang zum Weltrohstoffmarkt sowie eine beabsichtigte dauerhafte Versorgung von Rohstoffen aus europäischen Quellen als die zentralen Prämissen.\autocite[4]{dauke_rohstoff-_2011} Eine Einschätzung zum Rechtsrahmen eines funktionalen (Rohstoff-)Handelsgütersektors liefert OEHL, der insbesondere die Rolle dieses Rechtsrahmens in Bezug auf eine „Global Commodity Governance“ untersucht.\autocite{oehl_sustainable_2022} Im Bereich des Wirtschaftsrechts können insbesondere auch Freihandelsabkommen einen Beitrag zur Sicherung von Nachhaltigkeits- und Menschenrechtsaspekten leisten.\autocite[268]{priebe_lithiumabbau_2023}
Ein weiterer wichtiger Beitrag zur Erfassung des gegenwärtigen Literaturbestandes bilden zudem existierende und angekündigte Rohstoffadressierungen von Politik und Verwaltung, sei es in Form von Rohstoffstrategien, (internationalen) Partnerschaften im Rohstoffbereich oder sonstigen Initiativen zur Verbesserung der strategischen Ausrichtung. Im Bereich der Genehmigungsverfahren zum Rohstoffabbau hat eine Studie im Auftrag des BMWK\autocite{pavel_genehmigungsverfahren_2022} zur Verwaltungsrealität und der bisherigen Rechtsregelungen bereits erste Erkenntnisse gewonnen, auch in Bezug auf Forderungen betroffener Unternehmen und Fragen zur Rechtshierarchie EU-Deutschland. Darüber hinaus wird die Komplexität der Verfahren und des Rohstoffsektors als Hindernisse für effektive Verwaltungstätigkeit benannt. Im Hinblick auf die Schaffung von legitimer Rohstoffverwaltung unter globalen Gesichtspunkten sind Initiativen wie die Extractive Industries Transparency Initative (EITI) zu nennen.\autocite{feil_rohstoffkonflikte_2011} Auch Verflechtungen auf EU-Ebene zur Risikominimierung, z. B. Konvergenz von Verwaltungsvorschriften, wurden bereits als Strategieoption genannt.\autocite[47]{feil_rohstoffkonflikte_2011} Ferner wird auf Digitalisierung der Verwaltung gedrängt, aus automobilindustrieller Perspektive besonders in Hinsicht auf höhere Transparenz und Nachverfolgung von Lieferketten.\autocite[41]{kagermann_resilienz_2021} Auch Rufe nach einer stärkeren staatlichen Verwaltung der Widerstandsfähigkeit von Lieferketten können vernommen werden.\autocite{mortsiefer_autobauer_2022}
Die aufgezeigten Hintergründe verdeutlichen, dass ein Rohstoffverwaltungsrecht per se zum aktuellen Zeitpunkt nur rudimentär existiert und sich daher weitreichende Fragen ergeben, die nicht nur auf die eigentliche Herausbildung eines solchen (nationalen und europäischen) Rechtsbereiches abzielen, sondern unweigerlich auch auf die (verwaltungs-)politischen Beweggründe hinter der Schaffung des Untersuchungsgegenstandes, als auch die praktische Bedeutung für Akteure die durch solch einen Rechtsrahmen Beeinflussung erfahren.

\subsection{Kritische Betrachtung der vorhandenen Literatur und Identifikation der Forschungslücke}

Die vorhandene Literatur fokussiert sich insbesondere auf Versorgungsrisiken und geopolitisische Abhängigkeiten der Automobilindustrie, allen voran Lithium, Kobalt, Nickel und seltenen Erden  

Auch die Erkenntnis, dass Lieferketten diversifiziert werden müssen, Recycling und heimische Produktion vorangetrieben werden sollen und strategische Partnerschaften geschlossen werden, ist keineswegs unterbeleuchtet.\autocite{https://klardenker.kpmg.de/so-sichert-sich-die-automobilindustrie-kritische-rohstoffe-2/} Hierbei wird jedoch auch deutlich, dass diese Maßnahmen allein nicht ausreichend sind und die Erforderlichkeit der regulatorischen Harmonisierung besteht.

Nearshoring gewinnt an Bedeutung, insbesondere in Deutschland.

Jedoch ist festzustellen, dass die Literatur zum gegenwärtigen Zeitpunkt oft technisch oder wirtschaftlich orientiert bleibt, während rechtliche Aspekte -- insbesondere das Rohstoffverwaltungsrecht -- nur unzureichend, isoliert oder gar nicht betrachtet werden. Zwar lässt sich eine rege Auseinandersetzung mit dem CRMA feststellen, jedoch unterbleibt eine Einordnung in den größeren Rechtsrahmen. Es mangelt an einer systematischen rechtlichen Analyse, , wie bestehende und neue rechtliche Rahmenbedingungen mit den geopolitischen und wirtschaftlichen Trends interagieren. Die Verknüpfung von globalen Trends (Slowbalization etc.) mit nationalem und europäischem Rohstoffrecht ist kaum vorhanden.






\section{Beitrag der Dissertation}
Eine sichere Rohstoffversorgung scheint, wie dargelegt, also essenziell für das Fortbestehen der deutschen Automobilindustrie, die sich im Spannungsfeld geopolitischer (insb. hinsichtlich China), wirtschaftlicher (Globalisierungsfragen) und rechtlicher (Status des Rohstoffrechts) Herausforderungen befindet. Schäffer bezeichnete zudem 2023\autocite{Schäffer, EuZW 2023, 695} im Kontext unternehmerischer Herausforderungen im Spielfeld von Außenwirtschaft und Geopolitik eine Untersuchung der \glqq Effizienz des regulatorischen Umfelds und dessen Auswirkung auf die unternehmerische Entscheidungsfind[ung] umfassend [...] zu untersuchen\grqq als \glqq wünschenswert\grqq -- denn: die Anforderungen aus rechtlicher Sicht an Unternehmen steigen, und mit ihr entsprechende Risiken und Kosten, zeitgleich ist insbesondere die Union aus Kohärenzgründen zu Verschlankung und Vereinfachung der Gesamtheit an Regelungen angehalten (so zum Beispiel im Bereich der Berichtspflichten).\autocite{Schäffer, EuZW 2023, 695}
Die erwarteten Ergebnisse und der Beitrag dieser Arbeit sollen dazu beitragen, ein umfassendes Verständnis für die Herausforderungen, aber auch Chancen in der deutschen Automobilwirtschaft im Kontext der Rohstoffverwaltung insbesondere in Bezug auf globale sowie nationale wirtschaftliche und politische Entwicklungen zu schaffen. Durch die Verbindung mit Szenarien und Handlungsempfehlungen werden sowohl die industrielle Praxis, aber auch die verwaltungsrechtlichen Gegebenheiten zielführend miteinander vereint, sodass für diverse Akteure in Wirtschaft und Politik entsprechende Erkenntnisse durch diese Vermittlerrolle im Sinne einer verwaltungsindustriellen Zusammenarbeit gewonnen werden. Durch die Empfehlungen können insbesondere im politischen Kontext das Rohstoffverwaltungsrecht betreffende Entscheidungen effektiver getroffen werden, aber auch im automobilindustriellen Bereich durch eine etwaige Anpassung und Optimierung von Strategien in Bezug auf Rohstoffe, Lieferkettenmanagement und Nachhaltigkeit, aber auch die Erfüllung möglicher internationaler Verpflichtungen. Zwar liegt, wie aus dem vorläufigen Literaturüberblick hervorgeht, zwar bereits eine gewisse Aufmerksamkeit in rohstoffrechtlichen Vertiefungen vor, jedoch fehlt es wie gezeigt an einer umfassenden Auseinandersetzung mit einem Rohstoffverwaltungsrecht, insbesondere im Lichte der aufgezeigten Ansprüche.
Aufgrund der bereits beschriebenen wenig ausgeprägten Literaturverfügbarkeit schafft die Dissertation eine weitere Grundlage für zukünftige Forschungsvorhaben im Bereich des Rohstoffverwaltungsrecht, der politischen Ökonomie sowie für Aspekte der Nachhaltigkeit, und füllt zudem die wissenschaftliche Lücke, die durch die Akzentverschiebungen entsteht. Die Dissertation wird dazu zudem beitragen, bestehende wissenschaftliche Konzepte im Bereich des Rohstoffrechts zu erweitern und anzupassen
Die Entwicklung eines klar definiertem deutschen und europäischen Rohstoffverwaltungsrecht kann zudem dazu beitragen, Schwierigkeiten und Inkonsistenz hinsichtlich der Abgrenzung von (staatlicher) Regulierung und möglichen Eingriffen zu minimieren, ein Auseinanderfallen zwischen Policy und Rechtsrahmen zu verhindern, eine unkoordinierte Weiterentwicklung von Rechtsakten auf verschiedenen Stufen eines Multi-Level-Governance-Systems vorzubeugen, und Rechtssicherheit für beteiligte Akteure zu gewährleisten. Durch einen umfassenden Beitrag zum wissenschaftlichen Diskurs wird die Forschung nicht nur dazu beitragen, die regulatorischen Herausforderungen in der Automobilindustrie besser zu verstehen, sondern auch Impulse für zukünftige politische Entscheidungen und unternehmerische Strategien zu setzen. Somit stellt diese Arbeit einen wichtigen Schritt dar, um den Dialog zwischen Rechtstheorie, Politik und Wirtschaft zu intensivieren und konstruktive Impulse für eine effektivere und nachhaltigere Rohstoffverwaltung zu liefern. Die Dissertation identifiziert, quantifiziert, diskutiert und bewertet daher entsprechende Szenarien der Veränderungen von Rohstoffwertschöpfungsketten und ergründet die Folgen und Ansprüche für und an das Rohstoffrecht.

Die Dissertation schließt somit die Lücke der rechtlichen Perspektive auf globale Trends, genauer wie sich lowbalization, Decoupling und Nearshoring konkret auf das Rohstoffverwaltungsrecht auswirken, da dieser Aspekt bisher nicht behandelt wurde. Ferner wird untersucht, wie Transformationsstrategien in ein bestehendes Rohstoffrecht integriert werden können, insbesondere ob und wie das deutsche Rohstoffverwaltungsrecht auf die neuen Anforderungen der Automobilindustrie reagiert oder reagieren sollte (z. B. durch Anpassung des Bergrechts, neue Genehmigungsverfahren, internationale Partnerschaften). Schließlich wird auch eine Bewertung der Governance-Instrumente vorgenommen sowie Verbesserungspotential indentifiziert, sodass die Arbeit in ihrer Gesamtheit zur dogmatischen Fundierung eines modernen Rohstoffverwaltungsrechts beiträgt, welches auf Versorgungssicherheit, internationale Kooperation und Nachhaltigkeit ausgerichtet ist.

\section{TBD}

\section{Fazit}
Zwar sind bei der EU (und anderen westlichen Staaten) erste Schritte zum Aufbau einer mineralischen Lieferkette zu erkennen -- jedoch werden die Bemühungen nur langsam ausgeweitet, nicht zuletzt weil die Verantwortung an private Unternehmen abgegeben wird. Hierbei entsteht ein klassisches Koordinationsproblem zwischen Staat und Wirtschaft bzw. ein \glqq regulatorisches Investitionsdilemma\grqq, in welchem Unternehmen Investitionen zurückhalten, weil sie auf klare regulatorische Rahmenbedingungen warten, um Planungssicherheit zu haben -- gleichzeitig der Staat zögert, Regulierung einzuführen, weil er keine ausreichende wirtschaftliche Aktivität oder Nachfrage nach dem betreffenden Bereich sieht. Die regulatorische Lücke sollte also gefüllt werden, um einem \glqq First-mover disadvantage\grqq vorzubeugen, während parallel Unsicherheit über zukünftige poltische Maßnahmen abgebaut wird.

Nichtsdestotrotz können einerseits der Protektionismus seit des zweiten Amtsantritts von US-Präsident Trump und andererseits der Übergang in der EU zu \glqq grünen\grqq Technologien als Katalysatoren dienen, um die Bestrebungen nach einer autonomen und abgesicherten Versorgung zu gewährleisten.


\printbibliography 
\end{document}