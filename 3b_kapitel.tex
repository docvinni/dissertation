\documentclass[12pt,a4paper,oneside]{book} % 'oneside' für einseitigen Druck

% Kodierung, Sprache und Schrift
\usepackage[utf8]{inputenc} % Erlaubt die Verwendung von Umlauten
\usepackage[T1]{fontenc} % Bessere Schriftkodierung
\usepackage[ngerman]{babel} % Deutsche Lokalisierung

% Schriftarten
\usepackage{lmodern} % Modernere Schriftart, gut für Skalierbarkeit und Lesbarkeit

% Für Abbildungen
\usepackage{graphicx}
\graphicspath{{bilder/}} % Verzeichnis, in dem Bilder gespeichert sind

% Für Tabellen
\usepackage{booktabs}

% Für Links und PDF-Metadaten
\usepackage[hidelinks]{hyperref}
\hypersetup{
	pdftitle={Titel der Dissertation},
	pdfauthor={Autor},
	pdfsubject={Doktorarbeit in den Sozial- und Rechtswissenschaften},
	pdfkeywords={Schlüsselwörter},
}

\usepackage{array}
% Für Bibliographie - Anpassung für Geisteswissenschaften
%%\usepackage[style=authoryear-icomp,backend=biber]{biblatex}
%%\usepackage[backend=biber, style=authoryear-icomp]{biblatex}
%%\usepackage[backend=biber, style=verbose-trad1]{biblatex}
\usepackage[backend=biber, style=verbose-inote]{biblatex}
%%\usepackage{biblatex}
\addbibresource{literatur.bib} % Name der BibTeX-Datei
\DeclareNameAlias{author}{family-given} % Nachname des Autors zuerst

% Anpassung der Nummerierung mit Punkten
\renewcommand{\thechapter}{\arabic{chapter}.} % Kapitel: 1., 2., 3., ...
\renewcommand{\thesection}{\Alph{section}.} % Abschnitt: A., B., C., ...
\renewcommand{\thesubsection}{\Roman{subsection}.} % Unterabschnitt: I., II., III., ...
\renewcommand{\thesubsubsection}{\arabic{subsubsection}.} % Unterunterabschnitt: 1., 2., 3., ...
\renewcommand{\theparagraph}{\alph{paragraph}.} % Absatz: a., b., c., ...


% Für Fußnoten
%%\usepackage[bottom]{footmisc} % Fußnoten am Seitenende


% Anpassung der Kapitelüberschriften
%\usepackage{titlesec}
%\titleformat{\chapter}[hang]{\Huge\bfseries}{\thechapter\quad}{0pt}{\Huge\bfseries}

% Abstand der Fußnoten
\setlength{\footnotesep}{0.5cm}

% Tiefe der Nummerierung und des Inhaltsverzeichnisses
\setcounter{secnumdepth}{4} % Nummerierungstiefe einstellen
\setcounter{tocdepth}{4} % Inhaltsverzeichnistiefe einstellen

% Abstand zwischen Absätzen und kein Einzug
%\usepackage{parskip}
%\setlength{\parskip}{0.5em}
%\setlength{\parindent}{0pt}

% Für Abkürzungsverzeichnis
\usepackage[printonlyused]{acronym}

% Für Zitate und Theoreme (falls benötigt)
\usepackage{csquotes}

% Für Gesetzestexte, Zitate und andere strukturierte Texte
\usepackage{enumitem}

% Zeilenabstand auf 1.3
\usepackage{setspace}



% Beginn des Dokuments
\begin{document}
	% Hier beginnt der eigentliche Inhalt der Arbeit
	% ...
	
\chapter{Rohstoffverwaltung im Bezugsfeld zu anderen Dimensionen}

Rohstoffvorkommen finden sich oft in Ländern mit entsprechend hohen Risiken in umwelt-, sozial- und regierungsbezogenen Dimensionen finden\autocite{The social and environmental complexities of extracting energy transition metals} bzw. diese Länder eine mangelhafte nationale Rohstoffpolitik aufweisen, mit entsprechend negativen Folgen.\autocite{Sustainable energy transitions require enhanced resource governance}

Der Dodd-Frank-Act und die Konfliktmineralienverordnung der EU haben 

Darüber hinaus steigt die Zahl der Rechtsakte in der Union, die Unternehmen zu Lieferkettensorgfalten verpflichten, weiter an.


\section{Rohstoffe und Menschenrechte}

Rohstoffkontrollregime

\section{Rohstoffe und Umweltschutz}

Die Gewinnung von für die grüne Transformationen benötigter Rohstoffe bringt ein inneres Paradoxon mit sich: Auf der einen Seite sind sie unabdingbar für die Umstellung auf nachhaltige und umweltfreundliche Technologien, andererseits sind die Gewinnungs- und Verarbeitungsverfahren mitunter extrem umweltschädlich.\autocite{https://www.actu-environnement.com/ae/news/interview-christian-hocquard-terres-rares-applications-environnementales-impact-chine-10352.php4}

Bisher gibt es in der EU keine in Betrieb befindlichen Minen für Seltene Erden und nur zwei große Aufbereitungsanlagen. Die Überwindung von Umwelteinwänden gegen die Verarbeitung könnte noch schwieriger sein als der Zugang zu dem Mineral. 

Es wird u. A. kritisiert, dass Diskussionen zu (kritischen) Rohstoffen oftmals adverse ökologische und soziale Folgen in den (nicht-europäischen) Abbauländern nicht berücksichtigen.\autocite[15]{Kueblboeck_2023}

Lagerstätten schwerer Seltenerdmetalle treten häufig gemeinsam mit radioaktiven Elementen wie Uran und Thorium auf. Ihre Erschließung führt daher zwangsläufig zur Entstehung hoch radioaktiver Abfälle. Aus diesem Grund hat China die Förderung schwerer Seltenerdmetalle ab 2014 zunehmend in benachbarte Staaten, beispielsweise in das von Rebellen kontrollierte Gebiet in Myanmar, verlagert und konzentriert sich selbst vor allem auf die Weiterverarbeitung. Auf diese Weise sollen die ökologischen Belastungen der Rohstoffgewinnung ins Ausland verlagert werden. Für die Europäische Union, die nahezu ihren gesamten Bedarf an schweren Seltenerdmetallen importiert, bedeutet dies zugleich eine Auslagerung der damit verbundenen Umwelt- und Sozialprobleme.

Bei allen Abbauvorhaben muss daher beachtet werden: Es ist nicht möglich, lediglich den \glqq Nutzen\grqq in der EU anzusielden, denn sowohl Abbau als auch Verarbeitung kommen mit entsprechenden Herausforderungen an Umweltschutz. In anderen Worten: Deutschland und die EU stehen vor der strategischen Entscheidung, wie der Bedarf an kritischen Rohstoffen gedeckt werden kann. Grundsätzlich ergeben sich dabei drei Handlungsoptionen: Entweder werden Rohstoffe importiert und damit auch die damit verbundenen Abfälle und Emissionen aus deren Verarbeitung in Kauf genommen, oder es werden technologische Alternativen entwickelt, die den Einsatz dieser Rohstoffe überflüssig machen oder zumindest deren Einsatz auf ein Minimum reduzieren. Eine dritte Möglichkeit besteht im verstärkten Ausbau des Recyclings, da die betreffenden Rohstoffe bereits in vorhandenen Produkten verbaut sind und somit prinzipiell einer Rückgewinnung zugänglich gemacht werden können.

Rohstoffe und Klimaschutzrecht

\subsection{Tiefseebergbau}

Tiefseebergbau zählt zu 

The Metals Company Study

%menschenrechte rohstoff verwaltung

\section{Rohstoffe in der Entwicklungspolitik}

Da der Rohstoffsektor von zentraler Bedeutung für die Entwicklungspolitik ist, erfolgt die Aktivität der EU auf den globalen Rohstoffmärkten auch im Rahmen der Entwicklungszusammenarbeit.\autocite{Schorkopf, Rn. 53}



Im Rahmen der Art. 208-211 AEUV kann die Union auf eine spezielle Rechtsgrundlage für die Entwicklungszusammenarbeit zurückgreifen.

Schon bei der Betrachtung der Karte zur geographischen Lokalisierung von Rohstoffvorkommen (/ref ) fällt auf, dass sich ein bedeutender Teil in Entwicklungs- und Schwellenländern befindet.

Der Rohstoffsektor spielt insbesondere in Schwellen- und Entwicklungsländern eine Rolle.

Sektorprogramm Rohstoffe und Entwicklung

Die Verschränkung von Rohstoffpolitik und -verwaltung mit der Entwicklungspolitik der Union wird insbesondere durch den Terminus der Rohstoffdiplomatie angesprochen. Zudem ist im Rahmen der Art. 208ff. AEUV im Rahmen der Außenpolitik der Union (siehe dazu auch den entsprechenden Abschnitt) ein entsprechendes Gefüge aus Abkommen entstanden.

Die Rohstoffstrategie von 2008 verweist auf eine dreifache Aufgabe der rohstoffbezogenenen Entwicklungspolitik: Die Stärkung lokaler Staatsführung durch Governance-Verpflichtungen, Förderung eines günstigen Investitionsklimas insbesondee vor dem Hintergrund des fairen Marktzugangs, sowie des Ausbaus eines nachhaltigen Rohstoffsektors.\footnote{KOM(2008)699 S. 7f.}

Daher wird auch gefordert, dass die Union \glqq ihrer Verpflichtung zu entwicklungspolitischer Kohärenz Genüge tun\grqq sollte um sicherzustellen, dass dementsprechend die Förderung regionaler Wertschöpfung im Rohstoffsektor der nicht-eruopäischen Abbaupartnerländer im Vordergrund stehen solle.\autocite[15]{Kueblboeck_2023}

Im Sinne der Rohstoffverwaltung erstreckt sich diese somit bereits auf den Prozess vor der Unionseinfuhr von Mineralien.


\subsection{Rohstofffluch}

Gerade vor entwicklungspolitischen Hintergründen drängt sich der mittlerweile einschlägig bekannte \glqq Rohstofffluch\grqq auf -- also die Erscheinung, dass trotz des vorhandenen Rohstoffreichtums die jeweiligen Länder nicht von dessen Exploration profitieren. Das Phänomen des Rohstofffluchs ist umfassend untersucht und dokumentiert

Die Erkenntnis

\section{Tiefseebergbau}
Der Meeresboden bietet ebenfalls entsprechende Vorkommen an, nebst anderen Rohstoffen, kritischen Mineralien. Zu den rechtlichen Implikationen ist insbesondere auf den Beitrag von Jenisch zu verweisen

Der Tiefseebergbau umfasst hierbei den Abbau von Mineralien am Meeresboden auf der Hohen See außerhalb des Festlandsockels.\autocite{Weber kompakt, Rechtswörterbuch, Tiefseebergbau, beck-online} Die rechtlichen Voraussetzungen des Tiefseebergbaus stützen sich außerhalb der Wirkungsgrenzen des nationalen und europäischen Rechts (Küstenmeer, Ausschließliche Wirtschaftszone, Festlandsockel) also auf internationale Übereinkommen.[Aufgrund der vorhandenen Literatur zu den einschlägigen Grundlagen wird auf eine weitere Darstellung verzichtet, siehe exemplarisch]\autocite{Uwe Jenisch, Tiefseebergbau – Lizenzvergabe und Umweltschutz. Natur u. Recht (NuR) 2013, Nr. 12, S. 841-854.}

Gemäß dem Seerechtsübereinkommen (1982) in Verbindung mit dem Durchführungsübereinkommen (1994) war der Meeresboden zunächst als gemeinsames Erbe der Menschheit zu betrachten und der Tiefseebergbau unterlag der Verwaltung durch die Internationale Meeresbodenbehörde\autocite[Ausführlich zur Rolle der IMB]{NordÖR 2014, 421}, jedoch liegt seit dem Durchführungsübereinkommen kommerzielle Nutzung mittlerweile im Vordergrund.
Die Dimension des Tiefseebergbaus spielte bereits in der Vergangenheit in der Ausgestaltung deutscher und europäischer Rohstoffpolitik nur eine geringe Rolle, auch wegen eines fehlenden \glqq Meeresbewusstsein\grqq und im \glqq territorial-zentrierten Denken\grqq, denn andere Staaten würden diese Ressourcenquelle längst nutzen.\autocite{Jenisch} Jedoch sind insbesondere umweltbezogene Bedenken und fehlende effektive Regulierung hauptausschlaggebend für fehlendes Voranschreiten,\autocite{ZUR 2017, 323} denn trotz der Existenz eines \glqq Mining Codes\grqq gibt es keine entsprechend weiterentwickelten Bestimmungen des internationalen Wirtschaftsrechts.\autocite{NuR (2023) 45: 169–175, 170}

Hinsichtlich der unionalen Rohstoffverwaltung zum Tiefseebergbau ergeben sich die Zuständigkeiten der Union, ähnlich wie in den übrigen Dimension der rohstoffpolitischen Zuständigkeit, aus den Bestimmungen zur Handels-, Wettbewerbs- und Umweltpolitik, aber auch die Industriepolitik.\autocite[siehe ergänzend]{Jenisch}

Zudem scheint es, dass auf EU-Ebene kein politischer Wille\autocite{Jenisch, 430} (bspw. in Form von explizit europäischem Projekten zur Rohstoffsicherung über den Meeresboden im Rahmen einer Meerespolitik) vorhanden war, schon frühzeitig (respektive überhaupt) bestand, an einer effektiven und zukunftsgerichteten Rohstoffpolitik für den Meeresmineralienabau zu arbeiten. Auch innerhalb der EU herrscht Uneinigkeit hinsichtlich der Thematik.

Aktuell geht es stattdessen vielmehr um die Frage, ob, unter welchen Bedingungen und wann Tiefseebergbau erstmals real umgesetzt wird\autocite{NuR (2023) 45: 169–175}. Zudem lassen sich die Auswirkung des Tiefseebergbaus auf die bisherige Rohstoffbasis nur begrenzt abschätzen, auch weil ein Großteil der Rohstoffe dem internationalen Recht und damit der Regelung durch die IMB unterstehen und auch technische Modliäten noch nicht vollends finalisiert sind.\autocite{Commodity Top News 40, S. 9}  Nichtsdestotrotz bietet der Tiefsebergbau Potential zur Rohstoffsicherung

Mehrere deutsche Automobilhersteller stehen dem Tiefseebergbau zudem kritisch gegegnüber.

Aufgrund der aktuell nachrangigeren Bedeutung sowie offensichtlicher politischer Hürden scheint es angebracht, den Tiefseebergbau im Rahmen der Rohstoffverwaltung vorest nicht weiter zu betrachten. Grundsätzlich käme er infrage--aufgrund der mangelnden Vorkommen jedoch auch vermutlich vornehmlich im Bereich von Rohstoffpartnerschaften und -abkommen.

\section{Nichtbeachtung der Rohstoffthematik}
Unter aktuellen Gesichtspunkten mag das vorangegangene Desinteresse an einer EU-Rohstoffpolitik überraschen.

Der Mangel an spezifischen gesetzlichen Vorgaben zur Rohstoffverwaltung lässt sich durch mehrere Faktoren erklären, deren plausible Relevanz im Folgenden weiter untersucht und evaluiert werden soll. Aus rein historischer Sicht standen in Deutschland und der EU andere rechtliche und wirtschaftliche Prioritäten im Vordergrund - insbesondere in einer Zeit, in der elektrisch angetriebende Fahrzeuge und dementsprechend EV-Batterien und ihre Herkunft ein Schattendasein pflegte, und die langfristige Sicherung von Rohstoffquellen wurde oft als weniger dringlich angesehen, insbesondere in Zeiten wirtschaftlicher Prosperität oder stabiler internationaler Lieferketten – also insbesondere vor disruptiven Ereignissen wie der Covid-19-Krise oder dem russischen Angriffskrieg gegen die Ukraine und den entsprechenden Folgen. Hierbei kann der Fall der Rohstoffabhängigkeit überhaupt nur als entfernte Szenario-Blaupause dienen, denn die Abhängigkeit von Russland bei der Energieversorgung sei im Vergleich zur Abhängigkeit von China bei kritischen Mineralien gering.\autocite{Rachman, FT}


Ferner kann vermutet werden, dass die Nichtexistenz eines spezifischen Rechtsbereiches zur Rohstoffverwaltung in sich selbst begründet werden kann, sodass das Nichtvorhandensein rechtfertigt, hier auch nichtg tätig zu werden. Wie im Bereich des Bergrechts zu sehen, ist der Rechtsrahmen zudem als eher fragmentiert im Vergelich zu anderen Rechtsbereichen einzustufen, denn auch das Rohstoffverwaltungsrecht findet Anküpfungspunkte dort, aber auch im Umwelt- und Handelsrecht oder dem internationalen Wirtschafts(völker-)recht. Diese Fragmentierung führt dazu, dass es keine einheitlichen, spezifischen gesetzlichen Vorgaben gibt, die alle Aspekte der Rohstoffverwaltung umfassend regeln. Stattdessen existieren zahlreiche Einzelregelungen, die unterschiedliche Bereiche und Aspekte der Rohstoffnutzung abdecken, ohne jedoch eine kohärente Gesamtstrategie zu bilden.
Ferner: Langfristige, mitunter politisch begründete oder geprägte Rohstoffstrategien und nachhaltige Rohstoffverwaltung stehen häufig im Widerspruch zu eher kurzfristigen bzw. schneller umsetzbaren Zielen und insbesondere auch schneller wahrnehmbaren Erfolgen, auch da jede Art von rohstoffverwaltungsrechtlichen Akten einen entsprechenden Eingriff mitsichbringt. 
Es drängt sich zudem die Vermutung auf, dass das öffentliche Bewusstsein für die Bedeutung einer Rohstoffverwaltung, insbesondere in Politik und Wirtschaft, eher gering war und im Rahmen der Zeitenwende einen entsprechenden Bedeutungsgewinn erfahren hat. Ferner kann der Bereich des Rohtsoffverwaltungsrecht durchaus als ein technisches und spezialisiertes Thema wahrgenommen werden, sodass die Bedeutungskraft hinter anderen Themen zurücktritt - Stichwort Salienz.

Die Rohstoffverwaltung ist ein globales Thema, das internationale Kooperation und Abstimmung erfordert. Unterschiedliche nationale Interessen und Prioritäten erschweren die Entwicklung einheitlicher internationaler Regelungen und Standards. Zudem sind viele rohstoffreiche Länder außerhalb Europas angesiedelt, was die Einflussmöglichkeiten der deutschen und europäischen Gesetzgeber einschränkt und die Notwendigkeit internationaler Verhandlungen und Vereinbarungen erhöht.

Ein umfassendes Verständnis dieser Hintergründe ist essentiell, um die bestehenden Regelungen zu evaluieren und mögliche Wege für die Entwicklung kohärenter und effektiver rechtlicher Rahmenbedingungen zu identifizieren. Eine weitergehende wissenschaftliche Auseinandersetzung mit diesen Themen ist daher dringend notwendig, um die Herausforderungen der globalen Rohstoffversorgung nachhaltig und rechtlich fundiert zu bewältigen.

Beim BDI-Rohstoffkongress 2018 wurde zudem auch gefordert, ein entsprechendes Gesetz zum Weltraumbergbau aufzusetzen, um sich eine Vorreiterrole zu sichern -- passiert ist jedoch auch hier nichts.

% Rohstofffluch Frau


\section{Herausforderungen für das Rohstoffverwaltungsrecht auf EU-Ebene}
%Tabelle Rohstoffinitativen EU-Mitgliedsländer
Es muss erkannt werden, dass nicht zwangsläufig alle Mitgliedsstaaten Schemata zum Umgang mit kritischen Rohstoffen in Lieferketten ausformuliert haben und so auch Unternehmen über mögliche Risiken im Falle von (zeitweilig) gestörten oder versiegten Lieferketten aufzuklären beziehunsgweise entsprechende Verantwortung für solche zu übernehmen, nciht zuletzt aufgrund unterschiedlicher Relevanz und Betroffenheit einzelner Mitgliedsstaaten und daraus resultierendem, divergierendem Risikobewusstsein und entsprechender Vorsorge.

\subsection{Verfahrensautonomie der EU-Mitgliedsstaaten}
Grundsätzlich obliegt der Vollzug einzelner Titel den Mitgliedsstaaten, mit der Einschränkung durch entsprechende Vorgaben der Union, sodass die Verfahrensautonomie der Mitgliedsstaaten gegenüber der zunehmenden Europäisierung der nationalen (Verwaltungs-)Rechtssysteme als eine \glqq ungeklärte Kardinalsfrage\grqq \autocite{Ludiwgs, NVwZ 2018, 1417} betrachtet werden kann

Es bleibt weiterhin zu untersuchen, inwieweit eine unmittelbare Anwendbarkeit der rohstoffrechtlichen und -politischen Vorschriften gegeben ist, sodass sich im Verhältnis Bürger-Staat und Unternehmen-Staat auf eine entsprechende Unionsaktivität berufen werden könnte. Diese Prüfung erfolgt unberührt der Klage (auch durch natürliche oder juristische Person) auf Feststellung einer Vertragsverletzung wegen Untätigkeit nach Art. 286 AEUV.

\section{Die Politisierung des Rohstoffrechts}

Es ist mithin bekannt, dass ein Großteil der globalen mineralischen Rohstoffvorräte und insbesondere solche der seltenen Erden in diversen Staaten, darunter zahlreichen Entwicklungsländern, lagern, 

Somit kam es hier insbesondere in den letzten Jahre, ausgehend von ?, zu einer zunehmenden Politisierung im Bereich der Rohstoffe, die sich unweigerlich auch auf das Rohstoffrecht ausdehnt. Gründe hierfür sind einerseits die Verbindung des Rechtsbereiches mit internationalem Recht, welches zwangsläufig einer Politisierung unterliegt, andererseits aber auch die Erkenntnis der Politik, dass ein Handeln erforderlich sei, wie in den bereits ausführlich dargestellten Rohstoffstrategien deutlich wird.

Blickt man in die primärrechtlichen Vorgaben der Union, insbesondere im Rahmen der Handelskompetenz, so fällt auf dass die Politisierung dieses und anderer Rechtsbereiche durchaus vorgesehen ist: So wird die Union gem. Art. 205 AEUV bei ihrem Handeln auf internationaler Ebene von den Zielen und Bestimmungen aus Titel V Kap. 1 EUV geleitet und richtet es zudem daran aus. In Art. 21 EUV wird ein ähnliches Leitmotiv aufgegriffen, insbesondere unter Abs. II -- die Politisierung lässt sich hier problemlos aus den Zielen herauslesen.
%HIER BENÖTIGT ES NOCH ENTSPRECHENDER LITERATUR ZU 205 AEUV, 21 EUV DIE DAS BESTÄTIGEN
%DAUSE/LUDWIGS O. CALLIES HABEN BESTIMMT WAS


% auf räumliche Verteilung der Rostoffe eingehen

Es wird ersichtlich, dass die Verteilung von seltenen Erden höchst ungleich einzustufen ist, im Kontrast zu anderen mineralischen Rohstoffen. Deutlich wird auch, dass insbesondere als Entwicklungsländer bzw. \glqq least developed countries\grqq (LDC)\footnote{Eine allgemeine Definition für ein solches Entwicklungsland existiert nicht. Für den Gebrauch in dieser Arbeit wird daher die englische Abkürzung LDC im Sinne der least bzw. less developed countries genutzt mit dem Umfang der \glqq DAC-Liste der Entwicklungsländer und -gebiete\grqq des Bundesministeriums für wirtschaftliche Zusammenarbeit und Entwicklung.} einzustufende Staaten über solche Rohstoffvorkommen verfügen, viele Industrieländer\footnote{Bzw. \glqq developed countries\grqq folgend der Definition des International Monetary Fund (IMF)} hingegen nicht. 

Alle vier Risiken, die Kommission und EWSA schon 2008 identifizierten, haben sich mittlerweile verstärkt und entsprechende Politisierung erfahren:
\begin{itemize}
	\item Wettrennen um die Rohstoffverarbeitung
	\item Agglomerieren von Rohstoffen, auch durch Handelsmaßnahmen
	\item Wettstreit um den Zugang zu Lagerstätten und Infrastruktur, sowie
	\item Unterbrechung von Mineralien-Lieferketten
\end{itemize}
.\footnote{ABl. C277/93, 2009}




\section{Rohstoffe und internationales Wirtschaftsrecht}
Die territoriale Gebundenheit von Rohstoffen als Erzeugnisse der Urproduktion und ihre natürliche Knappheit machen ihre Beschaffung zu einer geopolitischen Herausforderung. Da gewaltsame Aneignung heute völkerrechtlich nicht mehr legitimierbar ist, bleibt der internationale Handel als zentraler Zugangsweg. Rohstoffe werden so zu einem Schlüsselfaktor internationaler Wirtschaftsbeziehungen und unterliegen deren rechtlicher Steuerung.\autocite{Terhechte, Rohstoffverwaltung, Rn. 5}

Bindungen mit Rohstoffbezug im Rahmen des internationalen Wirtschaftsrechts ergeben sich für die Union und die einzelnen Mitgliedsstaaten vorrangig aus dem GATT. Die Handels- und Zollpolitik der Union unterliegt den Vorgaben des GATT 1994 sowie entsprechender ergänzender Abkommen.\footnote{Dies bedeutet dann die Bindung an die klassischen Prinzipien der Meistbegünstigung sowie das Verbot willkürlicher Beschränkungen, sodass Maßnahmen, die den Zugang zu Rohstoffen aus Drittstaaten diskriminieren oder behindern, WTO-rechtlich zu Problemem führen können}. In Bezug auf Exportländer wie China ist die Rohstofffrage bereits Gegenstand mehrerer WTO-Verfahren gewesen (s.u.).

Die Union hat in ihren Rohstoffinitiativen und -strategien deutlich erwähnt, dass ein diskriminierungsfreier und gesicherter Zugang zu ausländischen Rohstoffvorkommen und generell den Rohstoffmärkten zu den vorrangigen Zielen zählt, da die Union aufgrund mangelnder Heimvorkommen auf Importe angewiesen ist. Diese lässt sich, wie dargestellt, auch nicht durch europäische Vorkommen vollends ersetzen. Gefahren durch Abhängigkeiten ergeben sich durch die ungleiche Verteilung der Verfügbarkeiten auf wenige Marktteilnehmer.

\subsection{Globale Rohstoffmärkte}

Bereits durch die Zollunion ist die Union befugt, gegenüber Drittländern mit einer Stimme aufzutreten. Rohstoffe, die in einen Mitgliedstaat eingeführt werden, gelten nach Art. 29 AEUV als Unionswaren und unterliegen damit keinem weiteren Zollregime innerhalb der EU. Die Zollpolitik gegenüber Drittländern wiederum fällt unter die Gemeinsame Handelspolitik gem. Art. 206 und 207 AEUV.\footnote{Konkretisierunggen dieser u. A. durch den EuGH: Commission v Council (Titanium Dioxide) (Rs. 45/86) und mit vergleichsweise aktuellem Bezug Singapore Free Trade Agreement (Gutachten 2/15, EU:C:2017:376)}

Die Stellung der Union auf internationalen Rohstoffmärkten wird aufgrund ihres geschlossenen Auftretens als Zollunion (Art. 28 AEUV) und die ausschließliche Zuständigkeit im Außenhandel (Art. 3 I e AEUV) mitunter als \glqq stark\grqq charakterisiert.\autocite{Schorkopf, Rohstoffverwaltung, Rn. 34.} Die ausschließliche Zuständigkeit bewirkt also eine instituinelle Stärke, zumindest bei rein struktureller Betrachtung. Die EU ist daher in der Lage, durch unionsweit einheitliche Handelsabkommen, etwa mit rohstoffreichen Drittstaaten wie Chile (z. B. modernisiertes EU-Chile-Assoziierungsabkommen, 2023), strategische Rohstoffinteressen kollektiv zu verfolgen. Auch das neue Abkommen mit Namibia (Memorandum of Understanding 2022) zum Aufbau einer grünen Wasserstoff- und Rohstoffpartnerschaft fällt unter diesen Mechanismus.

Die Union hat ihre handelspolitische Kompetenz in den letzten Jahren zunehmend für rohstoffpolitische Ziele strategisch instrumentalisiert. Maßgeblich ist hier die Rohstoffinitiative (KOM(2008) 699) sowie das Konzept der „Strategischen Partnerschaften“ im Kontext des Critical Raw Materials Act (CRMA), KOM(2023) 160 final. Die Union verfolgt explizit das Ziel, durch internationale Abkommen den Zugang zu kritischen Rohstoffen zu sichern -- etwa durch Konditionalität (Zugeständnisse beim Marktzugang gegen Rohstoffexportzusagen) oder sektorale Rohstoffdialoge (so mit Kanada, Argentinien, DRK).

Trotz der extensiven primärrechtlichen Grundlage ist die Stärke der Union nicht vollumfänglich; insbesondere durch die fehlende Binnenkonsolidierung der Rohstoffpolitik durch wenig harmoniserte Rohstoffgesetzgebung und eine gewisse reaktive Haltung der Union, strategische Initativen erst bei Auswirkungen externer Krise zu ergreifen, schmälert den Erfolg der Union auf den Rohstoffmärkten






\subsection{Zollrechtliche Betrachtung}
Aufgrund des Binnenmarktes und der daraus folgenden Zollunion (Art. 28 I AEUV) entfällt die innereuropäische Verwaltung von zollrechtlichen Rohstoffaspekten -- Einfuhren werden zu \glqq Unionswaren\grqq.\footnote{vgl. Art. 5 Nr. 23 UZK; Verordnung (EU) Nr. 952/2013} Sobald also kritische oder strategische Rohstoffe (z. B. Lithium, Nickel, Seltenerdmetalle) rechtmäßig in den freien Verkehr der EU überführt wurden, verlieren zollrechtliche Fragen für den innereuropäischen Warenverkehr ihre Relevanz. Der gesamte Automobil-Cluster – einschließlich seiner vorgelagerten Veredelungsstufen – profitiert dadurch von einem barrierefreien Zugang zu den benötigten Vormaterialien im Binnenmarkt.

Daraus ergibt sich aber eine gemeinsame Zollregelung gegenüber Drittstaaten. Wie bereits erkannt sind die Union und ihre Unternehmen hochgradig von Importen abhängig. Solche Rohstoffeinfuhren unterliegen hierbei dem Gemeinsamen Zolltarif (GZT) der Union\footnote{Art. 56 UZK i.V.m. Art. 31 AEUV} mit ursprungsbezogenen Regelungen.\footnote{Dies umfasst klassische Zölle, aber auch nichttarifäre Maßnahmen wie Umweltschutzauflagen oder Vorgaben zur Produktsicherheit.}

Die Union kann grundsätzlich (beispielsweise zur Förderung der Industrie) gezielt Zollaussetzungen oder autonome Zollkontingente beschließen (Art. 31 AEUV). Diese können für bestimmte kritische Rohstoffe temporäre Abgabenfreiheit gewähren -- zentrales Kriterium der Verhältnismäßigkeit ist die Verwirklichung der grundlegenden Ziele der Verbesserung der Wettbewerbsfähigkeit der Wirtschaft der Union.\footnote{vgl. Verordnung (EU) 2021/2278 des Rates, Ew. 6.}

Diese Maßnahmen könnte einer de-facto Rohstoffpolitik mit zollrechtlichen Mitteln, die auf Antrag über die Mitgliedstaaten oder von der Kommission selbst erfolgen können.\footnote{Siehe auch Verordnung (EU) 2023/1190 des Rates vom 19. Juni 2023}

Zollaussetzungen und autonome Zollkontingente sind kein Bestandteil des kodifizierten Zollrechts, sondern beruhen auf einer praxisbasierten Ausgestaltung zollpolitischer Industrieunterstützung gem. Art. 31 AEUV. Sie entfalten faktisch zentrale Steuerungswirkung im Rahmen der Rohstoffpolitik der EU, insbesondere bei kritischen Rohstoffen für industrielle Anwendungen.

%prüfen ob solche aussetzungen schon existieren

Im Rahmen von handelsrechtlichen Präferenzabkommen kann die Einfuhr strategisch wichtiger Rohstoffe zollbegünstigt oder zollfrei erfolgen, sofern die Ursprungsregeln eingehalten werden. Diese Verträge sind insbesondere für die Diversifizierungsstrategie der EU im Sinne des Derisking relevant, wie sie u. a. im Critical Raw Materials Act (CRMA) und im Aktionsplan für kritische Rohstoffe (COM(2020) 474 final) angelegt ist. Zollrechtlich sind diese Erleichterungen an die Einhaltung spezifischer Ursprungsregeln (vgl. Art. 60 UZK) sowie Nachweise wie Lieferantenerklärungen oder Ursprungszeugnisse geknüpft. Für die Automobilindustrie bedeutet dies: Lieferketten müssen nachweislich \glqq präferenzkonform\grqq gestaltet sein, was mit erheblichem administrativem Aufwand einhergeht.

Innerhalb des sog. \textit{Harmonisierten Systems}, bei dem Waren zur Einstufung numerische Codes zugeordnet werden, betrachtet mineralische Rohstoffe zudem nach ihrem Verwendungs- bzw. Weiterverarbeitungszweck. Die Union hat hierauf aufbauend die \textit{Kombinierte Nomenklatur} entwickelt mit weiteren Unterteilungen, insbesondere zur Gewährleistung des GZT. 



Mineralien werden hier als \glqq Teile mit allgemeiner Verwendungsmöglichkeit\grqq eingestuft, mit der Unterscheidung zwischen \glqq unedlen Metallen\grqq\footnote{Zu denen die meisten kritischen Rohstoffe zählen (Wolfram, Molybdän, Tantal, Magnesium, Cobalt, Bismut, Cadmium, Titan, Zirconium, Antimon, Mangan, Beryllium, Chrom, Germanium, Vanadium, Gallium, Hafnium, Indium, Niob (Columbium), Rhenium und Thallium; vgl. Abschnitt XV, 2., 3., Durchführungsverordnung 2024/2522 der Kommission)}

Innerhalb des HS-Systems erfolgt die Unterteilung beispielsweise nach Reinheit und Bearbeitungsgrad (z. B. Kapitel 26 ggü. Kapitel 71). ür das Rohstoffrecht und die Rohstoffverwaltung ergibt sich hieraus ein zentraler Handlungsbedarf: Einerseits sind präzise Kenntnisse der HS-Codierung notwendig, um Handelsflüsse strategischer Rohstoffe rechtssicher zu überwachen, exportkontrollrechtlich zu erfassen oder präferenzielle Handelsabkommen gezielt zu nutzen. Andererseits können Lücken oder unzureichende Differenzierung in der Codierung die Umsetzung rohstoffpolitischer Zielsetzungen – wie etwa das EU-Derisking durch Diversifizierung kritischer Rohstoffquellen – behindern. Entsprechend wächst das Interesse an einer Harmonisierung rohstoffpolitischer Klassifikationen mit dem internationalen Zollnomenklatursystem, etwa durch spezifische Subpositionen für besonders relevante Rohstoffe wie Seltene Erden oder Lithiumverbindungen.

Diese verstreute Erfassung erschwert die rechtssichere Identifikation und Nachverfolgbarkeit kritischer Rohstoffe im Außenhandel. Im Kontext der EU-Rohstoffstrategie und der Verordnung über kritische Rohstoffe (Critical Raw Materials Act – CRMA) gewinnt eine präzise Zuordnung an Bedeutung, da auf ihr Exportkontrollmaßnahmen, Investitionsprüfungen, Herkunftsnachweise und die Inanspruchnahme präferenzieller Ursprungsregeln aufbauen. Juristisch relevant ist dabei nicht nur die technische Klassifikation, sondern auch ihre Auswirkung auf handelsrechtliche Schutzinstrumente und WTO-Verpflichtungen. So sind tarifäre Maßnahmen, Exportbeschränkungen oder Subventionen an die spezifische Codierung geknüpft, wobei unklare oder zu breit gefasste Warennummern potenziell gegen das Diskriminierungsverbot nach Art. I und III GATT verstoßen oder zu nicht-notifizierten Handelshemmnissen i.S.v. Art. XI GATT führen können. Auch handelsrechtlich relevante Differenzierungen, wie zwischen Rohform und verarbeiteten Produkten, basieren auf HS-Codes und entscheiden über Marktzugänge im Rahmen von Freihandelsabkommen. Für die Rohstoffverwaltung ergibt sich daraus ein Spannungsverhältnis: Einerseits bedarf es einer internationalen Standardisierung zur Unterstützung strategischer Autonomie und Rohstoffsicherheit, etwa durch gezielte Unterpositionen für kritische Rohstoffe wie Lithium oder Gallium. Andererseits muss diese Standardisierung WTO-konform ausgestaltet werden, um protektionistische Tendenzen zu vermeiden und die multilaterale Handelssystematik nicht zu unterminieren. Die internationale Zollnomenklatur steht damit zunehmend im Fokus geopolitisch motivierter Rohstoffstrategien, was neue Anforderungen an ihre rechtliche Kohärenz und strategische Steuerungsfähigkeit stellt.

Demgegenüber erlaubt die Ausdifferenzierung der Klassifikation eine genaue Zuordung der Rohstoffeigenschaften, so z. B. hinsichtlich der beabsichtigten Weiterverarbeitung.\autocite{Schorkopf, Rohstoffverwaltung, Rn. 8.}

Die teilweise unsystematische Verteilung strategisch bedeutender Rohstoffe über verschiedene Warengruppen erschwert eine transparente Erfassung und gezielte Steuerung

\subsection{EU-rechtliche Ausnahmevorschriften}

Das Primärrecht hält Ausnahmevorschriften vor, die eine Einschränkung des freien Warenverkehrs von Rohstoffen über Art. 36 AEUV, Cassis-de-Dijon, Art. 347 AEUV und Art. 122 AEUV ermöglichen.\autocite{Schorkopf, Rohstoffverwaltung, Rn. 10}


\subsection{Exportrestriktionen und Exportkontrolle}



Exportkontrollen sind im Bereich der militärischen bzw. sog. \glqq Dual-Use-Güter\grqq hinreichend geläufig -- jedoch kommen hierzu mittlerweile auch solche Exportkontrollen, die zur Erreichung politischer Ziele dienen und in direktem Bezug zur Sicherstellung von technologischer und wirtschaftlicher Sicherheit stehen.\footnote{Text} Diese Maßnahmen sind nicht als kurzfristige Reaktionen zu verstehen, sondern fügen sich in das Bild eines größeren Wandels in der Politik, vor Allem in China und den USA, sodass Exportbeschränkungen als Instrument innerhalb wirtchaftlichen Abhängigkeiten genutzt werden.

Exportkontrollen stellen einen wesentlichen Teil der außenpolitischen Strategien China sund den USA dar.

Auch Deutschland bzw. die EU könnten hierbei unter Druck geraten, der Politik der USA zu folgen -- aber auch aus eigenem strategischen Interesse wird eine Mitgestaltung der Kontrollen auch unter Einbeziehung der europäischen Partner gefordert.\autocite{Medunic, FiliP. Deutschland muss Exportkontrollen strategischer gestalten, DGAp Memo Nr. 15, Juli 2024, S.1.}


\subsubsection{China}

China hatte schon vor vielen Jahren die Weitsicht, aus der Dominanz seiner Seltenen Erden Kapital zu schlagen: Deng Xiaoping, \textit{Überragender Führer} Chinas von 1978 bis 1989, wird das 1987 ausgesprochene Zitat \glqq Der Nahe Osten hat Öl. CHina hat seltende Erden\grqq zugeschrieben.\footnote{Eine Primärquelle ist hier nicht auszumachen; andere Quellen sprechen darüber hinaus von 1992.}

Auch 2008 bzw. 2011 wurden bereits Befürchtungen hinsichtlich der Versorgungssicherheit bei chinesischen Exportverboten\autocite{Top Commodty News 31, S. 1} und der Problematik illegaler chinesischer Verarbeitung\autocite{Top Commodty News 36} geäußert.

China besitzt neben Vorkommen kritischer und strategischer Rohstoffe durch die lokale Integration der Vorleistungswertschöpfungskette, also der Verarbeitung von Rohstoffen, entsprechende strategisches Potential zur Beeinflussung der globalen Versorgungslage -- und besitzt somit das sprichwörtliche \glqq Ass im Ärmel\grqq.\autocite{Rachman, Gideon: }

Chinesische Kontrollen erfolgten oftmals als Antwort bzw. \glqq Vergeltung\grqq auf Maßnahmen der USA: So veröffentlichte China im Oktober 2020 das \glqq PRC China Export Control Law\grqq in Reaktion auf die Entscheidung im August 2020 durch das US BIS, Huawei auf die sog. Entity-List zu setzen\footnote{BIS, Entitiy List}. Auch im weiteren Verlauf erfolgten chinesische Exportkontrollen auf mineralische Rohstoffe vergleichsweise schnell innerhalb weniger Tage in Reaktion auf US-Maßnahmen zu Halbleitern (Oktober 2023, Dezember 2024) und auf angekündigte Zollerhöhungen.\autocite{International Institute for Strategic Studies, Export controls: China and the United States’ use of export controls, 2010–25.} So begann China kurz dem nach dem von US-Präsident ausgerufenen \glqq Liberation Day\grqq am 2. April 2025 seine Karten zu spielen und verhängte Exportkontrollen für insgesamt sieben Seltenerdmetalle. Angesichts der US/Importabhängigkeit von chinesischen Mineralien sah Trump sich gezwungen, die Zollerhöhungen mehrfach zu verschieben. Damit hat China dauerhaft ein veritabel effektives Instrument geschaffen: sollten die USA Zölle oder ähnliche Beschränkungen einführen, kann China jederzeit die mächtige Stellschraube der Exportkontrolle nutzen -- oder zumindest damit drohen.

2010 reduzierte China die zulässigen Exportquoten für seltene Erden um 40\%. Die USA, die EU und Japan strebten daraufhin ein DSU-Verfahren\footnote{DS431} gegen China bei der WTO an.\autocite{EuZW 2012, 286} Das Panel urteilte, dass sich China mit dem Eintritt in die WTO der Verpflichtung unterwarf, Exportbeschränkungen jeglicher Art zu eliminieren, mit Ausnahme einiger Exportzölle für bestimmte Produkte, zu welchen die mineralischen Rohstoffe nicht zählen. China hingegen verwies auf die Beschränkungen aus Gründen des Umweltschutzes, welche jedoch abgewiesen wurden -- zwar kann ein WTO-Mitgliedsland über Intensität des Abbaus entscheiden, aber die handelsrechtlichen Regeln der WTO werden sofort nach Abbau gültih.\autocites{EuZW 2014, 684}{EuZW 2014, 283} Ein ähnliches Verfahren gegen China mit im Ergebnis gleicher Urteilsbegründung war bereits ab 2009 bis 2011 geführt worden.\footnote{DS395}


Ein weiteres Beispiel sind von Indonesien eingeführte Exportverbote für Nickelerz sowohl 2014 als auch 2020. Das Land gilt als weltweit führender Nickelförderer.\autocite[52]{DERA Rohstoffinformation Indonesien, } Nickel zählt hierbei zu den strategischen und kritischen Rohstoffen. Ziel war es in beiden Fällen, durch das Exportverbot eine lokale Wertschöpfung zu erzielen -- durch das Exportverbot stiegen jedoch auch erwartungsgemäß die Marktpreise und insbesondere China forcierte lokale Verarbeitung in Indonesien.\autocite{DERA} Chinas Beschränkungen für Seltene Erden stützen sich auf das Exportkontrollgesetz von 2020, das den Abfluss von Materialien mit doppeltem Verwendungszweck aus Gründen der nationalen Sicherheit einschränkt. Artikel 5 des Gesetzes übertrug sowohl dem Staatsrat als auch der Zentralen Militärkommission Durchsetzungsbefugnisse. Artikel 48 erlaubt es, das Gesetz zu nutzen, um auf Einschränkungen zu reagieren, die von anderen auferlegt werden. Die erste offene Anwendung für kritische Mineralien erfolgte im Juli 2023. Unter Berufung auf Erwägungen der nationalen Sicherheit verhängte das chinesische Handelsministerium Exportkontrollen für Gallium und Germanium. Später im Jahr 2023 schränkte das chinesische Handelsministerium Graphitprodukte ein, insbesondere solche, die in Lithium-Ionen-Batterieanoden verwendet werden. 

Die Europäische Kommission beantrage daher 2019 ergebnislose Konsultationen über die WTO mit Indonesien, sodass in einem DSU-Verfahren 2022 durch die WTO eine Verletzung des Art. XI:1 GATT festgestellt wurde\footnote{DS592} und die Beschränkung für ungültig erklärt.\footnote{Pressemitteilung der Kommission: WTO-Panelentscheidung gegen indonesische Ausfuhrbeschränkungen für Rohstoffe, 30. November 2022, IP/22/7314}

Grundsätzlich weisen die Verfahren der EU daher hohe Relevanz für die Sicherung der Versorgung der Union auf.

Bekanntermaßen ist das Streitschlichtungsverfahren der WTO jedoch seit 2019 praktisch wirkungslos bzw. blockiert,\footnote{Altemöller EuZW 2022, 207} da das WTO-Berufungsgremium nicht entsprechend besetzt wurde. Daher erweiterte die EU 2021 mit der Aktualisierung\footnote{Verordnung (EU) 2021/167} ihrer Verordnung zur Anwendung und Durchsetzung internationaler Handelsregeln\footnote{Verordnung (EU) Nr. 654/2014}, sodass die EU unabhängig von einer möglichen Berufung nach einem gewonnenen Verfahren vor dem WTO-Panel einseitig Vergeltungsmaßnahmen aussprechen kann. Auch kann sich vor ähnlichen Situationen im Rahmen anderer Abkommen abgesichert werden, sollte ein Drittstaat Funktionsmechanismen ausnutzen.\footnote{s. hierzu Ew. 4, VO 2021/167} Dies erscheint auch vor dem Hintergrund neuerlicher Verfahren angebracht.\footnote{Seit dem 30. Mai 2023 untersucht ein Panel der WTO eine Beschwerde Indonesiens zu Antidumpingmaßnahmen der EU bei Stahlimporten aus Indonesien (DSD616) in Reaktion auf das Verfahren 592.}
Weiterhin besteht die Problematik aufgrund des funktionslosen DSU-Mechanismus, dass Panel-Berichte nach einer Anfechtung \textit{into the void} verschoben werden, d. h. in die Leere des Verfahrens, da kein funktionierender Appelate Body besteht\autocite{Therrien-Tremblay, A.: Settling Disputes at the World Trade Organization: Alternatives to Appealing “Into the Void”, 2024} -- daher laufen auch alle zukünftigen Entscheidungen Gefahr, in diese Leere zu laufen.
Die Kommission unterstützte im Rahmen der OSA eine entsprechende Reform der WTO.\footnote{COM(2021) 66, S. 13.}

Auch wenn grundsätzlich die Ausfuhren in Drittländer keinen mengenmäßigen Beschränkungen unterliegen,\footnote{siehe Art. 1 Verordnung (EU) 2015/479} erlaubt die Union erlaubt explizit Beschränkungen der Ausfuhr, so auch von Rohstoffen. Dies umfasst Schutzmaßnahmen \glqq aufgrund einer außergewöhnlichen Entwicklung des Marktes\grqq, d. h. entweder die Prävention eine durch einen Mangels an \glqq lebenswichtigen Gütern\grqq Krise oder aber die Sicherstellung internationaler Verpflichtungen der Union bzw. den Mitgliedsstaaten.\footnote{Art. 6. (EU) 2015/479}

Art. 6 Abs. 1 lit. a erlaubt der Kommission, im Fall oder zur Prävention einer versorgungskritischen Lage geeignete Maßnahmen zu ergreifen, etwa durch Exportbeschränkungen, Genehmigungspflichten oder Zuteilungssysteme. Die zentrale Voraussetzung ist ein „Mangel an lebenswichtigen Gütern“, wobei dieser Begriff – in der Auslegungspraxis der Kommission – nicht allein auf Nahrungs- oder Medizinprodukte beschränkt ist. Auch strategisch systemrelevante Industriegüter können in die Reichweite des Art. 6 rücken, wenn ihre Knappheit systemische Folgen für Wirtschaft oder Gesellschaft auslöst --  Durchführungsverordnung (EU) 2020/402 zur Einführung von Ausfuhrgenehmigungen für persönliche Schutzausrüstungen, gestützt auf eben jene Verordnung 2015/479. Für kritische Rohstoffe – wie etwa Lithium, Seltene Erden oder Magnesium – gilt: Kommt es zu einem signifikanten Angebotsrückgang oder droht ein systemischer Produktionsausfall in strategischen Industrien (z. B. Automobil- oder Halbleiterproduktion), wäre die Anwendung von Art. 6 denkbar und rechtlich gedeckt. Dies würde ein faktisches Exportmoratorium ermöglichen, um die Binnenmarktversorgung zu priorisieren. Art. 6 Abs. 1 lit. b verweist auf die internationale Verpflichtungstreue der Union, etwa im Rahmen von WTO-Recht, bilateralen Handelsabkommen oder Multilateralen Rohstoffinitiativen (z. B. die Extractive Industries Transparency Initiative – EITI).

Die Klausel ist relevant, wenn Exportmaßnahmen erforderlich sind, um solchen Verpflichtungen nachzukommen. Denkbar ist dies etwa, wenn eine durch die Union eingegangene Vereinbarung zur nachhaltigen Nutzung von Rohstoffen, zur Kontrolle von Dual-Use-Gütern oder zur Transparenz in Lieferketten nur durch kontrollierte Ausfuhren umsetzbar ist. Im Rohstoffkontext betrifft dies etwa die EU-Verordnung 2017/821 (Konfliktmineralien) oder künftige Verpflichtungen aus Rohstoffpartnerschaften (z. B. mit Chile, Namibia). 

Die Vorschrift aus Art. 6 ist für die EU-Rohstoffpolitik latent strategisch bedeutsam, auch wenn sie bislang nur selten rohstoffbezogen angewendet wurde. Ihre Relevanz steigt jedoch im Kontext wachsender geopolitischer Spannungen und begrenzter globaler Rohstoffverfügbarkeit. Denn sie erlaubt es der Union, bei Versorgungsengpässen oder politischem Handlungsdruck kurzfristig und autonom zu reagieren – ohne neues Primärrecht schaffen zu müssen.

Im Licht des Critical Raw Materials Act (COM(2023) 160 final), der eine stärkere Binnenmarktsteuerung von Rohstoffen anstrebt, könnte Art. 6 zukünftig als Flankierungsinstrument genutzt werden -- z. B. um zu verhindern, dass strategische Rohstoffe, die im Rahmen eines strategischen Projekts gewonnen oder recycelt wurden, ungehindert exportiert werden, obwohl sie für den europäischen Binnenmarkt bestimmt sind.

Im Juni 2024 unterzeichnete Premierminister Li Qiang die Vorschriften zum Management von Seltenen Erden, in denen er das Staatseigentum an Seltenerdressourcen bekräftigte und darauf bestand, dass diese "gemäß den Beschlüssen der Partei" verwaltet werden. China kündigte daraufhin Exportkontrollen für Antimon und verwandte Produkte an und berief sich dabei erneut auf die nationale Sicherheit. Ende 2024 verhängte China umfassendere Beschränkungen für Wolfram-, Graphit-, Magnesium- und Aluminiumlegierungen, die für Halbleiter-, Batterie- und Verteidigungsanwendungen von entscheidender Bedeutung sind. m Gegensatz zu den vorherigen Beschränkungen, die ein Ausfuhrgenehmigungssystem vorsahen, kam dies einem vollständigen Verbot gleich, das ausdrücklich auf Ausfuhren in die Vereinigten Staaten abzielte. Der Schritt wurde weithin als Vergeltung gegen die US-Sanktionen im Zusammenhang mit Halbleitern interpretiert.

Im April 2025 erweiterte China die Liste der kritischen Rohstoffe, die von Exportkontrollen betroffen sind.\autocite{https://www.reuters.com/world/china/chinas-export-controls-are-curbing-critical-mineral-shipments-world-2025-04-20/} Dies folgte auf Beschränkungen für Gallium und Germanium (2023) sowie Antimon (2024) und war eine Reaktion auf die Ankündingung der USA, Zölle in Höhe von 125\% auf chinesische Waren zu erheben. Die Beschränkung galt weltweit, sodass nicht nur US-Importeure betroffen waren, und für die Rohstoffe und Rohstoffmischungen Dysprosium, Gadolinium, Lutetium, Samarium, Scandium und Yttrium.

Hierbei nutzte China die Rohstoffbeschränkungen zudem als Verhandlungsinstrument zur Erreichung von Zugeständnissen seitens der EU im Bereich der Technologieexportbeschränkungen.\autocite{Rohstoffe als Verhandlungswaffe}

Kurzfristige und unvorhergesehene Exportbeschränkungen Chinas bei Seltenen Erden sind auch hier keine Neuigkeit: So wurden chinesische Exporte dieser Mineralien nach Japan im September 2010 kurzfristig eingeschränkt, als Reaktion auf die Festsetzung eines chinesischen Schiffes in japanischen Gewässern.


Unternehmen, die die betroffenen Mineralien exportieren, sind verpflichtet eine entsprechende Exportlizenz zu beantragen.\autocite[Dessen Beantragung veranschlagt sechs Wochen;]{Wolf. Edda: China führt Exportauflagen für kritische Metalle ein, GTAI} Die Lizenzen galten für alle Exporte, nicht nur für US-Verkäufe. Die Lizenzausgabe war limitiert, was zum Teil auf die Verwirrung über die Tatsache zurückzuführen war, dass in den Codes des chinesischen Harmonisierten Systems nicht zwischen den Kategorien von Seltenerdmagneten unterschieden wird und das chinesische Handelsministerium MOFCOM zunächst keine Lizenzvergabeverfahren entwickelt hatte. Dies führte insbesondere in den ersten Tagen zu einem faktischen Erliegen des Handels mit Seltenen Erden. Auch 

Kritisch wurde zudem beurteilt, dass die Antragsteller mitunter vertrauliche Informationen zu den benötigten Rohstoffen mitteilen mussten, auch in höherem Detailgrad.

Im Rahmen eines Treffens am 7. Juni mit EU-Handelskommissar Šefčovič erklärte das MOFCOM, dass künftig Maßnahmen ergriffen würden, um die Bearbeitung der Anträge von EU-Unternehmen zu beschleunigen. Zu diesem Zweck soll ein sogenannter „grüner Kanal“ eingerichtet werden, der den Zugang zum chinesischen Markt erleichtert. Im Gegenzug wurde jedoch auch Chinas Erwartung geäußert, dass die EU den Export von Hochtechnologieprodukten nach China unter Einhaltung geltender Rechtsvorschriften ermögliche; diese Aussage bezieht sich vermutlich auf die restriktiven Exportregelungen der Niederlande, die unter dem Einfluss der Vereinigten Staaten eingeführt wurden und den Verkauf von Anlagen zur Halbleiterproduktion an China untersagen (ASML). 
Trotz der Vereinfachungszusagen erfuhren deutsche Unternehmen weiterhin Schwierigkeiten bei der Lieferung, auch durch die Erforderlichkeit der Bereitstellung von sensiblen Unternehmensdaten -- die Auflagen seien \glqq erkennbar politisch motiviert\grqq\autocite{Wirtschaft fordert mehr Tempo bei Rohstofffonds} und kämen einem \glqq faktischen Exportstopp\grqq gleich.\autocite{Seiwert, M.}

Der Fall zeigt, wie Exportkontrollen insbesondere auch den Automobilsektor beeinflussen. Zwar führten die  Maßnahmen insbesondere in den USA zu weitreichenden Folgen, jedoch auch im europäischen Raum. Die US-amerikanische Alliance for Automotive Innovation als auch die Vehicle Suppliers Association wandten sich an das Weiße Haus, und Ford stellte zeitweise die Produktion ein. Auch ACEA bestätigte Produktionspausen einiger Hersteller. China konnte somit die Auswirkungen der US-Zölle zumindest verkraften, nicht aber die USA die Wirkung der Exportkontrollen auf den Automobilsektor. Auch waren von den Exportbeschränkungen bisher nur insgesamt fünf Seltene Erden betroffen, sodass China noch weitere Möglichkeiten zur Einflussnahme nutzen kann.

Wenn China die Exporte wichtiger Mineralien weiter drosselt, könnten höhere Preise Anreize für Bergbau- und Verarbeitungsausgaben in anderen Ländern schaffen.

Als Reaktion auf die seit 2018 höheren US-Zölle auf Importe aus China und insbesondere die seit 2025 zunehmenden bilateralen Spannungen in anderen Bereichen hat China im 2024 Mineralien ins Visier genommen, die für die US-Verteidigung von entscheidender Bedeutung sind, darunter Gallium und Germanium. Im April 2025, nachdem die USA Zölle auf chinesische Waren angehoben hatten, kündigte China an, dass Exporteure von insgesamt zwölf weiteren Rohstoffen Lizenzen für den Export benötigen würden; später einigten sich USA und China auf eine Lockerung der Exportkontrollen.

Nichtsdestotrotz: je länger die Angebotsverschärfung anhält, desto mehr wird Chinas Status als zuverlässiger Lieferant beschäftigt, und desto attraktiver und notwendiger ist die Suche nach alternativen Bezugsquellen -- obgleich solche Reaktionen eher auf Jahre als auf Monate ausgelegt sind. Die Exportkontrollen auf seltene Erden läuteten daher eine \glqq neue Ära chinesischer Wirtschaftsstaatskunst\grqq und sei ein Beleg für eine Sanktionspolitik, die auch größere Volkswirtschaften unter Druck setzt.\autocite{Miller, C.: China’s weaponisation of rare earths is a new kind of trade war., Financial Times} Chinas vorherige Exportkontrollen seien eher als politische Signale denn als ökonomisch substantiell zu verstehen, jedoch gelte dies nicht mehr für die Maßnahmen 2025, die innerhalb weniger Wochen zu Produktionsausfällen in der gesamten Automobilindustrie führten.\autocite{Miller, C.: China’s weaponisation of rare earths is a new kind of trade war., Financial Times} Zudem sei deutlich geworden, wie unvorbereitet westliche Regierungen und Unternehmen auf die Instrumentalisierung der seltenen Erden waren -- in anderen Worten: \glqq Even those who cannot name a single rare earth element know that China dominates their production. Nevertheless, over the decade and a half since China first cut rare earth exports to Japan in 2011, the west has failed to find new suppliers. [...] This is a weapon they have been staring at for decades. They should not have been surprised when Beijing finally pulled the trigger.\grqq \autocite{Miller, C.: China’s weaponisation of rare earths is a new kind of trade war., Financial Times}

China hat somit seine Exportkontrollpolitik im Laufe der Jahre verschärft, und die Beschränkungen auf seltende Erden fügen sich als ein \glqq letzter Schritt\grqq in dieses Bild -- und obwohl China somit mittlerweile ein umfassendes Gefüge an Instrumenten aufgebaut habe, wurden diese lange nicht genutzt, jedoch zeige die zunehmende Einführung von Maßnahmen dass China anderen Märkten durchaus schaden kann und Druckmittel nutzt, sollten seine Interessen berührt sein.\autocite{https://beschaffung-aktuell.industrie.de/einkauf/seltene-erden-total-abhaengig-europas-suche-nach-alternativen/}

Die Exportrestriktionen Chinas stellen somit eine faktische Bedrohung für die Versorgungssicherheit europäischer Schlüsselindustrien dar und bilden damit eine zentrale Legitimationsgrundlage für die Kodifizierung eines europäischen Rohstoffverwaltungsrechts. Wie deutlich wurde, ist besonders der Automobilsektor betroffen: Permanentmagete in Elektromotoren bestehen nahezu ausschließlich aus chinesischen Materialien, Produktionslinien mehrerer Hersteller mussten kruzzeitig stillgelegt werden und waren nur eine Blaupause für vollumfängliche Produktionsausfälle die drohen, sollten Ausfuhren gänzlich eingestellt werden.
Diese Entwicklung offenbart die strukturelle Verwundbarkeit europäischer Wertschöpfungsketten und unterstreicht die Notwendigkeit eines rechtlich abgesicherten, strategisch gesteuerten Rohstoffregimes. Die Abhängigkeit von einem einzigen Drittstaat -- der bei bestimmten Rohstoffen bis zu 90 \% der globalen Raffinierungskapazitäten kontrolliert -- widerspricht den Grundprinzipien der europäischen Wirtschaftssicherheitsstrategie und gefährdet die technologische Souveränität der Union. Die rechtliche Reaktion in Form eines kodifizierten Rohstoffverwaltungsrechts ist daher nicht nur politisch opportun, sondern ordnungspolitisch geboten. Gegenargumente, die auf die Effizienz globaler Märkte oder die Gefahr protektionistischer Tendenzen verweisen, greifen zu kurz. Denn die chinesischen Maßnahmen sind kein Ausdruck marktüblicher Preisbildung, sondern gezielte geopolitische Instrumentalisierung von Ressourcenmacht. Die EU sollte hierauf nicht mit einem Decoupling, sondern mit Diversifizierung reagieren; die Kodifizierung eines europäischen Rohstoffverwaltungsrechts ist somit Ausdruck einer neuen Realität, denn die Sicherung industrieller Resilienz und technologischer Wettbewerbsfähigkeit erfordert eine aktive, rechtlich fundierte Steuerung der Rohstoffversorgung. Die chinesischen Exportkontrollen fungieren dabei als Katalysator für eine überfällige rechtliche Systematisierung. Ferner würde auch zum aktuellen Zeitpunkt der grüne Wandel bzw. der Übergang hinzu nachhaltigeren Antriebsformen verlangsamt werden oder gar zum Stillstand koommen, sollte die Versorgung zunächst mit chinesischen sowie auch generell und alternativen Mineralien nicht sichergestellt werden können.

In den vergangenen Jahren haben chinesische Unternehmen -- oftmals unter strategischer Anleitung staatlicher Institutionen – umfassende Kompetenzen in der Verarbeitung kritischer Rohstoffe entwickelt. Dabei konnten sie sich insbesondere durch niedrigere Produktionskosten international behaupten. Initiativen zur Etablierung alternativer Förder- oder Verarbeitungsstandorte außerhalb Chinas scheiterten häufig an wirtschaftlichen Rahmenbedingungen: Die Kosten lagen deutlich über dem chinesischen Niveau, was unter anderem auf strengere Umweltauflagen in Regionen wie Europa zurückzuführen ist.

Neben den Exportkontrollen kam es ab 2011 zudem zu einer umfassenden Regulierung des Seltenerd-Sektors mit dem Ziel, die Produktionskapazitäten einzuschränken. Hierbei wurden entsprechende Quoten auf eine ausgewählte Zahl von Verarbeitern aufgeteilt, wobei davon ausgegangen wird, dass die tatsächlichen Produktionszahlen über den offiziellen liegen dürften, auch da mitunter massive Überkapazitäten vorlagen.\autocite{Top Commodity News 57, S. 1ff}

%von OA: Erhebliche langfristige chinesische Exportbeschränkungen würden sich als äußerst störend erweisen. Die Preise würden steigen, und die Industrien, die die Mineralien verwenden, müssten möglicherweise ihre Produktion reduzieren, vor allem die Automobil-, Elektronik- und Verteidigungsindustrie. Dies würde den Druck erhöhen, die Versorgung anderswo zu sichern und Innovationen zu entwickeln, um den Einsatz dieser Metalle zu reduzieren. Dies würde jedoch Zeit und offizielle politische Unterstützung erfordern.

\subsubsection{EU}

Die Frage, inwieweit EU-Mitgliedsstaaten eigenmächtig Exportkontrollen für Rohstoffe einführen dürfen, berührt zentrale Prinzipien des EU-Binnenmarkts, des Außenwirtschaftsrechts sowie der Kompetenzverteilung zwischen Union und Mitgliedsstaaten.

Eine Einführung solcher Kontrollen käme zudem zunächst auf EU-Ebene infrage, um eine etwaige Ausfuhr von bereits in der Union vorhanden ROhstoffen zu unterbinden. Die gemeinsame Handelspolitik ist eine ausschließliche Zuständigkeit der EU. Sie umfasst Maßnahmen zur Regelung des Exports gegenüber Drittstaaten, einschließlich etwaiger Exportkontrollen. Im Sekundärrecht bildet der CRMA die Grundlage für die Gewährleistung der Versorgungssicherheit der EU mit kritischen und strategischen Rohstoffen. Der CRMA enthält keine expliziten Exportverbote oder -kontrollen für kritische Rohstoffe. Die EU könnte jedoch auf Grundlage von Art. 207 AEUV und unter Rückgriff auf die die Dual-Use-Verordnung (EU) 2021/821 Exportkontrollen einführen – etwa bei sicherheitsrelevanten Rohstoffen oder zur Verhinderung von strategischem Abfluss.

Die EU hat bislang keine expliziten Exportkontrollen für kritische Rohstoffe eingeführt, obwohl sie deren strategische Bedeutung anerkennt. Dies kann zu einem Spannungsverhältnis führen: DIe Unionsebene besitzt die Kompetenz, handelt abr (noch) nicht restriktiv, sodass sich die Mitgliedsstaaten gezwungen  sehen könnten, eigenmächtig Exportkontrollen einzuführen, um z. B. Recyclingprodukte oder Lagerbestände zu schützen. Vor dem Hintergrund der Improtabhängigkeit der Union ergibt sich das Risiko, dass Recyclingprodukte in Drittstaaten exportiert werden könnten, Sekundärrohstoffe der europäischen Industrie entzogen werden

Rohstoffe sind in der Regel nicht gelistet in der EU-Dual-Use-Verordnung, es sei denn, sie sind Teil sensitiver Technologien (z. B. seltene Erden für Rüstung oder Hochtechnologie). Daher könnten Mitgliedstaaten versuchen, nationale Exportkontrollen für bestimmte Rohstoffe zu rechtfertigen, etwa zur Sicherung der Versorgungssicherheit. Dennoch stehen solche Maßnahmen unter strenger Kontrolle durch die Kommission und den EuGH, sodass nationale Exporte zwangsläufig unionsrechtlich abgestützt sein müssen.\footnote{s. C-5/94 Hedley Lomas}, und der EuGH hat wiederholt die Notwendigkeit einer einheitlichen Außenhandelspolitik betont sowie die enge Auslegung der Ausnahmetatbestände aus Art. 36, 346 AEUV.


Trotz der EU-Kompetenz gibt es Ausnahmen, in denen nationale Exportkontrollen zulässig sind: Art. 346 AEUV erlaubt es Mitgliedstaaten, Maßnahmen zu ergreifen, die sie für den Schutz ihrer sicherheitspolitischen Interessen für erforderlich halten – etwa bei Rüstungsgütern oder kritischen Rohstoffen mit strategischer Bedeutung. Daraus ergibt sich bspw. auch die nationale Regelung im deutschen AWG und der deutschen AWV, dass zur Wahrung der öffentlichen Sicherheit Exportkontrollen umzusetzen sind.

Somit gilt auf europäischer Ebene: Nationale Exportkontrollen sind nur in engen Grenzen zulässig und daher als eher unwahrscheinlich einzustufen, obgleich Rohstoffe in den meisten Fällen nicht unter Dual-Use-Regelungen fallen und so potentiell Spielräume eröffnet und hierbei doch die Gefahr der Fragmentierung entsteht. Zu Rohstoffen wurden bisher keine nationalen Exprotkontrollen entwickelt -- jedoch im Bereich der Halbleitertechnologien, so beispielsweise von Frankreich\footnote{Arrêté du 2 février 2024 relatif aux exportations vers les pays tiers de biens et technologies associés à l'ordinateur quantique et à ses technologies habilitantes et d'équipements de conception, développement, production, test et inspection de composants électroniques avancés} und den Niederlanden\footnote{Regeling van de Minister voor Buitenlandse Handel en Ontwikkelingssamenwerking van 23 juni 2023, nr. MinBuza.2023.15246-27 houdende invoering van een vergunningplicht voor de uitvoer van geavanceerde productieapparatuur voor halfgeleiders die niet zijn genoemd in bijlage I van Verordening 2021/821 (Regeling geavanceerde productieapparatuur voor halfgeleiders)}.
Darüber hinaus ist zu beachten, dass die Union angesichts geopolitischer Spannungen auch Rohstoffexportregelungen weiter harmonisieren könnte, eas nationale Spielräume weiter einschränken könnte. Dieses Spannungsfeld ist ein zentraler Ansatzpunkt für eine kritische Analyse der Rohstoffgovernance in der EU. Denn: Eine auf EU-Level koordinierte Exportkontrolle erscheint vorteilhafter, jedoch können mitgliedsstaatliche Vorausbemühungen einen Wegbereiter hierfür darstellen.\autocite{Medunic, Nr 15 Juli 2024, S. 3}

Eine mögliche Forderung: Erweiterung des CRMA um exportseitige Schutzmechanismen, etwa durch eine EU-weite Exportgenehmigungspflicht für bestimmte Rohstoffe oder Recyclingprodukte. Insbesondere sollte die Union hierbei Exportkontrollen nicht nur als rüstungskontrollpolitisches INstrument betrachten, sondern -- ähnlich den USA -- als \glqq außenpolitisches Mittel im Rahmen der geostrategischen Rivalität mit China\grqq und im Sinne einer nationalen Sicherheit.\autocite{Medunic, Nr 15 Juli 2024, S. 3}

Zudem 

Die Strategie und die Bemühungen der Union, kritische und strategische Mineralien zu sichern, erschließt die Möglichkeit, zumindest die regionale Widerstandskraft zu stärken, erhöht aber zeitgleich auch das Risiko, dass dies als Rechtfertigung für weitere Vergeltungsmaßnahmen verstanden werden kann.\autocite[siehe auch]{Schroeder, Patrick: Letter: Race for critical minerals sparks call for new materials agency, FT, July 14 2025}



\subsection{Importrestriktionen}
Importrestriktionen \textit{per se} lassen sich zum aktuellen Zeitpunkt keine feststellen -- jedoch ist es durchaus denkbar, dass Importkontrollen durch die Union für eine forcierte Abhängigkeitsminimierung denkbar sind. Generell sind Einfuhrbechränkungen für industrielle Rohstoffe untypisch.\autocite{Schorkopf, Rohstoffverwaltung, Rn. 37}

Die Einführung von solchen Importverboten erfordert freilich auch hier wieder eine entsprechende Kompetenz der Union aus Art. 207 AEUV, sodass ein handelspolitisches Schutzinstrument im Sinne einer importregulierenden Maßnahme hierunter fallen würde; auch Art. 215 AEUV betreffend der Gemeinsamen Außen- und Sicherheitspolitik bietet hier durch Absatz I i. V. m. Artt. 21ff. EUV die Möglichkeit der Einführung restriktiver Maßnahmen nach Entscheidung gem. Art. 29 EUV durch den Rat. Darüber hinaus bietet die Verordnung über eine gemeinsame Einfuhrregelgung der Union\footnote{Verordnung (EU) 2015/478} über eine dann genauer zu prüfende \glqq bedeutende Schädigung\grqq grundsätzlich Möglichkeiten für einfuhrbezogene Schutzmaßnahmen.\footnote{Die Problematik der Einführung von Maßnahmen gegenüber einem WTO-Mitgliedsland wird entsprechend miteinbezogen und erfordert entsprechende Prüfung; ebenso existiert eine parallele Verordnung zu gemeinsamen Ausfuhrregelungen (Verordnung (EU) 2015/179).} Bei jeglichen Importrestriktionen ist hier, trotz oder gerade wegen Art. XIX GATT\footnote{Notstandsmaßnahmen bei Einfuhr bestimmter Waren}, auch die Vereinbarkeit mit Art. XI GATT\footnote{Allgemeine Beseitigung von mengenmäßigen Beschränkungen} genau zu prüfen sowie die infragekommende Auslegung der Artt. XX lit. b\footnote{Maßnahmen zum Schutze des Lebens und der Gesundheit von Menschen[...]; siehe zur komplexen Konstellation der Auslegung des Art. XX am Beispiel des WTO-Panelverfahren gegen chinesische Ausfuhrbeschränkungen Franke, M.: WTO, China-Raw Materials: Ein Beitrag zu fairem Rohstoffhandel?, Beiträge zum Transnationalen Wirtschaftsrecht 114, 2011; ebenfalls zur Auslegung Krenzler/Herrmann/Niestedt, EU-Außenwirtschafts- und Zollrecht, Rn. 109ff.}, XXI\footnote{Ausnahmen zur Wahrung der Sicherheit; in Bezug auf mineralische Rohstoffe grundsätzlich nur lit b. iii. einschlägig, erneut besteht hier deutlich eine Auslegungserforderlichkeit; zur Problematik der Anwendung aufgrund fehlender Rechtsprechung und Einschätzung des Ausscheidens des Art. XXI lit. b Zimmermann, C./Erben, J.: Gallium, Germanium und GATT: Neue Chinesische Ausfuhrkontrollen auf Halbleiterelemente, ZASA 2023, 195, 197}. Dies ist jedoch stark fallabhängig und kann daher an dieser Stelle nicht weiter verfolgt werden.
%Auslegung XXI GATT

\subsection{Zur Rolle der WTO}
Initial ist zunächst festzuhalten, dass die WTO (respektive das GATT) entgegen etwaiger institutioneller Annahmen nicht den Hauptmaßstab für den internationalen Rohstoffhandel darstellen, ursächlich auch durch die generelle Entwicklung des rohstofflichen Rechtsrahmens, der über spezifische Rohstoffübereinkommen anders gestaltet wurde als der auf Waren bezogene.\autocite{Schorkopf, Rohstoffverwaltung, Rn. 42}

Ihre vertragliche Handelspolitik verfolgt die Union auf multilateraler Ebene vor Allem im Rahmen der WTO.\autocite{Müller-Ibold/Herrmann: Die Entwicklung des Europäischen Außenwirtschaftsrechts (2020-2022), EuZW 2022, 1029} Sowohl die Union selbst als auch ihre Mitgliedsstaaten sind Mitglieder der WTO,\autocites[Gemeinsame Positionen der Mitgliedsstaaten und der Union werden aber fast ausschließlich durch die Kommission vertreten, siehe dazu und zur Kooperation der EU und Mitgliedsstaatn im WTO-Kontext]{Streinz, EUV/AEUV, AEUV Art. 218, Rn. 41}[zur dennoch unklaren unionsinternen Rechtssituation]{Tietje	Das Recht der Europäischen Union, AEUV Art. 220, Rn. 23} und sind damit an die Einhaltung der Verpflichtungen im Rahmen der jeweiligen Abkommen.\autocite[Siehe zur Einbindung in die EU-Handelspolitik und das Verhältnis zur WTO]{Krenzler/Herrmann/Niestedt, EU-Außenwirtschafts- und Zollrecht, Rn. 37f.} Auch im Bereich der mineralischen Rohstoffe bildet der zentrale Übereinkommens-Anknüpfungspunkt das GATT, auch durch explizit rohstoffbezogene Bestimmungen wie eine \glqq Bemühenspflicht zur Vermeidung von Ausfuhrsubventionen und Verbot der Unangemessenheit\grqq.\autocite{Herdegen IntWirtschaftsR/Herdegen, 13. Aufl. 2023, § 11. Rn. 1}

Auch in der WTO spiegelte sich die langwährende Nichtbeschäftigung mit Aspekten der Rohstoffpolitik, in den oben geziegten Fällen insbesondere der Fall der Auswirkungen von Handelsbeschränkungen auf Rohstoffe, wider.\autocite[ebenso]{Franke, TWR 114, 2011, S. 29} Generell wäre es aus \glqq institutioneller Perspektive\grqq zu erwarten, dass der Handel mit Rohstoffen insbesondere durch das WTO-Recht bzw. das GATT abgedeckt wird -- dies ist jedoch nicht der Fall, die WTO kennt keine eigenständige Rohstoffordnung.\autocite[Vielmehr waren fertige Erzeugnisse der Anknüpfungspunkt und Rohstoffe nur selektiv vorgesehen, obwohl es de jure von Anfang auf diese anwendbar war; ]{Schorkopf, Rohstoffverwaltung, Rn. 42 ff.}  Die WTO regelt daher nicht den Zugang zu Rohstofflagerstätten, Explorationsrechte oder Lizenzvergaben

Mengenmäßige Ausfuhrbeschränkungen sind auch durch das GATT 1994 verboten (XI GATT); zwar erlaubt Art. XI Abs. 2 lit. a Ausnahmen für zeitlich begrenzte Maßnahmen zur Verhinderung oder Linderung von Engpässen. Diese Ausnahme wird jedoch restriktiv ausgelegt.

Nichtsdestotrotz verbleibt die Problematik der Vereinbarkeit von Maßnahmen in ihrer Begründungsauslegung mit WTO-Recht (insofern ist es zu begrüßen, dass die Union dieses Verhältnis für Aus- und Einfuhrbeschränkungen per Verordnung regelt), aber auch unter aktuellen Gesichtspunkten die Durchsetzbarkeit. Es dürfte hinreichend deutlich sein, dass auf China keine rein liberale Handelspolitik zutrifft, und daher auch vor dem Hintergrund potenzieller zukünftiger Beschränkungen und Verfahren diese Obligationen entsprechend berücksichtigt, denn insbesondere im WTO-Verfahren wurde durch China selbst angegeben, dass der Ausbau der lokalen chinesischen Wertschöpfungskette der eigentliche Grund der Restriktionen ist.\footnote{Im englischsprachigen Original \glqq [...]  export restrictions will allow China to develop its economy in the future . . . [sic] The reason for this is that export restraints encourage the domestic consumption of these basic materials in the domestic economy\grqq, WT/DS394/R WT/DS395/R WT/DS398/R, S. 144.} 
Auch erneute Ausfuhrkontrollen Chinas\footnote{Im Unterschied zu vorherigen Maßnahmen aber nicht durch die Einführung von mengenmäßigen Beschränkungen für Exporte, sondern durch Genehmigungsanforderungen für Ausfuhren.}, diesmal auf Halbleiterrohstoffe\footnote{Gallium und Germanium, ebenfalls als kritische strategische Rohstoffe durch die eingestuft; siehe dazu https://www.reuters.com/markets/commodities/china-bans-exports-gallium-germanium-antimony-us-2024-12-03/}, zeigen, dass handelsrechtliche Maßnahmen weiterhin Bestand haben und mitunter konfrontativ zu verstehen sind.\autocite{ZASA 2023, 195} Ein Panel-Verfahren ist auch hier denkbar, eine etwaige chinesische Rechtfertigung z. B. aus Gründen der Sicherheit\footnote{s. dazu oben, aber ebenso: ZASA 2023, 195} bleibt abzuwarten. Zu beachten ist jedoch generell, insbesondere im Rahmen des Art. XXI lit. b (ii), (iii) GATT, dass ein Handelsverbot, das nur als Vorwand für Sicherheitsmaßnahmen dient, nicht die Ausnahme nach (ii) erfüllt, und eine Täuschung bzw. ein Missbrauch der Sicherheitsausnahme dann vorliegt, wenn  die Handelsmaßnahme absichtlich irreführend ist oder aberdie Sicherheitsausnahme, objektiv betrachtet, missbraucht wird, zugleich gestaltet sich die Darstellung der Beweislage entsprechend anspruchsvoll.\autocites{ZASA 2023, 195, 197}{Ikeda, K.: A Proposed Interpretation of GATT Article XXI (b) (ii) in Light of its Implications for Export Control} Weitere Schwierigkeiten ergeben sich aus der Weite und Charakter des nationalen Einschätzungsspielruams (\glqq nach seiner Auffassung\grqq, Art. XXI b GATT), droht hier doch die generelle Selbstentziehung des Mitgliedsstaates aus den handelsrechtlichen Verpflichtungen.\autocite[Ausführlich]{Herdegen, Internationales Wirtschaftsrecht, Rn. 81ff.}
Eine entsprechende Rechtsprechung oder zumindest eindeutige Auslegung ist daher wünschenswert, zumindest aber als sinnvoll einzustufen.
Das vorangegangene Beispiel illustriert hierbei, dass zwar grundsätzlich nationale Exportrestriktionen von WTO-Mitgliedern der Vertragskonformität und Normenkohärenz mit den WTO-Regeln unterliegen und die WTO zunächst als verbindlivher Rechtsrahmen von zentraler Bedeutung bleibt, denn konsensuelle Maßstäbe werden aufrechterhalten, sodass auch im internationalen Rohstoffhandel ein gewisser Grad an Rechtssicherheit erzielt wird. Die Unzulänglichkeiten im DSU-Mechanismus beeinträchtigen zwar die praktische Durchsetzung, mindern jedoch nicht die normative Bindungskraft des WTO-Rechts. Nichtsdestotrotz besteht die Gefahr dass Sicherheits- und wirtschaftspolitische Ausnahmetatbestände zur Legitimation protektionistischer Maßnahmen missbraucht werden. Ein selektiver Umgang mit WTO-Recht infolge wirtschaftspolitischer Interessen birgt die Gefahr, dass Vertrauensverluste in den multilateralen Streitbeilegungsmechanismus eintreten. Dies führt zu Rechtsunsicherheiten und könnte nachteilige Präzedenzfälle schaffen, die es anderen WTO-Mitgliedern erleichtern, ebenfalls von verbindlichen Verpflichtungen abzuweichen, auch im Lichte handelsbezogener Maßnahmen der Vereinigten Staaten.\footnote{Siehe dazu \ref{Kapitel XXX}} Insbesondere von der EU und ihren Mitgliedsstaaten ist aber keine Abweichung von der Beachtung der wertebasierten internationalen Handelsordnung und damit eine Abkehr vom WTO-Handelsregelregime zu erwarten, was in weiterer Vorausschau dann in Herausforderungen im Schnittfeld der Einhaltung multilateraler Handelsregeln einerseits und der Sicherung der Versorgungssicherheit mit Rohstoffen andererseits resultiert. Es ist zudem deutlich geworden, dass insbesondere die Anwendung der WTO-Normen im strategisch sensiblen Bereich der kritischen Rohstoffe nicht frei von politischen Erwägungen bleibt. Wirtschaftliche Akteure wiederum können zwar von einem grundsätzlich stabilen multilateralen Regelwerk ausgehen, die konkrete Durchsetzung ihrer Rechte hängt jedoch maßgeblich von der staatlichen Implementierung und internationalen Kooperation ab. Insbesondere in Sektoren mit strategischer Bedeutung können einseitige nationale Eingriffe und die Berufung auf Sicherheitsklauseln zu erheblichen Rechtsunsicherheiten und planungsrelevanten Risiken führen.

Insbesondere China rückt hier aufgrund der Rohstoffkonzentration vermehrt in den Vordergrund. Zwar ist es allgemein denkbar, dass China WTO-Regeln nur noch selektiv anwenden könnte angesichts aktuell\footnote{Stand Februar 2025 gem. %https://www.wto.org/english/tratop_e/dispu_e/dispu_by_country_e.htm}
52 laufender Verfahren gegen China, andererseits aber auch 29 Verfahren von China selbst initiiert wurden.

Erneut bietet wie in der gesamten Rohstoffpolitik die Rolle der mitgliedsstaatlichen Mitwirkung Potenzial für einen wenig harmonisierten handelsrechtlichen Rahmen in Bezug auf Rohstoffe und die WTO vis-à-vis der EU und ihren Mitgliedsstaaten. Bei Betrachtung der Außenfunktion fällt hierbei die divergierende interne rechtliche Einstufung der multilateralen Ebene auf, sodass sich hier ein Spannungsfeld zwischen globalem Einfluss, Zugang zu Rohstoffen und dessen Sicherung sowie das Verhältnis zwischen einer Stärkung der Union (nach außen hin) oder aber ein Auftreten in abgestufter Zusammenstellung der einzelnen Mitgliedsstaaten,\autocite{Dauses/Ludwigs, Handbuch des EU-Wirtschaftsrechts, A. I., Rn. 43} was der Harmonisierung einer europäischen Rohstoffverwaltung (auch oder gerade besonders auf) multilateraler Ebene nicht zuträglich ist.

Darüber hinaus schuf die WTO Ende 2024 eine Datenbank zu Handel, Zoll und politischen Maßnahmen im Zusammenhang kritischen Rohstoffen.\footnote{abrufbar unter www.critmin.org.}

Angesichts dieser begrenzten Steuerungsfähigkeit des WTO-Rechts entwickelt sich das internationale Rohstoffrecht zunehmend über bilaterale und plurilaterale Instrumente weiter: Die EU verfolgt gezielte Rohstoffpartnerschaften, Internationale Standards wie die OECD-Leitlinien für verantwortungsvolle Lieferketten mineralischer Rohstoffe oder ESG-Initiativen im Rohstoffsektor ersetzen z. T. verbindliche WTO-Vorgaben, und auch das Außenwirtschaftsrecht der Union entwickelt sich unabhängig vom WTO-Recht weiter, etwa durch autonome Handelsinstrumente, extraterritoriale Sorgfaltspflichten (CS3D), oder strategische Reservemechanismen (vgl. Art. 29 CRMA-Vorschlag).

Aus wirtschaftsverwaltungsrechtlicher Sicht kann das WTO-Recht derzeit nicht als hinreichender Rechtsrahmen für eine strategische Rohstoffverwaltung angesehen werden, da es funktional zu eng auf marktwirtschaftliche Liberalisierung ausgerichtet ist und viele rohstoffspezifische Problemlagen -– etwa Zugang zu Lagerstätten, strategische Autonomie oder Kreislaufwirtschaft –- nicht adressiert. Aus rohstoffverwaltungsrechtlicher Perspektive ist zudem zu berücksichtigen, dass Versorgungssicherheit, nachhaltige Extraktion, strategische Abhängigkeiten und geopolitische Interessenlagen zunehmend neue Steuerungsmechanismen erfordern, die das WTO-Regelwerk nicht bereitstellt. Rechtsakte wie der CRMA der EU stehen exemplarisch für diesen Wandel zu einer ressourcenstrategischen Governance außerhalb klassischer Freihandelsdogmatik. as WTO-Recht bildet nicht den primären Maßstab für den internationalen Rohstoffhandel, weil es aus historischen, systematischen und funktionellen Gründen keine spezialisierten Steuerungsmechanismen für mineralische Rohstoffe entwickelt hat. Vielmehr bestehen heute neue hybride Ordnungen, in denen wirtschaftsverwaltungsrechtliche und außenwirtschaftsrechtliche Instrumente jenseits des WTO-Rechts Anwendung finden. Für die Rohstoffpolitik der EU bedeutet dies, dass sie sich zunehmend auf autonome, multilaterale und bilaterale Instrumente stützt, um strategische Ziele der Versorgungssicherheit, Nachhaltigkeit und geopolitischen Resilienz zu verfolgen.

\subsection{Eine internationale Rohstofforganisation}
Vor dem Hintergrund dieser Erkenntnisse stellt sich die Frage, ob eine Einrichtung eine auf einem internationalen völkerrechtlichen Rohstoffübereinkommmen basierende Rohstoffagentur nicht sinnvoll wäre. Eine solche internationale oder nur europäische Institution zur Verwaltung einer entsprechenden Rohstoffpolitik existiert derzeit nicht, obwohl es rein aus systemischer Sicht sinnvoll erscheinen könnte, eine solche einzurichten.

Die Idee von Ausgleichslagern, wie in der Vergangenheit zu beobachten war, würde zunächst für eine größere Kontrolle über Preis- und Verfügbarkeitsschwankungen sorgen.

In der Geschichte der Rohstoffübereinkommen waren diese zunächst als marktordnende Gebilde gestaltet, die über entsprechende Intervention Preissteuerungen erzielten, insbesondere bei besonderen wirtschaftlichen Herausforderungen oder generellen Marktversagen im Bereich der Rohstoffe zu agieren um eine entsprechende Verteilung von Rohstoffen im Markt sicherzustellen und Preissprüngen vorzubeugen -- jedoch gilt diese mithin interventionistische Herangehensweise als nicht mehr zeitgemäß.\autocite[Und dazu auch widerlegt; siehe ausführlich]{Schorkopf, Rn. 43} Die noch existierenden orginiären Rohstoffübereinkommen mit der EU als Vertragspartner, basierend auf der Regelungsidee der Transparenz, beziehen sich jedoch ausschließlich auf Agrarrohstoffe; auch die Beteiligung an den sog. Internationalen Studiengruppen für bestimmte metallische Rohstoffe zählen hierzu.\autocite{Schorkopf, Rn. 44, 45}

Die Divergenz der Interessen wird insbesondere im Bereich der mineralischen kritischen Rohstoffe deutlich: Während die einen auf Wertschöpfung im Ursprungsland drängen, fordern die anderen verlässlichen Zugang zu günstigen Rohstoffen. Diese Asymmetrie verhindert die Herausbildung eines Konsenses, wie er etwa in der WTO oder bei den Bretton-Woods-Institutionen gelungen ist. Die Förderung und Veredelung vieler kritischer Rohstoffe (etwa Seltene Erden, Gallium, Kobalt) ist auf wenige Anbieterstaaten konzentriert (v. a. China). Diese besitzen erhebliche Marktmacht und haben kaum ökonomische Anreize, sich auf ein internationales Regime einzulassen, das ihre Preissetzungsspielräume oder Exportstrategien einschränkt. Im Gegensatz zum Energierecht (Energiecharta, OPEC) oder Agrarrecht (WTO-Agrarabkommen, FAO) existieren für kritische Rohstoffe bislang keine kodifizierten multilateralen Rechtsinstrumente. Der fragmentierte Rechtsrahmen (WTO, bilaterale Investitionsschutzabkommen, Umweltabkommen) schafft kein kohärentes Regulierungsregime.

Eine solch internationale und global ausgerichtete Rohstoffagentur könnte als koordinierende Kraft wirken und somit insbesondere die instrumentalisierte Rohstoffnutzung, geopolitische Spannungen und unvorteilhafte Auswirkungen auf Abbauländer (sozial, ökologisch, ökonomisch -- s. Rohstofffluch) zumindest mitigieren. Als zentrales Element, auch im Sinne einer Marktkoordination und durch einen datenbasierten Marktüberblick, kann eine solche Agentur zudem als neutraler Mittler und Koordinator bei Rohstoffpartnerschaften auftreten.\autocite{Letter: Race for critical minerals sparks call for new materials agency}

Auch eine Alternative zur WTO im Bereich von rohstoffbezogenener Streitbeilegung ist denkbar

Das Fehlen einer internationalen Rohstofforganisation für kritische Rohstoffe ist Ausdruck geopolitischer Rivalität, wirtschaftlicher Interessenunvereinbarkeit und völkerrechtlicher Strukturdefizite. Die Entwicklung des Rohstoffrechtsrahmens bleibt fragmentiert und reaktiv. Während sektorale Regelwerke (Agrar, Energie) historisch aus globalen Koordinierungsbedarfen heraus entstanden, fehlt es im Bereich kritischer Rohstoffe bislang an einem vergleichbaren kollektiven Regulierungsimpuls. Die Union muss daher alternative Strategien wie resiliente Lieferketten, strategische Partnerschaften und Binnenförderung priorisieren, ohne auf ein funktionierendes globales Rohstoffregime vertrauen zu können.

Wie zuvor beleuchtet bietet sich auch eine strategisch motivierte Absicherung der Exportkontrolle an: Ähnlich der \textit{Global Export Control Coalition} (GECC), die die Exportkontrollen nach 

sland und Belarus abstimmt, kann eine rohstoffspezifische Regelung entwickelt werden, sodass der Handel zwischen den den Mitgliedsstaaten einer Rohstoff-GECC weiter ermöglicht wird und für den den ex-Rohstoff-GECC-Export statt einem Listenansatz ein Lizenzansatz verfolgt wird.\autocite[in Anlehnung an den Vorschlag von]{Medunic, Nr. 15, Juli 2024, S. 4}



Die Einrichtung einer Rohstoffagentur ist nicht zuletzt auf den Willen und die Unterstützung sowohl durch Hauptlieferanten als auch Nettoimporteure angewiesen.



\section{Fazit}
Obgleich es notwendig ist, die Wertschöpfungskette kritischer Rohstoffe in der Union zu stärken, um die Versorgungssicherheit zu verbessern, bleiben die Lieferketten für diese Rohstoffe weltweit und unterliegen externen Einflüssen. Inwieweit diese externen Einflüsse, insbesondere aus einer wirtschaftspolitischen Sichtweise, relevant für das Rohstoffverwaltungsrecht sind, soll im folgenden Kapitel weiter betrachtet werden.

Insbesondere die Rohstoffverwaltung ist als ein Handlungsgebiet von nationalen Lösungen zu klassifizierenm geprägt durch nationale Alleingänge und keiner feststellbaren Tendenz zu einer Zentralisierung afu europäischer Ebene und erst recht keinen sekunderrechtlichen Strukturierung. Der Hauptgrund der mangelhaften bzw. nicht erfolgten Ausgestaltung der unionalen Rohstoffpolitik dürfte also darin liegen, dass die Mitgliedsstaaten sich auf nationale Alleingänge konzentrieren und dadruch eine eigenständige Versrogungspolitik betreiben und zudem nur die jeweiligen nationalen Unternehmen ansprechen.

Die Gestaltung eines stabilen Rechtsrahmens zur Sicherstellung der Rohstoffversrogung der Wirtschaft, die die Ziele der Union unmittelbar beeinflusst, ist also die wesentliche Aufgabe der europäischen Rohstoffpolitik.

Zur Erkenntnis gehört jedoch auch: (Rohstoffliche) Lieferketten lassen sich nur schwerlich kurzfristig diversifizieren.

Auch wenn Exploration und Abbau in der Union vorangetrieben und gar umgesetzt wird, dürfte die europäische Kapazität wohl kaum mit der chinesischen gleichziehen: 

Nichtsdestotrotz ist jeder Schritt heraus aus der chinesischen Abhängigkeit ein Schritt Richtung strategischer Autonomie. 

Die Auswirkungen der chinesischen Abhängigkeit auf Europa sind hinlänglich visualisiert worden -- die EU muss sich bewusst sein, dass \glqq alles auf die grüne Transformation gesetzt wurde und China könnte uns zum Schweigen bringen\grqq\autocite{Rachman, FT}


\end{document}