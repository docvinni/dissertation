\documentclass[12pt,a4paper,oneside]{book} % 'oneside' für einseitigen Druck

% Kodierung, Sprache und Schrift
\usepackage[utf8]{inputenc} % Erlaubt die Verwendung von Umlauten
\usepackage[T1]{fontenc} % Bessere Schriftkodierung
\usepackage[ngerman]{babel} % Deutsche Lokalisierung

% Schriftarten
\usepackage{lmodern} % Modernere Schriftart, gut für Skalierbarkeit und Lesbarkeit

% Für Abbildungen
\usepackage{graphicx}
\graphicspath{{bilder/}} % Verzeichnis, in dem Bilder gespeichert sind

% Für Tabellen
\usepackage{booktabs}

% Für Links und PDF-Metadaten
\usepackage[hidelinks]{hyperref}
\hypersetup{
	pdftitle={Titel der Dissertation},
	pdfauthor={Autor},
	pdfsubject={Doktorarbeit in den Sozial- und Rechtswissenschaften},
	pdfkeywords={Schlüsselwörter},
}

\usepackage{array}
% Für Bibliographie - Anpassung für Geisteswissenschaften
%%\usepackage[style=authoryear-icomp,backend=biber]{biblatex}
%%\usepackage[backend=biber, style=authoryear-icomp]{biblatex}
%%\usepackage[backend=biber, style=verbose-trad1]{biblatex}
\usepackage[backend=biber, style=verbose-inote]{biblatex}
%%\usepackage{biblatex}
\addbibresource{literatur.bib} % Name der BibTeX-Datei
\DeclareNameAlias{author}{family-given} % Nachname des Autors zuerst

% Anpassung der Nummerierung mit Punkten
\renewcommand{\thechapter}{\arabic{chapter}.} % Kapitel: 1., 2., 3., ...
\renewcommand{\thesection}{\Alph{section}.} % Abschnitt: A., B., C., ...
\renewcommand{\thesubsection}{\Roman{subsection}.} % Unterabschnitt: I., II., III., ...
\renewcommand{\thesubsubsection}{\arabic{subsubsection}.} % Unterunterabschnitt: 1., 2., 3., ...
\renewcommand{\theparagraph}{\alph{paragraph}.} % Absatz: a., b., c., ...


% Für Fußnoten
%%\usepackage[bottom]{footmisc} % Fußnoten am Seitenende


% Anpassung der Kapitelüberschriften
%\usepackage{titlesec}
%\titleformat{\chapter}[hang]{\Huge\bfseries}{\thechapter\quad}{0pt}{\Huge\bfseries}

% Abstand der Fußnoten
\setlength{\footnotesep}{0.5cm}

% Tiefe der Nummerierung und des Inhaltsverzeichnisses
\setcounter{secnumdepth}{4} % Nummerierungstiefe einstellen
\setcounter{tocdepth}{4} % Inhaltsverzeichnistiefe einstellen

% Abstand zwischen Absätzen und kein Einzug
%\usepackage{parskip}
%\setlength{\parskip}{0.5em}
%\setlength{\parindent}{0pt}

% Für Abkürzungsverzeichnis
\usepackage[printonlyused]{acronym}

% Für Zitate und Theoreme (falls benötigt)
\usepackage{csquotes}

% Für Gesetzestexte, Zitate und andere strukturierte Texte
\usepackage{enumitem}

% Zeilenabstand auf 1.3
\usepackage{setspace}



% Beginn des Dokuments
\begin{document}
	% Hier beginnt der eigentliche Inhalt der Arbeit
	% ...
	
	\chapter{Fazit}
	
Die Nichtbeachtung ergibt sich aus dem Zusammentreffen mehrerer Faktoren: Zum einen gab es aus geopolitischer Sicht keine Erfordelrichkeit, sich mit der Thematik gestörter Lieferketten auseinanderzusetzen, da sowohl geopolitische Ereignisse und handelspolitische Maßnahmen keinen kritischen Effekt auf die Rohstoffversorgung hatten.
Ferner geriet die Rohstoffversorgung insbesondere durch die Mobilitätswende vermehrt in den Fokus, jedoch nahm die Bedeutung von elektrisch angetriebenen Fahrzeugen erst ab dem Jahr XXXX Fahrt auf und somit die Nachfrage und Bedeutung nach Batterierohstoffen.
Auch die Erkundung nationaler Vorkommen und Verarbeitung war wirtschaftlich nicht attraktiv, sodass sie nicht weiter forciert wurde.
Schließlich erfolgte auch durch die Politik keine entsprechende Aktivität, da andere Themen mehr Salienz erfuhren und daher primär betrachtet wurden.

Alles, was es jetzt braucht, ist, dass die Vereinigten Staaten und die EU ernsthaft in diesen Bereich der Mineralien, der Verarbeitung und der Produktion einsteigen und höhere Preise zahlen müssen. Es sollte daran erinnert werden, dass die Quellen für Seltenerdrohstoffe zum größten Teil entweder in westlichen Ländern oder in Ländern mit engen Beziehungen zum Westen liegen. Der Westen könnte diese Seltenen Erden leicht an China abwürgen, da seine einzige Macht in der billigen Verarbeitung und Produktion liegt.

Die meisten Seltenen Erden befinden sich nicht in China (oder den USA). Die Sperre, die die Chinesen haben, liegt in der Verarbeitung von ihnen. Der Westen mag es derzeit nicht, Seltene Erden zu verarbeiten, weil die Verarbeitung schmutzig und umweltschädlich ist, und westliche Demokratien erlauben es NIMBY-affinen Wählern, neue Anlagen zu stoppen. Der Westen kann die Sperre für die Verarbeitung von Seltenen Erden relativ schnell knacken.

Freon Ukraine

Spätestens mit der Einführung chinesischer Exportkontrollen dürfte das Vertrauen in die Lieferkettenstabilität zerstört gewesen sein. Insofern ist zu hoffen, dass diese Zäsur stark genug war, ein entsprechendes nachhaltiges Umdenken zu bewirken. Es kommt nicht auf die technische Machbarkeit oder das geologische Vorkommen an, sondern vielmehr auf Skaleneffekte und lokale Kosten, ohne Selbstgefälligkeit zu erreichen.
	
	
\end{document}