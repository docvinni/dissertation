\documentclass[12pt,a4paper,oneside]{book} % 'oneside' für einseitigen Druck

% Kodierung, Sprache und Schrift
\usepackage[utf8]{inputenc} % Erlaubt die Verwendung von Umlauten
\usepackage[T1]{fontenc} % Bessere Schriftkodierung
\usepackage[ngerman]{babel} % Deutsche Lokalisierung

% Schriftarten
\usepackage{lmodern} % Modernere Schriftart, gut für Skalierbarkeit und Lesbarkeit

% Für Abbildungen
\usepackage{graphicx}
\graphicspath{{bilder/}} % Verzeichnis, in dem Bilder gespeichert sind

% Für Tabellen
\usepackage{booktabs}

% Für Links und PDF-Metadaten
\usepackage[hidelinks]{hyperref}
\hypersetup{
	pdftitle={Titel der Dissertation},
	pdfauthor={Autor},
	pdfsubject={Doktorarbeit in den Sozial- und Rechtswissenschaften},
	pdfkeywords={Schlüsselwörter},
}

\usepackage{array}
% Für Bibliographie - Anpassung für Geisteswissenschaften
%%\usepackage[style=authoryear-icomp,backend=biber]{biblatex}
%%\usepackage[backend=biber, style=authoryear-icomp]{biblatex}
%%\usepackage[backend=biber, style=verbose-trad1]{biblatex}
\usepackage[backend=biber, style=verbose-inote]{biblatex}
%%\usepackage{biblatex}
\addbibresource{literatur.bib} % Name der BibTeX-Datei
\DeclareNameAlias{author}{family-given} % Nachname des Autors zuerst

% Anpassung der Nummerierung mit Punkten
\renewcommand{\thechapter}{\arabic{chapter}.} % Kapitel: 1., 2., 3., ...
\renewcommand{\thesection}{\Alph{section}.} % Abschnitt: A., B., C., ...
\renewcommand{\thesubsection}{\Roman{subsection}.} % Unterabschnitt: I., II., III., ...
\renewcommand{\thesubsubsection}{\arabic{subsubsection}.} % Unterunterabschnitt: 1., 2., 3., ...
\renewcommand{\theparagraph}{\alph{paragraph}.} % Absatz: a., b., c., ...


% Für Fußnoten
%%\usepackage[bottom]{footmisc} % Fußnoten am Seitenende


% Anpassung der Kapitelüberschriften
%\usepackage{titlesec}
%\titleformat{\chapter}[hang]{\Huge\bfseries}{\thechapter\quad}{0pt}{\Huge\bfseries}

% Abstand der Fußnoten
\setlength{\footnotesep}{0.5cm}

% Tiefe der Nummerierung und des Inhaltsverzeichnisses
\setcounter{secnumdepth}{4} % Nummerierungstiefe einstellen
\setcounter{tocdepth}{4} % Inhaltsverzeichnistiefe einstellen

% Abstand zwischen Absätzen und kein Einzug
%\usepackage{parskip}
%\setlength{\parskip}{0.5em}
%\setlength{\parindent}{0pt}

% Für Abkürzungsverzeichnis
\usepackage[printonlyused]{acronym}

% Für Zitate und Theoreme (falls benötigt)
\usepackage{csquotes}

% Für Gesetzestexte, Zitate und andere strukturierte Texte
\usepackage{enumitem}

% Zeilenabstand auf 1.3
\usepackage{setspace}



% Beginn des Dokuments
\begin{document}
	% Hier beginnt der eigentliche Inhalt der Arbeit
	% ...

\chapter{Rohstoffverwaltungsrechtliche und -technische Instrumente auf europäischer und nationaler Ebene}

\section{Instrumente für eine EU-Rohstoffpolitik}
Für die Ausgestaltung einer unionalen Rohstoffpolitik kommen zunächst alle primärrechtlichen Instrumente nach Art. 288 AEUV infrage.

Insbesondere im Rohstoffbereich finden sich vermehrt Empfehlungen, Stellungnahmen und Mitteilungen, nutzt also rechtlich zunächst nicht verbindliche Instrumente und werden ausschließlich von der Kommission eingesetzt.

Die beiden einzigen \textit{eindeutig} dem Rohstoffrecht zuzuordnenden Rechtsakte besitzen den Charakter eine Verordnung, sind also im Rahmen des Art. 288 AEUV allgemein gültig, gelten verbindlich und unmittelbar in jedem Mitgliedsstaat. 

%im Verlauf der Entwicklung des CRMA und Konflikt-VO im Rechtssetzungvorgang der Kommission schauen wieso VO gewählt wurde

%Instrumentencharakter von Strategien, Weißbuch etc.

Die Weißbücher der Kommission (s.\,o.) fallen hingegen nicht in den klassischen Bereich der Instrumente, da diese sowohl nur von der Kommission alleinig erstellt und veröffentlicht werden als auch keinen rechtlich verbindlichen Charakter haben, sondern eher als eine Art \glqq Absichtserklärung\grqq verstanden werden können, auch da ihre Umsetzung unter Umständen von der anfänglichen Darstellung abweichen kann.\autocite[siehe hierzu]{Dauses/Ludwigs, O. Umweltrecht, Rn. 196} 

Anders wiederum sieht es auf Ebene des CRMA aus: Die dort fesgehaltenen Berichtspflichten

\subsection{Europäische Verordnung zu kritischen Rohstoffen (CRMA)}%\label{EU-Verordnung}
Die Kommissionspräsidenten kündigte ein Gesetz zu kritischen Rohstoffen im September 2022 an, im Rahmen ihrer State-of-the-Union-Rede vor dem Hintergrund des russischen Krieges gegen die Ukraine, der Covid-19-Pandemie und des Zusammenhangs von Abhängigkeit und Wettbewerbsfähigkeit .\footnote{SPEECH/22/5493}

CRMA: ein Koloss auf tönernen Füßen?

%Betrachtung COM(2020 474 final)
%EU Grundsätze für nachhaltuge Rohstoffe

Obwohl der Verabschiedungszeitplan zum CRMA mitunter \glqq ambitioniert\grqq bezeichnet wurde, konnte die Union noch im April/Mai 2024 nach Abschluss der Frist für Rückmeldungen Ende Juni 2023 den Akt in Kraft treten lassen und damit 14 Monate nach der Veröffentlichung des Vorschlags durch die Kommission\footnote{IP/24/2748} und somit noch rechtzeitig vor den Wahlen zum Europäischen Parlament im Juni 2024. Ferner hat die Kommission mit dem Instrumententyp der Verordnung nach Art. 288 AEUV S. 2 das weitreichendste Instrument gewählt, gilt der CRMA somit unmittelbar und verbindlich in jedem Mitgliedsstaat. Dementsprechend sind auch keine Umsetzungsrechtsakte erforderlich. Zwar ist der CRMA vorrangig auf Art. 114 AEUV gestützt, jedoch wäre auch im Rahmen der Umsetzung der Handelspolitik nach Art. 207 die Verordnung das INstrument der Wahl zur tatsächlichen Umsetzung der gemeinsamen Handelspolitik.\autocite{RdTW 2024, 216}

Der CRMA zeigt die Verbindung zur Handelspolitik bereits in den Erwägungsgründen: 

%https://single-market-economy.ec.europa.eu/publications/european-critical-raw-materials-act_en

%https://www.europarl.europa.eu/doceo/document/TA-9-2023-0454_DE.html#title2

Seit dem 23. Mai 2024 ist die Verordnung 2024/1252 \glqq zur Schaffung eines Rahmes zur Gewährleistung einer sicheren und nachhaltigen Versorgung mit kritischen Rohstoffen\grqq in Kraft getreten, nachdem die Europäische Kommission im März 2023 einen entsprechenden Vorschlag unterbreitete. Dies ist insbesondere herauszustellen, da im Vergleich zu anderen Rechtsakten der EU innerhalb des ordentlichen Gesetzgebungsverfahrens (s. u.) die kurze Dauer von etwas mehr als einem Jahr auf die erkannte Dringlichkeit des Aktes hinweist, denn die zügige Gesetzgebung unterstreicht die Priorität, welche die EU der Sicherung kritischer Rohstoffe für die industrielle Resilienz und technologische Souveränität zumindest in diesem konkreten Fall beimisst. Nichtsdestotrotz folgt diese Geschwindigkeit des Verfahrens dem Trend der Verkürzung der Dauer von Rechtsakten, die nach der ersten Lesung abgeschlossen werden.\footnote{So betrug die durchschnittliche Dauer im ordentlichen Gesetzgebungsverfahrens für Rechtsakte, die nach der ersten Lesung abgeschlossen wurden, im Zeitraum der 8. Wahlperiode (2014-2019) 18 Monate, während der 9. Wahlperiode hingegen nur 13 Monate; siehe hierzu Europäisches Parlament: Facts and Figures, Briefing, Mai 2023, BRI(2023)747102, S. 13.} 

Beim CRMA tritt die enge Verbindung zum veränderten Fokus der unionalen Handelspolitik hervor -- jedoch ist zu betonen, dass nicht allein auf die Handelsdimension bei der Sicherung der Versorgung abgestellt wird, sondern vielmehr auf mögliche Wettbewerbsverzerrungen wie sie durch für Marktteilnehmer divergierende Rechtsvorschirften und Zugangsbedingungen bedingt werden könnten, aber auch durch die tatsächliche Versorgungsrisikoüberwachung, ungleiche Begünstigungen nationaler Maßnahmen oder durch Hürden im Bereich des grenzüberschreitenden Warenverkehrs. Neu im Vergleich zu den eingangs beschriebenen Rechtsakten ist hierbei jedoch die Erkenntnis, dass die Kommission die Erforderlichkeit einer \textit{verbindlichen Steuerung} erkennt.\autocite{Schäffer/Hach, ZRP 2023, 207, 208} \textit{Müller-Ibold/Herrmann} merkten in Bezug zum CRMA 2022, ob durch die Aktivität der Union \glqq sich [mancher fragen wird], ob hier das Pendel nicht zu weit in eine neue Richtung ausschlägt\grqq.\autocite{Müller-Ibold/Herrmann, EuZW 2022, 1085, 1091} Diese Frage mag zunächst berechtigt erscheinen, bewegt sich die Union auf bisher nicht betretenem Boden, denn der CRMA markiert einen mehr oder minder bedeutenden Schritt für die Rohstoffpolitik und -verwaltung im Rahmen der strategischen Autonomie durch den Ausbau inländischer Kapazitäten und die Diversifikation externer Rohstoffquellen anstrebt, und bewegt sich zudem innerhalb des Spannungsfeldes der unionsrechtlichen Kompetenzen (wie der vorige Abschnitt zeigt) und dementsprechend der Autonomie der EU(-Organe), sowie der Marktliberalität. Somit stellt sich die \textit{Pendelfrage}, ob die Union mit diesem neuen Regulierungsrahmen über den verfassungsrechtlich zulässigen Rahmen hinausgeht oder aber zu sehr in eine Richtung reguliert. Der Frage ist aus drei Sichtweisen zu begegnen. Zum Einen hat die strategische Notwendigkeit der Ziele des CRMAs seit 2022 nicht abgenommen -- die Versorgung mit kritischen und strategischen Rohstoffen hängt weiterhin stark von einigen wenigen Drittstaaten ab, und eine Verringerung ist nicht zu erkennen (zumindest nicht, bis entsprechende Maßnahmen ihre Wirkung entfalten), und auch die Implikationen der Zeitenwende können weiterhin als dringlich angesehen werden. Der CRMA kann zudem problemlos auf Art. 173 AEUV (Industriepolitik) gestützt werden, der Maßnahmen zur Wettbewerbsfähigkeit und Widerstandsfähigkeit der EU-Industrie vorsieht. Diese Bestimmung erlaubt die Förderung eines „offenen und wettbewerbsfähigen“ Binnenmarktes und gibt der EU gewisse Eingriffsmöglichkeiten in Wirtschaftsbereiche, die als besonders strategisch gelten, ohne die Marktkräfte auszuhebeln. Man könnte gar die Kompetenzbasis durch Art. 122 I AEUv weiter ergänzen, der Notmaßnahmen zur Bewältigung von Krisen zulässt. Angesichts der hohen Abhängigkeit der EU von Drittstaaten und der damit verbundenen potenziellen Risiken ist diese Norm rechtlich vertretbar und erlaubt der EU, bei Bedarf Maßnahmen zur Sicherung von Rohstoffen zu ergreifen. Ferner stünden weitere Kompetenztitel zu einer Verfeinerung des Pendels zur Verfügung. Letztlich hat die Union auch in der Vergangenheit schon mehrfach die Ausschlagsstärken diverser Pendel neu kalibriert -- als ein prominentes Beispiel sei hier die Entwicklung der Gemeinsamen Agrarpolitik in deen 1960er Jahren zu erwähnen.

Art. 26 AEUV gibt die Grundlage für den Binnenmarkt und die Freiheit der Kapital- und Warenbewegung, Art. 63, Art. 34 AEUV. Das CRMA könnte jedoch als Einschränkung der Marktfreiheiten gewertet werden, falls durch diese Maßnahmen strategische Autonomie über marktwirtschaftliche Prinzipien gestellt wird.


Die handelspolitische Dimension weißt hier einen Doppelcharakter auf, die einerseits auf Drittländer und andererseits auf den Binnenmarkt gerichtet ist.\autocite{Paschke, Rdtw 2024, 206, 211f}


Es wird also deutlich, dass sich die kompetenzgebende Verknüpfung der Politiken in der unionalen Rohstoffpolitik eindeutig in den Ausführungen des CRMAs widerspiegeln und somit auch nochmals die Erforderlichkeit einer interdisziplinären Betrachtung sowohl innerhalb der rechtlichen Sichtweise aber auch unter Einbeziehung weiterer Disziplinen unerlässlich erscheint. 

%Vergleich mit anderen Legislativprozessen

Dieser \glqq Critical Raw Materials Act\grqq (CRMA) ist Teil des \textit(EU Green Deals) und wurde 2022 von der damaligen EU-Kommissionspräsidentin von der Leyen angekündigt \footnote{Ursula von der Leyen, "`State of the Union"' Rede, 14. September 2022.}, in Bezug zur informellen \glqq Versailles Decleration\grqq des Europäischen Rates von 2022.\footnote{Die Deklaration entstand vor dem Hintergrund des russischen Angriffskrieges auf die Ukraine im Februar 2022; in Kapitel III `"Building a more robust economic base'" wird unter dem Stichwort \textit{Critical raw materials} eine Sicherstellung der EU-Versorgung durch strategische Partnerschaften, Bevorratung und Förderung einer Kreislaufwirtschaft und Ressoruceneffizienz gefordert (Europäischer Rat, Informal meeting of the Heads of State or Government, Versailles Declaration, 11. März 2022, S. 7).} Die Kommissions-Generaldirektion Binnenmarkt, Industrie, Unternehmertum und KMU (DG GROW) ist betraut mit der Umsetzung des CRMA.

Im Allgemeinen unterscheidet die Verordnung zwischen \textit(strategischen) Rohstoffen einerseits und \textit(kritischen) Rohstoffen andererseits

Generell kann eine geographische Beschränkung auf einzelne Länder als kritisch angesehen werden, solange die Rohstoffe auch aus anderen Regionen von Unternehmen bezogen werden können, mit entsprechendem Folgen für die lokale wirtschaftliche Stabilität.\autocite[s. hierzu]{ruettinger_doddfrank_2015}

subsubsection{Der Vorschlag der Europäischen Kommission}
Ein erster Vorschlag der Kommission wurde im März 2023 veröffentlicht.

Zu betonen ist hierbei der Fokus der Verordnung auf \glqq nichtenergetisch[e], nichtlandwirtschaftlich[e] Rohstoff[e]\grqq mit entsprechender Bedeutung für die EU-Wirtschaft und Vorliegen eines Vorsorgungsrisikos. Der Entwurf orientiert sich zudem explizit an der offenen strategischen Autonomie\footnote{siehe hierzu Abschnitt Handelspolitik} der Union.\footnote{ErwGr 1}

Als Rechtsgrundlage des Entwurfs wurde Art. 114 AEUV gewählt
Die Wahl der Rechtsgrundlage deutet also darauf hin, dass über das Vehikel der Vervollständigung des Binnenmarktes und Harmonisierung erreicht wird.  Dies stellt formal die Hauptrechtsgrundlage dar, welche sich auf die Verwirklichung und das Funktionieren des Binnenmarkts bezieht.
Hinsichtlich der zwangsläufig auftretenden Frage nach Wahl der Kompetenzgrundlage hat die Union eine elegante Lösung gefunden. Die Abgrenzung der passenden Kompetenz ist nach dem Hauptzweck der Maßnahme zu wählen, in der Regel ist auch nur eine Kompetenzgrundlage zu nutzen anhand objektiver und nachprüfbarer Sachverhalte.\footnote{Bereits erläutert: EuGH, Rs. C-42/97, Rn. 39 ff., EuGH, Rs. C-300/89, Rn. 13, 17} Durch die Formulierung \glqq gestützt [...] insbesondere auf Artikel 114\grqq wird \textit{eine} Kompetenzgrundlage ausgewählt, aber gleichzeitig die Einschlägigkeit anderer infragekommender Kompetenzgrundlage nicht infrage gestellt. Die Wortwahl \glqq insbesondere\grqq lässt jedoch eine implizite Referenzierung anderer Politikbereiche erkennen, ohne aber eine  Mehrfachrechtsgrundlage im Sinne der Rechtsetzungstechnik zu wählen. Solange diese Aspekte nicht eigenständig tragend für die Zielsetzung des Rechtsakts sind, sondern Hilfs- bzw. Begleitfunktionen erfüllen, ist ihre Nichtbenennung als Rechtsgrundlage unionsrechtlich nicht zu beanstanden. \footnote{vgl. erneut Rs. C‑300/89, Rn. 17–19}

Die bewusst gewählte Formulierung lässt sich im systematischen Sinne als Ausdruck einer differenzierten Kompetenzstrategie der Union verstehen und dem Gebot der Bestimmtheit der Kompetenzgrundlage genügt, zugleich aber offen für die Mitberücksichtigung sachlich verwandter Kompetenzen bleibt, ohne diese zu präjudizieren oder zu untergraben, somit auch politisch und funktional offen gegenüber anderen einschlägigen Kompetenzbereichen bleibt. Dieses Vorgehen entspricht einer formell korrekten, inhaltlich pragmatischen Kompetenzwahl und kann dient  als formelhaft vorsichtige Sprachregelung, um etwaige spätere Differenzierungen – etwa im Rahmen von Rechtsstreitigkeiten – nicht vorwegzunehmen. Dies reflektiert den interdisziplinären Charakter moderner Unionspolitik – gerade in komplexen Bereichen wie der Rohstoffsicherung – und steht in Einklang mit der bisherigen Praxis der Kommission und der Rechtsprechung des EuGH, insbesondere vor dem Hintergrund der Komplexität moderner Legislativvorhaben gerecht wird, ohne das Primat der Hauptkompetenz zu relativieren. Ein ähnliches Vorgehen wurde etwa auch im Verordnungsvorschlag für die Batterieverordnung (COM(2020) 798 final) oder in der Verordnung (EU) 2017/821 über Konfliktmineralien gewählt, die ebenfalls auf Art. 114 AEUV gestützt wurden, obwohl auch hier klare Anknüpfungspunkte an Umwelt-, Industrie- und Außenhandelspolitik bestehen.

Im Bereich der Rohstoffverwaltung liefern die Ergebnisse der Ex-Post-Bewertung sowie die Konsultation der Interessenträger im Vorschlag der Verordnung Erkenntnisse: So kritisierten Unternehmen und Betreibe Verfahrens- und Verwaltungskosten sowie die Dauer von Genehmigungsverfahren im Allgemeinen.\footnote{COM(2023) 160 final, S. 10.}

Entsprechend berücksichtigt wird hierbei, dass durch den CRMA keine oder nur begrenzte weitere Verwaltungskosten für Unternehmen entstehen (so beispielsweise durch Berichtspflichten), wobei durch die Kommission ein Kostenausgleich durch die Profitierung von effizienteren Verwaltungsverfahren angenommen wird.\footnote{COM(2023) 160 final, S. 14.} Dieser lässt sich \textit{ex ante} nicht verifizieren, ist jedoch auch Bestandteil der in dieser Arbeit durchgeführten Befragung. %hier dann verweis einfügen

Ferner schlägt die Kommission vor, dass \glqq große Unternehmen, die [...] strategische Rohstoffe verwenden, ihre Lieferketten prüfen und [...] regelmäßige Stresstests ihrer Lieferketten strategischer Rohstoffe durchführen, um sicherzustellen, dass sie alle verschiedenen Szenarien berücksichtigen, die sich im Falle einer Unterbrechung auf ihre Versorgung auswirken könnten\grqq \footnote{COM(2023) 160 final, S. 14.} Hierbei ergibt sich die Frage nach dem Zuständigkeits- und Verantwortungsbereich: Inwieweit können, müssen oder sollten Unternehmen zu solchen Auflagen verpflichtet werden, und inwieweit sollten diese Aufgaben von staatlicher Seite übernommen werden?
% Abstimmungsverhalten

Ein weiterer zentraler Bestandteil des Entwurfs ist das \glqq strategische Projekt\grqq. %Art. 5ff.

Zwar kann die Verordnung als ein erster Beitrag zur Herausbildung eines europäischen Rohstoffverwaltungsrecht gesehen werden, erkennt aber dass keine aktive Harmonisierung der einzelnen mitgliedsstaatlichen Vorschriften beabsichtigt ist.\footnote{COM(2023) 160 final, S. 13.} Die Mitgliedstaaten der EU behalten somit weitgehend ihre Souveränität -- hierbei birgt sich aber das Risiko nach einer weiteren Fragmentierung der Rechtsvorschriften, unter Umständen mit entsprechenden Folgen für den EU-Binnenmarkt.


\paragraph{Die Wahl der Kompetenzgrundlage}
Der CRMA stützt sich vornehmlich auf Art. 114 AEUV, sodass hier zunächst die Binnenmarktpolitik der Union zum Tragen kommt. Wie im ersten Abschnitt festgestellt, agiert Art. 114 als Grundlage, da dieser Artikel der EU die Befugnis einräumt, Maßnahmen zur Angleichung nationaler Rechtsvorschriften zu erlassen, die den reibungslosen Funktionieren des Binnenmarkts gewährleisten. Die Harmonisierung im Bereich kritischer Rohstoffe – etwa hinsichtlich der Gewinnung, Verarbeitung, Wiederverwertung und des Handels – zielt darauf ab, interne Handelshemmnisse abzubauen und so gleiche Wettbewerbsbedingungen in der gesamten Union zu schaffen.

Art. 114 AEUV dient also dazu, verfahrensrechtliche und materielle Anforderungen im Bereich der Rohstoffpolitik zu harmonisieren, etwa durch:

einheitliche Genehmigungszeiträume für strategische Projekte (Art. 10 CRMA-Vorschlag),

gemeinsame Standards zur Nachhaltigkeit und Resilienz (Art. 26 ff. CRMA-Vorschlag),

koordinierte Anforderungen an das Recycling und die Rückgewinnung (Art. 28 ff.).

Der Regelungszweck liegt im Abbau von regulatorischen Unterschieden, die den freien Verkehr von Rohstoffen, Zwischenprodukten und Technologien innerhalb des Binnenmarkts behindern können. Auch die Schaffung von verlässlichen Lieferketten zählt zu den wirtschaftspolitischen Zielen im Rahmen des Binnenmarkts.

Im Bereich der Rohstoffpolitik kann Art. 197 AEUV die Umsetzung von Art. 114 AEUV flankieren, etwa wenn Mitgliedstaaten bei der Zulassung strategischer Projekte, der Anwendung technischer Vorschriften oder bei Genehmigungsprozessen unterstützt werden. Der CRMA sieht etwa in Art. 8 ff. die Einrichtung einer „Behördenplattform für kritische Rohstoffe“ vor, welche den Austausch zwischen Mitgliedstaaten fördern und Verwaltungsverfahren beschleunigen soll.
Diese Mechanismen lassen sich eindeutig auf Art. 197 AEUV zurückführen, wobei der Rechtsakt formal dennoch auf Art. 114 AEUV gestützt bleibt. Art. 197 AEUV ist insofern keine originäre Rechtsgrundlage, sondern eine verfahrensrechtliche Ergänzungskompetenz zur Unterstützung der Durchführung unionsrechtlicher Vorgaben.

%Methodik zur Auswahl strategischer Rohstoffe
%Vergleich der Liste strategischer Rohstoffe

\paragraph{Liste kritischer und strategischer Rohstoffe}

Das Konzept einer Liste kritischer Rohstoffe wie im CRMA ist nicht neu, lag eine solche bereits 2011 als Folge der Rohstoffinitiaive von 2008 -- vor mit insgesamt 14 Mineralien, die mittlerweile auf 34 angewachsen ist.\footnote{Zwischenzeitlich erfolgten drei Aktualisierungen: 2014 (COM(2014) 297) bei der die Liste auf 20 Mineralien erweitert wurde, 2017 (COM(2017) 490) und Erweiterung auf 27 sowie 2020 auf 30 (COM(2020) ???)}. Die Aktualisierung 2023 erfolgte im Rahmen des CRMA%PRÜFEN
Diese regelmäßige, aber mindestens alle drei jahre stattfindende Aktualisierung der Liste wird entsprechend der Ankündigung der Kommission umgesetzt.\footnote{COM(2011) 25 S. 16}

\glqq Kritisch\grqq bezeichnet hierbei einen Rohstoff, bei dem in den nächsten zehn Jahren ab Betrachtungszeitpunkt ein besonders großes Versorgungsengpassrisikos besteht und darüber hinaus als essentiell für die Wertschöpfungskette eingestuft werden.\footnote{COM(2011) 25, S. 13.} 

Der CRMA führte hierbei dann erstmals auch den Begriff der \textit{strategischen} Rohstoffe ein, die als bedeutsam für Wirtschaftszweige der Union (Beispiele?) eingestuft werden.

Somit ergibt sich zum aktuellen Zeitpunkt folgende Übersicht:

\begin{tabular}{|>{\raggedright}p{5cm}|c|c|}
	\hline
	\textbf{Mineral} & \textbf{Kritisch} & \textbf{Strategisch} \\
	\hline
	Antimon & x & \\
	\hline
	Arsen & x & \\
	\hline
	Bauxit & x & \\
	\hline
	Baryt & x & \\
	\hline
	Beryllium & x & \\
	\hline
	Bismut & x & x \\
	\hline
	Bor – metallurgische Qualität & x & x \\
	\hline
	Kobalt & x & x \\
	\hline
	Kokskohle & x & \\
	\hline
	Kupfer & x & x \\
	\hline
	Feldspat & x & \\
	\hline
	Flussspat & x & \\
	\hline
	Gallium & x & x \\
	\hline
	Germanium & x & x \\
	\hline
	Hafnium & x & \\
	\hline
	Helium & x & \\
	\hline
	Schwere seltene Erden & x & \\
	\hline
	Leichte seltene Erden & x & \\
	\hline
	Lithium – Batteriequalität & x & x \\
	\hline
	Magnesium & x & \\
	\hline
	Magnesiummetall & & x \\
	\hline
	Mangan – Batteriequalität & & x \\
	\hline
	Natürlicher Grafit – Batteriequalität & x & x \\
	\hline
	Nickel – Batteriequalität & x & x \\
	\hline
	Niob & x & \\
	\hline
	Phosphorit & x & \\
	\hline
	Phosphor & x & \\
	\hline
	Metalle der Platingruppe & x & x \\
	\hline
	Scandium & x & \\
	\hline
	Siliciummetall & x & x \\
	\hline
	Strontium & x & \\
	\hline
	Tantal & x & \\
	\hline
	Titanmetall & x & x \\
	\hline
	Wolfram & x & x \\
	\hline
	Vanadium & x & \\
	\hline
\end{tabular}



Die Unterscheidung ist dahingehend relevant, dass aufgrund der Eigenschaft des kritischen bzw. strategischen Rohstoffs die entsprechenden Verfahren und Projekte daran angeknüpft bzw. in dieser unterschieden werden.

Eine neue Aktualisierung ist, bei Beibehaltung des Dreijahresrhythmus, für 2026 und 2029 zu erwarten. Hierbei wird insbesondere von Interesse sein, ob die Liste weiter anwächst oder gar Rohstoffe wieder von der Liste gestrichen werden.

Frau\autocite{Frau 2024, NVwZ 2024, 1874, 1875} stellt hierbei treffend heraus, dass es als ungünstig einzustufen ist, dass der CRMA konstant (insgesamt 26 Mal) den \glqq Bereich kritischer Rohstoffe\grqq in Bezug auf strategische Projekte nennt, obgleich diese Projekte hier (nur und immer) zur \glqq Sicherung der Versorgung der Union mit \textit{strategischen} Rohstoffen\grqq\footnote{So eindeutig CRMA, Art. 6 I lit. a} beitragen.

%weitere Listen?

In dieser Hinsicht ist die EU anderen staatlichen Akteuren voraus: Beispielsweise existiert auf US-Ebene keine einheitliche Liste kritischer (oder strategischer) Mineralien. Stattdessen führen daher sowohl das Verteidigungsministerium, das Energieministerium sowie das Innenministerium ihre jeweils eigenen Listen. Von den insgesamt 70 Mineralien werden nur 13 von allen Behörden als \glqq kritisch\grqq eingestuft.\autocite{Baskaran, G.; Schwart, M.: Three U.S. Government Lists: Which Minerals Are the Most Critical?, CSIS} Das Vorhandensein mehrerer, teils widersprüchlicher Listen führt zu vermeidbarer Komplexität und Unsicherheit und untergräbt so Bemühungen, (private) Investitionen in den Lieferketten kritischer Mineralien sowohl im Inland als auch international zu fördern, sodass der EU zumindest hier erfolgreiche Harmonisierung bescheinigt werden kann.

Es wird zudem kritisiert, dass zwar ROhstoffe als kritisch eingestuft werden, die Forschung bei dieser Entwicklung jedoch in den Rückstand geraten ist: So würden insbesondere hcohsaliente Minerlaien wie Kobalt oder Lithium besonders intensiv betrachtet, andere ähnlich dringend benötigte jedoch nicht.\autocite{ÖAW, Studie EP}


\subsubsection{Stellungnahme des Europäischen Wirtschafts- und Sozialaussschuss}

Die Bedeutung einer entsprechenden Infrastruktur für eine Rohstoffverwaltung erkennt auch der Ausschuss, indem er den "Aufbau von Verwaltungskapazitäten in den öffentlichen Verwaltungen der EU-Mitgliedsstaaten" fordert % C 349/142,3.
und 

\subsubsection{Lesung im Europäischen Parlament}


Die Verabschiedung im Europäischen Parlament erfolgte bereits im Dezember 2023.

\subsubsection{Betrachtung des Verordnungstextes}
Im Rahmen des Gesetzgebungsverfahren erfuhr der Vorschlag der Kommission entsprechende Veränderungen.

Frau\autocite{Frau, NVwZ 2024, 1874}


Der CRMA führte nachprüfbare Ziele für strategische Rohstoffe ein:
\begin{itemize}
	\item Stärkung der 
\end{itemize}


Ein weiterer zentraler Punkt des CRMA ist die Verankerung der \textit{strategischen Projekte}, die einen Dreh- und Angelpunkt des Akts darstellen.
Die Länge der jeweiligen Genehmigungsverfahren ist hierbei festgelegt und beträgt projektabhängig entweder 15 Monate (für solche, die lediglich Verarbeitung oder Recycling betreffen) bzw. 27 Monate für Projekte im Bereich Gewinnung, eine entsprechende Verkürzung dieser Genehmigungsverfahrensdauer um 3 Monate ist bei bereits genehmigten Projekten vorgesehen, wobei Fristverlängerungen in Ausnahmefällen durchaus möglich sind.\footnote{CRMA, Art. 11} Diese Verfahrensdauerfestlegung wird mitunter als ambitoniert angesehen.\autocite{Quantz, ZfPC 2024, 1}



\subsubsection{Zum Rohstoffbegriff im CRMA}
Es ist hervorzuheben, dass mit dem CRMA erstmals der Begriff des Rohstoffes legal definiert wird, was somit auch für zukünftige Rechtsakte eine wichtige Grundlage darstellt. Im Sinne des CRMA bezeichnet ein Rohstoff \glqq einen verarbeiteten oder unverarbeiteten Stoff, der als Input für die Herstellung von Zwischen- oder Endprodukte verwendet wird, mit Ausnahme von Stoffen, die überwiegend als Lebensmittel, Futtermittel oder Brennstoff verwendet werden\grqq. Der Ausschluss der energetisch und landwirtschaftlich geprägten Rohstoffe rührt daher, dass aufgrund der sonstigen Weite des Rohstoffbegriffs\footnote{Diese Problamtik wurde bereits in dieser Arbeit und auch von anderer Seite ausführlich thematisiert, so z. B. Frau 2025, S. 12, } das Versorgungsrisiko und die wirtschaftliche Bedeutung (letztendlich also die \glqq Kritikalität\grqq) in strategischen Sektoren im Mittelpunkt stehen soll.\footnote{vgl. Erwägungsgrund 1, sowie S. 1 KOM2023 160 final.}
Auch die Definition von Exploration, Gewinnung, Mineralvorkommen, Verarbeitung, Recycling, Versorgungsrisiko, Genehmigungsverfahren, Rohstofflieferketten und weitere bilden eine entsprechende Definitionsgrundlage.\footnote{vgl. CRMA, Art. 2, Nr. 2ff.}
%Rechtsanalyse

\subsubsection{Strategische Projekte}
Die strategischen Projekte stellen eine der Hauptsäulen im CRMA dar. Bis Ende August 2024\footnote{Enddatum des ersten Aufrufs der Kommission} erhielt die Kommission 170 Vorschläge für strategische Projekte. Eine Entscheidung erfolgte dann bereits 2025:

Die Tabelle zeigt den Umsetzungsstand nationaler Rohstoffstrategien in den EU-Mitgliedstaaten im Kontext des Critical Raw Materials Act (CRMA). Sie verdeutlicht die erhebliche Heterogenität innerhalb der Union hinsichtlich der politischen Priorisierung, strategischen Ausgestaltung und praktischen Beteiligung an CRMA-Projekten. Während rohstoffintensive Industrienationen wie Deutschland, Frankreich oder Schweden bereits umfassende nationale Strategien implementiert und sich aktiv an strategischen Projekten beteiligt haben, bleibt die Reaktion vieler anderer Mitgliedstaaten bislang verhalten oder aus. Insbesondere kleinere oder rohstoffarme Länder weisen keine erkennbare Rohstoffpolitik auf, was auf eine potenzielle Unterschätzung der Versorgungssicherheitsproblematik hindeutet. Diese Fragmentierung stellt eine Herausforderung für die Harmonisierung europäischer Industrie- und Rohstoffpolitik dar und hat unmittelbare Auswirkungen auf sektorale Strategien, etwa im Bereich der Automobilproduktion.

\begin{table}[htbp]
	\centering
	\caption{Umsetzungsstand nationaler Rohstoffstrategien in den EU-Mitgliedstaaten im Kontext des CRMA (Stand: August 2025)}
	\begin{tabular}{|p{3cm}|p{2.5cm}|p{2.5cm}|p{4cm}|p{6cm}|}
		\hline
		\textbf{Land} & \textbf{Nationale Rohstoffstrategie} & \textbf{CRMA-Projekte} & \textbf{Schwerpunkt} & \textbf{Bemerkungen} \\
		\hline
		Deutschland & Ja & Ja & Primärgewinnung, Verarbeitung, Substitution & Nationales Explorationsprogramm, Batterieprojekte (Lithium, Graphit) \\
		Frankreich & Ja & Ja & Verarbeitung, internationale Kooperation & Beteiligung an NGC Battery Materials, Fokus auf Graphit \\
		Italien & Teilweise & Ja & Recycling, Verarbeitung & Teilnahme an CRMA-Projekten, keine umfassende nationale Strategie \\
		Spanien & Teilweise & Ja & Primärgewinnung & Lithium- und Kupferprojekte, Fokus auf Bergbau \\
		Polen & Teilweise & Ja & Verarbeitung, Recycling & Teilnahme an strategischen Projekten, sektorale Ansätze \\
		Schweden & Ja & Ja & Primärgewinnung, Kreislaufwirtschaft & Starke nationale Strategie, Fokus auf Seltene Erden und Batteriemetalle \\
		Finnland & Ja & Ja & Primärgewinnung, Verarbeitung & Aktive Rohstoffpolitik, Beteiligung an mehreren Projekten \\
		Portugal & Teilweise & Ja & Primärgewinnung & Lithiumprojekte, aber begrenzte strategische Tiefe \\
		Estland & Nein & Ja & Verarbeitung & Teilnahme an CRMA-Projekten, keine erkennbare nationale Strategie \\
		Tschechien & Nein & Ja & Verarbeitung & Einbindung in EU-Projekte, keine nationale Strategie bekannt \\
		Griechenland & Nein & Ja & Primärgewinnung & Einzelprojekte, keine umfassende Strategie \\
		Rumänien & Nein & Ja & Primärgewinnung & Teilnahme an EU-Projekten, keine nationale Strategie \\
		Belgien & Teilweise & Ja & Recycling, Verarbeitung & Fokus auf Sekundärrohstoffe, keine umfassende Strategie \\
		Österreich & Nein & Nein & Keine erkennbare Aktivität & Keine Beteiligung an CRMA-Projekten, keine Strategie veröffentlicht \\
		Ungarn & Nein & Nein & Keine erkennbare Aktivität & Keine bekannten Projekte oder Strategien \\
		Irland & Nein & Nein & Keine erkennbare Aktivität & Keine öffentliche Rohstoffstrategie \\
		Kroatien & Nein & Nein & Keine erkennbare Aktivität & Keine Beteiligung an CRMA-Projekten \\
		Slowakei & Nein & Nein & Keine erkennbare Aktivität & Keine Strategie oder Projekte bekannt \\
		Slowenien & Nein & Nein & Keine erkennbare Aktivität & Keine erkennbare nationale Rohstoffpolitik \\
		Luxemburg & Nein & Nein & Keine erkennbare Aktivität & Kein Rohstoffsektor, keine Strategie \\
		Malta & Nein & Nein & Keine erkennbare Aktivität & Kein industrieller Rohstoffbedarf \\
		Zypern & Nein & Nein & Keine erkennbare Aktivität & Keine Beteiligung an CRMA-Projekten \\
		Bulgarien & Teilweise & Nein & Primärgewinnung & Einzelne Bergbauaktivitäten, keine Strategie \\
		Litauen & Nein & Nein & Keine erkennbare Aktivität & Keine Strategie oder Projekte bekannt \\
		Lettland & Nein & Nein & Keine erkennbare Aktivität & Keine erkennbare Rohstoffpolitik \\
		Dänemark & Teilweise & Nein & Internationale Kooperation (Grönland) & Indirekte Beteiligung über Projekte in Grönland \\
		\hline
	\end{tabular}
	\label{tab:crma_umsetzung}
\end{table}

\subsubsection{Verwaltungstätitgkeit im CRMA}

Zur Verwaltung der Genehmigungsverfahren zu den strategischen Projekten werden von den Mitgliedsstaaten eine oder mehrere nationale Anlaufstellen benannt oder eingerichtet, jedoch müssen diese nicht zwangsläufig Behörden sein.\footnote{CRMA, Art. 9} Auch die Wahl der zuständigen Verwaltungsebene obliegt den Mitgliedsstaaten.\footnote{CRMA, Ew. 28f.}

Die Kommission ermöglicht den Mitgliedsstaaten weiterhin eine deutliche Repräsentierung im Bereich der kritischen Rohstoffen, sodass eine Mitwirkung und Einbindung weiterhin sichergestellt wird.

Auch für Unternehmen entstehen entsprechende Verwaltungsauflagen durch Art. 24 CRMA: Die nach Abs. I durch die Mitgliedsstaaten identifizierten \glqq großen\grqq Unternehmen, u. A. solche im Bereich des Automobilsektors, müssen mindestens alle drei Jahre eine Risikobewertung zu ihren Lieferketten für stratgische Rohstoffe durchführen. Dies umfasst eine geographische Analyse sowie eine zu Versorgungsfaktoren und schließlich eine Betrachtung der Anfälligkeit für Versorgungsuntebrechungen.\footnote{Art. 24 II lit. a-c} Die Ergebnisse fließen dann in Risikominderungsmaßnahmen ein.\footnote{Art. 24 IV, V CRMA}


Art. 35 CRMA schafft drüber hinaus einen aus Vertretern von Mitgliedsstaaten und Kommission zusammengesetzten\footnote{Art. 36 I CRMA} und regelmäßig tagenden\footnote{Art. 36 V CRMA} Europäischen Ausschuss für kritische Rohstoffe, welcher einerseits die Kommission berät und andererseits seine im CRMA festgelegten Aufgaben übernimmt: Hierunter fallen die Bewertung und Beschleunigung von Genhemigungsverfahren und strategischen Projekten, Kontrolle der mitgliedsstaatlichen Umsetzungsverpflichtungen, die Liste zu den kritischen und strategischen Rohstoffen und Koordinierung im Rahmen des CRMA.
Diese Konstruktion eines Rohstoffausschusses stützt sich – neben Art. 114 AEUV – auch funktional auf die Prinzipien der Verwaltungszusammenarbeit.

\subsubsection{Parallelen zur REACH-Verordnung}
Die europäische ROhstoffpolitik steht zudem im Spannungsfeld zwischen Umwelt- und Gesundheitsschutz und Versorgungssicherheit -- in diesem Kontext tritt die sog. REACH-Verordnung\footnote{EG Nr. 1907/2006} als zentraler Rechtsakt hervor, dessen Regelungsinhalt sich teilweise mit dem des CRMA überschneidet. Beide Regelungen betreffen rein inhaltlich dieselben Stoffe, insbesondere Mietalle und Mineralien wie Lithium, Nickel, Kobalt, jedoch aus unterschiedlichen Perspektiven: Während die REACH-VO auf die Bewertung und Adressierung chemischer Risiken abzielt, ist der CRMA bekanntermaßen versorgungsorientiert. 

ahlreiche Stoffe, die unter REACH als besonders besorgniserregend (SVHC) eingestuft sind, finden sich auch auf der Liste der kritischen Rohstoffe gemäß CRMA. So ist etwa Lithium sowohl  strategisch bedeutsam, als auch Gegenstand toxikologischer Bewertungen im Rahmen der REACH-VO, sodass dass hier potentielle Regelungskonflikte entstehen könnte, etwa wenn ein Stoff aus Gründen des Gesundheitsschutzes beschränkt oder verboten wird, obwohl er für die Umsetzung der grünen Transformation unverzichtbar ist. Ein weiteres zentrales Spannungsfeld betrifft die Zulassungsverfahren nach REACH. Die Verordnung sieht vor, dass bestimmte gefährliche Stoffe nur nach vorheriger Zulassung verwendet werden dürfen (Art. 57 ff. REACH). Dies kann zu einem faktischen Verwendungsverbot führen, wenn keine Zulassung erteilt wird. Der CRMA hingegen verfolgt das Ziel, die Verfügbarkeit genau dieser Stoffe zu fördern, etwa durch beschleunigte Genehmigungsverfahren für strategische Projekte (Art. 8 CRMA) oder durch die Festlegung von Benchmarks für die heimische Förderung und Verarbeitung.

CRMA und REACH sind unionsrechtlich gleichrangig und gelten daher nebeneinander, ein ausdrücklicher Vorrang ist nicht normiert. Vielmehr ergibt sich aus dem Zusammenspiel beider Regelwerke ein Koordinierungserfordernis, das insbesondere die Europäische Kommission betrifft. In der Praxis bedeutet dies, dass bei der Bewertung von Zulassungsanträgen nach REACH auch asymmetrische Interessenlagen – wie etwa die strategische Bedeutung eines Stoffes – berücksichtigt werden sollten. Der CRMA enthält hierzu in Art. 24 Abs. 2 eine Öffnungsklausel, wonach die Kommission Maßnahmen zur Unterstützung von Unternehmen ergreifen kann, die strategische Rohstoffe verwenden. 

Ein weiterer Berührungspunkt liegt im Bereich der Daten- und Informationspflichten. Beide Verordnungen setzen auf eine starke Rolle der Industrie bei der Bereitstellung von Informationen: REACH verlangt umfassende Daten zur Stoffbewertung, während der CRMA ein Monitoring der Lieferketten und eine Risikoanalyse durch Unternehmen vorsieht (Art. 17 CRMA). Hier besteht die Gefahr von Doppelregulierung und administrativer Überlastung, was eine Harmonisierung der Berichtspflichten nahelegt.

Beide Verordnungen zielen auf eine nachhaltige Nutzung von Ressourcen ab, wenn auch mit unterschiedlicher Schwerpunktsetzung, sodass such deutliche Parallelen insbesondere im direkten Stoffbezug ergeben. Zugleich offenbaren sich Regelungslücken und Zielkonflikte, die eine kohärente Rechtsanwendung und gegebenenfalls eine Weiterentwicklung des europäischen Stoffrechts erforderlich machen. Für die Zukunft wäre eine systematische Integration strategischer Rohstoffaspekte in das REACH-Regime denkbar – etwa durch eine Ausnahmeregelung für Stoffe mit hoher Versorgungskritikalität oder durch eine abgestimmte Risikobewertung unter Einbeziehung wirtschaftlicher und geopolitischer Faktoren.

\subsubsection{Mangelnde Sanktionierung bei Zielverfehlung}

Dem CRMA mangelt es an klaren Konsequenzen für den Fall, dass die Ziele nicht eingehalten werden können. Zwar können die Ziele eher als Motivation denn als Mindestziele interpretiert werden \autocite{finden, existiert} -- nichtsdestotrotz nimmt so der CRMA die Gestalt eines zahnlosen Tigers an, sodass die Verbindlichkeit infrage gestellt werden könnte.


\subsubsection{Fazit zum CRMA}
Durch die Verabschiedung wurde die Kooperation im Bereich der Rohstoffverwaltung auf europäischer Ebene gestärkt, die Bedeutung nochmals unterstrichen und die in den 2000er-Jahren begonenen Entwicklung einer unionalen Rohstoffpolitik fortgesetzt, weiter ausgeprägt und auf aktuelle Umstände zugeschnitten. Die eingangs beschriebene Phase der Bedeutungslosigkeit und geringen Prioritisierung scheint überwunden, was auch der vergleichsweise rasche Gesetzgebungsprozess des CRMA verdeutlicht.

Der CRMA sei daher zu begrüßen, auch wenn man keine kurzfristigen Effekte kurz nach den Inkrafttreten erwarten konnte würde der CRMA dennoch mittel- und langfristig einen ausschlaggebenden Beitrag zur offenen strategischen Autonomie und zum Binnenmarkt liefern, trotz der enthaltenden Zielkonflikte,\autocite{Schäffer/Hach, ZRP 2023, 210f.}

Erneut tritt die Politisierung des Rechtsbereiches hervor: Die Unterscheidung zwischen kritischen und strategischen Rohstoffen ist nicht per se an Eigenschaften der jeweiligen Mineralien gebunden, sondern (wenigstens im Falle des CRMA) Resultat einer entsprechenden Entcheidung bzw. Definition durch die Kommission--diese (wirtschafts-)politische Entscheidung sollte sich hierbei natürlich an einer objektiven EInschätzung orientieren.\autocite{Frau 2024, NVwZ, 1874, 1876}

Durch den CRMA sind zudem sowohl die Begrifflichkeiten und die umfassten Mineralien der kritischen sowie strategischen Rohstoffen nochmals respektive erstmals primärrechtlich fixiert.

Somit geht der CRMA insbesondere im Bereich der Ziele nicht über ein Wunschdenken hinaus: Die 

%Zielkonflikte

Zudem betont die Kommission in ihrer Folgenabschätzung zum CRMA, dass es in vielen Mitgliedstaaten an Kapazitäten, Daten und politischem Willen fehlt, um die Versorgungssicherheit systematisch zu adressieren

Obwohl der CRMA eine unionsweite Verpflichtung zur strategischen Rohstoffsicherung vorsieht, ist die tatsächliche Umsetzung in den Mitgliedstaaten bislang fragmentiert. Während rohstoffintensive Industrienationen wie Deutschland oder Frankreich das Thema aktiv adressieren, bleibt die politische und administrative Reaktion in anderen Mitgliedstaaten verhalten oder gar aus. Diese Heterogenität birgt Risiken für die Kohärenz der europäischen Industriepolitik und erschwert die Harmonisierung sektoraler Strategien, etwa im Automobilbereich

\subsection{Konfliktmineralien}
Erneut existiert keine einheitliche Definition zu den sog. Konfliktmineralien - also solche, 

Meist werden die Elemente Zinn, Tantal, Wolfram und Gold als Konfliktmineralien eingestuft, 

Die Regulierung von Konfliktmineralien und ihre Integration in das Wirtschaftsverwaltungsrecht stellt 

\subsubsection{Die Konfliktmineralien-VO}
Die Verordnung (EU) 2017/821, auch bekannt als "Konfliktmineralien-Verordnung" (VO) bindet Unionseinführer (UE) bestimmter sog. "Konfliktmineralien", genauer Wolfram, Tantal, Zinn ("3T" nach den englischen Bezeichnungen Tungsten und Tin) und Gold ("3TG") an Sorgfaltspflichten, die sich auf die entsprechenden Lieferketten auswirken, und kann als das am weitesten entwickelte Regelungsregime auf euiropäischer Ebene eingestuft werden\autocite{Kalls, ZfPW 2024, 181, 199}. Diese Sorgfaltspflichten sollen sicherstellen, dass UE mit ihrem Handel der genannten Metalle und deren Erze, also die als Konfliktmineralien verstandenen Rohstoffe, nicht zur Finanzierung bewaffneter Konflikte oder zu Menschenrechtsverletzungen beitragen und somit die "Verknüpfung zwischen Konflikten und illegalem Mineralabbau durchbrochen wird"\footnote{2017/821}. Die Verordnung gilt seit dem 1. Januar 2021 und erstreckt sich im Anwenungsbereich auf Importeure innerhalb der Europäischen Union, wie in Art. 1 Konfliktmineralien-VO geregelt. Unionseinführer i. S. d. Verordnung sind natürliche oder juristische Personen, die die vom Anwendungsbereich umfassten Minerale, Metalle und deren Erze ab in denen in Anhang 1 festgelegten Mengenschwellen in das Zollgebiet der Union einführen.\footnote{Art. 2 lit. l), VO 2017/821.} 

Der entsprechende Entwurf der Verordnung wurde von der Kommission bereits 2014 an Rat und Parlament übermittelt, hingegen stimmte der Rat erst 2017 ohne Gegenstimmen dem Vorschlag zu.\footnote{ST 7937 2017 INIT} Das Parlament nahm insgesamt 60 Änderungen zum Kommissionsvorschlag an,\footnote{8645/15} und erkannte bereits hier die Problematik wie "die richtige Balance zwischen operationeller Flexibilität bei der Umsetzung (...) und angemessener Einbeziehung des Gesetzgebers bei wichtigen praktischen Fragen (...) gefunden werden kann."\footnote{A8-0141/2015} Dies lässt sich als Erkenntnis festhalten: eine flexible Umsetzung von Rechtsnormen notwendig, um auf die dynamischen und oft volatilen Bedingungen im Rohstoffsektor angemessen reagieren zu können. Rohstoffmärkte sind von globalen Entwicklungen, geopolitischen Risiken und technologischen Innovationen geprägt, die schnelle Anpassungen erfordern. Eine starre rechtliche Struktur würde das Risiko bergen, dass Unternehmen nicht in der Lage sind, auf solche Veränderungen zeitnah zu reagieren, was ihre Wettbewerbsfähigkeit und letztlich die Versorgungssicherheit gefährden könnte. Gleichzeitig ist es jedoch unerlässlich, dass der Gesetzgeber bei der Ausgestaltung und Anpassung des Rohstoffverwaltungsrechts eine aktive Rolle spielt. Dies gewährleistet, dass wirtschaftliche Interessen nicht auf Kosten von Rechtsstaatlichkeit, Umweltstandards oder ethischen Verpflichtungen verfolgt werden. Insbesondere im Kontext der Konfliktmineralien-Verordnung zeigt sich die Notwendigkeit, legislative Vorgaben so zu gestalten, dass sie praktikabel sind, ohne die Kontrolle über kritische Aspekte wie Menschenrechtsverletzungen oder Umweltzerstörung zu verlieren. Im Rohstoffverwaltungsrecht ist diese Balance daher nicht nur eine Frage der rechtlichen Präzision, sondern auch ein wesentlicher Faktor für die wirtschaftliche Stabilität und ethische Integrität der gesamten Rohstoffwirtschaft.
Ein weiterer Punkt, der insbesondere den multilateralen Aspekt umfasst, wird in der Stellungnahme des Entwicklungsausschusses\footnote{A8-0141/2015, S. 53} angeführt: Eine Verodnung wie die Konfliktmineralien-VO können nicht als ein "in sich geschlossendes handelspolitisches Instrument angesehen werden", sondern müsse aus einer weiter gefassten Perspektive auch international betrachtet werden.

Die jeweiligen Konflikt- und Risikogebiete nach Art. 2 lit. f werden, einer jährlichen Aktualisierung unterliegend, auf der \textit{Conflict-Affected and High Risk Areas}-Liste (CAHRAS) der EU bekanntgegeben.\footnote{Die Liste ist abrufbar unter https://www.cahraslist.net/cahras. Stand 2024 umfasst die Liste die Staaten Afghanistan, Äthiopien, Burkina Faso, Burundi, DR Kongo, Eritrea, Indien, Jemen, Kamerun, Kolumbien, Libanon, Libyen, Mali, Mosambik, Myanmar, Niger, Nigeria, Pakistan, die Philippinen, Russland, Simbabwe, Somalia, Sudan, Südsudan, Ukraine, Venezuela, sowie die Zentralafrikanische Republik.} Durch den Verweis auf die CAHRAS-Liste wird ermöglicht, den geographischen Anwendungsbereich entsprechend anpassen zu können.

Hinsichtlich der Sorgfaltspflichten orientiert sich die Konfliktmineralien-VO eng an den, \textit{per se} rfechtlich unverbindlichen, OECD-Leitsätzen \autocite{OECDleitfaden2019} für die Erfüllung dieser, insbesondere durch die Bestrebungen der Kommission, diese Leitsätze stärker zu unterstützen wie auch die Anwendung der Leitsätze durch Unternehmen und die "Erfüllung der Sorgfaltspflicht", auch in nicht-OECD Staaten.\footnote{s. Präambel 9, VO 2017/821.} Daraus lässt sich schließen, dass eine Erfüllung der Vorgaben und Ziele der Konfliktmineralien-VO hinreichend durch eine Orientierung an den und eine Erfüllung der OECD-Leitsätzen erreicht werden kann, was letztendlich die Erfüllung die Sorgfaltspflicht betrifft. Letztlich werden also die Lieferketten der EU-Einführer gemäß einer Due-Dilligence-Prüfung betrachtet.\autocite[Rn. 390]{ruttloff_lieferkettensorgfaltspflichtengesetz_2022}
%wie gestaltet sich das im Umsetzungsrechtsakt

%genaue Analyse Konflikt-VO, Art 9


Zum 7. Mai 2020 trat das entsprechende deutsche Durchführungsgesetz (Mineralische-Rohstoffe-Sorgfaltspflichten-Gesetz, MinRohSorgG) zur Verordnung 2017/821 in Kraft \footnote{Gesetz vom 29.04.2020 - BGBl. I 2020, Nr. 21 vom 06.05.2020, S. 864.}, es gab zuvor keine entsprechenden Rechtsakte, sodass die Thematik vergleichsweise spät in den Bereich des Rohstoffrechts aufgenommen wurde. Die Implementierung der EU-Verordnung in den deutschen Rechtsrahmen zeigt, dass eine Verstärkung des Rohstoffwirtschaftsverwaltungsrechts erforderlich ist, um die Sorgfaltspflichten der Unternehmen wirksam durchzusetzen.

Hierbei agiert die Bundesanstalt für Geowissenschaften und Rohstoffe (BGR) als benannte Nationale Behörde und hat mit der deutschen "Kontrollstelle EU-Sorgfaltspflichten in Rohstofflieferketten" (DEKSOR) eine entsprechende Kontrollstelle geschaffen, die mit der Anwendung der Konfliktmineralien-VO betraut ist und Unionseinführer auf Einhaltung der Regelungen kontrolliert. 

Insofern hat die Union hier durch Vorgabe der Schaffung der Verpflcihtung der Mitgliedsstaaten zur Einrichtung einer nationalen Vollzugsbehörde die Richtlinien für die nationalen Verwaltungsstrukturen vorgeformt und intensiviert bzw. der mitgliedsstaatliche Vollzug einerseits mit Maßgaben konfrontiert, aber auch dementsprechend mit Insrumenten zum tatsächlichen Vollzug ausgestattet.

Die Relevanz des Rechtsaktes, zumindest aus deutscher Perspektive, kann gemischt betrachtet werden. Für den Berichtszeitraum 2022 \autocite{deutsche\_kontrollstelle\_eu-sorgfaltspflichten\_in\_rohstofflieferketten\_jahresbericht\_2023} (sowie teilweise 2023) wurden insgesamt 2.402 Unionseinführer erfasst, wovon aber lediglich 150 überhaupt die erforderliche Mengenschwelle überschritten (6\%) und davon 15 Unternehmen von nachträglichen Kontrollen betroffen waren. Unionseinführer können hierbei auch Privatpersonen sein, worauf ca. 2.000 der 2.402 zurückgehen. Ferner entfiel mehr als die Hälfte der Unionseinführer auf Zinnprodukte, bei Überschreitung der Mengenschwelle jedoch der überwiegende Teil auf Wolfram entfällt. Der Bericht beschreibt ein weiteres Problem, was sich insbesondere aus der zollrechtlichen Definition eines "Ursprungslandes" ergibt, also dem Land, in dem der letzte Verarbeitungsschritt stattfand – somit können Konfliktmineralien durchaus aus nicht vom geographischen Regelungsbereich der Verordnung umfassten Drittländern erfolgen, sodass die Nachverfolgbarkeit erschwert wird. Die Rückverfolgbarkeit über die gesamte Lieferkette ist daher nur deutlich eingeschränkt möglich.
Zudem verfügen die DEKSOR und BGR nur über eine stark eingeschränkte Kompetenz. So ist zwangsläufig eine Bewertung der Kommission erforderlich, die die nationalen Behörden zum Verhängen von Strafen bei Nichteinhaltung befugt (Art. 17 III Konfliktmineralien-VO). Auch der DEKSOR-Bericht nennt das Nichtvorhandensein einer Befugnis bei Unklarheiten über die Überschreitung der Mengenschwelle Auskunft zu erhalten als nachteilig, setzt eine "Auskunftspflicht [$\ldots$] voraus, dass ein Unionsführer tatsächlich die Mengenschwelle überschritten hat". Dies schränkt die Durchsetzungskraft dieses rohstoffverwaltungsrechtlichen Akts ein; ferner kann daraus eine mangelnde Umsetzung der erwähnten Sorgfaltspflicht abgeleitet werden, was die DEKSOR in ihrem Bericht bestätigt, denn die jeweiligen Offenlegungspflichten der Unionseinführer über den Mengenschwellen werden auch bisher nur von einer Minderheit in vollem Umfang erfüllt \autocite[25]{deutsche\_kontrollstelle\_eu-sorgfaltspflichten\_in\_rohstofflieferketten\_jahresbericht\_2023}[]. Bei nachträglichen Kontrollen waren zudem Nachweise oft nicht ausreichend oder wurden nur zeitverzögert bereitgestellt, was insbesondere dem § 6 MinRohSorgG zuwiderläuft bzw. erkennen lässt, dass lediglich den Auskunftspflichten aus Art. 6 Konfliktmineralien-VO nachgekommen wird, nicht aber den Risikomanagementpflichten (Art. 5 Konfliktmineralien-VO), den Pflichten in Bezug auf das Managementsystem (Art. 4) und den generellen Offenlegungspflichten nach Art. 7.

DEKSOR: "Im Rahmen der nachträglichen Kontrollen ist zudem aufgefallen,
dass es für die Unternehmen eine Herausforderung ist, die erlangten und auf aktuellem Stand gehaltenen Informationen bezüglich der Lieferkette nach Art. 7 Abs. 2 VO von vorgelagerten Hütten und Raffinerien zu erhalten."

Gemäß Art. 11 der Konfliktmineralien-VO i. V. m. § 3 III MinRohSorgG ist die DEKSOR zur Durchführung "geeigneter nachträglicher Kontrollen" befugt, sodass entweder auf Grundlage eines "risikobasierte[n] Ansatz[es]" oder bei Vorliegen entsprechender Hinweise auf Verstöße (auch durch Dritte) nachträgliche Kontrollen durchgeführt werden können, in denen insbesondere ein Einhaltung der Sorgfaltspflichten und Prüfpflichten nachgegangen wird.

Es wird von der DEKSOR zudem berichtet, dass sich hierbei auf ein mögliches Risiko für die Geschäftstätigkeit der involvierten Unternehmen bei Offenlegung der Lieferketten im Sinne der Wahrung von Geschäftsgeheimnissen und anderen wettbewerbsrelevanten Informationen berufen wird, was als ein Widerspruch zum Transparenzziel der Konfliktmineralien-VO gewertet werden kann, wobei auch die OECD-Leitsätze eine Offenlegung zumindest gegenüber staatlichen Stellen als verpflichtend darstellen.\autocite[vgl.][37-38]{deutsche\_kontrollstelle\_eu-sorgfaltspflichten\_in\_rohstofflieferketten\_jahresbericht\_2023} Insbesondere in Verbindung mit der Frage der Rohstoffversorgungssicherheit scheint es sinnvoll, dass sich die Kommission bei ihrer Überprüfung im Dreijahresrhythmus nach Art. 17 II Konfliktmineralien-VO 
%hier noch genauer die Artikel anschauen hinsichtlich Wortlaut 

Es wird zudem festgestellt, dass die entsprechenden Verstoßverfahren den von der DEKSOR erwarteten Umfang und Komplexität überschritten, was in verzögerten Verfahrensdauern resultiert. 

Die beschränkten Befugnisse der nationalen Behörden weisen auf eine Schwäche im aktuellen Rohstoffverwaltungsrecht hin. Es zeigt sich die Notwendigkeit, die Durchsetzungsmechanismen zu stärken, damit die Behörden effektiver handeln und die Einhaltung der Sorgfaltspflichten sicherstellen können. Wenn nur eine Minderheit der Unternehmen die Pflichten einhält, besteht die Gefahr, dass die Wirksamkeit der Verordnung und der zugrunde liegenden Rechtsnorm untergraben wird. Dies könnte das Vertrauen in das Rohstoffverwaltungsrecht schädigen und zu einer Fragmentierung der Rechtsdurchsetzung führen. Eine Einführung von Maßnahmen zur Compliancesteigerung, z. B. Anreizsysteme, sind denkbar -- zumindest muss aber der Fokus auf wirksamen Durchsetzungsmechanismen liegen, nicht nur im Bereich der Konfliktmineralien-VO.

Es lässt sich außerdem argumentieren, dass das MinRohSorgG möglicherweise nur eine ungenügende Eingriffsgrundlage bei einer unzureichenden Erfüllung der unternehmerischen Auskunfts- und Mitwirkungspflichten bietet.  Dies wird deutlich, wenn man die Sanktionsmechanismen, Kontrollmöglichkeiten und Durchsetzungskraft dieses Gesetzes betrachtet. Als Bußgeld ist lediglich ein Zwangsgeld in Höhe von "bis zu 50.000 Euro im Verwaltungszwangsverfahren vorgesehen (§ 9).

%Verwaltungszwangverfahren: Eingriffshürde betrachten

Strafrechtliche Sanktionen wie im benachbarten Umweltstrafrecht sind nicht vorgesehen. m Vergleich zum MinRohSorgG bietet das deutsche Lieferkettensorgfaltspflichtengesetz, das seit 2023 in Kraft ist, deutlich schärfere Sanktionen. Es sieht nicht nur höhere Bußgelder vor, sondern ermöglicht auch den Ausschluss von Unternehmen aus öffentlichen Vergabeverfahren. 

Schließlich stellt die DEKSOR fest, dass Versuche von Unternehmen wahrgenommen werden, die erforderlichen Nachweise zum Beleg der Erfüllung der Sorgfaltspflichtsvorschriften auf andere Bereiche auszulagern, die nicht 

%Umgehungstatbestände



Diese Verzahnung bzw. Verbindung von Rechtsakten zeigt, dass dieses Vorgehen insbesondere im rohstoffverwaltungsrechtlichen Bereich sinnvoll erscheint. Die Übernahme internationaler Standards in das europäische Rechtssystem stärkt die Integration globaler Best Practices in die nationale Gesetzgebung und fördert die Kohärenz im Rohstoffverwaltungsrecht. Im Beispiel der Konfliktmineralien-VO erleichtert die Orientierung an den OECD-Leitsätzen sowohl zunächst die Anforderungen der EU-Verordnung, aber auch potentieller weiterer (internationaler), nicht zwangsläufig staatlicher Regelungen nachzukommen, sodass hier mit der Schaffung einer sektor- und rechtsübergreifenden Grundlage begonnen wird. Langfristig entstehen hier Vorteile sowohl auf Unternehmens- und Verwaltungsseite: Durch die Nutzung anerkannter und immer weiter verbreiteter Richtlinien und Leitsätzen wird der Compliance- und Verwaltungsaufwand auf beiderlei Seiten auf ein erforderliches Minimum reduziert, mit den entsprechenden positiven Effekten. Ferner wird widersprüchlichen Regelungen oder Schlupflöchern vorgebeugt. Den zunächst rechtlich nicht verbindlichen Leitsätzen kommt durch den Verweis in der Konfliktmineralien-VO schließlich trotz ihres Soft-Law-Status durch ihre normative Einbettung eine mittelbare Rechtswirkung zukommt: Die Konfliktmineralien-VO hebt die Leitsätze aus ihrem ursprünglichen Soft Law-Kontext heraus und verleiht ihnen innerhalb des Anwendungsbereichs (und nur hier) der Verordnung eine rechtliche Relevanz -- "von 'Soft Law' zu 'Hard Law'"\autocite[42]{teicke_gesetzliche_2018}. Ferner kann inn Fällen, in denen die Verordnung unspezifisch ist oder Interpretationsspielraum lässt, die OECD-Leitsätze zur Konkretisierung herangezogen werden. Auf der anderen Seite lässt sich hingegen argumentieren, dass die als Soft Law konzipierten Leitsätze durch den schlichten Verweis in Art. 4 lit. b) Konfliktmineralien-VO nicht formell durch ein Rechtsetzungsakt in "harte" Rechtsnormen überführt wurden, denn es kann argumentiert werden, dass der Verweis als ein schlichter Hinweis auf eine bewährte bzw. empfohlene Praxis verstanden werden kann, ohne aber eine rechtliche Bindungswirking (mittelbar oder unmittelbar) zu entfalten -- "Soft Law bleibt Soft Law". Zudem: Die Verordnung weist keine zwingende Normenkonkretisierung auf, das heißt die Verordnung selbst gibt den Adressaten einen gewissen Spielraum, \textit{wie} die Sorgfaltspflichten genau umzusetzen sind, sodass der Verweis schlicht als eine Empfehlung oder Vorschlag interpretiert werden kann, der keine strikte rechtliche Bindung (sofern diese Begriffskonstellation überhaupt als existierend gewertet werden kann) entfaltet und ggf. durch UE alternative, ebenfalls geeignete Maßnahmen zur Einhaltung der Sorgfaltspflichten ergriffen werden könnten (Orientierungshilfen-Argument). Denn schließlich bleibt auch unklar, inwieweit ein Verstoß gegen die OECD-Leitsätze, ohne gleichzeitig gegen die formellen Vorschriften der Verordnung zu verstoßen, tatsächlich zu rechtlichen Konsequenzen führen würde. Da die Leitsätze nicht selbst Teil des bindenden Rechts sind, könnte argumentiert werden, dass ihre Nichtbeachtung, solange die formellen Pflichten der Verordnung erfüllt werden, keine Sanktionen nach sich ziehen sollte. Dies schwächt das Argument für eine mittelbare Rechtswirkung ab. Diese Auslegungsfrage lässt sich auch in der gegenwärtigen Literaturlage wiederfinden: So argumentiert Grado\autocite{grado_eu_2018}, dass die Umsetzung der OECD-Leitsätze in der EU-Verordnung als ein Schritt in Richtung einer stärkeren Verbindlichkeit von Soft Law betrachtet werden kann, aber dass dies nicht gleichbedeutend mit einer vollständigen Rechtswirkung ist. Die Verordnung hat zwar das Potenzial, die Einhaltung der Leitsätze zu fördern, jedoch bleibt sie in ihrer Fähigkeit, diese Leitsätze in hartes Recht zu verwandeln, begrenzt. Im Kontext möglicher Konflikte zum deutschen ABG-Recht wird ebenfalls darauf hingewiesen, dass ein pauschaler Verweis auf die OECD-Leitsätze aus Transparenzgründen nicht zu empfehlen ist und zudem "wenig harmonisch" gestaltet worden sei.\autocite[42]{teicke_gesetzliche_2018}
Die Frage der unklaren mittelbaren Rechtswirkung hängt daher auch stark von der Auslegung und Anwendung im Rahmen des Rohstoffverwaltungsrechts ab. Weiterhin ist also noch nicht ersichtlich, inwieweit die Konfliktmineralien-VO diese Fragen klären und zukünftigen Entwicklungen beispielhaft dienen kann, insbesondere hinsichtlich der Soft-Law-Bindungsfrage.

%im Legislativprozess der VO checken, wie ein Verweis auf die Leitsätze argumentiert war

Dieses Vorgehen der Verzahnung kann also als ein beispielhaftes Leitbild für die weitere Entwicklung eines interdisziplinären und anwendungsbetontem Rohstoff- und Rohstoffverwaltungsrecht verstanden werden. Weitere Beispiele für solche Leitsätze lassen sich finden: %!

Hingegen existiert, wie oben ausführlich dargestellt, im Bereich der Verzahnung von zunächst rechtlich nicht bindenden mit sekundärrechtlichen Akten die "Soft-Law-Problematik" der mittelbaren Anwendbarkeit.

Ferner lässt die VO ausreichenden Gestaltungsspielraum für die die jeweiligen nationalen Umsetzungsmaßnahmen und insbesondere die Einführung von Regeln bei Verstößen gegen die Verordnung.\footnote{\textit{Siehe hierzu} Erwägungsgrund 20, VO 2017/821 sowie Art. 16.} So liegt beispielsweise die Höchststrafe 

Die Durchführungsbefugnisse hingegen "sollten der Kommission übertragen werden".\footnote{Erwägungsgrund 21, VO 2017/821.} 

%Analysieren: https://dip.bundestag.de/vorgang/.../255348

Die Verordnung somit zielt darauf ab, Transparenz und Verantwortlichkeit in den Lieferketten der Unternehmen zu erhöhen, indem sie sie verpflichtet, Informationen über die Herkunft und den Handel der von ihnen verwendeten Mineralien offenzulegen. Diese Verpflichtungen stehen in direktem Zusammenhang mit den möglichen Grundsätzen des Rohstoffwirtschaftsverwaltungsrechts, und das RWvR kann erneut als rechtlicher Rahmen zur Umsetzung sowie Einhaltung der Regelungen agieren. Interessant gestaltet sich hierbei nicht nur die Erwähnungsgrund 13 genannte Transparenzschaffung zur Vertrauensbildung gegenüber den Wirtschaftsbeteiligten, sondern auch eine Öffentlichmachung des rohstoffverwaltungsrechtlichen Aspektes -- es ließe sich argumentieren, dass durch die öffentliche Salienz hinsichtlich der Verfahren und Strategien sowie die eigentliche Erfüllung auch hier eine Art "Demokratisierung" des Rohstoffrechts stattfindet, auch aus Unternehmenssicht durch die eindeutige Verwaltungs-Unternehmens-Beziehung in dieser Angelegenheit und die zunächst eindeutige Regelung des Bereiches.

Die Regulierung von Konfliktmineralien ist ein entscheidender Bestandteil eines umfassenden Rohstoffwirtschaftsverwaltungsrechts, das sowohl die wirtschaftlichen Interessen als auch die ethischen Verpflichtungen Deutschlands und der EU berücksichtigt. Die bisher bestehenden gesetzlichen Regelungen, insbesondere die Verordnung  2017/821 und das Lieferkettensorgfaltspflichtengesetz, bieten eine Grundlage, die jedoch durch weitere rechtliche Entwicklungen und eine verbesserte Durchsetzung ergänzt werden muss. Ein integrierter Ansatz im Rahmen eines nationalen Rohstoffwirtschaftsverwaltungsrechts kann dazu beitragen, diese Herausforderungen zu bewältigen und gleichzeitig die Rohstoffversorgung weiterhin sicherzustellen und negative Effekte bei Unternehmen, aber auch wie schon beschrieben in den konfliktmineralienexportierenden Staaten zu vermeiden.

Nichtsdestotrotz kann festgehalten werden, dass es sich bei der Konfliktmineralien-VO um einen sehr kleinen und limitierten Regelungsbereich handelt, was sich insbesondere in der Zahl der erfassten Fälle widerspiegelt und daher nun eine geringere Relevanz für die Wirtschaft angenommen werden kann. Darüber hinaus ist die Konfliktmineralien-VO eindeutig dem internationalen Teil des Rohstoffverwaltungsrechts zuzurechnen und fällt ebenso in den Bereich des Wirtschaftsvölkerrechts, nicht zuletzt durch die eindeutigen Menschenrechtsbezogenen Aspekte im Rechtsakt. Ferner spricht die Konfliktmineralien-VO eindeutig nicht den Bereich der Rohstoffversorgung an, kann aber ohne Weiteres der Rohstoffverwaltung zumindest angerechnet werden.

\paragraph{Nearshoring und Konfliktmineralien}

Es liegt nahe, dass Unternehmen im Bereich der Konfliktmineralien eine Verlagerung von Importen, nichtg zwangsläufig in Form des Nearshorings, in Betracht ziehen können, um die Auskunftspflichten im Rahmen der Konfliktmineralien-VO zu reduzieren oder gar gänzlich zu vermeiden und so den unternehmerischen Dokumentationsaufwand einzustellen. Im DEKSOR-Report wird beschrieben, dass aus unternehmerischer Sicht die Aufwendungen für die Erfüllung der Sorgfaltspflichten nicht im Verhältnis zum erforderlichen Umsetzungsaufwand stehen würden, sodass entsprechende Verlagerungen in Betracht gezogen werden um die Tatbestandsmerkmale der Konfliktmineralien-VO bzw. des MinRohSorgG zu verlassen -- insbesondere durch ein Nearshoring des Rohstoffbezugs hin zu einem direkten Import aus EU-Mitgliedsstaaten. Auch die DEKSOR erkennt hier richtigerweise, dass hier ein Umgehungstatbestand infrage kommen könnte; dieser sei zwar nicht illegal, könne aber nicht als Sorgfaltsmaßnahme des Unternehmens selbst betrachtet werden, sondern als eine Verlagerung dieser Sorgfaltspflicht auf den dann infrage kommenden Unionseinführer in das Zollgebiet der Union.\autocite[vgl.][43]{deutsche\_kontrollstelle\_eu-sorgfaltspflichten\_in\_rohstofflieferketten\_jahresbericht\_2023}

%Bewertung von Umgehungstatbeständen
Berufen auf Vertrauen des Unionseinführers/Verhütter/Rohstoffproduzent

Auch die DEKSOR stellte hierbei fest, dass das Sorgfaltsvertrauen der (nachgelagerten) Unternehmen trotz Zertifizierungen nicht immer begründet sei, denn "eine Verlagerung der Sorgfaltspflichten auf andere Unternehmen verschiebt nur das Problem anstatt im Verbund auf transparente Lieferketten hinzuwirken" \autocite[43]{deutsche\_kontrollstelle\_eu-sorgfaltspflichten\_in\_rohstofflieferketten\_jahresbericht\_2023}

Nearshoring als Umgehungsstrategie

Juristisch gesehen kann die Verlagerung von Beschaffungsprozessen in stabilere, näher gelegene Regionen als eine direkte Reaktion auf die Anforderungen der Konfliktmineralien-Verordnung gesehen werden. Diese Verlagerung ist im Wesentlichen eine Form des Nearshoring, da sie den geografischen Schwerpunkt der Beschaffung näher an die Heimatmärkte bringt

Nur eine indirekte Verbindung zum Nearshoring weisen sekundäre Effekte der Konfliktmineralien-VO auf Unternehmen auf -- Compliance in Lieferketten ist im engeren Verständnis kein Bestandteil eines Nearshorings, kann aber durchaus als ein solches interpretiert werden, da letztendlich durch die Anwendung von Sorgfaltspflichten wie im Rahmen der Konfliktmineralien-VO Unternehmen dazu angehalten sind, ihre Lieferketten so zu gestalten, dass sie die mitunter strengen Anforderungen der VO erfüllen und somit über Nearshoring auf eine Lösung zurückgreifen, um Compliance-Kosten zu senken, die Kontrolle über die Lieferkette zu verbessern und Risiken zu minimieren, die mit weit entfernten, instabilen Lieferquellen verbunden sind. So wird Nearshoring zu einer strategischen Entscheidung, die durch rechtliche Vorgaben motiviert ist und die betriebliche Resilienz und Nachhaltigkeit fördert. Dies bestätigt auch die Beobachtung, dass die sog. \textit{Corporate Social Responsibility} (CSR), in der unternehmerischen Praxis immer weiter an Bedeutung gewinnt und sogar nur Empfehlungen im Sinne eines Soft Laws übernommen werden, die sich teilweise bis auf Lieferanten erstrecken.\autocite[39]{teicke_gesetzliche_2018}

Hier lassen sich aus rohstoffverwaltungsrechtlicher Perspektive weitere Vorteile zur Erfolgsmaximierung und Umsetzungsvereinfachung erkennen, denn Soft Law kann eine bedeutende und sinnvolle Ergänzung sowie Erweiterung des Rohstoffverwaltungsrechts darstellen, indem es flexiblere, anpassungsfähigere und dennoch wirksame Rahmenbedingungen schafft, die sowohl den regulatorischen Anforderungen als auch den praktischen Bedürfnissen der Wirtschaft und der staatlichen Verwaltung gerecht werden. „Hard Law“-Regelungen bieten zwar Rechtssicherheit, sind jedoch oft starr und weniger anpassungsfähig an die dynamischen Entwicklungen in globalen Lieferketten und Rohstoffmärkten. Soft Law, das typischerweise in Form von unverbindlichen Leitlinien, Empfehlungen, Verhaltenskodizes oder Branchenstandards existiert, bietet hier eine notwendige Flexibilität. Es ermöglicht eine schnellere Anpassung an technologische Innovationen, Marktveränderungen und neue ethische Anforderungen. Soft Law kann somit Lücken im bestehenden Rechtssystem füllen und eine dynamische Anpassung ermöglichen, die für eine effektive und zeitnahe Regulierung im Rohstoffsektor unerlässlich ist. Für staatliche Behörden bietet Soft Law den Vorteil, regulatorische Ziele durch kooperative Ansätze zu erreichen, anstatt ausschließlich auf Zwangsmaßnahmen zurückzugreifen. Soft Law kann als Instrument zur Förderung der Selbstregulierung innerhalb von Industrien dienen, wobei staatliche Kontrolle durch Beratung, Überwachung und Förderung von Best Practices ergänzt wird. Ein solcher kooperativer Ansatz reduziert den administrativen Aufwand und ermöglicht eine effizientere Allokation staatlicher Ressourcen. Zudem kann Soft Law als „Laboratorium“ für neue Regulierungsansätze fungieren, die, sobald sie sich in der Praxis bewährt haben, in verbindliche Rechtsnormen überführt werden können. Dies fördert eine schrittweise und praxisnahe Weiterentwicklung des Rohstoffverwaltungsrechts. Darüber hinaus kann CSR als ein international geprägtes Feld wahrgenommen werden, sodass Soft Law hier eine Art Brücke zwischen diversen nationalen und internationalen Rechtsordnungen schlagen kann.


\subsection{ACI VO}
Die Verordnung 


\subsection{Verwaltungsvollzug durch die Union}
Ferner, wie im CRMA deutlich wird, ist ein Großteil des EU-Rohstoffrechts der Verwaltungskoordinierung zuzuschlagen: insbesondere die Berichterstattung durch die Mitgliedsstaaten mit entsprechender Verwaltung durch die Kommission tritt hierbei hervor, wobei diese Verwaltung auch ihre Wirkung gegenüber den Wirtschaftsbeteiligten entfaltet. Die Union weist hier fast keine eigene Vollzugskompetenz auf und ist stattdessen auf den mittelbaren durch die Mitgliedsstaaten angewiesen. Der Vollzug der indirekten Verwaltung findet also nur in der Informationsgewinnung der Kommission und die entsprechende Nutzung dieser Informationen Ausdruck, was nahtlos in die Kontrollfunktion der Kommission übergeht.

Die Berichterstattung ist somit als zentrales Instrument der EU zur Verwaltung des Rohstoffrechts wahrzunehmen. Hierbei ist die Kommission ebenfalls, den Verwaltungsaufwand zu reduzieren, was beispielsweise durch delegierte Rechtsakte erreicht wird\footnote{vgl. hierzu exemplarisch CRMA Art. 34 S. 2} oder aber durch die Schaffung einheitlicher digitaler Schnittstellen zur Online-Bereitstellung.\footnote{CRMA Rn. 31}, wie auch bereits in der Vergangenheit durch die Kommission beabsichtigt.\footnote{Nicht im konkreten Rohstoffbezug, aber allgemein VO (EU) 2018/1724, der CRMA verweist in Rn. 31 auch ausdrücklich auf Art. 6 I der Verordnung.}
%auf Vollzug Unionsveraltungsrecht eingehen

Weitere Berichts- oder Informationsstellen auf mitgliedsstaatlicher Ebene wie die nationalen Rohstoffagenturen stehen hier ebenfalls der Kommission zur Verfügunug.
\begin{table}[h!]
	\centering
	\begin{tabular}{|l|l|}
		\hline
		\textbf{Mitgliedsland} & \textbf{Name der Rohstoffagentur} \\ \hline
		Deutschland            & Deutsche Rohstoffagentur (DERA) \\ \hline
		Frankreich             & Bureau de Recherches Géologiques et Minières (BRGM) \\ \hline
		Finnland               & Geological Survey of Finland (GTK) \\ \hline
		Schweden               & Geological Survey of Sweden (SGU) \\ \hline
		Portugal               & Laboratório Nacional de Energia e Geologia (LNEG) \\ \hline
		Österreich             & Geologische Bundesanstalt (GBA) \\ \hline
		Polen                  & Polish Geological Institute – National Research Institute (PIG-PIB) \\ \hline
		Tschechien             & Czech Geological Survey (ČGS) \\ \hline
	\end{tabular}
	\caption{EU-Mitgliedsländer mit Rohstoffagenturen}
	\label{tab:rohstoffagenturen}
\end{table}

%Dichte der Überwachung des Rohstoffrechts






\subsection{Rohstoffabkommen und Rohstoffpartnerschaften}
Die internationale Prägung des Marktes für mineralische Rohstoffe bedarf aus den bekannten Gründen keiner weiteren Einordnung.

Rohstoffabkommen und -partnerschaften lassen sich aus zweierlei Hinsicht betrachten: als Instrument der Rohstoffpolitik und als außengerichtete Komponente der Rohstoffpolitik und -verwaltung. Bei diesen Abkommen und Partnerschaften handelt es sich um zwischenstaatliche Kooperationsformate, die nicht primär handelsrechtlich verfasst sind, sondern einen Mix aus diplomatischen, wirtschaftlichen und entwicklungspolitischen Maßnahmen darstellen. Anders als klassische Handelsverträge zielen sie auf dauerhafte, strukturelle Kooperationen, etwa im Bereich Exploration, nachhaltige Gewinnung, Rückverfolgbarkeit und Technologietransfer. Juristisch lassen sich Rohstoffpartnerschaften teils als völkerrechtliche Vereinbarungen, teils als politische Absichtserklärungen einordnen, wobei ihre rechtliche Verbindlichkeit und institutionelle Ausgestaltung im Einzelfall stark variiert.

Internationale Rohstoffabkommen\footnote{Oder auch -übereinkommen -- hinsichtlich der völkerrechtlichen Bindungswirkung ist die Bezeichnung nicht ausschlaggebend, sondern vielmehr die vertragliche Rechtsnatur; siehe hierzu Art. 2 I lit. a des Wiener Übereinkommens über das Recht der Verträge, BGbl. 1985 II S. 926ff., 928 sowie ausführlich generell zur Stellung und Wirkung völkerrechtlicher Verträge innerhalb der nationalen Rechtsordnung der Vetragsstaaten Wissenschaftliche Dienste des Deutschen Bundestages, Rechtsverbindlichkeit völkerrechtlicher Abkommen sowie der Beschlüsse von Einrichtungen der VN, WD 2 - 3000 - 086/19, S. 4, 2019.} sind als völkerrechtlicher Vertrag einzstufen und binden zunächst Anbieter und Nachfrager im Sinne eines angemessenen Ausgleichs der jeweiligen Interessen.\autocite{Schorkopf, Rn. 43}

Art. 207 AEUV, Art. 21 EUV


Insbesondere durch Betätigung des UNCTAD\footnote{United Nations Conference on Trade and Development} bilden Rohstoffabkommen ein Instrument zur Sicherung sowohgl des Preisniveaus und der Rohstoffversorgung, insbesondere auch vor dem Hintergrund dass mehrheitlich Entwicklungsländer von im Rohstoffexport erzielten Einnahmen abhängig sind.\autocite{Herdegen IntWirtschaftsR/Herdegen, 13. Aufl. 2023, § 11. Rn. 3,} Rohstoffabkommen können hierbei an sich auch als internationale Organisation gestaltet sein.

Rohstoffabkommen treten hier selbstverständlich auch mit Mineralien-Bezug auf -- in der Vergangenheit oft inklusive einer direkten Steuerung des Marktes, was im Verlauf auch zum Scheitern solcher mineralischen Rohstoffabkommen geführt hat.\autocite{ntergouvernementale Rohstoffregimes im Zwielicht — Lehren aus dem Zinn-Debakel Ludwig Gramlich Verfassung und Recht in Übersee / Law and Politics in Africa, Asia and Latin America Vol. 20, No. 4 (1987), pp. 486-514} 

Die EU verfolgt im Rahmen ihrer Rohstoffpolitik verschiedene Typen von Partnerschaften: Strategische Partnerschaften mit Drittstaaten, etwa mit Kanada, Kasachstan oder Namibia, zielen auf langfristige Kooperationsformate zur Sicherstellung nachhaltiger Versorgung, während Technologie- und Kapazitätsaufbauprogramme (z. B. durch die EU-Rohstoffallianz, ERMA)  dem Know-how-Transfer und der Förderung lokaler Wertschöpfungsketten dienen. Partnerschaften unter dem EU \glqq Global Gateway Programm\grqq spielen aus Rohstoffsicht noch eine untergeordnete Rolle.\footnote{Zwar hat die Kommission im Rahmen des Global Gateways mit der DRK Ende 2023 ein MoU unterzeichnet, jedoch sind seitdem keine nennenswerten Enwticklungen festzustellen.}

Aktuell 

COM(2012) 82 final 

Die Rolle der strategischen Rohstoffpartnerschaften und ihre Bedeutung für die Lieferketten kritischer Rohstoffe wurde zudem durch die Kommission auch im Kontext strategischer Projekte in der Vorstellung des Industrial Action Plan for the European automotive sector erneut unterstrichen.\footnote{COM(2025) 95 final, S. 13.}

%RohPolPress


\subsection{Charakterisierung der EU-Rohstoffgesetzgebung}

Kommission: Die Kommission verfolgt 
Sie wird hierbei insbesondere vom Europäischen Green Deal beeinflusst

Die rohstoffpolitischen Aktivitäten der Kommission beschränkten sich zunächst nur auf Mitteilungen in Form von Strategien; erst mit der Verabschiedung des CRMA und der Konfliktmineralien-Verordnung sind hier die ersten konkreten Rechtsakte in Erscheinung getreten, sodass eine gewisse Diskrepanz zwischen Forderung und Strategieempfehlungen auf der einen Seite und tatsächlicher Umsetzung auf der anderen zu erkennen ist.

Parlament: 

Darüber hinaus ist das Parlament als ein vollwertiger Akteur im Bereich der Rohstoffpolitik zu klassifizieren

Rat:



Durch die fehlenden Durchsetzungskompetenzen beschränkt sich der rohstoffverwaltungsrechtscharakter der Union fast ausschließlich auf die rechtsetzende Tätigkeit, die Umsetzung und Durchführung obliegt den Mitgliedsstaaten, ebenso die Kontrolle des letzteren durch die betroffenen Akteure. Dies deckt die Erkenntnis, dass sich die Unionsaktivität auf ein Minimum beschränkt und der überwiegende Teil der Rohstoffpolitik von den Mitgliedsstaaten geprägt ist, was die Erörterungen im Folgenden Abschnitt rechtfertigt.


\subsection{Interimsfazit}
Der vorangegangene Teil hat die Verbindung von mineralischen Rohstoffe im Sinne dieser Arbeit und der Europäischen Union verdeutlicht. In der historischen Betrachtung konnte festgestellt werden, dass die Union durchaus bereits vor der \glqq Zeitenwende\grqq im rohstoffrechtlichen Bereich aktiv war.
Die Frage nach der Zuständigkeit \textit{per se} ist abschließend zu bejahen -- jedoch ist hierbei festzustellen, dass die Union von ihrer Kompetenz nicht vollumfänglich Gebrauch macht und sich die weitere Ausgestaltung in den Rechtsrahmen der Mitgliedsstaaten bewegt.
Die Organe der Union haben bisher nahezu alle Formen der Ausübung der Zuständigkeit gewählt, hierbei sticht jedoch der CRMA als erste Verordnung abzielend auf mineralische Rohstoffe im Sinne dieser Arbeit hervor.

Inwieweit nun die limitierte Aktivität der Union im Rahmen der geteilten Zuständigkeit durch die Mitgliedsstaaten aufgefüllt wird, soll im Folgenden genauer betrachtet
werden.

Rohstoffrechtliche Aktivität in Bezug auf kritische Mineralien zählt zu den jüngeren Betätigungsfeldern mitgliedsstaatlicher Politik
%hier näher darauf eingehen, welche Staaten hier wie aktiv sind




\section{Rohstoffe und Nationalstaaten}



\subsection{Zum Verhältnis nationaler Regelungen und Unionsrecht}
Unter Berücksichtigung des Schutzbereiches der unionsrechtlichen Regelungen zu Rohstoffen ist in Bezug auf nationale Maßnahmen der Vollständigkeit halber zu betrachten, wie weit der Handlungsspielraum der Mitgliedsstaaten ausfällt.

\subsection{Bergbau in den Mitgliedsstaaten der Union}

Da die Sicherung der Versorgungssicherheit auf die gesamte Union erstreckt, ist auch die Erschließung von mineralischen Vorkommen in der gesamten EU zu beobachten.	
%https://www.europarl.europa.eu/RegData/etudes/STUD/2022/729156/IPOL_STU(2022)729156_EN.pdf

Jedoch ist mit einer Dauer von 10-15 Jahren zwischen ENtdeckung und Abbau zu rechnen.

%https://fennia.journal.fi/article/view/87223

Im Januar 2023 gab der schwedische LKAB-Konzern bekannt, dass nahe der schwedischen Stadt Kiruna ein Vorkommen an seltenen Erden entdeckt worden sei 

Dass dieses Vorkommen keineswegs neu ist, zeigt die Experteneinschätzung, dass das Vorkommen bereits seit dem Zweiten Weltkrieg längst bekannt sei und bei Rentabilität eine Erschließung schon längst erfolgt worden wäre.\autocite{VDi Nachrichten: Seltene Erden: Deutschland importiert lieber, als selbst zu fördern}


Jedoch 

Auch stößt der geforderte Abbau in den Mitgliedsstaaten nicht immer auf Unterstützung: Gibt es hier Klagen?

\section{Rohstoffe und Deutschland}

\subsection{Grundgesetz}

\subsubsection{BBergG}
Parallelen zwischen dem deutschen Bergrecht und dem unionalen CRMA sind zu erkennen

\subsection{Interimsfazit}

\section{Die Entstehung von Rohstoffstrategien}

Dennoch entwickelten die meisten westlichen Regierungen Strategien für kritische Mineralien und entschieden sich dann, sie nicht zu finanzieren. Die Hersteller sprechen von Resilienz, doch manche halten nur einen Wochenvorrat an Seltenerdmagneten auf Lager.

\subsection{Auf nationaler Ebene}
Stand xxxx haben xxxx europäische Länder eine nationale Rohstoffstrategie entwickelt

\subsubsection{Die Rohstoffstrategie der Bundesregierung}

\subsection{Auf europäischer Ebene}

Im Rahmen des Modells des EU-Verwaltungsrechts obliegt der Vollzug und die Umsetzung grundsätzlich den Mitgliedsstaaten

\section{Wechselwirkungen zwischen deutschem und europäischen Rohstoffrecht}
Im Rahmen des Modells des EU-Verwaltungsrechts obliegt der Vollzug und die Umsetzung grundsätzlich den Mitgliedsstaaten

\section{Betrachtung weiterer ausgewählter Rechtsakte}

Dies dient insbesondere dem Vergleich 

\subsection{Dodd-Frank-Act}
Bereits 2010 und damit deutlich vor Legislativinititaven der EU hatten die USA über den \textit{Dodd-Frank-Act} \footnote{Dodd-Frank Wall Street Reform and Consumer Protection Act, Art. 1502-1504 (\textit{sections})} Transparenzpflichten im Umgang mit Konfliktmineralien in Lieferketten eingeführt. Der Gesetzesakt ist nicht primär auf den Bereich des Rohstoffrechts ausgerichtet, sondern enthält eher nebenbei Regelungen zu Konfliktmnineralien, insbesondere diejenigen des Art. 1502.  Hierbei ist der Ansatz zu beachten, dass nicht der Bezug der entsprechenden Rohstoffe eingeschränkt wird, sondern ein Reputationsrisiko für Unternehmen bei Rohstoffbezug in Konfliktregionen durch die Veröffentlichungspflichten entsteht.\autocite[Rn. 415]{ruttloff_lieferkettensorgfaltspflichtengesetz_2022} Eine relevante Schnittmenge zwischen dem Dodd-Frank-Act und dem LkSG kann jedoch nicht festgestellt werden, denn Unternehmen die bereits den Regelungen zur Transparenz des Art. 1502 unterliegen, können vom LkSG höchstens auf indirektem Wege profitieren, da der Handel mit Konfliktmineralien und die damit verbundene Vermeidung der Finanzierung bewaffneter Konflikte nicht im Fokus des LkSG stehen.\autocite[Rn 423]{ruttloff_lieferkettensorgfaltspflichtengesetz_2022} Stattdessen ist der Anknüpfungspunkte in EU-Verordnung 2017/821 zu suchen, die in Erwägungsgründen explizit eine dem Dodd-Frank-Act-ähnliche Rechtsvorschrift erwähnt.\footnote{2017/821, Erwägungsgrund 9: "[...] forderte das Europäische Parlament die Union auf, mit dem US-amerikanischen Gesetz über Konfliktminerale, Artikel 1502 des Dodd-Frank-Gesetzes (...), vergleichbare Rechtsvorschriften zu erlassen.} 

\subsection{Kimberley-Prozess}
Das \textit{Kimberley Process Cetification} Schema stellt einen Prozess zur Unterbindung des Handels mit Diamanten dar, durch dessen Verkauf kriegerische Auseinandersetzungen und Konflikte finanziert werden (\glqq Blutdiamanten\grqq oder Konfliktdiamanten)


Der Prozess kann grundsätzlich als Vorbild für eine entsprechende \glqq Kopie\grqq für den Bereich der seltenen Erden bzw. kritischen/strategischen Rohstoffe dienen -- so z. B. insbesondere hinsichtlich des Konflikts in der Demokratischen Republik Kongo (DRK/DRC), was ebenfalls von der EU-Außenbeauftragten Kallas erwähnt wurde.\footnote{\glqq We need something like this for essential raw materials […] if a country is attacking another country, taking over the mines and selling these raw materials as if they were its own and can finance its war with these raw materials\grqq; European Newsroom: Blindspots, raw materials, aid visibility: Kallas on development work and the EU’s presence in Africa, 14.02.2025. %https://europeannewsroom.com/blindspots-raw-materials-aid-visibility-kallas-on-development-work-and-the-eus-presence-in-africa/} 




\section{Rohstoffe und Menschenrechte}

Rohstoffkontrollregime

\section{Rohstoffe und Umweltschutz}

Es wird u. A. kritisiert, dass Diskussionen zu (kritischen) Rohstoffen oftmals adverse ökologische und soziale Folgen in den (nicht-europäischen) Abbauländern nicht berücksichtigen.\autocite[15]{Kueblboeck_2023}

Lagerstätten schwerer Seltenerdmetalle treten häufig gemeinsam mit radioaktiven Elementen wie Uran und Thorium auf. Ihre Erschließung führt daher zwangsläufig zur Entstehung hoch radioaktiver Abfälle. Aus diesem Grund hat China die Förderung schwerer Seltenerdmetalle ab 2014 zunehmend in benachbarte Staaten, beispielsweise in das von Rebellen kontrollierte Gebiet in Myanmar, verlagert und konzentriert sich selbst vor allem auf die Weiterverarbeitung. Auf diese Weise sollen die ökologischen Belastungen der Rohstoffgewinnung ins Ausland verlagert werden. Für die Europäische Union, die nahezu ihren gesamten Bedarf an schweren Seltenerdmetallen importiert, bedeutet dies zugleich eine Auslagerung der damit verbundenen Umwelt- und Sozialprobleme.

Bei allen Abbauvorhaben muss daher beachtet werden: Es ist nicht möglich, lediglich den \glqq Nutzen\grqq in der EU anzusielden, denn sowohl Abbau als auch Verarbeitung kommen mit entsprechenden Herausforderungen an Umweltschutz. In anderen Worten: Deutschland und die EU stehen vor der strategischen Entscheidung, wie der Bedarf an kritischen Rohstoffen gedeckt werden kann. Grundsätzlich ergeben sich dabei drei Handlungsoptionen: Entweder werden Rohstoffe importiert und damit auch die damit verbundenen Abfälle und Emissionen aus deren Verarbeitung in Kauf genommen, oder es werden technologische Alternativen entwickelt, die den Einsatz dieser Rohstoffe überflüssig machen oder zumindest deren Einsatz auf ein Minimum reduzieren. Eine dritte Möglichkeit besteht im verstärkten Ausbau des Recyclings, da die betreffenden Rohstoffe bereits in vorhandenen Produkten verbaut sind und somit prinzipiell einer Rückgewinnung zugänglich gemacht werden können.

Rohstoffe und Klimaschutzrecht

%menschenrechte rohstoff verwaltung

\section{Rohstoffe in der Entwicklungspolitik}

Da der Rohstoffsektor von zentraler Bedeutung für die Entwicklungspolitik ist, erfolgt die Aktivität der EU auf den globalen Rohstoffmärkten auch im Rahmen der Entwicklungszusammenarbeit.\autocite{Schorkopf, Rn. 53}



Im Rahmen der Art. 208-211 AEUV kann die Union auf eine spezielle Rechtsgrundlage für die Entwicklungszusammenarbeit zurückgreifen.

Schon bei der Betrachtung der Karte zur geographischen Lokalisierung von Rohstoffvorkommen (/ref ) fällt auf, dass sich ein bedeutender Teil in Entwicklungs- und Schwellenländern befindet.

Der Rohstoffsektor spielt insbesondere in Schwellen- und Entwicklungsländern eine Rolle.

Sektorprogramm Rohstoffe und Entwicklung

Die Verschränkung von Rohstoffpolitik und -verwaltung mit der Entwicklungspolitik der Union wird insbesondere durch den Terminus der Rohstoffdiplomatie angesprochen. Zudem ist im Rahmen der Art. 208ff. AEUV im Rahmen der Außenpolitik der Union (siehe dazu auch den entsprechenden Abschnitt) ein entsprechendes Gefüge aus Abkommen entstanden.

Die Rohstoffstrategie von 2008 verweist auf eine dreifache Aufgabe der rohstoffbezogenenen Entwicklungspolitik: Die Stärkung lokaler Staatsführung durch Governance-Verpflichtungen, Förderung eines günstigen Investitionsklimas insbesondee vor dem Hintergrund des fairen Marktzugangs, sowie des Ausbaus eines nachhaltigen Rohstoffsektors.\footnote{KOM(2008)699 S. 7f.}

Daher wird auch gefordert, dass die Union \glqq ihrer Verpflichtung zu entwicklungspolitischer Kohärenz Genüge tun\grqq sollte um sicherzustellen, dass dementsprechend die Förderung regionaler Wertschöpfung im Rohstoffsektor der nicht-eruopäischen Abbaupartnerländer im Vordergrund stehen solle.\autocite[15]{Kueblboeck_2023}

Im Sinne der Rohstoffverwaltung erstreckt sich diese somit bereits auf den Prozess vor der Unionseinfuhr von Mineralien.

\subsection{Rohstofffluch}

Gerade vor entwicklungspolitischen Hintergründen drängt sich der mittlerweile einschlägig bekannte \glqq Rohstofffluch\grqq auf -- also die Erscheinung, dass trotz des vorhandenen Rohstoffreichtums die jeweiligen Länder nicht von dessen Exploration profitieren. Das Phänomen des Rohstofffluchs ist umfassend untersucht und dokumentiert

Die Erkenntnis




\section{Nichtbeachtung der Rohstoffthematik}
Der Mangel an spezifischen gesetzlichen Vorgaben zur Rohstoffverwaltung lässt sich durch mehrere Faktoren erklären, deren plausible Relevanz im Folgenden weiter untersucht und evaluiert werden soll. Aus rein historischer Sicht standen in Deutschland und der EU andere rechtliche und wirtschaftliche Prioritäten im Vordergrund - insbesondere in einer Zeit, in der elektrisch angetriebende Fahrzeuge und dementsprechend EV-Batterien und ihre Herkunft ein Schattendasein pflegte, und die langfristige Sicherung von Rohstoffquellen wurde oft als weniger dringlich angesehen, insbesondere in Zeiten wirtschaftlicher Prosperität oder stabiler internationaler Lieferketten – also insbesondere vor disruptiven Ereignissen wie der Covid-19-Krise oder dem russischen Angriffskrieg gegen die Ukraine und den entsprechenden Folgen.
Ferner kann vermutet werden, dass die Nichtexistenz eines spezifischen Rechtsbereiches zur Rohstoffverwaltung in sich selbst begründet werden kann, sodass das Nichtvorhandensein rechtfertigt, hier auch nichtg tätig zu werden. Wie im Bereich des Bergrechts zu sehen, ist der Rechtsrahmen zudem als eher fragmentiert im Vergelich zu anderen Rechtsbereichen einzustufen, denn auch das Rohstoffverwaltungsrecht findet Anküpfungspunkte dort, aber auch im Umwelt- und Handelsrecht oder dem internationalen Wirtschafts(völker-)recht. Diese Fragmentierung führt dazu, dass es keine einheitlichen, spezifischen gesetzlichen Vorgaben gibt, die alle Aspekte der Rohstoffverwaltung umfassend regeln. Stattdessen existieren zahlreiche Einzelregelungen, die unterschiedliche Bereiche und Aspekte der Rohstoffnutzung abdecken, ohne jedoch eine kohärente Gesamtstrategie zu bilden.
Ferner: Langfristige, mitunter politisch begründete oder geprägte Rohstoffstrategien und nachhaltige Rohstoffverwaltung stehen häufig im Widerspruch zu eher kurzfristigen bzw. schneller umsetzbaren Zielen und insbesondere auch schneller wahrnehmbaren Erfolgen, auch da jede Art von rohstoffverwaltungsrechtlichen Akten einen entsprechenden Eingriff mitsichbringt. 
Es drängt sich zudem die Vermutung auf, dass das öffentliche Bewusstsein für die Bedeutung einer Rohstoffverwaltung, insbesondere in Politik und Wirtschaft, eher gering war und im Rahmen der Zeitenwende einen entsprechenden Bedeutungsgewinn erfahren hat. Ferner kann der Bereich des Rohtsoffverwaltungsrecht durchaus als ein technisches und spezialisiertes Thema wahrgenommen werden, sodass die Bedeutungskraft hinter anderen Themen zurücktritt - Stichwort Salienz.

Die Rohstoffverwaltung ist ein globales Thema, das internationale Kooperation und Abstimmung erfordert. Unterschiedliche nationale Interessen und Prioritäten erschweren die Entwicklung einheitlicher internationaler Regelungen und Standards. Zudem sind viele rohstoffreiche Länder außerhalb Europas angesiedelt, was die Einflussmöglichkeiten der deutschen und europäischen Gesetzgeber einschränkt und die Notwendigkeit internationaler Verhandlungen und Vereinbarungen erhöht.

Ein umfassendes Verständnis dieser Hintergründe ist essentiell, um die bestehenden Regelungen zu evaluieren und mögliche Wege für die Entwicklung kohärenter und effektiver rechtlicher Rahmenbedingungen zu identifizieren. Eine weitergehende wissenschaftliche Auseinandersetzung mit diesen Themen ist daher dringend notwendig, um die Herausforderungen der globalen Rohstoffversorgung nachhaltig und rechtlich fundiert zu bewältigen.

Beim BDI-Rohstoffkongress 2018 wurde zudem auch gefordert, ein entsprechendes Gesetz zum Weltraumbergbau aufzusetzen, um sich eine Vorreiterrole zu sichern -- passiert ist jedoch auch hier nichts.

% Rohstofffluch Frau


\section{Herausforderungen für das Rohstoffverwaltungsrecht auf EU-Ebene}
%Tabelle Rohstoffinitativen EU-Mitgliedsländer
Es muss erkannt werden, dass nicht zwangsläufig alle Mitgliedsstaaten Schemata zum Umgang mit kritischen Rohstoffen in Lieferketten ausformuliert haben und so auch Unternehmen über mögliche Risiken im Falle von (zeitweilig) gestörten oder versiegten Lieferketten aufzuklären beziehunsgweise entsprechende Verantwortung für solche zu übernehmen, nciht zuletzt aufgrund unterschiedlicher Relevanz und Betroffenheit einzelner Mitgliedsstaaten und daraus resultierendem, divergierendem Risikobewusstsein und entsprechender Vorsorge.

\subsection{Verfahrensautonomie der EU-Mitgliedsstaaten}
Grundsätzlich obliegt der Vollzug einzelner Titel den Mitgliedsstaaten, mit der Einschränkung durch entsprechende Vorgaben der Union, sodass die Verfahrensautonomie der Mitgliedsstaaten gegenüber der zunehmenden Europäisierung der nationalen (Verwaltungs-)Rechtssysteme als eine \glqq ungeklärte Kardinalsfrage\grqq \autocite{Ludiwgs, NVwZ 2018, 1417} betrachtet werden kann

Es bleibt weiterhin zu untersuchen, inwieweit eine unmittelbare Anwendbarkeit der rohstoffrechtlichen und -politischen Vorschriften gegeben ist, sodass sich im Verhältnis Bürger-Staat und Unternehmen-Staat auf eine entsprechende Unionsaktivität berufen werden könnte. Diese Prüfung erfolgt unberührt der Klage (auch durch natürliche oder juristische Person) auf Feststellung einer Vertragsverletzung wegen Untätigkeit nach Art. 286 AEUV.

\section{Die Politisierung des Rohstoffrechts}

Es ist mithin bekannt, dass ein Großteil der globalen mineralischen Rohstoffvorräte und insbesondere solche der seltenen Erden in diversen Staaten, darunter zahlreichen Entwicklungsländern, lagern, 

Somit kam es hier insbesondere in den letzten Jahre, ausgehend von ?, zu einer zunehmenden Politisierung im Bereich der Rohstoffe, die sich unweigerlich auch auf das Rohstoffrecht ausdehnt. Gründe hierfür sind einerseits die Verbindung des Rechtsbereiches mit internationalem Recht, welches zwangsläufig einer Politisierung unterliegt, andererseits aber auch die Erkenntnis der Politik, dass ein Handeln erforderlich sei, wie in den bereits ausführlich dargestellten Rohstoffstrategien deutlich wird.

Blickt man in die primärrechtlichen Vorgaben der Union, insbesondere im Rahmen der Handelskompetenz, so fällt auf dass die Politisierung dieses und anderer Rechtsbereiche durchaus vorgesehen ist: So wird die Union gem. Art. 205 AEUV bei ihrem Handeln auf internationaler Ebene von den Zielen und Bestimmungen aus Titel V Kap. 1 EUV geleitet und richtet es zudem daran aus. In Art. 21 EUV wird ein ähnliches Leitmotiv aufgegriffen, insbesondere unter Abs. II -- die Politisierung lässt sich hier problemlos aus den Zielen herauslesen.
%HIER BENÖTIGT ES NOCH ENTSPRECHENDER LITERATUR ZU 205 AEUV, 21 EUV DIE DAS BESTÄTIGEN
%DAUSE/LUDWIGS O. CALLIES HABEN BESTIMMT WAS


% auf räumliche Verteilung der Rostoffe eingehen

Es wird ersichtlich, dass die Verteilung von seltenen Erden höchst ungleich einzustufen ist, im Kontrast zu anderen mineralischen Rohstoffen. Deutlich wird auch, dass insbesondere als Entwicklungsländer bzw. \glqq least developed countries\grqq (LDC)\footnote{Eine allgemeine Definition für ein solches Entwicklungsland existiert nicht. Für den Gebrauch in dieser Arbeit wird daher die englische Abkürzung LDC im Sinne der least bzw. less developed countries genutzt mit dem Umfang der \glqq DAC-Liste der Entwicklungsländer und -gebiete\grqq des Bundesministeriums für wirtschaftliche Zusammenarbeit und Entwicklung.} einzustufende Staaten über solche Rohstoffvorkommen verfügen, viele Industrieländer\footnote{Bzw. \glqq developed countries\grqq folgend der Definition des International Monetary Fund (IMF)} hingegen nicht. 

Alle vier Risiken, die Kommission und EWSA schon 2008 identifizierten, haben sich mittlerweile verstärkt und entsprechende Politisierung erfahren:
\begin{itemize}
	\item Wettrennen um die Rohstoffverarbeitung
	\item Agglomerieren von Rohstoffen, auch durch Handelsmaßnahmen
	\item Wettstreit um den Zugang zu Lagerstätten und Infrastruktur, sowie
	\item Unterbrechung von Mineralien-Lieferketten
\end{itemize}
.\footnote{ABl. C277/93, 2009}




\section{Rohstoffe und internationales Wirtschaftsrecht}
Die territoriale Gebundenheit von Rohstoffen als Erzeugnisse der Urproduktion und ihre natürliche Knappheit machen ihre Beschaffung zu einer geopolitischen Herausforderung. Da gewaltsame Aneignung heute völkerrechtlich nicht mehr legitimierbar ist, bleibt der internationale Handel als zentraler Zugangsweg. Rohstoffe werden so zu einem Schlüsselfaktor internationaler Wirtschaftsbeziehungen und unterliegen deren rechtlicher Steuerung.\autocite{Terhechte, Rohstoffverwaltung, Rn. 5}

Bindungen mit Rohstoffbezug im Rahmen des internationalen Wirtschaftsrechts ergeben sich für die Union und die einzelnen Mitgliedsstaaten vorrangig aus dem GATT. Die Handels- und Zollpolitik der Union unterliegt den Vorgaben des GATT 1994 sowie entsprechender ergänzender Abkommen.\footnote{Dies bedeutet dann die Bindung an die klassischen Prinzipien der Meistbegünstigung sowie das Verbot willkürlicher Beschränkungen, sodass Maßnahmen, die den Zugang zu Rohstoffen aus Drittstaaten diskriminieren oder behindern, WTO-rechtlich zu Problemem führen können}. In Bezug auf Exportländer wie China ist die Rohstofffrage bereits Gegenstand mehrerer WTO-Verfahren gewesen (s.u.).

Die Union hat in ihren Rohstoffinitiativen und -strategien deutlich erwähnt, dass ein diskriminierungsfreier und gesicherter Zugang zu ausländischen Rohstoffvorkommen und generell den Rohstoffmärkten zu den vorrangigen Zielen zählt, da die Union aufgrund mangelnder Heimvorkommen auf Importe angewiesen ist. Diese lässt sich, wie dargestellt, auch nicht durch europäische Vorkommen vollends ersetzen. Gefahren durch Abhängigkeiten ergeben sich durch die ungleiche Verteilung der Verfügbarkeiten auf wenige Marktteilnehmer.

\subsection{Globale Rohstoffmärkte}

Bereits durch die Zollunion ist die Union befugt, gegenüber Drittländern mit einer Stimme aufzutreten. Rohstoffe, die in einen Mitgliedstaat eingeführt werden, gelten nach Art. 29 AEUV als Unionswaren und unterliegen damit keinem weiteren Zollregime innerhalb der EU. Die Zollpolitik gegenüber Drittländern wiederum fällt unter die Gemeinsame Handelspolitik gem. Art. 206 und 207 AEUV.\footnote{Konkretisierunggen dieser u. A. durch den EuGH: Commission v Council (Titanium Dioxide) (Rs. 45/86) und mit vergleichsweise aktuellem Bezug Singapore Free Trade Agreement (Gutachten 2/15, EU:C:2017:376)}

Die Stellung der Union auf internationalen Rohstoffmärkten wird aufgrund ihres geschlossenen Auftretens als Zollunion (Art. 28 AEUV) und die ausschließliche Zuständigkeit im Außenhandel (Art. 3 I e AEUV) mitunter als \glqq stark\grqq charakterisiert.\autocite{Schorkopf, Rohstoffverwaltung, Rn. 34.} Die ausschließliche Zuständigkeit bewirkt also eine instituinelle Stärke, zumindest bei rein struktureller Betrachtung. Die EU ist daher in der Lage, durch unionsweit einheitliche Handelsabkommen, etwa mit rohstoffreichen Drittstaaten wie Chile (z. B. modernisiertes EU-Chile-Assoziierungsabkommen, 2023), strategische Rohstoffinteressen kollektiv zu verfolgen. Auch das neue Abkommen mit Namibia (Memorandum of Understanding 2022) zum Aufbau einer grünen Wasserstoff- und Rohstoffpartnerschaft fällt unter diesen Mechanismus.

Die Union hat ihre handelspolitische Kompetenz in den letzten Jahren zunehmend für rohstoffpolitische Ziele strategisch instrumentalisiert. Maßgeblich ist hier die Rohstoffinitiative (KOM(2008) 699) sowie das Konzept der „Strategischen Partnerschaften“ im Kontext des Critical Raw Materials Act (CRMA), KOM(2023) 160 final. Die Union verfolgt explizit das Ziel, durch internationale Abkommen den Zugang zu kritischen Rohstoffen zu sichern -- etwa durch Konditionalität (Zugeständnisse beim Marktzugang gegen Rohstoffexportzusagen) oder sektorale Rohstoffdialoge (so mit Kanada, Argentinien, DRK).

Trotz der extensiven primärrechtlichen Grundlage ist die Stärke der Union nicht vollumfänglich; insbesondere durch die fehlende Binnenkonsolidierung der Rohstoffpolitik durch wenig harmoniserte Rohstoffgesetzgebung und eine gewisse reaktive Haltung der Union, strategische Initativen erst bei Auswirkungen externer Krise zu ergreifen, schmälert den Erfolg der Union auf den Rohstoffmärkten






\subsection{Zollrechtliche Betrachtung}
Aufgrund des Binnenmarktes und der daraus folgenden Zollunion (Art. 28 I AEUV) entfällt die innereuropäische Verwaltung von zollrechtlichen Rohstoffaspekten -- Einfuhren werden zu \glqq Unionswaren\grqq.\footnote{vgl. Art. 5 Nr. 23 UZK; Verordnung (EU) Nr. 952/2013} Sobald also kritische oder strategische Rohstoffe (z. B. Lithium, Nickel, Seltenerdmetalle) rechtmäßig in den freien Verkehr der EU überführt wurden, verlieren zollrechtliche Fragen für den innereuropäischen Warenverkehr ihre Relevanz. Der gesamte Automobil-Cluster – einschließlich seiner vorgelagerten Veredelungsstufen – profitiert dadurch von einem barrierefreien Zugang zu den benötigten Vormaterialien im Binnenmarkt.

Daraus ergibt sich aber eine gemeinsame Zollregelung gegenüber Drittstaaten. Wie bereits erkannt sind die Union und ihre Unternehmen hochgradig von Importen abhängig. Solche Rohstoffeinfuhren unterliegen hierbei dem Gemeinsamen Zolltarif (GZT) der Union\footnote{Art. 56 UZK i.V.m. Art. 31 AEUV} mit ursprungsbezogenen Regelungen.\footnote{Dies umfasst klassische Zölle, aber auch nichttarifäre Maßnahmen wie Umweltschutzauflagen oder Vorgaben zur Produktsicherheit.}

Die Union kann grundsätzlich (beispielsweise zur Förderung der Industrie) gezielt Zollaussetzungen oder autonome Zollkontingente beschließen (Art. 31 AEUV). Diese können für bestimmte kritische Rohstoffe temporäre Abgabenfreiheit gewähren -- zentrales Kriterium der Verhältnismäßigkeit ist die Verwirklichung der grundlegenden Ziele der Verbesserung der Wettbewerbsfähigkeit der Wirtschaft der Union.\footnote{vgl. Verordnung (EU) 2021/2278 des Rates, Ew. 6.}

Diese Maßnahmen könnte einer de-facto Rohstoffpolitik mit zollrechtlichen Mitteln, die auf Antrag über die Mitgliedstaaten oder von der Kommission selbst erfolgen können.\footnote{Siehe auch Verordnung (EU) 2023/1190 des Rates vom 19. Juni 2023}

Zollaussetzungen und autonome Zollkontingente sind kein Bestandteil des kodifizierten Zollrechts, sondern beruhen auf einer praxisbasierten Ausgestaltung zollpolitischer Industrieunterstützung gem. Art. 31 AEUV. Sie entfalten faktisch zentrale Steuerungswirkung im Rahmen der Rohstoffpolitik der EU, insbesondere bei kritischen Rohstoffen für industrielle Anwendungen.

%prüfen ob solche aussetzungen schon existieren

Im Rahmen von handelsrechtlichen Präferenzabkommen kann die Einfuhr strategisch wichtiger Rohstoffe zollbegünstigt oder zollfrei erfolgen, sofern die Ursprungsregeln eingehalten werden. Diese Verträge sind insbesondere für die Diversifizierungsstrategie der EU im Sinne des Derisking relevant, wie sie u. a. im Critical Raw Materials Act (CRMA) und im Aktionsplan für kritische Rohstoffe (COM(2020) 474 final) angelegt ist. Zollrechtlich sind diese Erleichterungen an die Einhaltung spezifischer Ursprungsregeln (vgl. Art. 60 UZK) sowie Nachweise wie Lieferantenerklärungen oder Ursprungszeugnisse geknüpft. Für die Automobilindustrie bedeutet dies: Lieferketten müssen nachweislich \glqq präferenzkonform\grqq gestaltet sein, was mit erheblichem administrativem Aufwand einhergeht.

Innerhalb des sog. \textit{Harmonisierten Systems}, bei dem Waren zur Einstufung numerische Codes zugeordnet werden, betrachtet mineralische Rohstoffe zudem nach ihrem Verwendungs- bzw. Weiterverarbeitungszweck. Die Union hat hierauf aufbauend die \textit{Kombinierte Nomenklatur} entwickelt mit weiteren Unterteilungen, insbesondere zur Gewährleistung des GZT. 



Mineralien werden hier als \glqq Teile mit allgemeiner Verwendungsmöglichkeit\grqq eingestuft, mit der Unterscheidung zwischen \glqq unedlen Metallen\grqq\footnote{Zu denen die meisten kritischen Rohstoffe zählen (Wolfram, Molybdän, Tantal, Magnesium, Cobalt, Bismut, Cadmium, Titan, Zirconium, Antimon, Mangan, Beryllium, Chrom, Germanium, Vanadium, Gallium, Hafnium, Indium, Niob (Columbium), Rhenium und Thallium; vgl. Abschnitt XV, 2., 3., Durchführungsverordnung 2024/2522 der Kommission)}

Innerhalb des HS-Systems erfolgt die Unterteilung beispielsweise nach Reinheit und Bearbeitungsgrad (z. B. Kapitel 26 ggü. Kapitel 71). ür das Rohstoffrecht und die Rohstoffverwaltung ergibt sich hieraus ein zentraler Handlungsbedarf: Einerseits sind präzise Kenntnisse der HS-Codierung notwendig, um Handelsflüsse strategischer Rohstoffe rechtssicher zu überwachen, exportkontrollrechtlich zu erfassen oder präferenzielle Handelsabkommen gezielt zu nutzen. Andererseits können Lücken oder unzureichende Differenzierung in der Codierung die Umsetzung rohstoffpolitischer Zielsetzungen – wie etwa das EU-Derisking durch Diversifizierung kritischer Rohstoffquellen – behindern. Entsprechend wächst das Interesse an einer Harmonisierung rohstoffpolitischer Klassifikationen mit dem internationalen Zollnomenklatursystem, etwa durch spezifische Subpositionen für besonders relevante Rohstoffe wie Seltene Erden oder Lithiumverbindungen.

Diese verstreute Erfassung erschwert die rechtssichere Identifikation und Nachverfolgbarkeit kritischer Rohstoffe im Außenhandel. Im Kontext der EU-Rohstoffstrategie und der Verordnung über kritische Rohstoffe (Critical Raw Materials Act – CRMA) gewinnt eine präzise Zuordnung an Bedeutung, da auf ihr Exportkontrollmaßnahmen, Investitionsprüfungen, Herkunftsnachweise und die Inanspruchnahme präferenzieller Ursprungsregeln aufbauen. Juristisch relevant ist dabei nicht nur die technische Klassifikation, sondern auch ihre Auswirkung auf handelsrechtliche Schutzinstrumente und WTO-Verpflichtungen. So sind tarifäre Maßnahmen, Exportbeschränkungen oder Subventionen an die spezifische Codierung geknüpft, wobei unklare oder zu breit gefasste Warennummern potenziell gegen das Diskriminierungsverbot nach Art. I und III GATT verstoßen oder zu nicht-notifizierten Handelshemmnissen i.S.v. Art. XI GATT führen können. Auch handelsrechtlich relevante Differenzierungen, wie zwischen Rohform und verarbeiteten Produkten, basieren auf HS-Codes und entscheiden über Marktzugänge im Rahmen von Freihandelsabkommen. Für die Rohstoffverwaltung ergibt sich daraus ein Spannungsverhältnis: Einerseits bedarf es einer internationalen Standardisierung zur Unterstützung strategischer Autonomie und Rohstoffsicherheit, etwa durch gezielte Unterpositionen für kritische Rohstoffe wie Lithium oder Gallium. Andererseits muss diese Standardisierung WTO-konform ausgestaltet werden, um protektionistische Tendenzen zu vermeiden und die multilaterale Handelssystematik nicht zu unterminieren. Die internationale Zollnomenklatur steht damit zunehmend im Fokus geopolitisch motivierter Rohstoffstrategien, was neue Anforderungen an ihre rechtliche Kohärenz und strategische Steuerungsfähigkeit stellt.

Demgegenüber erlaubt die Ausdifferenzierung der Klassifikation eine genaue Zuordung der Rohstoffeigenschaften, so z. B. hinsichtlich der beabsichtigten Weiterverarbeitung.\autocite{Schorkopf, Rohstoffverwaltung, Rn. 8.}

Die teilweise unsystematische Verteilung strategisch bedeutender Rohstoffe über verschiedene Warengruppen erschwert eine transparente Erfassung und gezielte Steuerung

\subsection{EU-rechtliche Ausnahmevorschriften}

Das Primärrecht hält Ausnahmevorschriften vor, die eine Einschränkung des freien Warenverkehrs von Rohstoffen über Art. 36 AEUV, Cassis-de-Dijon, Art. 347 AEUV und Art. 122 AEUV ermöglichen.\autocite{Schorkopf, Rohstoffverwaltung, Rn. 10}


\subsection{Exportrestriktionen und Exportkontrolle}



Exportkontrollen sind im Bereich der militärischen bzw. sog. \glqq Dual-Use-Güter\grqq hinreichend geläufig -- jedoch kommen hierzu mittlerweile auch solche Exportkontrollen, die zur Erreichung politischer Ziele dienen und in direktem Bezug zur Sicherstellung von technologischer und wirtschaftlicher Sicherheit stehen.\footnote{Text} Diese Maßnahmen sind nicht als kurzfristige Reaktionen zu verstehen, sondern fügen sich in das Bild eines größeren Wandels in der Politik, vor Allem in China und den USA, sodass Exportbeschränkungen als Instrument innerhalb wirtchaftlichen Abhängigkeiten genutzt werden.

Exportkontrollen stellen einen wesentlichen Teil der außenpolitischen Strategien China sund den USA dar.

Auch Deutschland bzw. die EU könnten hierbei unter Druck geraten, der Politik der USA zu folgen -- aber auch aus eigenem strategischen Interesse wird eine Mitgestaltung der Kontrollen auch unter Einbeziehung der europäischen Partner gefordert.\autocite{Medunic, FiliP. Deutschland muss Exportkontrollen strategischer gestalten, DGAp Memo Nr. 15, Juli 2024, S.1.}


\subsubsection{China}

Deng Xiaoping, \textit{Überragender Führer} Chinas von 1978 bis 1989, wird das 1987 ausgesprochene Zitat \glqq Der Nahe Osten hat Öl. CHina hat seltende Erden\grqq zugeschrieben.\footnote{Eine Primärquelle ist hier nicht auszumachen; andere Quellen sprechen darüber hinaus von 1992.}

China besitzt neben Vorkommen kritischer und strategischer Rohstoffe durch die lokale Integration der Vorleistungswertschöpfungskette, also der Verarbeitung von Rohstoffen, entsprechende strategisches Potential zur Beeinflussung der globalen Versorgungslage.

Chinesische Kontrollen erfolgten oftmals als Antwort bzw. \glqq Vergeltung\grqq auf Maßnahmen der USA: So veröffentlichte China im Oktober 2020 das \glqq PRC China Export Control Law\grqq in Reaktion auf die Entscheidung im August 2020 durch das US BIS, Huawei auf die sog. Entity-List zu setzen\footnote{BIS, Entitiy List}. Auch im weiteren Verlauf erfolgten chinesische Exportkontrollen auf mineralische Rohstoffe vergleichsweise schnell innerhalb weniger Tage in Reaktion auf US-Maßnahmen zu Halbleitern (Oktober 2023, Dezember 2024) und auf angekündigte Zollerhöhungen.\autocite{International Institute for Strategic Studies, Export controls: China and the United States’ use of export controls, 2010–25.}

2010 reduzierte China die zulässigen Exportquoten für seltene Erden um 40\%.

Die USA, die EU und Japan strebten daraufhin ein DSU-Verfahren\footnote{DS431}  gegen China bei der WTO an.\autocite{EuZW 2012, 286}

Das Panel urteilte, dass sich China mit dem Eintritt in die WTO der Verpflichtung unterwarf, Exportbeschränkungen jeglicher Art zu eliminieren, mit Ausnahme einiger Exportzölle für bestimmte Produkte, zu welchen die mineralischen Rohstoffe nicht zählen. China hingegen verwies auf die Beschränkungen aus Gründen des Umweltschutzes, welche jedoch abgewiesen wurden -- zwar kann ein WTO-Mitgliedsland über Intensität des Abbaus entscheiden, aber die handelsrechtlichen Regeln der WTO werden sofort nach Abbau gültih.\autocites{EuZW 2014, 684}{EuZW 2014, 283}

Ein ähnliches Verfahren gegen China mit im Ergebnis gleicher Urteilsbegründung war bereits von 2009 bis 2011 geführt worden.\footnote{DS395}

Ein weiteres Beispiel sind von Indonesien eingeführte Exportverbote für Nickelerz sowohl 2014 als auch 2020. Das Land gilt als weltweit führender Nickelförderer.\autocite[52]{DERA Rohstoffinformation Indonesien, } Nickel zählt hierbei zu den strategischen und kritischen Rohstoffen. Ziel war es in beiden Fällen, durch das Exportverbot eine lokale Wertschöpfung zu erzielen -- durch das Exportverbot stiegen jedoch auch erwartungsgemäß die Marktpreise und insbesondere China forcierte lokale Verarbeitung in Indonesien.\autocite{DERA} Chinas Beschränkungen für Seltene Erden stützen sich auf das Exportkontrollgesetz von 2020, das den Abfluss von Materialien mit doppeltem Verwendungszweck aus Gründen der nationalen Sicherheit einschränkt. Artikel 5 des Gesetzes übertrug sowohl dem Staatsrat als auch der Zentralen Militärkommission Durchsetzungsbefugnisse. Artikel 48 erlaubt es, das Gesetz zu nutzen, um auf Einschränkungen zu reagieren, die von anderen auferlegt werden. Die erste offene Anwendung für kritische Mineralien erfolgte im Juli 2023. Unter Berufung auf Erwägungen der nationalen Sicherheit verhängte das chinesische Handelsministerium Exportkontrollen für Gallium und Germanium. Später im Jahr 2023 schränkte das chinesische Handelsministerium Graphitprodukte ein, insbesondere solche, die in Lithium-Ionen-Batterieanoden verwendet werden. 

Die Europäische Kommission beantrage daher 2019 ergebnislose Konsultationen über die WTO mit Indonesien, sodass in einem DSU-Verfahren 2022 durch die WTO eine Verletzung des Art. XI:1 GATT festgestellt wurde\footnote{DS592} und die Beschränkung für ungültig erklärt.\footnote{Pressemitteilung der Kommission: WTO-Panelentscheidung gegen indonesische Ausfuhrbeschränkungen für Rohstoffe, 30. November 2022, IP/22/7314}

Grundsätzlich weisen die Verfahren der EU daher hohe Relevanz für die Sicherung der Versorgung der Union auf.

Bekanntermaßen ist das Streitschlichtungsverfahren der WTO jedoch seit 2019 praktisch wirkungslos bzw. blockiert,\footnote{Altemöller EuZW 2022, 207} da das WTO-Berufungsgremium nicht entsprechend besetzt wurde. Daher erweiterte die EU 2021 mit der Aktualisierung\footnote{Verordnung (EU) 2021/167} ihrer Verordnung zur Anwendung und Durchsetzung internationaler Handelsregeln\footnote{Verordnung (EU) Nr. 654/2014}, sodass die EU unabhängig von einer möglichen Berufung nach einem gewonnenen Verfahren vor dem WTO-Panel einseitig Vergeltungsmaßnahmen aussprechen kann. Auch kann sich vor ähnlichen Situationen im Rahmen anderer Abkommen abgesichert werden, sollte ein Drittstaat Funktionsmechanismen ausnutzen.\footnote{s. hierzu Ew. 4, VO 2021/167} Dies erscheint auch vor dem Hintergrund neuerlicher Verfahren angebracht.\footnote{Seit dem 30. Mai 2023 untersucht ein Panel der WTO eine Beschwerde Indonesiens zu Antidumpingmaßnahmen der EU bei Stahlimporten aus Indonesien (DSD616) in Reaktion auf das Verfahren 592.}
Weiterhin besteht die Problematik aufgrund des funktionslosen DSU-Mechanismus, dass Panel-Berichte nach einer Anfechtung \textit{into the void} verschoben werden, d. h. in die Leere des Verfahrens, da kein funktionierender Appelate Body besteht\autocite{Therrien-Tremblay, A.: Settling Disputes at the World Trade Organization: Alternatives to Appealing “Into the Void”, 2024} -- daher laufen auch alle zukünftigen Entscheidungen Gefahr, in diese Leere zu laufen.
Die Kommission unterstützte im Rahmen der OSA eine entsprechende Reform der WTO.\footnote{COM(2021) 66, S. 13.}

Auch wenn grundsätzlich die Ausfuhren in Drittländer keinen mengenmäßigen Beschränkungen unterliegen,\footnote{siehe Art. 1 Verordnung (EU) 2015/479} erlaubt die Union erlaubt explizit Beschränkungen der Ausfuhr, so auch von Rohstoffen. Dies umfasst Schutzmaßnahmen \glqq aufgrund einer außergewöhnlichen Entwicklung des Marktes\grqq, d. h. entweder die Prävention eine durch einen Mangels an \glqq lebenswichtigen Gütern\grqq Krise oder aber die Sicherstellung internationaler Verpflichtungen der Union bzw. den Mitgliedsstaaten.\footnote{Art. 6. (EU) 2015/479}

Art. 6 Abs. 1 lit. a erlaubt der Kommission, im Fall oder zur Prävention einer versorgungskritischen Lage geeignete Maßnahmen zu ergreifen, etwa durch Exportbeschränkungen, Genehmigungspflichten oder Zuteilungssysteme. Die zentrale Voraussetzung ist ein „Mangel an lebenswichtigen Gütern“, wobei dieser Begriff – in der Auslegungspraxis der Kommission – nicht allein auf Nahrungs- oder Medizinprodukte beschränkt ist. Auch strategisch systemrelevante Industriegüter können in die Reichweite des Art. 6 rücken, wenn ihre Knappheit systemische Folgen für Wirtschaft oder Gesellschaft auslöst --  Durchführungsverordnung (EU) 2020/402 zur Einführung von Ausfuhrgenehmigungen für persönliche Schutzausrüstungen, gestützt auf eben jene Verordnung 2015/479. Für kritische Rohstoffe – wie etwa Lithium, Seltene Erden oder Magnesium – gilt: Kommt es zu einem signifikanten Angebotsrückgang oder droht ein systemischer Produktionsausfall in strategischen Industrien (z. B. Automobil- oder Halbleiterproduktion), wäre die Anwendung von Art. 6 denkbar und rechtlich gedeckt. Dies würde ein faktisches Exportmoratorium ermöglichen, um die Binnenmarktversorgung zu priorisieren. Art. 6 Abs. 1 lit. b verweist auf die internationale Verpflichtungstreue der Union, etwa im Rahmen von WTO-Recht, bilateralen Handelsabkommen oder Multilateralen Rohstoffinitiativen (z. B. die Extractive Industries Transparency Initiative – EITI).

Die Klausel ist relevant, wenn Exportmaßnahmen erforderlich sind, um solchen Verpflichtungen nachzukommen. Denkbar ist dies etwa, wenn eine durch die Union eingegangene Vereinbarung zur nachhaltigen Nutzung von Rohstoffen, zur Kontrolle von Dual-Use-Gütern oder zur Transparenz in Lieferketten nur durch kontrollierte Ausfuhren umsetzbar ist. Im Rohstoffkontext betrifft dies etwa die EU-Verordnung 2017/821 (Konfliktmineralien) oder künftige Verpflichtungen aus Rohstoffpartnerschaften (z. B. mit Chile, Namibia). 

Die Vorschrift aus Art. 6 ist für die EU-Rohstoffpolitik latent strategisch bedeutsam, auch wenn sie bislang nur selten rohstoffbezogen angewendet wurde. Ihre Relevanz steigt jedoch im Kontext wachsender geopolitischer Spannungen und begrenzter globaler Rohstoffverfügbarkeit. Denn sie erlaubt es der Union, bei Versorgungsengpässen oder politischem Handlungsdruck kurzfristig und autonom zu reagieren – ohne neues Primärrecht schaffen zu müssen.

Im Licht des Critical Raw Materials Act (COM(2023) 160 final), der eine stärkere Binnenmarktsteuerung von Rohstoffen anstrebt, könnte Art. 6 zukünftig als Flankierungsinstrument genutzt werden – z. B. um zu verhindern, dass strategische Rohstoffe, die im Rahmen eines strategischen Projekts gewonnen oder recycelt wurden, ungehindert exportiert werden, obwohl sie für den europäischen Binnenmarkt bestimmt sind.

Im Juni 2024 unterzeichnete Premierminister Li Qiang die Vorschriften zum Management von Seltenen Erden, in denen er das Staatseigentum an Seltenerdressourcen bekräftigte und darauf bestand, dass diese "gemäß den Beschlüssen der Partei" verwaltet werden. China kündigte daraufhin Exportkontrollen für Antimon und verwandte Produkte an und berief sich dabei erneut auf die nationale Sicherheit. Ende 2024 verhängte China umfassendere Beschränkungen für Wolfram-, Graphit-, Magnesium- und Aluminiumlegierungen, die für Halbleiter-, Batterie- und Verteidigungsanwendungen von entscheidender Bedeutung sind. m Gegensatz zu den vorherigen Beschränkungen, die ein Ausfuhrgenehmigungssystem vorsahen, kam dies einem vollständigen Verbot gleich, das ausdrücklich auf Ausfuhren in die Vereinigten Staaten abzielte. Der Schritt wurde weithin als Vergeltung gegen die US-Sanktionen im Zusammenhang mit Halbleitern interpretiert.

Im April 2025 erweiterte China die Liste der kritischen Rohstoffe, die von Exportkontrollen betroffen sind.\autocite{https://www.reuters.com/world/china/chinas-export-controls-are-curbing-critical-mineral-shipments-world-2025-04-20/} Dies folgte auf Beschränkungen für Gallium und Germanium (2023) sowie Antimon (2024) und war eine Reaktion auf die Ankündingung der USA, Zölle in Höhe von 125\% auf chinesische Waren zu erheben. Die Beschränkung galt weltweit, sodass nicht nur US-Importeure betroffen waren, und für die Rohstoffe und Rohstoffmischungen Dysprosium, Gadolinium, Lutetium, Samarium, Scandium und Yttrium.



Unternehmen, die die betroffenen Mineralien exportieren, sind verpflichtet eine entsprechende Exportlizenz zu beantragen.\autocite[Dessen Beantragung veranschlagt sechs Wochen;]{Wolf. Edda: China führt Exportauflagen für kritische Metalle ein, GTAI} Die Lizenzen galten für alle Exporte, nicht nur für US-Verkäufe. Die Lizenzausgabe war limitiert, was zum Teil auf die Verwirrung über die Tatsache zurückzuführen war, dass in den Codes des chinesischen Harmonisierten Systems nicht zwischen den Kategorien von Seltenerdmagneten unterschieden wird und das chinesische Handelsministerium MOFCOM zunächst keine Lizenzvergabeverfahren entwickelt hatte. Dies führte insbesondere in den ersten Tagen zu einem faktischen Erliegen des Handels mit Seltenen Erden

Kritisch wurde zudem beurteilt, dass die Antragsteller mitunter vertrauliche Informationen zu den benötigten Rohstoffen mitteilen mussten, auch in höherem Detailgrad.

Im Rahmen eines Treffens am 7. Juni mit EU-Handelskommissar Šefčovič erklärte das MOFCOM, dass künftig Maßnahmen ergriffen würden, um die Bearbeitung der Anträge von EU-Unternehmen zu beschleunigen. Zu diesem Zweck soll ein sogenannter „grüner Kanal“ eingerichtet werden, der den Zugang zum chinesischen Markt erleichtert. Im Gegenzug wurde jedoch auch Chinas Erwartung geäußert, dass die EU den Export von Hochtechnologieprodukten nach China unter Einhaltung geltender Rechtsvorschriften ermögliche; diese Aussage bezieht sich vermutlich auf die restriktiven Exportregelungen der Niederlande, die unter dem Einfluss der Vereinigten Staaten eingeführt wurden und den Verkauf von Anlagen zur Halbleiterproduktion an China untersagen (ASML)



Der Fall zeigt, wie Exportkontrollen insbesondere auch den Automobilsektor beeinflussen. Zwar führten die  Maßnahmen insbesondere in den USA zu weitreichenden Folgen, jedoch auch im europäischen Raum. Die US-amerikanische Alliance for Automotive Innovation als auch die Vehicle Suppliers Association wandten sich an das Weiße Haus, und Ford stellte zeitweise die Produktion ein. Auch ACEA bestätigte Produktionspausen einiger Hersteller. China konnte somit die Auswirkungen der US-Zölle zumindest verkraften, nicht aber die USA die Wirkung der Exportkontrollen auf den Automobilsektor.



Wenn China die Exporte wichtiger Mineralien weiter drosselt, könnten höhere Preise Anreize für Bergbau- und Verarbeitungsausgaben in anderen Ländern schaffen.

Als Reaktion auf die seit 2018 höheren US-Zölle auf Importe aus China und insbesondere die seit 2025 zunehmenden bilateralen Spannungen in anderen Bereichen hat China im 2024 Mineralien ins Visier genommen, die für die US-Verteidigung von entscheidender Bedeutung sind, darunter Gallium und Germanium. Im April 2025, nachdem die USA Zölle auf chinesische Waren angehoben hatten, kündigte China an, dass Exporteure von insgesamt zwölf weiteren Rohstoffen Lizenzen für den Export benötigen würden; später einigten sich USA und China auf eine Lockerung der Exportkontrollen.

Nichtsdestotrotz: je länger die Angebotsverschärfung anhält, desto mehr wird Chinas Status als zuverlässiger Lieferant beschäftigt, und desto attraktiver und notwendiger ist die Suche nach alternativen Bezugsquellen -- obgleich solche Reaktionen eher auf Jahre als auf Monate ausgelegt sind. Die Exportkontrollen auf seltene Erden läuteten daher eine \glqq neue Ära chinesischer Wirtschaftsstaatskunst\grqq und sei ein Beleg für eine Sanktionspolitik, die auch größere Volkswirtschaften unter Druck setzt.\autocite{Miller, C.: China’s weaponisation of rare earths is a new kind of trade war., Financial Times} Chinas vorherige Exportkontrollen seien eher als politische Signale denn als ökonomisch substantiell zu verstehen, jedoch gelte dies nicht mehr für die Maßnahmen 2025, die innerhalb weniger Wochen zu Produktionsausfällen in der gesamten Automobilindustrie führten.\autocite{Miller, C.: China’s weaponisation of rare earths is a new kind of trade war., Financial Times} Zudem sei deutlich geworden, wie unvorbereitet westliche Regierungen und Unternehmen auf die Instrumentalisierung der seltenen Erden waren -- in anderen Worten: \glqq Even those who cannot name a single rare earth element know that China dominates their production. Nevertheless, over the decade and a half since China first cut rare earth exports to Japan in 2011, the west has failed to find new suppliers. [...] This is a weapon they have been staring at for decades. They should not have been surprised when Beijing finally pulled the trigger.\grqq \autocite{Miller, C.: China’s weaponisation of rare earths is a new kind of trade war., Financial Times}

China hat somit seine Exportkontrollpolitik im Laufe der Jahre verschärft, und die Beschränkungen auf seltende Erden fügen sich als ein \glqq letzter Schritt\grqq in dieses Bild -- und obwohl China somit mittlerweile ein umfassendes Gefüge an Instrumenten aufgebaut habe, wurden diese lange nicht genutzt, jedoch zeige die zunehmende Einführung von Maßnahmen dass China anderen Märkten durchaus schaden kann und Druckmittel nutzt, sollten seine Interessen berührt sein.\autocite{https://beschaffung-aktuell.industrie.de/einkauf/seltene-erden-total-abhaengig-europas-suche-nach-alternativen/}

Die Exportrestriktionen Chinas stellen somit eine faktische Bedrohung für die Versorgungssicherheit europäischer Schlüsselindustrien dar und bilden damit eine zentrale Legitimationsgrundlage für die Kodifizierung eines europäischen Rohstoffverwaltungsrechts. Wie deutlich wurde, ist besonders der Automobilsektor betroffen: Permanentmagete in Elektromotoren bestehen nahezu ausschließlich aus chinesischen Materialien, Produktionslinien mehrerer Hersteller mussten kruzzeitig stillgelegt werden und waren nur eine Blaupause für vollumfängliche Produktionsausfälle die drohen, sollten Ausfuhren gänzlich eingestellt werden.
Diese Entwicklung offenbart die strukturelle Verwundbarkeit europäischer Wertschöpfungsketten und unterstreicht die Notwendigkeit eines rechtlich abgesicherten, strategisch gesteuerten Rohstoffregimes. Die Abhängigkeit von einem einzigen Drittstaat -- der bei bestimmten Rohstoffen bis zu 90 \% der globalen Raffinierungskapazitäten kontrolliert -- widerspricht den Grundprinzipien der europäischen Wirtschaftssicherheitsstrategie und gefährdet die technologische Souveränität der Union. Die rechtliche Reaktion in Form eines kodifizierten Rohstoffverwaltungsrechts ist daher nicht nur politisch opportun, sondern ordnungspolitisch geboten. Gegenargumente, die auf die Effizienz globaler Märkte oder die Gefahr protektionistischer Tendenzen verweisen, greifen zu kurz. Denn die chinesischen Maßnahmen sind kein Ausdruck marktüblicher Preisbildung, sondern gezielte geopolitische Instrumentalisierung von Ressourcenmacht. Die EU sollte hierauf nicht mit einem Decoupling, sondern mit Diversifizierung reagieren; die Kodifizierung eines europäischen Rohstoffverwaltungsrechts ist somit Ausdruck einer neuen Realität, denn die Sicherung industrieller Resilienz und technologischer Wettbewerbsfähigkeit erfordert eine aktive, rechtlich fundierte Steuerung der Rohstoffversorgung. Die chinesischen Exportkontrollen fungieren dabei als Katalysator für eine überfällige rechtliche Systematisierung.



%von OA: Erhebliche langfristige chinesische Exportbeschränkungen würden sich als äußerst störend erweisen. Die Preise würden steigen, und die Industrien, die die Mineralien verwenden, müssten möglicherweise ihre Produktion reduzieren, vor allem die Automobil-, Elektronik- und Verteidigungsindustrie. Dies würde den Druck erhöhen, die Versorgung anderswo zu sichern und Innovationen zu entwickeln, um den Einsatz dieser Metalle zu reduzieren. Dies würde jedoch Zeit und offizielle politische Unterstützung erfordern.

\subsubsection{EU}

Die Frage, inwieweit EU-Mitgliedsstaaten eigenmächtig Exportkontrollen für Rohstoffe einführen dürfen, berührt zentrale Prinzipien des EU-Binnenmarkts, des Außenwirtschaftsrechts sowie der Kompetenzverteilung zwischen Union und Mitgliedsstaaten.

Eine Einführung solcher Kontrollen käme zudem zunächst auf EU-Ebene infrage, um eine etwaige Ausfuhr von bereits in der Union vorhanden ROhstoffen zu unterbinden. Die gemeinsame Handelspolitik ist eine ausschließliche Zuständigkeit der EU. Sie umfasst Maßnahmen zur Regelung des Exports gegenüber Drittstaaten, einschließlich etwaiger Exportkontrollen. Im Sekundärrecht bildet der CRMA die Grundlage für die Gewährleistung der Versorgungssicherheit der EU mit kritischen und strategischen Rohstoffen. Der CRMA enthält keine expliziten Exportverbote oder -kontrollen für kritische Rohstoffe. Die EU könnte jedoch auf Grundlage von Art. 207 AEUV und unter Rückgriff auf die die Dual-Use-Verordnung (EU) 2021/821 Exportkontrollen einführen – etwa bei sicherheitsrelevanten Rohstoffen oder zur Verhinderung von strategischem Abfluss.

Die EU hat bislang keine expliziten Exportkontrollen für kritische Rohstoffe eingeführt, obwohl sie deren strategische Bedeutung anerkennt. Dies kann zu einem Spannungsverhältnis führen: DIe Unionsebene besitzt die Kompetenz, handelt abr (noch) nicht restriktiv, sodass sich die Mitgliedsstaaten gezwungen  sehen könnten, eigenmächtig Exportkontrollen einzuführen, um z. B. Recyclingprodukte oder Lagerbestände zu schützen. Vor dem Hintergrund der Improtabhängigkeit der Union ergibt sich das Risiko, dass Recyclingprodukte in Drittstaaten exportiert werden könnten, Sekundärrohstoffe der europäischen Industrie entzogen werden

Rohstoffe sind in der Regel nicht gelistet in der EU-Dual-Use-Verordnung, es sei denn, sie sind Teil sensitiver Technologien (z. B. seltene Erden für Rüstung oder Hochtechnologie). Daher könnten Mitgliedstaaten versuchen, nationale Exportkontrollen für bestimmte Rohstoffe zu rechtfertigen, etwa zur Sicherung der Versorgungssicherheit. Dennoch stehen solche Maßnahmen unter strenger Kontrolle durch die Kommission und den EuGH, sodass nationale Exporte zwangsläufig unionsrechtlich abgestützt sein müssen.\footnote{s. C-5/94 Hedley Lomas}, und der EuGH hat wiederholt die Notwendigkeit einer einheitlichen Außenhandelspolitik betont sowie die enge Auslegung der Ausnahmetatbestände aus Art. 36, 346 AEUV.


Trotz der EU-Kompetenz gibt es Ausnahmen, in denen nationale Exportkontrollen zulässig sind: Art. 346 AEUV erlaubt es Mitgliedstaaten, Maßnahmen zu ergreifen, die sie für den Schutz ihrer sicherheitspolitischen Interessen für erforderlich halten – etwa bei Rüstungsgütern oder kritischen Rohstoffen mit strategischer Bedeutung. Daraus ergibt sich bspw. auch die nationale Regelung im deutschen AWG und der deutschen AWV, dass zur Wahrung der öffentlichen Sicherheit Exportkontrollen umzusetzen sind.

Somit gilt auf europäischer Ebene: Nationale Exportkontrollen sind nur in engen Grenzen zulässig und daher als eher unwahrscheinlich einzustufen, obgleich Rohstoffe in den meisten Fällen nicht unter Dual-Use-Regelungen fallen und so potentiell Spielräume eröffnet und hierbei doch die Gefahr der Fragmentierung entsteht. Zu Rohstoffen wurden bisher keine nationalen Exprotkontrollen entwickelt -- jedoch im Bereich der Halbleitertechnologien, so beispielsweise von Frankreich\footnote{Arrêté du 2 février 2024 relatif aux exportations vers les pays tiers de biens et technologies associés à l'ordinateur quantique et à ses technologies habilitantes et d'équipements de conception, développement, production, test et inspection de composants électroniques avancés} und den Niederlanden\footnote{Regeling van de Minister voor Buitenlandse Handel en Ontwikkelingssamenwerking van 23 juni 2023, nr. MinBuza.2023.15246-27 houdende invoering van een vergunningplicht voor de uitvoer van geavanceerde productieapparatuur voor halfgeleiders die niet zijn genoemd in bijlage I van Verordening 2021/821 (Regeling geavanceerde productieapparatuur voor halfgeleiders)}.
Darüber hinaus ist zu beachten, dass die Union angesichts geopolitischer Spannungen auch Rohstoffexportregelungen weiter harmonisieren könnte, eas nationale Spielräume weiter einschränken könnte. Dieses Spannungsfeld ist ein zentraler Ansatzpunkt für eine kritische Analyse der Rohstoffgovernance in der EU. Denn: Eine auf EU-Level koordinierte Exportkontrolle erscheint vorteilhafter, jedoch können mitgliedsstaatliche Vorausbemühungen einen Wegbereiter hierfür darstellen.\autocite{Medunic, Nr 15 Juli 2024, S. 3}

Eine mögliche Forderung: Erweiterung des CRMA um exportseitige Schutzmechanismen, etwa durch eine EU-weite Exportgenehmigungspflicht für bestimmte Rohstoffe oder Recyclingprodukte. Insbesondere sollte die Union hierbei Exportkontrollen nicht nur als rüstungskontrollpolitisches INstrument betrachten, sondern -- ähnlich den USA -- als \glqq außenpolitisches Mittel im Rahmen der geostrategischen Rivalität mit China\grqq und im Sinne einer nationalen Sicherheit.\autocite{Medunic, Nr 15 Juli 2024, S. 3}

Zudem 

Die Strategie und die Bemühungen der Union, kritische und strategische Mineralien zu sichern, erschließt die Möglichkeit, zumindest die regionale Widerstandskraft zu stärken, erhöht aber zeitgleich auch das Risiko, dass dies als Rechtfertigung für weitere Vergeltungsmaßnahmen verstanden werden kann.\autocite[siehe auch]{Schroeder, Patrick: Letter: Race for critical minerals sparks call for new materials agency, FT, July 14 2025}



\subsection{Importrestriktionen}
Importrestriktionen \textit{per se} lassen sich zum aktuellen Zeitpunkt keine feststellen -- jedoch ist es durchaus denkbar, dass Importkontrollen durch die Union für eine forcierte Abhängigkeitsminimierung denkbar sind. Generell sind Einfuhrbechränkungen für industrielle Rohstoffe untypisch.\autocite{Schorkopf, Rohstoffverwaltung, Rn. 37}

Die Einführung von solchen Importverboten erfordert freilich auch hier wieder eine entsprechende Kompetenz der Union aus Art. 207 AEUV, sodass ein handelspolitisches Schutzinstrument im Sinne einer importregulierenden Maßnahme hierunter fallen würde; auch Art. 215 AEUV betreffend der Gemeinsamen Außen- und Sicherheitspolitik bietet hier durch Absatz I i. V. m. Artt. 21ff. EUV die Möglichkeit der Einführung restriktiver Maßnahmen nach Entscheidung gem. Art. 29 EUV durch den Rat. Darüber hinaus bietet die Verordnung über eine gemeinsame Einfuhrregelgung der Union\footnote{Verordnung (EU) 2015/478} über eine dann genauer zu prüfende \glqq bedeutende Schädigung\grqq grundsätzlich Möglichkeiten für einfuhrbezogene Schutzmaßnahmen.\footnote{Die Problematik der Einführung von Maßnahmen gegenüber einem WTO-Mitgliedsland wird entsprechend miteinbezogen und erfordert entsprechende Prüfung; ebenso existiert eine parallele Verordnung zu gemeinsamen Ausfuhrregelungen (Verordnung (EU) 2015/179).} Bei jeglichen Importrestriktionen ist hier, trotz oder gerade wegen Art. XIX GATT\footnote{Notstandsmaßnahmen bei Einfuhr bestimmter Waren}, auch die Vereinbarkeit mit Art. XI GATT\footnote{Allgemeine Beseitigung von mengenmäßigen Beschränkungen} genau zu prüfen sowie die infragekommende Auslegung der Artt. XX lit. b\footnote{Maßnahmen zum Schutze des Lebens und der Gesundheit von Menschen[...]; siehe zur komplexen Konstellation der Auslegung des Art. XX am Beispiel des WTO-Panelverfahren gegen chinesische Ausfuhrbeschränkungen Franke, M.: WTO, China-Raw Materials: Ein Beitrag zu fairem Rohstoffhandel?, Beiträge zum Transnationalen Wirtschaftsrecht 114, 2011; ebenfalls zur Auslegung Krenzler/Herrmann/Niestedt, EU-Außenwirtschafts- und Zollrecht, Rn. 109ff.}, XXI\footnote{Ausnahmen zur Wahrung der Sicherheit; in Bezug auf mineralische Rohstoffe grundsätzlich nur lit b. iii. einschlägig, erneut besteht hier deutlich eine Auslegungserforderlichkeit; zur Problematik der Anwendung aufgrund fehlender Rechtsprechung und Einschätzung des Ausscheidens des Art. XXI lit. b Zimmermann, C./Erben, J.: Gallium, Germanium und GATT: Neue Chinesische Ausfuhrkontrollen auf Halbleiterelemente, ZASA 2023, 195, 197}. Dies ist jedoch stark fallabhängig und kann daher an dieser Stelle nicht weiter verfolgt werden.
%Auslegung XXI GATT

\subsection{Zur Rolle der WTO}
Initial ist zunächst festzuhalten, dass die WTO (respektive das GATT) entgegen etwaiger institutioneller Annahmen nicht den Hauptmaßstab für den internationalen Rohstoffhandel darstellen, ursächlich auch durch die generelle Entwicklung des rohstofflichen Rechtsrahmens, der über spezifische Rohstoffübereinkommen anders gestaltet wurde als der auf Waren bezogene.\autocite{Schorkopf, Rohstoffverwaltung, Rn. 42}

Ihre vertragliche Handelspolitik verfolgt die Union auf multilateraler Ebene vor Allem im Rahmen der WTO.\autocite{Müller-Ibold/Herrmann: Die Entwicklung des Europäischen Außenwirtschaftsrechts (2020-2022), EuZW 2022, 1029} Sowohl die Union selbst als auch ihre Mitgliedsstaaten sind Mitglieder der WTO,\autocites[Gemeinsame Positionen der Mitgliedsstaaten und der Union werden aber fast ausschließlich durch die Kommission vertreten, siehe dazu und zur Kooperation der EU und Mitgliedsstaatn im WTO-Kontext]{Streinz, EUV/AEUV, AEUV Art. 218, Rn. 41}[zur dennoch unklaren unionsinternen Rechtssituation]{Tietje	Das Recht der Europäischen Union, AEUV Art. 220, Rn. 23} und sind damit an die Einhaltung der Verpflichtungen im Rahmen der jeweiligen Abkommen.\autocite[Siehe zur Einbindung in die EU-Handelspolitik und das Verhältnis zur WTO]{Krenzler/Herrmann/Niestedt, EU-Außenwirtschafts- und Zollrecht, Rn. 37f.} Auch im Bereich der mineralischen Rohstoffe bildet der zentrale Übereinkommens-Anknüpfungspunkt das GATT, auch durch explizit rohstoffbezogene Bestimmungen wie eine \glqq Bemühenspflicht zur Vermeidung von Ausfuhrsubventionen und Verbot der Unangemessenheit\grqq.\autocite{Herdegen IntWirtschaftsR/Herdegen, 13. Aufl. 2023, § 11. Rn. 1}

Auch in der WTO spiegelte sich die langwährende Nichtbeschäftigung mit Aspekten der Rohstoffpolitik, in den oben geziegten Fällen insbesondere der Fall der Auswirkungen von Handelsbeschränkungen auf Rohstoffe, wider.\autocite[ebenso]{Franke, TWR 114, 2011, S. 29} Generell wäre es aus \glqq institutioneller Perspektive\grqq zu erwarten, dass der Handel mit Rohstoffen insbesondere durch das WTO-Recht bzw. das GATT abgedeckt wird -- dies ist jedoch nicht der Fall, die WTO kennt keine eigenständige Rohstoffordnung.\autocite[Vielmehr waren fertige Erzeugnisse der Anknüpfungspunkt und Rohstoffe nur selektiv vorgesehen, obwohl es de jure von Anfang auf diese anwendbar war; ]{Schorkopf, Rohstoffverwaltung, Rn. 42 ff.}  Die WTO regelt daher nicht den Zugang zu Rohstofflagerstätten, Explorationsrechte oder Lizenzvergaben

Mengenmäßige Ausfuhrbeschränkungen sind auch durch das GATT 1994 verboten (XI GATT); zwar erlaubt Art. XI Abs. 2 lit. a Ausnahmen für zeitlich begrenzte Maßnahmen zur Verhinderung oder Linderung von Engpässen. Diese Ausnahme wird jedoch restriktiv ausgelegt.

Nichtsdestotrotz verbleibt die Problematik der Vereinbarkeit von Maßnahmen in ihrer Begründungsauslegung mit WTO-Recht (insofern ist es zu begrüßen, dass die Union dieses Verhältnis für Aus- und Einfuhrbeschränkungen per Verordnung regelt), aber auch unter aktuellen Gesichtspunkten die Durchsetzbarkeit. Es dürfte hinreichend deutlich sein, dass auf China keine rein liberale Handelspolitik zutrifft, und daher auch vor dem Hintergrund potenzieller zukünftiger Beschränkungen und Verfahren diese Obligationen entsprechend berücksichtigt, denn insbesondere im WTO-Verfahren wurde durch China selbst angegeben, dass der Ausbau der lokalen chinesischen Wertschöpfungskette der eigentliche Grund der Restriktionen ist.\footnote{Im englischsprachigen Original \glqq [...]  export restrictions will allow China to develop its economy in the future . . . [sic] The reason for this is that export restraints encourage the domestic consumption of these basic materials in the domestic economy\grqq, WT/DS394/R WT/DS395/R WT/DS398/R, S. 144.} 
Auch erneute Ausfuhrkontrollen Chinas\footnote{Im Unterschied zu vorherigen Maßnahmen aber nicht durch die Einführung von mengenmäßigen Beschränkungen für Exporte, sondern durch Genehmigungsanforderungen für Ausfuhren.}, diesmal auf Halbleiterrohstoffe\footnote{Gallium und Germanium, ebenfalls als kritische strategische Rohstoffe durch die eingestuft; siehe dazu https://www.reuters.com/markets/commodities/china-bans-exports-gallium-germanium-antimony-us-2024-12-03/}, zeigen, dass handelsrechtliche Maßnahmen weiterhin Bestand haben und mitunter konfrontativ zu verstehen sind.\autocite{ZASA 2023, 195} Ein Panel-Verfahren ist auch hier denkbar, eine etwaige chinesische Rechtfertigung z. B. aus Gründen der Sicherheit\footnote{s. dazu oben, aber ebenso: ZASA 2023, 195} bleibt abzuwarten. Zu beachten ist jedoch generell, insbesondere im Rahmen des Art. XXI lit. b (ii), (iii) GATT, dass ein Handelsverbot, das nur als Vorwand für Sicherheitsmaßnahmen dient, nicht die Ausnahme nach (ii) erfüllt, und eine Täuschung bzw. ein Missbrauch der Sicherheitsausnahme dann vorliegt, wenn  die Handelsmaßnahme absichtlich irreführend ist oder aberdie Sicherheitsausnahme, objektiv betrachtet, missbraucht wird, zugleich gestaltet sich die Darstellung der Beweislage entsprechend anspruchsvoll.\autocites{ZASA 2023, 195, 197}{Ikeda, K.: A Proposed Interpretation of GATT Article XXI (b) (ii) in Light of its Implications for Export Control} Weitere Schwierigkeiten ergeben sich aus der Weite und Charakter des nationalen Einschätzungsspielruams (\glqq nach seiner Auffassung\grqq, Art. XXI b GATT), droht hier doch die generelle Selbstentziehung des Mitgliedsstaates aus den handelsrechtlichen Verpflichtungen.\autocite[Ausführlich]{Herdegen, Internationales Wirtschaftsrecht, Rn. 81ff.}
Eine entsprechende Rechtsprechung oder zumindest eindeutige Auslegung ist daher wünschenswert, zumindest aber als sinnvoll einzustufen.
Das vorangegangene Beispiel illustriert hierbei, dass zwar grundsätzlich nationale Exportrestriktionen von WTO-Mitgliedern der Vertragskonformität und Normenkohärenz mit den WTO-Regeln unterliegen und die WTO zunächst als verbindlivher Rechtsrahmen von zentraler Bedeutung bleibt, denn konsensuelle Maßstäbe werden aufrechterhalten, sodass auch im internationalen Rohstoffhandel ein gewisser Grad an Rechtssicherheit erzielt wird. Die Unzulänglichkeiten im DSU-Mechanismus beeinträchtigen zwar die praktische Durchsetzung, mindern jedoch nicht die normative Bindungskraft des WTO-Rechts. Nichtsdestotrotz besteht die Gefahr dass Sicherheits- und wirtschaftspolitische Ausnahmetatbestände zur Legitimation protektionistischer Maßnahmen missbraucht werden. Ein selektiver Umgang mit WTO-Recht infolge wirtschaftspolitischer Interessen birgt die Gefahr, dass Vertrauensverluste in den multilateralen Streitbeilegungsmechanismus eintreten. Dies führt zu Rechtsunsicherheiten und könnte nachteilige Präzedenzfälle schaffen, die es anderen WTO-Mitgliedern erleichtern, ebenfalls von verbindlichen Verpflichtungen abzuweichen, auch im Lichte handelsbezogener Maßnahmen der Vereinigten Staaten.\footnote{Siehe dazu \ref{Kapitel XXX}} Insbesondere von der EU und ihren Mitgliedsstaaten ist aber keine Abweichung von der Beachtung der wertebasierten internationalen Handelsordnung und damit eine Abkehr vom WTO-Handelsregelregime zu erwarten, was in weiterer Vorausschau dann in Herausforderungen im Schnittfeld der Einhaltung multilateraler Handelsregeln einerseits und der Sicherung der Versorgungssicherheit mit Rohstoffen andererseits resultiert. Es ist zudem deutlich geworden, dass insbesondere die Anwendung der WTO-Normen im strategisch sensiblen Bereich der kritischen Rohstoffe nicht frei von politischen Erwägungen bleibt. Wirtschaftliche Akteure wiederum können zwar von einem grundsätzlich stabilen multilateralen Regelwerk ausgehen, die konkrete Durchsetzung ihrer Rechte hängt jedoch maßgeblich von der staatlichen Implementierung und internationalen Kooperation ab. Insbesondere in Sektoren mit strategischer Bedeutung können einseitige nationale Eingriffe und die Berufung auf Sicherheitsklauseln zu erheblichen Rechtsunsicherheiten und planungsrelevanten Risiken führen.

Insbesondere China rückt hier aufgrund der Rohstoffkonzentration vermehrt in den Vordergrund. Zwar ist es allgemein denkbar, dass China WTO-Regeln nur noch selektiv anwenden könnte angesichts aktuell\footnote{Stand Februar 2025 gem. %https://www.wto.org/english/tratop_e/dispu_e/dispu_by_country_e.htm}
52 laufender Verfahren gegen China, andererseits aber auch 29 Verfahren von China selbst initiiert wurden.

Erneut bietet wie in der gesamten Rohstoffpolitik die Rolle der mitgliedsstaatlichen Mitwirkung Potenzial für einen wenig harmonisierten handelsrechtlichen Rahmen in Bezug auf Rohstoffe und die WTO vis-à-vis der EU und ihren Mitgliedsstaaten. Bei Betrachtung der Außenfunktion fällt hierbei die divergierende interne rechtliche Einstufung der multilateralen Ebene auf, sodass sich hier ein Spannungsfeld zwischen globalem Einfluss, Zugang zu Rohstoffen und dessen Sicherung sowie das Verhältnis zwischen einer Stärkung der Union (nach außen hin) oder aber ein Auftreten in abgestufter Zusammenstellung der einzelnen Mitgliedsstaaten,\autocite{Dauses/Ludwigs, Handbuch des EU-Wirtschaftsrechts, A. I., Rn. 43} was der Harmonisierung einer europäischen Rohstoffverwaltung (auch oder gerade besonders auf) multilateraler Ebene nicht zuträglich ist.

Darüber hinaus schuf die WTO Ende 2024 eine Datenbank zu Handel, Zoll und politischen Maßnahmen im Zusammenhang kritischen Rohstoffen.\footnote{abrufbar unter www.critmin.org.}

Angesichts dieser begrenzten Steuerungsfähigkeit des WTO-Rechts entwickelt sich das internationale Rohstoffrecht zunehmend über bilaterale und plurilaterale Instrumente weiter: Die EU verfolgt gezielte Rohstoffpartnerschaften, Internationale Standards wie die OECD-Leitlinien für verantwortungsvolle Lieferketten mineralischer Rohstoffe oder ESG-Initiativen im Rohstoffsektor ersetzen z. T. verbindliche WTO-Vorgaben, und auch das Außenwirtschaftsrecht der Union entwickelt sich unabhängig vom WTO-Recht weiter, etwa durch autonome Handelsinstrumente, extraterritoriale Sorgfaltspflichten (CS3D), oder strategische Reservemechanismen (vgl. Art. 29 CRMA-Vorschlag).

Aus wirtschaftsverwaltungsrechtlicher Sicht kann das WTO-Recht derzeit nicht als hinreichender Rechtsrahmen für eine strategische Rohstoffverwaltung angesehen werden, da es funktional zu eng auf marktwirtschaftliche Liberalisierung ausgerichtet ist und viele rohstoffspezifische Problemlagen -– etwa Zugang zu Lagerstätten, strategische Autonomie oder Kreislaufwirtschaft –- nicht adressiert. Aus rohstoffverwaltungsrechtlicher Perspektive ist zudem zu berücksichtigen, dass Versorgungssicherheit, nachhaltige Extraktion, strategische Abhängigkeiten und geopolitische Interessenlagen zunehmend neue Steuerungsmechanismen erfordern, die das WTO-Regelwerk nicht bereitstellt. Rechtsakte wie der CRMA der EU stehen exemplarisch für diesen Wandel zu einer ressourcenstrategischen Governance außerhalb klassischer Freihandelsdogmatik. as WTO-Recht bildet nicht den primären Maßstab für den internationalen Rohstoffhandel, weil es aus historischen, systematischen und funktionellen Gründen keine spezialisierten Steuerungsmechanismen für mineralische Rohstoffe entwickelt hat. Vielmehr bestehen heute neue hybride Ordnungen, in denen wirtschaftsverwaltungsrechtliche und außenwirtschaftsrechtliche Instrumente jenseits des WTO-Rechts Anwendung finden. Für die Rohstoffpolitik der EU bedeutet dies, dass sie sich zunehmend auf autonome, multilaterale und bilaterale Instrumente stützt, um strategische Ziele der Versorgungssicherheit, Nachhaltigkeit und geopolitischen Resilienz zu verfolgen.

\subsection{Eine internationale Rohstofforganisation}
Vor dem Hintergrund dieser Erkenntnisse stellt sich die Frage, ob eine Einrichtung eine auf einem internationalen völkerrechtlichen Rohstoffübereinkommmen basierende Rohstoffagentur nicht sinnvoll wäre. Eine solche internationale oder nur europäische Institution zur Verwaltung einer entsprechenden Rohstoffpolitik existiert derzeit nicht, obwohl es rein aus systemischer Sicht sinnvoll erscheinen könnte, eine solche einzurichten.

Die Idee von Ausgleichslagern, wie in der Vergangenheit zu beobachten war, würde zunächst für eine größere Kontrolle über Preis- und Verfügbarkeitsschwankungen sorgen.

In der Geschichte der Rohstoffübereinkommen waren diese zunächst als marktordnende Gebilde gestaltet, die über entsprechende Intervention Preissteuerungen erzielten, insbesondere bei besonderen wirtschaftlichen Herausforderungen oder generellen Marktversagen im Bereich der Rohstoffe zu agieren um eine entsprechende Verteilung von Rohstoffen im Markt sicherzustellen und Preissprüngen vorzubeugen -- jedoch gilt diese mithin interventionistische Herangehensweise als nicht mehr zeitgemäß.\autocite[Und dazu auch widerlegt; siehe ausführlich]{Schorkopf, Rn. 43} Die noch existierenden orginiären Rohstoffübereinkommen mit der EU als Vertragspartner, basierend auf der Regelungsidee der Transparenz, beziehen sich jedoch ausschließlich auf Agrarrohstoffe; auch die Beteiligung an den sog. Internationalen Studiengruppen für bestimmte metallische Rohstoffe zählen hierzu.\autocite{Schorkopf, Rn. 44, 45}

Die Divergenz der Interessen wird insbesondere im Bereich der mineralischen kritischen Rohstoffe deutlich: Während die einen auf Wertschöpfung im Ursprungsland drängen, fordern die anderen verlässlichen Zugang zu günstigen Rohstoffen. Diese Asymmetrie verhindert die Herausbildung eines Konsenses, wie er etwa in der WTO oder bei den Bretton-Woods-Institutionen gelungen ist. Die Förderung und Veredelung vieler kritischer Rohstoffe (etwa Seltene Erden, Gallium, Kobalt) ist auf wenige Anbieterstaaten konzentriert (v. a. China). Diese besitzen erhebliche Marktmacht und haben kaum ökonomische Anreize, sich auf ein internationales Regime einzulassen, das ihre Preissetzungsspielräume oder Exportstrategien einschränkt. Im Gegensatz zum Energierecht (Energiecharta, OPEC) oder Agrarrecht (WTO-Agrarabkommen, FAO) existieren für kritische Rohstoffe bislang keine kodifizierten multilateralen Rechtsinstrumente. Der fragmentierte Rechtsrahmen (WTO, bilaterale Investitionsschutzabkommen, Umweltabkommen) schafft kein kohärentes Regulierungsregime.

Eine solch internationale und global ausgerichtete Rohstoffagentur könnte als koordinierende Kraft wirken und somit insbesondere die instrumentalisierte Rohstoffnutzung, geopolitische Spannungen und unvorteilhafte Auswirkungen auf Abbauländer (sozial, ökologisch, ökonomisch -- s. Rohstofffluch) zumindest mitigieren. Als zentrales Element, auch im Sinne einer Marktkoordination und durch einen datenbasierten Marktüberblick, kann eine solche Agentur zudem als neutraler Mittler und Koordinator bei Rohstoffpartnerschaften auftreten

Auch eine Alternative zur WTO im Bereich von rohstoffbezogenener Streitbeilegung ist denkbar

Das Fehlen einer internationalen Rohstofforganisation für kritische Rohstoffe ist Ausdruck geopolitischer Rivalität, wirtschaftlicher Interessenunvereinbarkeit und völkerrechtlicher Strukturdefizite. Die Entwicklung des Rohstoffrechtsrahmens bleibt fragmentiert und reaktiv. Während sektorale Regelwerke (Agrar, Energie) historisch aus globalen Koordinierungsbedarfen heraus entstanden, fehlt es im Bereich kritischer Rohstoffe bislang an einem vergleichbaren kollektiven Regulierungsimpuls. Die Union muss daher alternative Strategien wie resiliente Lieferketten, strategische Partnerschaften und Binnenförderung priorisieren, ohne auf ein funktionierendes globales Rohstoffregime vertrauen zu können.

Wie zuvor beleuchtet bietet sich auch eine strategisch motivierte Absicherung der Exportkontrolle an: Ähnlich der \textit{Global Export Control Coalition} (GECC), die die Exportkontrollen nach Russland und Belarus abstimmt, kann eine rohstoffspezifische Regelung entwickelt werden, sodass der Handel zwischen den den Mitgliedsstaaten einer Rohstoff-GECC weiter ermöglicht wird und für den den ex-Rohstoff-GECC-Export statt einem Listenansatz ein Lizenzansatz verfolgt wird.\autocite[in Anlehnung an den Vorschlag von]{Medunic, Nr. 15, Juli 2024, S. 4}

Die Einrichtung einer Rohstoffagentur ist nicht zuletzt auf den Willen und die Unterstützung sowohl durch Hauptlieferanten als auch Nettoimporteure angewiesen.



\section{Fazit}
Obgleich es notwendig ist, die Wertschöpfungskette kritischer Rohstoffe in der Union zu stärken, um die Versorgungssicherheit zu verbessern, bleiben die Lieferketten für diese Rohstoffe weltweit und unterliegen externen Einflüssen. Inwieweit diese externen Einflüsse, insbesondere aus einer wirtschaftspolitischen Sichtweise, relevant für das Rohstoffverwaltungsrecht sind, soll im folgenden Kapitel weiter betrachtet werden.

Insbesondere die Rohstoffverwaltung ist als ein Handlungsgebiet von nationalen Lösungen zu klassifizierenm geprägt durch nationale Alleingänge und keiner feststellbaren Tendenz zu einer Zentralisierung afu europäischer Ebene und erst recht keinen sekunderrechtlichen Strukturierung. Der Hauptgrund der mangelhaften bzw. nicht erfolgten Ausgestaltung der unionalen Rohstoffpolitik dürfte also darin liegen, dass die Mitgliedsstaaten sich auf nationale Alleingänge konzentrieren und dadruch eine eigenständige Versrogungspolitik betreiben und zudem nur die jeweiligen nationalen Unternehmen ansprechen.

Die Gestaltung eines stabilen Rechtsrahmens zur Sicherstellung der Rohstoffversrogung der Wirtschaft, die die Ziele der Union unmittelbar beeinflusst, ist also die wesentliche Aufgabe der europäischen Rohstoffpolitik.

Zur Erkenntnis gehört jedoch auch: (Rohstoffliche) Lieferketten lassen sich nur schwerlich kurzfristig diversifizieren.

Auch wenn Exploration und Abbau in der Union vorangetrieben und gar umgesetzt wird, dürfte die europäische Kapazität wohl kaum mit der chinesischen gleichziehen: 

Nichtsdestotrotz ist jeder Schritt heraus aus der chinesischen Abhängigkeit ein Schritt Richtung strategischer Autonomie.



\end{document}