\documentclass[12pt,a4paper,oneside]{book} % 'oneside' für einseitigen Druck

% Kodierung, Sprache und Schrift
\usepackage[utf8]{inputenc} % Erlaubt die Verwendung von Umlauten
\usepackage[T1]{fontenc} % Bessere Schriftkodierung
\usepackage[ngerman]{babel} % Deutsche Lokalisierung

% Schriftarten
\usepackage{lmodern} % Modernere Schriftart, gut für Skalierbarkeit und Lesbarkeit

% Für Abbildungen
\usepackage{graphicx}
\graphicspath{{bilder/}} % Verzeichnis, in dem Bilder gespeichert sind

% Für Tabellen
\usepackage{booktabs}

% Für Links und PDF-Metadaten
\usepackage[hidelinks]{hyperref}
\hypersetup{
	pdftitle={Titel der Dissertation},
	pdfauthor={Autor},
	pdfsubject={Doktorarbeit in den Sozial- und Rechtswissenschaften},
	pdfkeywords={Schlüsselwörter},
}

\usepackage{array}
% Für Bibliographie - Anpassung für Geisteswissenschaften
%%\usepackage[style=authoryear-icomp,backend=biber]{biblatex}
%%\usepackage[backend=biber, style=authoryear-icomp]{biblatex}
%%\usepackage[backend=biber, style=verbose-trad1]{biblatex}
\usepackage[backend=biber, style=verbose-inote]{biblatex}
%%\usepackage{biblatex}
\addbibresource{literatur.bib} % Name der BibTeX-Datei
\DeclareNameAlias{author}{family-given} % Nachname des Autors zuerst

% Anpassung der Nummerierung mit Punkten
\renewcommand{\thechapter}{\arabic{chapter}.} % Kapitel: 1., 2., 3., ...
\renewcommand{\thesection}{\Alph{section}.} % Abschnitt: A., B., C., ...
\renewcommand{\thesubsection}{\Roman{subsection}.} % Unterabschnitt: I., II., III., ...
\renewcommand{\thesubsubsection}{\arabic{subsubsection}.} % Unterunterabschnitt: 1., 2., 3., ...
\renewcommand{\theparagraph}{\alph{paragraph}.} % Absatz: a., b., c., ...


% Für Fußnoten
%%\usepackage[bottom]{footmisc} % Fußnoten am Seitenende


% Anpassung der Kapitelüberschriften
%\usepackage{titlesec}
%\titleformat{\chapter}[hang]{\Huge\bfseries}{\thechapter\quad}{0pt}{\Huge\bfseries}

% Abstand der Fußnoten
\setlength{\footnotesep}{0.5cm}

% Tiefe der Nummerierung und des Inhaltsverzeichnisses
\setcounter{secnumdepth}{4} % Nummerierungstiefe einstellen
\setcounter{tocdepth}{4} % Inhaltsverzeichnistiefe einstellen

% Abstand zwischen Absätzen und kein Einzug
%\usepackage{parskip}
%\setlength{\parskip}{0.5em}
%\setlength{\parindent}{0pt}

% Für Abkürzungsverzeichnis
\usepackage[printonlyused]{acronym}

% Für Zitate und Theoreme (falls benötigt)
\usepackage{csquotes}

% Für Gesetzestexte, Zitate und andere strukturierte Texte
\usepackage{enumitem}

% Zeilenabstand auf 1.3
\usepackage{setspace}



% Beginn des Dokuments
\begin{document}
	% Hier beginnt der eigentliche Inhalt der Arbeit
	% ...

\chapter{Rohstoffverwaltungsrechtliche und -technische Instrumente auf europäischer und nationaler Ebene}

\section{Instrumente für eine EU-Rohstoffpolitik}
Für die Ausgestaltung einer unionalen Rohstoffpolitik kommen zunächst alle primärrechtlichen Instrumente nach Art. 288 AEUV infrage.

Insbesondere im Rohstoffbereich finden sich vermehrt Empfehlungen, Stellungnahmen und Mitteilungen, nutzt also rechtlich zunächst nicht verbindliche Instrumente und werden ausschließlich von der Kommission eingesetzt.

Die beiden einzigen \textit{eindeutig} dem Rohstoffrecht zuzuordnenden Rechtsakte besitzen den Charakter eine Verordnung, sind also im Rahmen des Art. 288 AEUV allgemein gültig, gelten verbindlich und unmittelbar in jedem Mitgliedsstaat. 

%im Verlauf der Entwicklung des CRMA und Konflikt-VO im Rechtssetzungvorgang der Kommission schauen wieso VO gewählt wurde

%Instrumentencharakter von Strategien, Weißbuch etc.

Die Weißbücher der Kommission (s.\,o.) fallen hingegen nicht in den klassischen Bereich der Instrumente, da diese sowohl nur von der Kommission alleinig erstellt und veröffentlicht werden als auch keinen rechtlich verbindlichen Charakter haben, sondern eher als eine Art \glqq Absichtserklärung\grqq verstanden werden können, auch da ihre Umsetzung unter Umständen von der anfänglichen Darstellung abweichen kann.\autocite[siehe hierzu]{Dauses/Ludwigs, O. Umweltrecht, Rn. 196} 

Anders wiederum sieht es auf Ebene des CRMA aus: Die dort fesgehaltenen Berichtspflichten

\subsection{Europäische Verordnung zu kritischen Rohstoffen (CRMA)}%\label{EU-Verordnung}
Die Kommissionspräsidenten kündigte ein Gesetz zu kritischen Rohstoffen im September 2022 an, im Rahmen ihrer State-of-the-Union-Rede vor dem Hintergrund des russischen Krieges gegen die Ukraine, der Covid-19-Pandemie und des Zusammenhangs von Abhängigkeit und Wettbewerbsfähigkeit .\footnote{SPEECH/22/5493}

CRMA: ein Koloss auf tönernen Füßen?

%Betrachtung COM(2020 474 final)
%EU Grundsätze für nachhaltuge Rohstoffe

Obwohl der Verabschiedungszeitplan zum CRMA mitunter \glqq ambitioniert\grqq bezeichnet wurde, konnte die Union noch im April/Mai 2024 nach Abschluss der Frist für Rückmeldungen Ende Juni 2023 den Akt in Kraft treten lassen und damit 14 Monate nach der Veröffentlichung des Vorschlags durch die Kommission\footnote{IP/24/2748} und somit noch rechtzeitig vor den Wahlen zum Europäischen Parlament im Juni 2024. Ferner hat die Kommission mit dem Instrumententyp der Verordnung nach Art. 288 AEUV S. 2 das weitreichendste Instrument gewählt, gilt der CRMA somit unmittelbar und verbindlich in jedem Mitgliedsstaat. Dementsprechend sind auch keine Umsetzungsrechtsakte erforderlich. Zwar ist der CRMA vorrangig auf Art. 114 AEUV gestützt, jedoch wäre auch im Rahmen der Umsetzung der Handelspolitik nach Art. 207 die Verordnung das INstrument der Wahl zur tatsächlichen Umsetzung der gemeinsamen Handelspolitik.\autocite{RdTW 2024, 216}

Der CRMA zeigt die Verbindung zur Handelspolitik bereits in den Erwägungsgründen: 

%https://single-market-economy.ec.europa.eu/publications/european-critical-raw-materials-act_en

%https://www.europarl.europa.eu/doceo/document/TA-9-2023-0454_DE.html#title2

Seit dem 23. Mai 2024 ist die Verordnung 2024/1252 \glqq zur Schaffung eines Rahmes zur Gewährleistung einer sicheren und nachhaltigen Versorgung mit kritischen Rohstoffen\grqq in Kraft getreten, nachdem die Europäische Kommission im März 2023 einen entsprechenden Vorschlag unterbreitete. Dies ist insbesondere herauszustellen, da im Vergleich zu anderen Rechtsakten der EU innerhalb des ordentlichen Gesetzgebungsverfahrens (s. u.) die kurze Dauer von etwas mehr als einem Jahr auf die erkannte Dringlichkeit des Aktes hinweist, denn die zügige Gesetzgebung unterstreicht die Priorität, welche die EU der Sicherung kritischer Rohstoffe für die industrielle Resilienz und technologische Souveränität zumindest in diesem konkreten Fall beimisst. Nichtsdestotrotz folgt diese Geschwindigkeit des Verfahrens dem Trend der Verkürzung der Dauer von Rechtsakten, die nach der ersten Lesung abgeschlossen werden.\footnote{So betrug die durchschnittliche Dauer im ordentlichen Gesetzgebungsverfahrens für Rechtsakte, die nach der ersten Lesung abgeschlossen wurden, im Zeitraum der 8. Wahlperiode (2014-2019) 18 Monate, während der 9. Wahlperiode hingegen nur 13 Monate; siehe hierzu Europäisches Parlament: Facts and Figures, Briefing, Mai 2023, BRI(2023)747102, S. 13.} 

Beim CRMA tritt die enge Verbindung zum veränderten Fokus der unionalen Handelspolitik hervor -- jedoch ist zu betonen, dass nicht allein auf die Handelsdimension bei der Sicherung der Versorgung abgestellt wird, sondern vielmehr auf mögliche Wettbewerbsverzerrungen wie sie durch für Marktteilnehmer divergierende Rechtsvorschirften und Zugangsbedingungen bedingt werden könnten, aber auch durch die tatsächliche Versorgungsrisikoüberwachung, ungleiche Begünstigungen nationaler Maßnahmen oder durch Hürden im Bereich des grenzüberschreitenden Warenverkehrs. Neu im Vergleich zu den eingangs beschriebenen Rechtsakten ist hierbei jedoch die Erkenntnis, dass die Kommission die Erforderlichkeit einer \textit{verbindlichen Steuerung} erkennt.\autocite{Schäffer/Hach, ZRP 2023, 207, 208} \textit{Müller-Ibold/Herrmann} merkten in Bezug zum CRMA 2022, ob durch die Aktivität der Union \glqq sich [mancher fragen wird], ob hier das Pendel nicht zu weit in eine neue Richtung ausschlägt\grqq.\autocite{Müller-Ibold/Herrmann, EuZW 2022, 1085, 1091} Diese Frage mag zunächst berechtigt erscheinen, bewegt sich die Union auf bisher nicht betretenem Boden, denn der CRMA markiert einen mehr oder minder bedeutenden Schritt für die Rohstoffpolitik und -verwaltung im Rahmen der strategischen Autonomie durch den Ausbau inländischer Kapazitäten und die Diversifikation externer Rohstoffquellen anstrebt, und bewegt sich zudem innerhalb des Spannungsfeldes der unionsrechtlichen Kompetenzen (wie der vorige Abschnitt zeigt) und dementsprechend der Autonomie der EU(-Organe), sowie der Marktliberalität. Somit stellt sich die \textit{Pendelfrage}, ob die Union mit diesem neuen Regulierungsrahmen über den verfassungsrechtlich zulässigen Rahmen hinausgeht oder aber zu sehr in eine Richtung reguliert. Der Frage ist aus drei Sichtweisen zu begegnen. Zum Einen hat die strategische Notwendigkeit der Ziele des CRMAs seit 2022 nicht abgenommen -- die Versorgung mit kritischen und strategischen Rohstoffen hängt weiterhin stark von einigen wenigen Drittstaaten ab, und eine Verringerung ist nicht zu erkennen (zumindest nicht, bis entsprechende Maßnahmen ihre Wirkung entfalten), und auch die Implikationen der Zeitenwende können weiterhin als dringlich angesehen werden. Der CRMA kann zudem problemlos auf Art. 173 AEUV (Industriepolitik) gestützt werden, der Maßnahmen zur Wettbewerbsfähigkeit und Widerstandsfähigkeit der EU-Industrie vorsieht. Diese Bestimmung erlaubt die Förderung eines „offenen und wettbewerbsfähigen“ Binnenmarktes und gibt der EU gewisse Eingriffsmöglichkeiten in Wirtschaftsbereiche, die als besonders strategisch gelten, ohne die Marktkräfte auszuhebeln. Man könnte gar die Kompetenzbasis durch Art. 122 I AEUv weiter ergänzen, der Notmaßnahmen zur Bewältigung von Krisen zulässt. Angesichts der hohen Abhängigkeit der EU von Drittstaaten und der damit verbundenen potenziellen Risiken ist diese Norm rechtlich vertretbar und erlaubt der EU, bei Bedarf Maßnahmen zur Sicherung von Rohstoffen zu ergreifen. Ferner stünden weitere Kompetenztitel zu einer Verfeinerung des Pendels zur Verfügung. Letztlich hat die Union auch in der Vergangenheit schon mehrfach die Ausschlagsstärken diverser Pendel neu kalibriert -- als ein prominentes Beispiel sei hier die Entwicklung der Gemeinsamen Agrarpolitik in deen 1960er Jahren zu erwähnen.

Art. 26 AEUV gibt die Grundlage für den Binnenmarkt und die Freiheit der Kapital- und Warenbewegung, Art. 63, Art. 34 AEUV. Das CRMA könnte jedoch als Einschränkung der Marktfreiheiten gewertet werden, falls durch diese Maßnahmen strategische Autonomie über marktwirtschaftliche Prinzipien gestellt wird.


Die handelspolitische Dimension weißt hier einen Doppelcharakter auf, die einerseits auf Drittländer und andererseits auf den Binnenmarkt gerichtet ist.\autocite{Paschke, Rdtw 2024, 206, 211f}


Es wird also deutlich, dass sich die kompetenzgebende Verknüpfung der Politiken in der unionalen Rohstoffpolitik eindeutig in den Ausführungen des CRMAs widerspiegeln und somit auch nochmals die Erforderlichkeit einer interdisziplinären Betrachtung sowohl innerhalb der rechtlichen Sichtweise aber auch unter Einbeziehung weiterer Disziplinen unerlässlich erscheint. 

%Vergleich mit anderen Legislativprozessen

Dieser \glqq Critical Raw Materials Act\grqq (CRMA) ist Teil des \textit(EU Green Deals) und wurde 2022 von der damaligen EU-Kommissionspräsidentin von der Leyen angekündigt \footnote{Ursula von der Leyen, "`State of the Union"' Rede, 14. September 2022.}, in Bezug zur informellen \glqq Versailles Decleration\grqq des Europäischen Rates von 2022.\footnote{Die Deklaration entstand vor dem Hintergrund des russischen Angriffskrieges auf die Ukraine im Februar 2022; in Kapitel III `"Building a more robust economic base'" wird unter dem Stichwort \textit{Critical raw materials} eine Sicherstellung der EU-Versorgung durch strategische Partnerschaften, Bevorratung und Förderung einer Kreislaufwirtschaft und Ressoruceneffizienz gefordert (Europäischer Rat, Informal meeting of the Heads of State or Government, Versailles Declaration, 11. März 2022, S. 7).} Die Kommissions-Generaldirektion Binnenmarkt, Industrie, Unternehmertum und KMU (DG GROW) ist betraut mit der Umsetzung des CRMA.

Im Allgemeinen unterscheidet die Verordnung zwischen \textit(strategischen) Rohstoffen einerseits und \textit(kritischen) Rohstoffen andererseits

Generell kann eine geographische Beschränkung auf einzelne Länder als kritisch angesehen werden, solange die Rohstoffe auch aus anderen Regionen von Unternehmen bezogen werden können, mit entsprechendem Folgen für die lokale wirtschaftliche Stabilität.\autocite[s. hierzu]{ruettinger_doddfrank_2015}

subsubsection{Der Vorschlag der Europäischen Kommission}
Ein erster Vorschlag der Kommission wurde im März 2023 veröffentlicht.

Zu betonen ist hierbei der Fokus der Verordnung auf \glqq nichtenergetisch[e], nichtlandwirtschaftlich[e] Rohstoff[e]\grqq mit entsprechender Bedeutung für die EU-Wirtschaft und Vorliegen eines Vorsorgungsrisikos. Der Entwurf orientiert sich zudem explizit an der offenen strategischen Autonomie\footnote{siehe hierzu Abschnitt Handelspolitik} der Union.\footnote{ErwGr 1}

Als Rechtsgrundlage des Entwurfs wurde Art. 114 AEUV gewählt
Die Wahl der Rechtsgrundlage deutet also darauf hin, dass über das Vehikel der Vervollständigung des Binnenmarktes und Harmonisierung erreicht wird.  Dies stellt formal die Hauptrechtsgrundlage dar, welche sich auf die Verwirklichung und das Funktionieren des Binnenmarkts bezieht.
Hinsichtlich der zwangsläufig auftretenden Frage nach Wahl der Kompetenzgrundlage hat die Union eine elegante Lösung gefunden. Die Abgrenzung der passenden Kompetenz ist nach dem Hauptzweck der Maßnahme zu wählen, in der Regel ist auch nur eine Kompetenzgrundlage zu nutzen anhand objektiver und nachprüfbarer Sachverhalte.\footnote{Bereits erläutert: EuGH, Rs. C-42/97, Rn. 39 ff., EuGH, Rs. C-300/89, Rn. 13, 17} Durch die Formulierung \glqq gestützt [...] insbesondere auf Artikel 114\grqq wird \textit{eine} Kompetenzgrundlage ausgewählt, aber gleichzeitig die Einschlägigkeit anderer infragekommender Kompetenzgrundlage nicht infrage gestellt. Die Wortwahl \glqq insbesondere\grqq lässt jedoch eine implizite Referenzierung anderer Politikbereiche erkennen, ohne aber eine  Mehrfachrechtsgrundlage im Sinne der Rechtsetzungstechnik zu wählen. Solange diese Aspekte nicht eigenständig tragend für die Zielsetzung des Rechtsakts sind, sondern Hilfs- bzw. Begleitfunktionen erfüllen, ist ihre Nichtbenennung als Rechtsgrundlage unionsrechtlich nicht zu beanstanden. \footnote{vgl. erneut Rs. C‑300/89, Rn. 17–19}

Die bewusst gewählte Formulierung lässt sich im systematischen Sinne als Ausdruck einer differenzierten Kompetenzstrategie der Union verstehen und dem Gebot der Bestimmtheit der Kompetenzgrundlage genügt, zugleich aber offen für die Mitberücksichtigung sachlich verwandter Kompetenzen bleibt, ohne diese zu präjudizieren oder zu untergraben, somit auch politisch und funktional offen gegenüber anderen einschlägigen Kompetenzbereichen bleibt. Dieses Vorgehen entspricht einer formell korrekten, inhaltlich pragmatischen Kompetenzwahl und kann dient  als formelhaft vorsichtige Sprachregelung, um etwaige spätere Differenzierungen – etwa im Rahmen von Rechtsstreitigkeiten – nicht vorwegzunehmen. Dies reflektiert den interdisziplinären Charakter moderner Unionspolitik – gerade in komplexen Bereichen wie der Rohstoffsicherung – und steht in Einklang mit der bisherigen Praxis der Kommission und der Rechtsprechung des EuGH, insbesondere vor dem Hintergrund der Komplexität moderner Legislativvorhaben gerecht wird, ohne das Primat der Hauptkompetenz zu relativieren. Ein ähnliches Vorgehen wurde etwa auch im Verordnungsvorschlag für die Batterieverordnung (COM(2020) 798 final) oder in der Verordnung (EU) 2017/821 über Konfliktmineralien gewählt, die ebenfalls auf Art. 114 AEUV gestützt wurden, obwohl auch hier klare Anknüpfungspunkte an Umwelt-, Industrie- und Außenhandelspolitik bestehen.

Im Bereich der Rohstoffverwaltung liefern die Ergebnisse der Ex-Post-Bewertung sowie die Konsultation der Interessenträger im Vorschlag der Verordnung Erkenntnisse: So kritisierten Unternehmen und Betreibe Verfahrens- und Verwaltungskosten sowie die Dauer von Genehmigungsverfahren im Allgemeinen.\footnote{COM(2023) 160 final, S. 10.}

Entsprechend berücksichtigt wird hierbei, dass durch den CRMA keine oder nur begrenzte weitere Verwaltungskosten für Unternehmen entstehen (so beispielsweise durch Berichtspflichten), wobei durch die Kommission ein Kostenausgleich durch die Profitierung von effizienteren Verwaltungsverfahren angenommen wird.\footnote{COM(2023) 160 final, S. 14.} Dieser lässt sich \textit{ex ante} nicht verifizieren, ist jedoch auch Bestandteil der in dieser Arbeit durchgeführten Befragung. %hier dann verweis einfügen

Ferner schlägt die Kommission vor, dass \glqq große Unternehmen, die [...] strategische Rohstoffe verwenden, ihre Lieferketten prüfen und [...] regelmäßige Stresstests ihrer Lieferketten strategischer Rohstoffe durchführen, um sicherzustellen, dass sie alle verschiedenen Szenarien berücksichtigen, die sich im Falle einer Unterbrechung auf ihre Versorgung auswirken könnten\grqq \footnote{COM(2023) 160 final, S. 14.} Hierbei ergibt sich die Frage nach dem Zuständigkeits- und Verantwortungsbereich: Inwieweit können, müssen oder sollten Unternehmen zu solchen Auflagen verpflichtet werden, und inwieweit sollten diese Aufgaben von staatlicher Seite übernommen werden?
% Abstimmungsverhalten

Ein weiterer zentraler Bestandteil des Entwurfs ist das \glqq strategische Projekt\grqq. %Art. 5ff.

Zwar kann die Verordnung als ein erster Beitrag zur Herausbildung eines europäischen Rohstoffverwaltungsrecht gesehen werden, erkennt aber dass keine aktive Harmonisierung der einzelnen mitgliedsstaatlichen Vorschriften beabsichtigt ist.\footnote{COM(2023) 160 final, S. 13.} Die Mitgliedstaaten der EU behalten somit weitgehend ihre Souveränität -- hierbei birgt sich aber das Risiko nach einer weiteren Fragmentierung der Rechtsvorschriften, unter Umständen mit entsprechenden Folgen für den EU-Binnenmarkt.


\paragraph{Die Wahl der Kompetenzgrundlage}
Der CRMA stützt sich vornehmlich auf Art. 114 AEUV, sodass hier zunächst die Binnenmarktpolitik der Union zum Tragen kommt. Wie im ersten Abschnitt festgestellt, agiert Art. 114 als Grundlage, da dieser Artikel der EU die Befugnis einräumt, Maßnahmen zur Angleichung nationaler Rechtsvorschriften zu erlassen, die den reibungslosen Funktionieren des Binnenmarkts gewährleisten. Die Harmonisierung im Bereich kritischer Rohstoffe – etwa hinsichtlich der Gewinnung, Verarbeitung, Wiederverwertung und des Handels – zielt darauf ab, interne Handelshemmnisse abzubauen und so gleiche Wettbewerbsbedingungen in der gesamten Union zu schaffen.

Art. 114 AEUV dient also dazu, verfahrensrechtliche und materielle Anforderungen im Bereich der Rohstoffpolitik zu harmonisieren, etwa durch:

einheitliche Genehmigungszeiträume für strategische Projekte (Art. 10 CRMA-Vorschlag),

gemeinsame Standards zur Nachhaltigkeit und Resilienz (Art. 26 ff. CRMA-Vorschlag),

koordinierte Anforderungen an das Recycling und die Rückgewinnung (Art. 28 ff.).

Der Regelungszweck liegt im Abbau von regulatorischen Unterschieden, die den freien Verkehr von Rohstoffen, Zwischenprodukten und Technologien innerhalb des Binnenmarkts behindern können. Auch die Schaffung von verlässlichen Lieferketten zählt zu den wirtschaftspolitischen Zielen im Rahmen des Binnenmarkts.

Im Bereich der Rohstoffpolitik kann Art. 197 AEUV die Umsetzung von Art. 114 AEUV flankieren, etwa wenn Mitgliedstaaten bei der Zulassung strategischer Projekte, der Anwendung technischer Vorschriften oder bei Genehmigungsprozessen unterstützt werden. Der CRMA sieht etwa in Art. 8 ff. die Einrichtung einer „Behördenplattform für kritische Rohstoffe“ vor, welche den Austausch zwischen Mitgliedstaaten fördern und Verwaltungsverfahren beschleunigen soll.
Diese Mechanismen lassen sich eindeutig auf Art. 197 AEUV zurückführen, wobei der Rechtsakt formal dennoch auf Art. 114 AEUV gestützt bleibt. Art. 197 AEUV ist insofern keine originäre Rechtsgrundlage, sondern eine verfahrensrechtliche Ergänzungskompetenz zur Unterstützung der Durchführung unionsrechtlicher Vorgaben.

%Methodik zur Auswahl strategischer Rohstoffe
%Vergleich der Liste strategischer Rohstoffe

\paragraph{Liste kritischer und strategischer Rohstoffe}

Das Konzept einer Liste kritischer Rohstoffe wie im CRMA ist nicht neu, lag eine solche bereits 2011 als Folge der Rohstoffinitiaive von 2008 -- vor mit insgesamt 14 Mineralien, die mittlerweile auf 34 angewachsen ist.\footnote{Zwischenzeitlich erfolgten drei Aktualisierungen: 2014 (COM(2014) 297) bei der die Liste auf 20 Mineralien erweitert wurde, 2017 (COM(2017) 490) und Erweiterung auf 27 sowie 2020 auf 30 (COM(2020) ???)}. Die Aktualisierung 2023 erfolgte im Rahmen des CRMA%PRÜFEN
Diese regelmäßige, aber mindestens alle drei jahre stattfindende Aktualisierung der Liste wird entsprechend der Ankündigung der Kommission umgesetzt.\footnote{COM(2011) 25 S. 16}

\glqq Kritisch\grqq bezeichnet hierbei einen Rohstoff, bei dem in den nächsten zehn Jahren ab Betrachtungszeitpunkt ein besonders großes Versorgungsengpassrisikos besteht und darüber hinaus als essentiell für die Wertschöpfungskette eingestuft werden.\footnote{COM(2011) 25, S. 13.} 

Der CRMA führte hierbei dann erstmals auch den Begriff der \textit{strategischen} Rohstoffe ein, die als bedeutsam für Wirtschaftszweige der Union (Beispiele?) eingestuft werden.

Somit ergibt sich zum aktuellen Zeitpunkt folgende Übersicht:

\begin{tabular}{|>{\raggedright}p{5cm}|c|c|}
	\hline
	\textbf{Mineral} & \textbf{Kritisch} & \textbf{Strategisch} \\
	\hline
	Antimon & x & \\
	\hline
	Arsen & x & \\
	\hline
	Bauxit & x & \\
	\hline
	Baryt & x & \\
	\hline
	Beryllium & x & \\
	\hline
	Bismut & x & x \\
	\hline
	Bor – metallurgische Qualität & x & x \\
	\hline
	Kobalt & x & x \\
	\hline
	Kokskohle & x & \\
	\hline
	Kupfer & x & x \\
	\hline
	Feldspat & x & \\
	\hline
	Flussspat & x & \\
	\hline
	Gallium & x & x \\
	\hline
	Germanium & x & x \\
	\hline
	Hafnium & x & \\
	\hline
	Helium & x & \\
	\hline
	Schwere seltene Erden & x & \\
	\hline
	Leichte seltene Erden & x & \\
	\hline
	Lithium – Batteriequalität & x & x \\
	\hline
	Magnesium & x & \\
	\hline
	Magnesiummetall & & x \\
	\hline
	Mangan – Batteriequalität & & x \\
	\hline
	Natürlicher Grafit – Batteriequalität & x & x \\
	\hline
	Nickel – Batteriequalität & x & x \\
	\hline
	Niob & x & \\
	\hline
	Phosphorit & x & \\
	\hline
	Phosphor & x & \\
	\hline
	Metalle der Platingruppe & x & x \\
	\hline
	Scandium & x & \\
	\hline
	Siliciummetall & x & x \\
	\hline
	Strontium & x & \\
	\hline
	Tantal & x & \\
	\hline
	Titanmetall & x & x \\
	\hline
	Wolfram & x & x \\
	\hline
	Vanadium & x & \\
	\hline
\end{tabular}



Die Unterscheidung ist dahingehend relevant, dass aufgrund der Eigenschaft des kritischen bzw. strategischen Rohstoffs die entsprechenden Verfahren und Projekte daran angeknüpft bzw. in dieser unterschieden werden.

Eine neue Aktualisierung ist, bei Beibehaltung des Dreijahresrhythmus, für 2026 und 2029 zu erwarten. Hierbei wird insbesondere von Interesse sein, ob die Liste weiter anwächst oder gar Rohstoffe wieder von der Liste gestrichen werden.

Frau\autocite{Frau 2024, NVwZ 2024, 1874, 1875} stellt hierbei treffend heraus, dass es als ungünstig einzustufen ist, dass der CRMA konstant (insgesamt 26 Mal) den \glqq Bereich kritischer Rohstoffe\grqq in Bezug auf strategische Projekte nennt, obgleich diese Projekte hier (nur und immer) zur \glqq Sicherung der Versorgung der Union mit \textit{strategischen} Rohstoffen\grqq\footnote{So eindeutig CRMA, Art. 6 I lit. a} beitragen.

%weitere Listen?

In dieser Hinsicht ist die EU anderen staatlichen Akteuren voraus: Beispielsweise existiert auf US-Ebene keine einheitliche Liste kritischer (oder strategischer) Mineralien. Stattdessen führen daher sowohl das Verteidigungsministerium, das Energieministerium sowie das Innenministerium ihre jeweils eigenen Listen. Von den insgesamt 70 Mineralien werden nur 13 von allen Behörden als \glqq kritisch\grqq eingestuft.\autocite{Baskaran, G.; Schwart, M.: Three U.S. Government Lists: Which Minerals Are the Most Critical?, CSIS} Das Vorhandensein mehrerer, teils widersprüchlicher Listen führt zu vermeidbarer Komplexität und Unsicherheit und untergräbt so Bemühungen, (private) Investitionen in den Lieferketten kritischer Mineralien sowohl im Inland als auch international zu fördern, sodass der EU zumindest hier erfolgreiche Harmonisierung bescheinigt werden kann.
Die meistverbreiteste ist jedoch die Liste, die im Rahmen des Energy Act 2020 entstand und von der US Geological Survey aktualisiert wird. Die USGS hat hierbei ein Modell entwickelt, um die potenziellen Auswirkungen von Unterbrechungen des Außenhandels mit mineralischen Rohstoffen auf die US-Wirtschaft abzuschätzen. Hierbei werden auch die wirtschaftlichen Risiken und den Kosten von Initiativen zur Verringerung der Risiken miteinbezogen. Um festzustellen, ob die Auswirkungen von Handelsstörungen auf die US-Wirtschaft signifikant sind, werden eine Bewertung der wirtschaftlichen Auswirkungen und eine Bewertung der Szenariowahrscheinlichkeiten durchgeführt; jeder Rohstoff wird dann hinsichtlich des Risikos einer Handelsunterbrechung analysiert. Eine hohe Wahrscheinlichkeit führt dementsprechend zur Aufnahme auf die Liste.

Es wird zudem kritisiert, dass zwar ROhstoffe als kritisch eingestuft werden, die Forschung bei dieser Entwicklung jedoch in den Rückstand geraten ist: So würden insbesondere hcohsaliente Minerlaien wie Kobalt oder Lithium besonders intensiv betrachtet, andere ähnlich dringend benötigte jedoch nicht.\autocite{ÖAW, Studie EP}


\subsubsection{Stellungnahme des Europäischen Wirtschafts- und Sozialaussschuss}

Die Bedeutung einer entsprechenden Infrastruktur für eine Rohstoffverwaltung erkennt auch der Ausschuss, indem er den "Aufbau von Verwaltungskapazitäten in den öffentlichen Verwaltungen der EU-Mitgliedsstaaten" fordert % C 349/142,3.
und 

\subsubsection{Lesung im Europäischen Parlament}


Die Verabschiedung im Europäischen Parlament erfolgte bereits im Dezember 2023.

\subsubsection{Betrachtung des Verordnungstextes}
Grundlegend stützt sich der CRMA auf vier Säulen: Zum Einen erfolgt eine Definition der kritischen und strategischen Rohstoffe über entsprechende Listen, des Weiteren ist eine verbessertes Monitoring und Risikomanagement vorgesehen, die nahtlos in eine verbesserte Resilienz der Wertschöpfungskette der kritischen Mineralien übergeht und schließlich auf einheitliche Bedingungen für Mineralien (Koordinierung, Verfügbarkeit, Verarbeitung, Kreislaufwirtschaft) hinwirken soll.

Im Rahmen des Gesetzgebungsverfahren erfuhr der Vorschlag der Kommission entsprechende Veränderungen.

Frau\autocite{Frau, NVwZ 2024, 1874}


Der CRMA führte nachprüfbare Ziele für strategische Rohstoffe ein:
\begin{itemize}
	\item Stärkung der 
\end{itemize}


Ein weiterer zentraler Punkt des CRMA ist die Verankerung der \textit{strategischen Projekte}, die einen Dreh- und Angelpunkt des Akts darstellen.
Die Länge der jeweiligen Genehmigungsverfahren ist hierbei festgelegt und beträgt projektabhängig entweder 15 Monate (für solche, die lediglich Verarbeitung oder Recycling betreffen) bzw. 27 Monate für Projekte im Bereich Gewinnung, eine entsprechende Verkürzung dieser Genehmigungsverfahrensdauer um 3 Monate ist bei bereits genehmigten Projekten vorgesehen, wobei Fristverlängerungen in Ausnahmefällen durchaus möglich sind.\footnote{CRMA, Art. 11} Diese Verfahrensdauerfestlegung wird mitunter als ambitoniert angesehen.\autocite{Quantz, ZfPC 2024, 1}


Darüber hinaus ist der Aufbau einer Rohstoff-Kreislaufwirtschaft ein weiterer wesentlicher Bestandteil des CRMA. Jedoch ist der Anteil des Recyclings an der Bedarfsdeckung bei den kritischen Rohstoffen unbedeutend.\autocite{Bericht zur Rohstoffsituation in Deutschland 2023}


Interessanterweise enthält der CRMA (im Gegensatz zum Chips Act) keinerlei beihilfebezogenen rechtlichen Rahmenbedingungen, stattdessen wird in Kapitel 3 der existierende 


\subsubsection{Zum Rohstoffbegriff im CRMA}
Es ist hervorzuheben, dass mit dem CRMA erstmals der Begriff des Rohstoffes legal definiert wird, was somit auch für zukünftige Rechtsakte eine wichtige Grundlage darstellt. Im Sinne des CRMA bezeichnet ein Rohstoff \glqq einen verarbeiteten oder unverarbeiteten Stoff, der als Input für die Herstellung von Zwischen- oder Endprodukte verwendet wird, mit Ausnahme von Stoffen, die überwiegend als Lebensmittel, Futtermittel oder Brennstoff verwendet werden\grqq. Der Ausschluss der energetisch und landwirtschaftlich geprägten Rohstoffe rührt daher, dass aufgrund der sonstigen Weite des Rohstoffbegriffs\footnote{Diese Problamtik wurde bereits in dieser Arbeit und auch von anderer Seite ausführlich thematisiert, so z. B. Frau 2025, S. 12, } das Versorgungsrisiko und die wirtschaftliche Bedeutung (letztendlich also die \glqq Kritikalität\grqq) in strategischen Sektoren im Mittelpunkt stehen soll.\footnote{vgl. Erwägungsgrund 1, sowie S. 1 KOM2023 160 final.}
Auch die Definition von Exploration, Gewinnung, Mineralvorkommen, Verarbeitung, Recycling, Versorgungsrisiko, Genehmigungsverfahren, Rohstofflieferketten und weitere bilden eine entsprechende Definitionsgrundlage.\footnote{vgl. CRMA, Art. 2, Nr. 2ff.}
%Rechtsanalyse

\subsubsection{Strategische Projekte}
Die strategischen Projekte stellen eine der Hauptsäulen im CRMA dar. Bis Ende August 2024\footnote{Enddatum des ersten Aufrufs der Kommission} erhielt die Kommission 170 Vorschläge für strategische Projekte. Eine Entscheidung erfolgte dann bereits 2025:

Die Tabelle zeigt den Umsetzungsstand nationaler Rohstoffstrategien in den EU-Mitgliedstaaten im Kontext des Critical Raw Materials Act (CRMA). Sie verdeutlicht die erhebliche Heterogenität innerhalb der Union hinsichtlich der politischen Priorisierung, strategischen Ausgestaltung und praktischen Beteiligung an CRMA-Projekten. Während rohstoffintensive Industrienationen wie Deutschland, Frankreich oder Schweden bereits umfassende nationale Strategien implementiert und sich aktiv an strategischen Projekten beteiligt haben, bleibt die Reaktion vieler anderer Mitgliedstaaten bislang verhalten oder aus. Insbesondere kleinere oder rohstoffarme Länder weisen keine erkennbare Rohstoffpolitik auf, was auf eine potenzielle Unterschätzung der Versorgungssicherheitsproblematik hindeutet. Diese Fragmentierung stellt eine Herausforderung für die Harmonisierung europäischer Industrie- und Rohstoffpolitik dar und hat unmittelbare Auswirkungen auf sektorale Strategien, etwa im Bereich der Automobilproduktion.

\begin{table}[htbp]
	\centering
	\caption{Umsetzungsstand nationaler Rohstoffstrategien in den EU-Mitgliedstaaten im Kontext des CRMA (Stand: August 2025)}
	\begin{tabular}{|p{3cm}|p{2.5cm}|p{2.5cm}|p{4cm}|p{6cm}|}
		\hline
		\textbf{Land} & \textbf{Nationale Rohstoffstrategie} & \textbf{CRMA-Projekte} & \textbf{Schwerpunkt} & \textbf{Bemerkungen} \\
		\hline
		Deutschland & Ja & Ja & Primärgewinnung, Verarbeitung, Substitution & Nationales Explorationsprogramm, Batterieprojekte (Lithium, Graphit) \\
		Frankreich & Ja & Ja & Verarbeitung, internationale Kooperation & Beteiligung an NGC Battery Materials, Fokus auf Graphit \\
		Italien & Teilweise & Ja & Recycling, Verarbeitung & Teilnahme an CRMA-Projekten, keine umfassende nationale Strategie \\
		Spanien & Teilweise & Ja & Primärgewinnung & Lithium- und Kupferprojekte, Fokus auf Bergbau \\
		Polen & Teilweise & Ja & Verarbeitung, Recycling & Teilnahme an strategischen Projekten, sektorale Ansätze \\
		Schweden & Ja & Ja & Primärgewinnung, Kreislaufwirtschaft & Starke nationale Strategie, Fokus auf Seltene Erden und Batteriemetalle \\
		Finnland & Ja & Ja & Primärgewinnung, Verarbeitung & Aktive Rohstoffpolitik, Beteiligung an mehreren Projekten \\
		Portugal & Teilweise & Ja & Primärgewinnung & Lithiumprojekte, aber begrenzte strategische Tiefe \\
		Estland & Nein & Ja & Verarbeitung & Teilnahme an CRMA-Projekten, keine erkennbare nationale Strategie \\
		Tschechien & Nein & Ja & Verarbeitung & Einbindung in EU-Projekte, keine nationale Strategie bekannt \\
		Griechenland & Nein & Ja & Primärgewinnung & Einzelprojekte, keine umfassende Strategie \\
		Rumänien & Nein & Ja & Primärgewinnung & Teilnahme an EU-Projekten, keine nationale Strategie \\
		Belgien & Teilweise & Ja & Recycling, Verarbeitung & Fokus auf Sekundärrohstoffe, keine umfassende Strategie \\
		Österreich & Nein & Nein & Keine erkennbare Aktivität & Keine Beteiligung an CRMA-Projekten, keine Strategie veröffentlicht \\
		Ungarn & Nein & Nein & Keine erkennbare Aktivität & Keine bekannten Projekte oder Strategien \\
		Irland & Nein & Nein & Keine erkennbare Aktivität & Keine öffentliche Rohstoffstrategie \\
		Kroatien & Nein & Nein & Keine erkennbare Aktivität & Keine Beteiligung an CRMA-Projekten \\
		Slowakei & Nein & Nein & Keine erkennbare Aktivität & Keine Strategie oder Projekte bekannt \\
		Slowenien & Nein & Nein & Keine erkennbare Aktivität & Keine erkennbare nationale Rohstoffpolitik \\
		Luxemburg & Nein & Nein & Keine erkennbare Aktivität & Kein Rohstoffsektor, keine Strategie \\
		Malta & Nein & Nein & Keine erkennbare Aktivität & Kein industrieller Rohstoffbedarf \\
		Zypern & Nein & Nein & Keine erkennbare Aktivität & Keine Beteiligung an CRMA-Projekten \\
		Bulgarien & Teilweise & Nein & Primärgewinnung & Einzelne Bergbauaktivitäten, keine Strategie \\
		Litauen & Nein & Nein & Keine erkennbare Aktivität & Keine Strategie oder Projekte bekannt \\
		Lettland & Nein & Nein & Keine erkennbare Aktivität & Keine erkennbare Rohstoffpolitik \\
		Dänemark & Teilweise & Nein & Internationale Kooperation (Grönland) & Indirekte Beteiligung über Projekte in Grönland \\
		\hline
	\end{tabular}
	\label{tab:crma_umsetzung}
\end{table}

\subsubsection{Verwaltungstätitgkeit im CRMA}

Zur Verwaltung der Genehmigungsverfahren zu den strategischen Projekten werden von den Mitgliedsstaaten eine oder mehrere nationale Anlaufstellen benannt oder eingerichtet, jedoch müssen diese nicht zwangsläufig Behörden sein.\footnote{CRMA, Art. 9} Auch die Wahl der zuständigen Verwaltungsebene obliegt den Mitgliedsstaaten.\footnote{CRMA, Ew. 28f.}

Die Kommission ermöglicht den Mitgliedsstaaten weiterhin eine deutliche Repräsentierung im Bereich der kritischen Rohstoffen, sodass eine Mitwirkung und Einbindung weiterhin sichergestellt wird.

Auch für Unternehmen entstehen entsprechende Verwaltungsauflagen durch Art. 24 CRMA: Die nach Abs. I durch die Mitgliedsstaaten identifizierten \glqq großen\grqq Unternehmen, u. A. solche im Bereich des Automobilsektors, müssen mindestens alle drei Jahre eine Risikobewertung zu ihren Lieferketten für stratgische Rohstoffe durchführen. Dies umfasst eine geographische Analyse sowie eine zu Versorgungsfaktoren und schließlich eine Betrachtung der Anfälligkeit für Versorgungsuntebrechungen.\footnote{Art. 24 II lit. a-c} Die Ergebnisse fließen dann in Risikominderungsmaßnahmen ein.\footnote{Art. 24 IV, V CRMA}


Art. 35 CRMA schafft drüber hinaus einen aus Vertretern von Mitgliedsstaaten und Kommission zusammengesetzten\footnote{Art. 36 I CRMA} und regelmäßig tagenden\footnote{Art. 36 V CRMA} Europäischen Ausschuss für kritische Rohstoffe, welcher einerseits die Kommission berät und andererseits seine im CRMA festgelegten Aufgaben übernimmt: Hierunter fallen die Bewertung und Beschleunigung von Genhemigungsverfahren und strategischen Projekten, Kontrolle der mitgliedsstaatlichen Umsetzungsverpflichtungen, die Liste zu den kritischen und strategischen Rohstoffen und Koordinierung im Rahmen des CRMA.
Diese Konstruktion eines Rohstoffausschusses stützt sich – neben Art. 114 AEUV – auch funktional auf die Prinzipien der Verwaltungszusammenarbeit.

Verwaltungsbezogenen Anforderungen ergeben sich insbesondere auch aus den Monitoringpflichten des CRMA (Kapitel 4), da die EInführung von entsprechenden Überwachungsmechanismen in den Wertschöpfungsketten hinsichtlich möglicher Lieferschwierigkeiten vorgesehen ist -- und beispielsweise durch eine gemeinsame EU-Vorratshaltung von Rohstoffen mitigiert werden könnte.

\subsubsection{Parallelen zur REACH-Verordnung}
Die europäische ROhstoffpolitik steht zudem im Spannungsfeld zwischen Umwelt- und Gesundheitsschutz und Versorgungssicherheit -- in diesem Kontext tritt die sog. REACH-Verordnung\footnote{EG Nr. 1907/2006} als zentraler Rechtsakt hervor, dessen Regelungsinhalt sich teilweise mit dem des CRMA überschneidet. Beide Regelungen betreffen rein inhaltlich dieselben Stoffe, insbesondere Mietalle und Mineralien wie Lithium, Nickel, Kobalt, jedoch aus unterschiedlichen Perspektiven: Während die REACH-VO auf die Bewertung und Adressierung chemischer Risiken abzielt, ist der CRMA bekanntermaßen versorgungsorientiert. 

ahlreiche Stoffe, die unter REACH als besonders besorgniserregend (SVHC) eingestuft sind, finden sich auch auf der Liste der kritischen Rohstoffe gemäß CRMA. So ist etwa Lithium sowohl  strategisch bedeutsam, als auch Gegenstand toxikologischer Bewertungen im Rahmen der REACH-VO, sodass dass hier potentielle Regelungskonflikte entstehen könnte, etwa wenn ein Stoff aus Gründen des Gesundheitsschutzes beschränkt oder verboten wird, obwohl er für die Umsetzung der grünen Transformation unverzichtbar ist. Ein weiteres zentrales Spannungsfeld betrifft die Zulassungsverfahren nach REACH. Die Verordnung sieht vor, dass bestimmte gefährliche Stoffe nur nach vorheriger Zulassung verwendet werden dürfen (Art. 57 ff. REACH). Dies kann zu einem faktischen Verwendungsverbot führen, wenn keine Zulassung erteilt wird. Der CRMA hingegen verfolgt das Ziel, die Verfügbarkeit genau dieser Stoffe zu fördern, etwa durch beschleunigte Genehmigungsverfahren für strategische Projekte (Art. 8 CRMA) oder durch die Festlegung von Benchmarks für die heimische Förderung und Verarbeitung.

CRMA und REACH sind unionsrechtlich gleichrangig und gelten daher nebeneinander, ein ausdrücklicher Vorrang ist nicht normiert. Vielmehr ergibt sich aus dem Zusammenspiel beider Regelwerke ein Koordinierungserfordernis, das insbesondere die Europäische Kommission betrifft. In der Praxis bedeutet dies, dass bei der Bewertung von Zulassungsanträgen nach REACH auch asymmetrische Interessenlagen – wie etwa die strategische Bedeutung eines Stoffes – berücksichtigt werden sollten. Der CRMA enthält hierzu in Art. 24 Abs. 2 eine Öffnungsklausel, wonach die Kommission Maßnahmen zur Unterstützung von Unternehmen ergreifen kann, die strategische Rohstoffe verwenden. 

Ein weiterer Berührungspunkt liegt im Bereich der Daten- und Informationspflichten. Beide Verordnungen setzen auf eine starke Rolle der Industrie bei der Bereitstellung von Informationen: REACH verlangt umfassende Daten zur Stoffbewertung, während der CRMA ein Monitoring der Lieferketten und eine Risikoanalyse durch Unternehmen vorsieht (Art. 17 CRMA). Hier besteht die Gefahr von Doppelregulierung und administrativer Überlastung, was eine Harmonisierung der Berichtspflichten nahelegt.

Beide Verordnungen zielen auf eine nachhaltige Nutzung von Ressourcen ab, wenn auch mit unterschiedlicher Schwerpunktsetzung, sodass such deutliche Parallelen insbesondere im direkten Stoffbezug ergeben. Zugleich offenbaren sich Regelungslücken und Zielkonflikte, die eine kohärente Rechtsanwendung und gegebenenfalls eine Weiterentwicklung des europäischen Stoffrechts erforderlich machen. Für die Zukunft wäre eine systematische Integration strategischer Rohstoffaspekte in das REACH-Regime denkbar – etwa durch eine Ausnahmeregelung für Stoffe mit hoher Versorgungskritikalität oder durch eine abgestimmte Risikobewertung unter Einbeziehung wirtschaftlicher und geopolitischer Faktoren.

\subsubsection{Mangelnde Sanktionierung bei Zielverfehlung}

Dem CRMA mangelt es an klaren Konsequenzen für den Fall, dass die Ziele nicht eingehalten werden können. Zwar können die Ziele eher als Motivation denn als Mindestziele interpretiert werden \autocite{finden, existiert} -- nichtsdestotrotz nimmt so der CRMA die Gestalt eines zahnlosen Tigers an, sodass die Verbindlichkeit infrage gestellt werden könnte.

\subsubsection{IPCEI zur Rohstoffen}
Die Einrichtung eines „Important Project of Common European Interest“ (IPCEI) für kritische Rohstoffe gilt als politisch und wirtschaftlich hochwahrscheinlich, jedoch rechtlich und administrativ herausfordernd.

Die Europäische Kommission hat im Frühjahr 2025 eine Konsultation zur industriellen Zusammenarbeit bei der Beschaffung und dem Recycling kritischer Rohstoffe eingeleitet, mit dem Ziel, die Vereinbarkeit solcher Kooperationsprojekte mit dem EU-Beihilferecht zu prüfen.\footnote{https://ec.europa.eu/commission/presscorner/api/files/document/print/de/ip_25_911/IP_25_911_DE.pdf} Jedoch ist bereits der Ausblick der Kommission sehr vage formuliert: So \glqq könnte es darauf hinauslaufen, dass die Kommission Unternehmen Leitlinien zur Vereinbarkeit von
Kooperationsprojekten in diesem Bereich mit den EU-Wettbewerbsvorschriften an die Hand gibt\grqq, dies hänge jedoch \glqq von der Auswertung der eingehenden Beiträge ab\grqq. Die politische Debatte wird flankiert von Forderungen aus Industrieverbänden wie dem BDI, der ein eigenständiges IPCEI für kritische Rohstoffe als notwendige Ergänzung zur bestehenden Förderarchitektur ansieht.\footnote{https://bdi.eu/spezial/europawahl/power-up-europe/handlungsempfehlungen-eu-rohstoffpolitik}

Noch zu Beginn ging man vom Aufbau eines IPCEI zu kritischen Rohstoffen aus, jedoch bereits mit der Erwartung, dass dies nicht zeitnah umgesetzt werden würde.\autocite{Rohde_PWC}

Zwar existiert ein IPCEI zur Batterielieferkette. Es zeigt exemplarisch, wie ein sektorales IPCEI zur Rohstoffsicherung beitragen kann, denn es umfasst die gesamte Wertschöpfungskettevon der Rohstoffgewinnung über die Zellproduktion bis zum Recycling und hat Investitionen in Höhe von über 13 Milliarden Euro ausgelöst.\footnote{https://competition-policy.ec.europa.eu/state-aid/ipcei/approved-ipceis/batteries-value-chain_en} Die dort entwickelten Governance-Strukturen, Fördermechanismen und Nachhaltigkeitskriterien bieten grundsätzlich modellhafte Ansätze für ein zukünftiges IPCEI im Bereich kritischer Rohstoffe.

Gleichzeitig bestehen strukturelle Hürden, die ein IPCEI für kritische Rohstoffe erschweren: Erstens ist die Rohstoffgewinnung stark standortgebunden und unterliegt komplexen Genehmigungsverfahren, die nicht ohne weiteres in ein transnationales IPCEI integriert werden können. Zweitens fehlt bislang eine kohärente Abstimmung mit anderen EU-Regelwerken, etwa der Taxonomie-Verordnung, dem Chemikalienrecht und den Lieferkettenvorgaben, was Zielkonflikte erzeugt. Drittens ist die Finanzierung fragmentiert: Der CRMA selbst enthält keine direkten Investitionsmittel, sodass ein IPCEI auf die Mobilisierung bestehender Fonds (z. B. InvestEU, EIB, EFRE) angewiesen wäre – ein Prozess, der politisch konsensfähig und administrativ belastbar ausgestaltet werden muss.

Insgesamt ist ein IPCEI für kritische Rohstoffe nicht nur wahrscheinlich, sondern ordnungspolitisch geboten, um die strategische Autonomie der EU zu stärken. Die Erfahrungen aus der Batterie-Wertschöpfungskette zeigen, dass sektorale IPCEIs ein effektives Instrument zur Bündelung von Ressourcen, zur Risikoteilung und zur Innovationsförderung darstellen – vorausgesetzt, die rechtlichen, finanziellen und politischen Rahmenbedingungen werden frühzeitig harmonisiert.

\subsubsection{Finanzierung des CRMA}

Grundlegend mangelt es dem CRMA an einem klaren Finanzkonzept: So sind keine konkreten Summen zur Zielerreichung vereinbart, noch gibt es kein CRMA-zugeschnittenes oder eigenes Finanzkonzept.

Zudem wird kritisiert, dass trotz der Existenz von EU-Förderprogrammen oftmals \glqq zu föderalistisch und fragmentiert\grqq vorgegangen werde, eine zentrale Stelle zur Bündelung wird zudem gefordert.\autocite{Je länger die EU untätig bleibt, umso verwundbarer macht sie sich}

\subsubsection{Parallelen zur EU-Chips-Verordnung}




\subsubsection{Fazit zum CRMA}
Durch die Verabschiedung wurde die Kooperation im Bereich der Rohstoffverwaltung auf europäischer Ebene gestärkt, die Bedeutung nochmals unterstrichen und die in den 2000er-Jahren begonenen Entwicklung einer unionalen Rohstoffpolitik fortgesetzt, weiter ausgeprägt und auf aktuelle Umstände zugeschnitten. Die eingangs beschriebene Phase der Bedeutungslosigkeit und geringen Prioritisierung scheint überwunden, was auch der vergleichsweise rasche Gesetzgebungsprozess des CRMA verdeutlicht.

Der CRMA sei daher zu begrüßen, auch wenn man keine kurzfristigen Effekte kurz nach den Inkrafttreten erwarten konnte würde der CRMA dennoch mittel- und langfristig einen ausschlaggebenden Beitrag zur offenen strategischen Autonomie und zum Binnenmarkt liefern, trotz der enthaltenden Zielkonflikte,\autocite{Schäffer/Hach, ZRP 2023, 210f.}

Erneut tritt die Politisierung des Rechtsbereiches hervor: Die Unterscheidung zwischen kritischen und strategischen Rohstoffen ist nicht per se an Eigenschaften der jeweiligen Mineralien gebunden, sondern (wenigstens im Falle des CRMA) Resultat einer entsprechenden Entcheidung bzw. Definition durch die Kommission--diese (wirtschafts-)politische Entscheidung sollte sich hierbei natürlich an einer objektiven EInschätzung orientieren.\autocite{Frau 2024, NVwZ, 1874, 1876} Mitunter wird dem CRMA eine mangelnde Weite attestiert, denn er benenne zwar die Herausforderungen, verlöre sich aber dann in der Bürokratie.\autocite{Lunday Politico}

Durch den CRMA sind zudem sowohl die Begrifflichkeiten und die umfassten Mineralien der kritischen sowie strategischen Rohstoffen nochmals respektive erstmals primärrechtlich fixiert.

Somit geht der CRMA insbesondere im Bereich der Ziele nicht über ein Wunschdenken hinaus: Die 

%Zielkonflikte

Zudem betont die Kommission in ihrer Folgenabschätzung zum CRMA, dass es in vielen Mitgliedstaaten an Kapazitäten, Daten und politischem Willen fehlt, um die Versorgungssicherheit systematisch zu adressieren

Obwohl der CRMA eine unionsweite Verpflichtung zur strategischen Rohstoffsicherung vorsieht, ist die tatsächliche Umsetzung in den Mitgliedstaaten bislang fragmentiert. Während rohstoffintensive Industrienationen wie Deutschland oder Frankreich das Thema aktiv adressieren, bleibt die politische und administrative Reaktion in anderen Mitgliedstaaten verhalten oder gar aus. Diese Heterogenität birgt Risiken für die Kohärenz der europäischen Industriepolitik und erschwert die Harmonisierung sektoraler Strategien, etwa im Automobilbereich

Die EU hat mit dem CRMA also einen vergleichsweise weichen und weniger strategisch und institutionsbezogenen Weg der Rohstoffverwaltung eingeschlagen, denn der Großteil der eigentlichen Handlungen wird der mitgliedsstaatlichen Diskretion überlassen. Zwar bietet der CRMA Koordinationsansätze und -hilfen, jedoch mangelt es (noch) an einer unionsübergreifenden Kontrollinstanz sowie überhaupt Konsequenzen bei Nichtbeachtung. Dies wird flankiert durch die mangelnde Einheitlichkeit der europäischen Rohstoffverwaltung, Nicht-Existenz einer Vorratshaltung (wie beispielsweise in den USA).

Weiterhin trifft der CRMA auch keine Unterteilung innerhalb der kritischen und strategischen Rohstoffe hinsichtlich einer Priorisierung der einzelnen Mineralien

Bevor überhaupt eine Entwicklung einer europäischen Rohstoffgewinnung und -kreislaufwirtschaft begonnen hat, droht die Gefahr eine Weiterentwicklung eines Rohstoffrechts, welches umfangreicher und komplizierter wird und somit eine Rohstoffpolitik im Keim ersticken könnte.

Keine hinreichende Klärung der Zuständigkeit

\subsection{Konfliktmineralien}
Erneut existiert keine einheitliche Definition zu den sog. Konfliktmineralien - also solche, 

Meist werden die Elemente Zinn, Tantal, Wolfram und Gold als Konfliktmineralien eingestuft, 

Die Regulierung von Konfliktmineralien und ihre Integration in das Wirtschaftsverwaltungsrecht stellt 

\subsubsection{Die Konfliktmineralien-VO}
Die Verordnung (EU) 2017/821, auch bekannt als "Konfliktmineralien-Verordnung" (VO) bindet Unionseinführer (UE) bestimmter sog. "Konfliktmineralien", genauer Wolfram, Tantal, Zinn ("3T" nach den englischen Bezeichnungen Tungsten und Tin) und Gold ("3TG") an Sorgfaltspflichten, die sich auf die entsprechenden Lieferketten auswirken, und kann als das am weitesten entwickelte Regelungsregime auf euiropäischer Ebene eingestuft werden\autocite{Kalls, ZfPW 2024, 181, 199}. Diese Sorgfaltspflichten sollen sicherstellen, dass UE mit ihrem Handel der genannten Metalle und deren Erze, also die als Konfliktmineralien verstandenen Rohstoffe, nicht zur Finanzierung bewaffneter Konflikte oder zu Menschenrechtsverletzungen beitragen und somit die "Verknüpfung zwischen Konflikten und illegalem Mineralabbau durchbrochen wird"\footnote{2017/821}. Die Verordnung gilt seit dem 1. Januar 2021 und erstreckt sich im Anwenungsbereich auf Importeure innerhalb der Europäischen Union, wie in Art. 1 Konfliktmineralien-VO geregelt. Unionseinführer i. S. d. Verordnung sind natürliche oder juristische Personen, die die vom Anwendungsbereich umfassten Minerale, Metalle und deren Erze ab in denen in Anhang 1 festgelegten Mengenschwellen in das Zollgebiet der Union einführen.\footnote{Art. 2 lit. l), VO 2017/821.} 

Der entsprechende Entwurf der Verordnung wurde von der Kommission bereits 2014 an Rat und Parlament übermittelt, hingegen stimmte der Rat erst 2017 ohne Gegenstimmen dem Vorschlag zu.\footnote{ST 7937 2017 INIT} Das Parlament nahm insgesamt 60 Änderungen zum Kommissionsvorschlag an,\footnote{8645/15} und erkannte bereits hier die Problematik wie "die richtige Balance zwischen operationeller Flexibilität bei der Umsetzung (...) und angemessener Einbeziehung des Gesetzgebers bei wichtigen praktischen Fragen (...) gefunden werden kann."\footnote{A8-0141/2015} Dies lässt sich als Erkenntnis festhalten: eine flexible Umsetzung von Rechtsnormen notwendig, um auf die dynamischen und oft volatilen Bedingungen im Rohstoffsektor angemessen reagieren zu können. Rohstoffmärkte sind von globalen Entwicklungen, geopolitischen Risiken und technologischen Innovationen geprägt, die schnelle Anpassungen erfordern. Eine starre rechtliche Struktur würde das Risiko bergen, dass Unternehmen nicht in der Lage sind, auf solche Veränderungen zeitnah zu reagieren, was ihre Wettbewerbsfähigkeit und letztlich die Versorgungssicherheit gefährden könnte. Gleichzeitig ist es jedoch unerlässlich, dass der Gesetzgeber bei der Ausgestaltung und Anpassung des Rohstoffverwaltungsrechts eine aktive Rolle spielt. Dies gewährleistet, dass wirtschaftliche Interessen nicht auf Kosten von Rechtsstaatlichkeit, Umweltstandards oder ethischen Verpflichtungen verfolgt werden. Insbesondere im Kontext der Konfliktmineralien-Verordnung zeigt sich die Notwendigkeit, legislative Vorgaben so zu gestalten, dass sie praktikabel sind, ohne die Kontrolle über kritische Aspekte wie Menschenrechtsverletzungen oder Umweltzerstörung zu verlieren. Im Rohstoffverwaltungsrecht ist diese Balance daher nicht nur eine Frage der rechtlichen Präzision, sondern auch ein wesentlicher Faktor für die wirtschaftliche Stabilität und ethische Integrität der gesamten Rohstoffwirtschaft.
Ein weiterer Punkt, der insbesondere den multilateralen Aspekt umfasst, wird in der Stellungnahme des Entwicklungsausschusses\footnote{A8-0141/2015, S. 53} angeführt: Eine Verodnung wie die Konfliktmineralien-VO können nicht als ein "in sich geschlossendes handelspolitisches Instrument angesehen werden", sondern müsse aus einer weiter gefassten Perspektive auch international betrachtet werden.

Die jeweiligen Konflikt- und Risikogebiete nach Art. 2 lit. f werden, einer jährlichen Aktualisierung unterliegend, auf der \textit{Conflict-Affected and High Risk Areas}-Liste (CAHRAS) der EU bekanntgegeben.\footnote{Die Liste ist abrufbar unter https://www.cahraslist.net/cahras. Stand 2024 umfasst die Liste die Staaten Afghanistan, Äthiopien, Burkina Faso, Burundi, DR Kongo, Eritrea, Indien, Jemen, Kamerun, Kolumbien, Libanon, Libyen, Mali, Mosambik, Myanmar, Niger, Nigeria, Pakistan, die Philippinen, Russland, Simbabwe, Somalia, Sudan, Südsudan, Ukraine, Venezuela, sowie die Zentralafrikanische Republik.} Durch den Verweis auf die CAHRAS-Liste wird ermöglicht, den geographischen Anwendungsbereich entsprechend anpassen zu können.

Hinsichtlich der Sorgfaltspflichten orientiert sich die Konfliktmineralien-VO eng an den, \textit{per se} rfechtlich unverbindlichen, OECD-Leitsätzen \autocite{OECDleitfaden2019} für die Erfüllung dieser, insbesondere durch die Bestrebungen der Kommission, diese Leitsätze stärker zu unterstützen wie auch die Anwendung der Leitsätze durch Unternehmen und die "Erfüllung der Sorgfaltspflicht", auch in nicht-OECD Staaten.\footnote{s. Präambel 9, VO 2017/821.} Daraus lässt sich schließen, dass eine Erfüllung der Vorgaben und Ziele der Konfliktmineralien-VO hinreichend durch eine Orientierung an den und eine Erfüllung der OECD-Leitsätzen erreicht werden kann, was letztendlich die Erfüllung die Sorgfaltspflicht betrifft. Letztlich werden also die Lieferketten der EU-Einführer gemäß einer Due-Dilligence-Prüfung betrachtet.\autocite[Rn. 390]{ruttloff_lieferkettensorgfaltspflichtengesetz_2022}
%wie gestaltet sich das im Umsetzungsrechtsakt

%genaue Analyse Konflikt-VO, Art 9


Zum 7. Mai 2020 trat das entsprechende deutsche Durchführungsgesetz (Mineralische-Rohstoffe-Sorgfaltspflichten-Gesetz, MinRohSorgG) zur Verordnung 2017/821 in Kraft \footnote{Gesetz vom 29.04.2020 - BGBl. I 2020, Nr. 21 vom 06.05.2020, S. 864.}, es gab zuvor keine entsprechenden Rechtsakte, sodass die Thematik vergleichsweise spät in den Bereich des Rohstoffrechts aufgenommen wurde. Die Implementierung der EU-Verordnung in den deutschen Rechtsrahmen zeigt, dass eine Verstärkung des Rohstoffwirtschaftsverwaltungsrechts erforderlich ist, um die Sorgfaltspflichten der Unternehmen wirksam durchzusetzen.

Hierbei agiert die Bundesanstalt für Geowissenschaften und Rohstoffe (BGR) als benannte Nationale Behörde und hat mit der deutschen "Kontrollstelle EU-Sorgfaltspflichten in Rohstofflieferketten" (DEKSOR) eine entsprechende Kontrollstelle geschaffen, die mit der Anwendung der Konfliktmineralien-VO betraut ist und Unionseinführer auf Einhaltung der Regelungen kontrolliert. 

Insofern hat die Union hier durch Vorgabe der Schaffung der Verpflcihtung der Mitgliedsstaaten zur Einrichtung einer nationalen Vollzugsbehörde die Richtlinien für die nationalen Verwaltungsstrukturen vorgeformt und intensiviert bzw. der mitgliedsstaatliche Vollzug einerseits mit Maßgaben konfrontiert, aber auch dementsprechend mit Insrumenten zum tatsächlichen Vollzug ausgestattet.

Die Relevanz des Rechtsaktes, zumindest aus deutscher Perspektive, kann gemischt betrachtet werden. Für den Berichtszeitraum 2022 \autocite{deutsche\_kontrollstelle\_eu-sorgfaltspflichten\_in\_rohstofflieferketten\_jahresbericht\_2023} (sowie teilweise 2023) wurden insgesamt 2.402 Unionseinführer erfasst, wovon aber lediglich 150 überhaupt die erforderliche Mengenschwelle überschritten (6\%) und davon 15 Unternehmen von nachträglichen Kontrollen betroffen waren. Unionseinführer können hierbei auch Privatpersonen sein, worauf ca. 2.000 der 2.402 zurückgehen. Ferner entfiel mehr als die Hälfte der Unionseinführer auf Zinnprodukte, bei Überschreitung der Mengenschwelle jedoch der überwiegende Teil auf Wolfram entfällt. Der Bericht beschreibt ein weiteres Problem, was sich insbesondere aus der zollrechtlichen Definition eines "Ursprungslandes" ergibt, also dem Land, in dem der letzte Verarbeitungsschritt stattfand – somit können Konfliktmineralien durchaus aus nicht vom geographischen Regelungsbereich der Verordnung umfassten Drittländern erfolgen, sodass die Nachverfolgbarkeit erschwert wird. Die Rückverfolgbarkeit über die gesamte Lieferkette ist daher nur deutlich eingeschränkt möglich.
Zudem verfügen die DEKSOR und BGR nur über eine stark eingeschränkte Kompetenz. So ist zwangsläufig eine Bewertung der Kommission erforderlich, die die nationalen Behörden zum Verhängen von Strafen bei Nichteinhaltung befugt (Art. 17 III Konfliktmineralien-VO). Auch der DEKSOR-Bericht nennt das Nichtvorhandensein einer Befugnis bei Unklarheiten über die Überschreitung der Mengenschwelle Auskunft zu erhalten als nachteilig, setzt eine "Auskunftspflicht [$\ldots$] voraus, dass ein Unionsführer tatsächlich die Mengenschwelle überschritten hat". Dies schränkt die Durchsetzungskraft dieses rohstoffverwaltungsrechtlichen Akts ein; ferner kann daraus eine mangelnde Umsetzung der erwähnten Sorgfaltspflicht abgeleitet werden, was die DEKSOR in ihrem Bericht bestätigt, denn die jeweiligen Offenlegungspflichten der Unionseinführer über den Mengenschwellen werden auch bisher nur von einer Minderheit in vollem Umfang erfüllt \autocite[25]{deutsche\_kontrollstelle\_eu-sorgfaltspflichten\_in\_rohstofflieferketten\_jahresbericht\_2023}[]. Bei nachträglichen Kontrollen waren zudem Nachweise oft nicht ausreichend oder wurden nur zeitverzögert bereitgestellt, was insbesondere dem § 6 MinRohSorgG zuwiderläuft bzw. erkennen lässt, dass lediglich den Auskunftspflichten aus Art. 6 Konfliktmineralien-VO nachgekommen wird, nicht aber den Risikomanagementpflichten (Art. 5 Konfliktmineralien-VO), den Pflichten in Bezug auf das Managementsystem (Art. 4) und den generellen Offenlegungspflichten nach Art. 7.

DEKSOR: "Im Rahmen der nachträglichen Kontrollen ist zudem aufgefallen,
dass es für die Unternehmen eine Herausforderung ist, die erlangten und auf aktuellem Stand gehaltenen Informationen bezüglich der Lieferkette nach Art. 7 Abs. 2 VO von vorgelagerten Hütten und Raffinerien zu erhalten."

Gemäß Art. 11 der Konfliktmineralien-VO i. V. m. § 3 III MinRohSorgG ist die DEKSOR zur Durchführung "geeigneter nachträglicher Kontrollen" befugt, sodass entweder auf Grundlage eines "risikobasierte[n] Ansatz[es]" oder bei Vorliegen entsprechender Hinweise auf Verstöße (auch durch Dritte) nachträgliche Kontrollen durchgeführt werden können, in denen insbesondere ein Einhaltung der Sorgfaltspflichten und Prüfpflichten nachgegangen wird.

Es wird von der DEKSOR zudem berichtet, dass sich hierbei auf ein mögliches Risiko für die Geschäftstätigkeit der involvierten Unternehmen bei Offenlegung der Lieferketten im Sinne der Wahrung von Geschäftsgeheimnissen und anderen wettbewerbsrelevanten Informationen berufen wird, was als ein Widerspruch zum Transparenzziel der Konfliktmineralien-VO gewertet werden kann, wobei auch die OECD-Leitsätze eine Offenlegung zumindest gegenüber staatlichen Stellen als verpflichtend darstellen.\autocite[vgl.][37-38]{deutsche\_kontrollstelle\_eu-sorgfaltspflichten\_in\_rohstofflieferketten\_jahresbericht\_2023} Insbesondere in Verbindung mit der Frage der Rohstoffversorgungssicherheit scheint es sinnvoll, dass sich die Kommission bei ihrer Überprüfung im Dreijahresrhythmus nach Art. 17 II Konfliktmineralien-VO 
%hier noch genauer die Artikel anschauen hinsichtlich Wortlaut 

Es wird zudem festgestellt, dass die entsprechenden Verstoßverfahren den von der DEKSOR erwarteten Umfang und Komplexität überschritten, was in verzögerten Verfahrensdauern resultiert. 

Die beschränkten Befugnisse der nationalen Behörden weisen auf eine Schwäche im aktuellen Rohstoffverwaltungsrecht hin. Es zeigt sich die Notwendigkeit, die Durchsetzungsmechanismen zu stärken, damit die Behörden effektiver handeln und die Einhaltung der Sorgfaltspflichten sicherstellen können. Wenn nur eine Minderheit der Unternehmen die Pflichten einhält, besteht die Gefahr, dass die Wirksamkeit der Verordnung und der zugrunde liegenden Rechtsnorm untergraben wird. Dies könnte das Vertrauen in das Rohstoffverwaltungsrecht schädigen und zu einer Fragmentierung der Rechtsdurchsetzung führen. Eine Einführung von Maßnahmen zur Compliancesteigerung, z. B. Anreizsysteme, sind denkbar -- zumindest muss aber der Fokus auf wirksamen Durchsetzungsmechanismen liegen, nicht nur im Bereich der Konfliktmineralien-VO.

Es lässt sich außerdem argumentieren, dass das MinRohSorgG möglicherweise nur eine ungenügende Eingriffsgrundlage bei einer unzureichenden Erfüllung der unternehmerischen Auskunfts- und Mitwirkungspflichten bietet.  Dies wird deutlich, wenn man die Sanktionsmechanismen, Kontrollmöglichkeiten und Durchsetzungskraft dieses Gesetzes betrachtet. Als Bußgeld ist lediglich ein Zwangsgeld in Höhe von "bis zu 50.000 Euro im Verwaltungszwangsverfahren vorgesehen (§ 9).

%Verwaltungszwangverfahren: Eingriffshürde betrachten

Strafrechtliche Sanktionen wie im benachbarten Umweltstrafrecht sind nicht vorgesehen. m Vergleich zum MinRohSorgG bietet das deutsche Lieferkettensorgfaltspflichtengesetz, das seit 2023 in Kraft ist, deutlich schärfere Sanktionen. Es sieht nicht nur höhere Bußgelder vor, sondern ermöglicht auch den Ausschluss von Unternehmen aus öffentlichen Vergabeverfahren. 

Schließlich stellt die DEKSOR fest, dass Versuche von Unternehmen wahrgenommen werden, die erforderlichen Nachweise zum Beleg der Erfüllung der Sorgfaltspflichtsvorschriften auf andere Bereiche auszulagern, die nicht 

%Umgehungstatbestände



Diese Verzahnung bzw. Verbindung von Rechtsakten zeigt, dass dieses Vorgehen insbesondere im rohstoffverwaltungsrechtlichen Bereich sinnvoll erscheint. Die Übernahme internationaler Standards in das europäische Rechtssystem stärkt die Integration globaler Best Practices in die nationale Gesetzgebung und fördert die Kohärenz im Rohstoffverwaltungsrecht. Im Beispiel der Konfliktmineralien-VO erleichtert die Orientierung an den OECD-Leitsätzen sowohl zunächst die Anforderungen der EU-Verordnung, aber auch potentieller weiterer (internationaler), nicht zwangsläufig staatlicher Regelungen nachzukommen, sodass hier mit der Schaffung einer sektor- und rechtsübergreifenden Grundlage begonnen wird. Langfristig entstehen hier Vorteile sowohl auf Unternehmens- und Verwaltungsseite: Durch die Nutzung anerkannter und immer weiter verbreiteter Richtlinien und Leitsätzen wird der Compliance- und Verwaltungsaufwand auf beiderlei Seiten auf ein erforderliches Minimum reduziert, mit den entsprechenden positiven Effekten. Ferner wird widersprüchlichen Regelungen oder Schlupflöchern vorgebeugt. Den zunächst rechtlich nicht verbindlichen Leitsätzen kommt durch den Verweis in der Konfliktmineralien-VO schließlich trotz ihres Soft-Law-Status durch ihre normative Einbettung eine mittelbare Rechtswirkung zukommt: Die Konfliktmineralien-VO hebt die Leitsätze aus ihrem ursprünglichen Soft Law-Kontext heraus und verleiht ihnen innerhalb des Anwendungsbereichs (und nur hier) der Verordnung eine rechtliche Relevanz -- "von 'Soft Law' zu 'Hard Law'"\autocite[42]{teicke_gesetzliche_2018}. Ferner kann inn Fällen, in denen die Verordnung unspezifisch ist oder Interpretationsspielraum lässt, die OECD-Leitsätze zur Konkretisierung herangezogen werden. Auf der anderen Seite lässt sich hingegen argumentieren, dass die als Soft Law konzipierten Leitsätze durch den schlichten Verweis in Art. 4 lit. b) Konfliktmineralien-VO nicht formell durch ein Rechtsetzungsakt in "harte" Rechtsnormen überführt wurden, denn es kann argumentiert werden, dass der Verweis als ein schlichter Hinweis auf eine bewährte bzw. empfohlene Praxis verstanden werden kann, ohne aber eine rechtliche Bindungswirking (mittelbar oder unmittelbar) zu entfalten -- "Soft Law bleibt Soft Law". Zudem: Die Verordnung weist keine zwingende Normenkonkretisierung auf, das heißt die Verordnung selbst gibt den Adressaten einen gewissen Spielraum, \textit{wie} die Sorgfaltspflichten genau umzusetzen sind, sodass der Verweis schlicht als eine Empfehlung oder Vorschlag interpretiert werden kann, der keine strikte rechtliche Bindung (sofern diese Begriffskonstellation überhaupt als existierend gewertet werden kann) entfaltet und ggf. durch UE alternative, ebenfalls geeignete Maßnahmen zur Einhaltung der Sorgfaltspflichten ergriffen werden könnten (Orientierungshilfen-Argument). Denn schließlich bleibt auch unklar, inwieweit ein Verstoß gegen die OECD-Leitsätze, ohne gleichzeitig gegen die formellen Vorschriften der Verordnung zu verstoßen, tatsächlich zu rechtlichen Konsequenzen führen würde. Da die Leitsätze nicht selbst Teil des bindenden Rechts sind, könnte argumentiert werden, dass ihre Nichtbeachtung, solange die formellen Pflichten der Verordnung erfüllt werden, keine Sanktionen nach sich ziehen sollte. Dies schwächt das Argument für eine mittelbare Rechtswirkung ab. Diese Auslegungsfrage lässt sich auch in der gegenwärtigen Literaturlage wiederfinden: So argumentiert Grado\autocite{grado_eu_2018}, dass die Umsetzung der OECD-Leitsätze in der EU-Verordnung als ein Schritt in Richtung einer stärkeren Verbindlichkeit von Soft Law betrachtet werden kann, aber dass dies nicht gleichbedeutend mit einer vollständigen Rechtswirkung ist. Die Verordnung hat zwar das Potenzial, die Einhaltung der Leitsätze zu fördern, jedoch bleibt sie in ihrer Fähigkeit, diese Leitsätze in hartes Recht zu verwandeln, begrenzt. Im Kontext möglicher Konflikte zum deutschen ABG-Recht wird ebenfalls darauf hingewiesen, dass ein pauschaler Verweis auf die OECD-Leitsätze aus Transparenzgründen nicht zu empfehlen ist und zudem "wenig harmonisch" gestaltet worden sei.\autocite[42]{teicke_gesetzliche_2018}
Die Frage der unklaren mittelbaren Rechtswirkung hängt daher auch stark von der Auslegung und Anwendung im Rahmen des Rohstoffverwaltungsrechts ab. Weiterhin ist also noch nicht ersichtlich, inwieweit die Konfliktmineralien-VO diese Fragen klären und zukünftigen Entwicklungen beispielhaft dienen kann, insbesondere hinsichtlich der Soft-Law-Bindungsfrage.

%im Legislativprozess der VO checken, wie ein Verweis auf die Leitsätze argumentiert war

Dieses Vorgehen der Verzahnung kann also als ein beispielhaftes Leitbild für die weitere Entwicklung eines interdisziplinären und anwendungsbetontem Rohstoff- und Rohstoffverwaltungsrecht verstanden werden. Weitere Beispiele für solche Leitsätze lassen sich finden: %!

Hingegen existiert, wie oben ausführlich dargestellt, im Bereich der Verzahnung von zunächst rechtlich nicht bindenden mit sekundärrechtlichen Akten die "Soft-Law-Problematik" der mittelbaren Anwendbarkeit.

Ferner lässt die VO ausreichenden Gestaltungsspielraum für die die jeweiligen nationalen Umsetzungsmaßnahmen und insbesondere die Einführung von Regeln bei Verstößen gegen die Verordnung.\footnote{\textit{Siehe hierzu} Erwägungsgrund 20, VO 2017/821 sowie Art. 16.} So liegt beispielsweise die Höchststrafe 

Die Durchführungsbefugnisse hingegen "sollten der Kommission übertragen werden".\footnote{Erwägungsgrund 21, VO 2017/821.} 

%Analysieren: https://dip.bundestag.de/vorgang/.../255348

Die Verordnung somit zielt darauf ab, Transparenz und Verantwortlichkeit in den Lieferketten der Unternehmen zu erhöhen, indem sie sie verpflichtet, Informationen über die Herkunft und den Handel der von ihnen verwendeten Mineralien offenzulegen. Diese Verpflichtungen stehen in direktem Zusammenhang mit den möglichen Grundsätzen des Rohstoffwirtschaftsverwaltungsrechts, und das RWvR kann erneut als rechtlicher Rahmen zur Umsetzung sowie Einhaltung der Regelungen agieren. Interessant gestaltet sich hierbei nicht nur die Erwähnungsgrund 13 genannte Transparenzschaffung zur Vertrauensbildung gegenüber den Wirtschaftsbeteiligten, sondern auch eine Öffentlichmachung des rohstoffverwaltungsrechtlichen Aspektes -- es ließe sich argumentieren, dass durch die öffentliche Salienz hinsichtlich der Verfahren und Strategien sowie die eigentliche Erfüllung auch hier eine Art "Demokratisierung" des Rohstoffrechts stattfindet, auch aus Unternehmenssicht durch die eindeutige Verwaltungs-Unternehmens-Beziehung in dieser Angelegenheit und die zunächst eindeutige Regelung des Bereiches.

Die Regulierung von Konfliktmineralien ist ein entscheidender Bestandteil eines umfassenden Rohstoffwirtschaftsverwaltungsrechts, das sowohl die wirtschaftlichen Interessen als auch die ethischen Verpflichtungen Deutschlands und der EU berücksichtigt. Die bisher bestehenden gesetzlichen Regelungen, insbesondere die Verordnung  2017/821 und das Lieferkettensorgfaltspflichtengesetz, bieten eine Grundlage, die jedoch durch weitere rechtliche Entwicklungen und eine verbesserte Durchsetzung ergänzt werden muss. Ein integrierter Ansatz im Rahmen eines nationalen Rohstoffwirtschaftsverwaltungsrechts kann dazu beitragen, diese Herausforderungen zu bewältigen und gleichzeitig die Rohstoffversorgung weiterhin sicherzustellen und negative Effekte bei Unternehmen, aber auch wie schon beschrieben in den konfliktmineralienexportierenden Staaten zu vermeiden.

Nichtsdestotrotz kann festgehalten werden, dass es sich bei der Konfliktmineralien-VO um einen sehr kleinen und limitierten Regelungsbereich handelt, was sich insbesondere in der Zahl der erfassten Fälle widerspiegelt und daher nun eine geringere Relevanz für die Wirtschaft angenommen werden kann. Darüber hinaus ist die Konfliktmineralien-VO eindeutig dem internationalen Teil des Rohstoffverwaltungsrechts zuzurechnen und fällt ebenso in den Bereich des Wirtschaftsvölkerrechts, nicht zuletzt durch die eindeutigen Menschenrechtsbezogenen Aspekte im Rechtsakt. Ferner spricht die Konfliktmineralien-VO eindeutig nicht den Bereich der Rohstoffversorgung an, kann aber ohne Weiteres der Rohstoffverwaltung zumindest angerechnet werden.

\paragraph{Nearshoring und Konfliktmineralien}

Es liegt nahe, dass Unternehmen im Bereich der Konfliktmineralien eine Verlagerung von Importen, nichtg zwangsläufig in Form des Nearshorings, in Betracht ziehen können, um die Auskunftspflichten im Rahmen der Konfliktmineralien-VO zu reduzieren oder gar gänzlich zu vermeiden und so den unternehmerischen Dokumentationsaufwand einzustellen. Im DEKSOR-Report wird beschrieben, dass aus unternehmerischer Sicht die Aufwendungen für die Erfüllung der Sorgfaltspflichten nicht im Verhältnis zum erforderlichen Umsetzungsaufwand stehen würden, sodass entsprechende Verlagerungen in Betracht gezogen werden um die Tatbestandsmerkmale der Konfliktmineralien-VO bzw. des MinRohSorgG zu verlassen -- insbesondere durch ein Nearshoring des Rohstoffbezugs hin zu einem direkten Import aus EU-Mitgliedsstaaten. Auch die DEKSOR erkennt hier richtigerweise, dass hier ein Umgehungstatbestand infrage kommen könnte; dieser sei zwar nicht illegal, könne aber nicht als Sorgfaltsmaßnahme des Unternehmens selbst betrachtet werden, sondern als eine Verlagerung dieser Sorgfaltspflicht auf den dann infrage kommenden Unionseinführer in das Zollgebiet der Union.\autocite[vgl.][43]{deutsche\_kontrollstelle\_eu-sorgfaltspflichten\_in\_rohstofflieferketten\_jahresbericht\_2023}

%Bewertung von Umgehungstatbeständen
Berufen auf Vertrauen des Unionseinführers/Verhütter/Rohstoffproduzent

Auch die DEKSOR stellte hierbei fest, dass das Sorgfaltsvertrauen der (nachgelagerten) Unternehmen trotz Zertifizierungen nicht immer begründet sei, denn "eine Verlagerung der Sorgfaltspflichten auf andere Unternehmen verschiebt nur das Problem anstatt im Verbund auf transparente Lieferketten hinzuwirken" \autocite[43]{deutsche\_kontrollstelle\_eu-sorgfaltspflichten\_in\_rohstofflieferketten\_jahresbericht\_2023}

Nearshoring als Umgehungsstrategie

Juristisch gesehen kann die Verlagerung von Beschaffungsprozessen in stabilere, näher gelegene Regionen als eine direkte Reaktion auf die Anforderungen der Konfliktmineralien-Verordnung gesehen werden. Diese Verlagerung ist im Wesentlichen eine Form des Nearshoring, da sie den geografischen Schwerpunkt der Beschaffung näher an die Heimatmärkte bringt

Nur eine indirekte Verbindung zum Nearshoring weisen sekundäre Effekte der Konfliktmineralien-VO auf Unternehmen auf -- Compliance in Lieferketten ist im engeren Verständnis kein Bestandteil eines Nearshorings, kann aber durchaus als ein solches interpretiert werden, da letztendlich durch die Anwendung von Sorgfaltspflichten wie im Rahmen der Konfliktmineralien-VO Unternehmen dazu angehalten sind, ihre Lieferketten so zu gestalten, dass sie die mitunter strengen Anforderungen der VO erfüllen und somit über Nearshoring auf eine Lösung zurückgreifen, um Compliance-Kosten zu senken, die Kontrolle über die Lieferkette zu verbessern und Risiken zu minimieren, die mit weit entfernten, instabilen Lieferquellen verbunden sind. So wird Nearshoring zu einer strategischen Entscheidung, die durch rechtliche Vorgaben motiviert ist und die betriebliche Resilienz und Nachhaltigkeit fördert. Dies bestätigt auch die Beobachtung, dass die sog. \textit{Corporate Social Responsibility} (CSR), in der unternehmerischen Praxis immer weiter an Bedeutung gewinnt und sogar nur Empfehlungen im Sinne eines Soft Laws übernommen werden, die sich teilweise bis auf Lieferanten erstrecken.\autocite[39]{teicke_gesetzliche_2018}

Hier lassen sich aus rohstoffverwaltungsrechtlicher Perspektive weitere Vorteile zur Erfolgsmaximierung und Umsetzungsvereinfachung erkennen, denn Soft Law kann eine bedeutende und sinnvolle Ergänzung sowie Erweiterung des Rohstoffverwaltungsrechts darstellen, indem es flexiblere, anpassungsfähigere und dennoch wirksame Rahmenbedingungen schafft, die sowohl den regulatorischen Anforderungen als auch den praktischen Bedürfnissen der Wirtschaft und der staatlichen Verwaltung gerecht werden. „Hard Law“-Regelungen bieten zwar Rechtssicherheit, sind jedoch oft starr und weniger anpassungsfähig an die dynamischen Entwicklungen in globalen Lieferketten und Rohstoffmärkten. Soft Law, das typischerweise in Form von unverbindlichen Leitlinien, Empfehlungen, Verhaltenskodizes oder Branchenstandards existiert, bietet hier eine notwendige Flexibilität. Es ermöglicht eine schnellere Anpassung an technologische Innovationen, Marktveränderungen und neue ethische Anforderungen. Soft Law kann somit Lücken im bestehenden Rechtssystem füllen und eine dynamische Anpassung ermöglichen, die für eine effektive und zeitnahe Regulierung im Rohstoffsektor unerlässlich ist. Für staatliche Behörden bietet Soft Law den Vorteil, regulatorische Ziele durch kooperative Ansätze zu erreichen, anstatt ausschließlich auf Zwangsmaßnahmen zurückzugreifen. Soft Law kann als Instrument zur Förderung der Selbstregulierung innerhalb von Industrien dienen, wobei staatliche Kontrolle durch Beratung, Überwachung und Förderung von Best Practices ergänzt wird. Ein solcher kooperativer Ansatz reduziert den administrativen Aufwand und ermöglicht eine effizientere Allokation staatlicher Ressourcen. Zudem kann Soft Law als „Laboratorium“ für neue Regulierungsansätze fungieren, die, sobald sie sich in der Praxis bewährt haben, in verbindliche Rechtsnormen überführt werden können. Dies fördert eine schrittweise und praxisnahe Weiterentwicklung des Rohstoffverwaltungsrechts. Darüber hinaus kann CSR als ein international geprägtes Feld wahrgenommen werden, sodass Soft Law hier eine Art Brücke zwischen diversen nationalen und internationalen Rechtsordnungen schlagen kann.


\subsection{ACI VO}
Die Verordnung 


\subsection{Verwaltungsvollzug durch die Union}
Ferner, wie im CRMA deutlich wird, ist ein Großteil des EU-Rohstoffrechts der Verwaltungskoordinierung zuzuschlagen: insbesondere die Berichterstattung durch die Mitgliedsstaaten mit entsprechender Verwaltung durch die Kommission tritt hierbei hervor, wobei diese Verwaltung auch ihre Wirkung gegenüber den Wirtschaftsbeteiligten entfaltet. Die Union weist hier fast keine eigene Vollzugskompetenz auf und ist stattdessen auf den mittelbaren durch die Mitgliedsstaaten angewiesen. Der Vollzug der indirekten Verwaltung findet also nur in der Informationsgewinnung der Kommission und die entsprechende Nutzung dieser Informationen Ausdruck, was nahtlos in die Kontrollfunktion der Kommission übergeht.

Die Berichterstattung ist somit als zentrales Instrument der EU zur Verwaltung des Rohstoffrechts wahrzunehmen. Hierbei ist die Kommission ebenfalls, den Verwaltungsaufwand zu reduzieren, was beispielsweise durch delegierte Rechtsakte erreicht wird\footnote{vgl. hierzu exemplarisch CRMA Art. 34 S. 2} oder aber durch die Schaffung einheitlicher digitaler Schnittstellen zur Online-Bereitstellung.\footnote{CRMA Rn. 31}, wie auch bereits in der Vergangenheit durch die Kommission beabsichtigt.\footnote{Nicht im konkreten Rohstoffbezug, aber allgemein VO (EU) 2018/1724, der CRMA verweist in Rn. 31 auch ausdrücklich auf Art. 6 I der Verordnung.}
%auf Vollzug Unionsveraltungsrecht eingehen

Weitere Berichts- oder Informationsstellen auf mitgliedsstaatlicher Ebene wie die nationalen Rohstoffagenturen stehen hier ebenfalls der Kommission zur Verfügunug.
\begin{table}[h!]
	\centering
	\begin{tabular}{|l|l|}
		\hline
		\textbf{Mitgliedsland} & \textbf{Name der Rohstoffagentur} \\ \hline
		Deutschland            & Deutsche Rohstoffagentur (DERA) \\ \hline
		Frankreich             & Bureau de Recherches Géologiques et Minières (BRGM) \\ \hline
		Finnland               & Geological Survey of Finland (GTK) \\ \hline
		Schweden               & Geological Survey of Sweden (SGU) \\ \hline
		Portugal               & Laboratório Nacional de Energia e Geologia (LNEG) \\ \hline
		Österreich             & Geologische Bundesanstalt (GBA) \\ \hline
		Polen                  & Polish Geological Institute – National Research Institute (PIG-PIB) \\ \hline
		Tschechien             & Czech Geological Survey (ČGS) \\ \hline
	\end{tabular}
	\caption{EU-Mitgliedsländer mit Rohstoffagenturen}
	\label{tab:rohstoffagenturen}
\end{table}

%Dichte der Überwachung des Rohstoffrechts






\subsection{Rohstoffabkommen und Rohstoffpartnerschaften -- neuer Wein in alten Schläuchen?}
Rohstoffpartnerschaften gelten in der europäischen Rohstoffpolitik als strategisches Instrument zur Diversifizierung von Lieferketten und zur Reduzierung geopolitischer Abhängigkeiten, insbesondere gegenüber China. Seit 2021 hat die EU über ein Dutzend solcher Partnerschaften mit rohstoffreichen Drittstaaten geschlossen, etwa mit Kanada, Namibia, Chile und Kasachstan. Ziel ist es, durch politische Kooperationen den Zugang zu kritischen Rohstoffen zu sichern, lokale Wertschöpfung zu fördern und nachhaltige Bergbaupraktiken zu etablieren.


Die internationale Prägung des Marktes für mineralische Rohstoffe bedarf aus den bekannten Gründen keiner weiteren Einordnung.

Rohstoffabkommen und -partnerschaften lassen sich aus zweierlei Hinsicht betrachten: als Instrument der Rohstoffpolitik und als außengerichtete Komponente der Rohstoffpolitik und -verwaltung. Bei diesen Abkommen und Partnerschaften handelt es sich um zwischenstaatliche Kooperationsformate, die nicht primär handelsrechtlich verfasst sind, sondern einen Mix aus diplomatischen, wirtschaftlichen und entwicklungspolitischen Maßnahmen darstellen. Anders als klassische Handelsverträge zielen sie auf dauerhafte, strukturelle Kooperationen, etwa im Bereich Exploration, nachhaltige Gewinnung, Rückverfolgbarkeit und Technologietransfer. Juristisch lassen sich Rohstoffpartnerschaften teils als völkerrechtliche Vereinbarungen, teils als politische Absichtserklärungen einordnen, wobei ihre rechtliche Verbindlichkeit und institutionelle Ausgestaltung im Einzelfall stark variiert.

Internationale Rohstoffabkommen\footnote{Oder auch -übereinkommen -- hinsichtlich der völkerrechtlichen Bindungswirkung ist die Bezeichnung nicht ausschlaggebend, sondern vielmehr die vertragliche Rechtsnatur; siehe hierzu Art. 2 I lit. a des Wiener Übereinkommens über das Recht der Verträge, BGbl. 1985 II S. 926ff., 928 sowie ausführlich generell zur Stellung und Wirkung völkerrechtlicher Verträge innerhalb der nationalen Rechtsordnung der Vetragsstaaten Wissenschaftliche Dienste des Deutschen Bundestages, Rechtsverbindlichkeit völkerrechtlicher Abkommen sowie der Beschlüsse von Einrichtungen der VN, WD 2 - 3000 - 086/19, S. 4, 2019.} sind als völkerrechtlicher Vertrag einzstufen und binden zunächst Anbieter und Nachfrager im Sinne eines angemessenen Ausgleichs der jeweiligen Interessen.\autocite{Schorkopf, Rn. 43}

Art. 207 AEUV, Art. 21 EUV


Insbesondere durch Betätigung des UNCTAD\footnote{United Nations Conference on Trade and Development} bilden Rohstoffabkommen ein Instrument zur Sicherung sowohgl des Preisniveaus und der Rohstoffversorgung, insbesondere auch vor dem Hintergrund dass mehrheitlich Entwicklungsländer von im Rohstoffexport erzielten Einnahmen abhängig sind.\autocite{Herdegen IntWirtschaftsR/Herdegen, 13. Aufl. 2023, § 11. Rn. 3,} Rohstoffabkommen können hierbei an sich auch als internationale Organisation gestaltet sein.

Rohstoffabkommen treten hier selbstverständlich auch mit Mineralien-Bezug auf -- in der Vergangenheit oft inklusive einer direkten Steuerung des Marktes, was im Verlauf auch zum Scheitern solcher mineralischen Rohstoffabkommen geführt hat.\autocite{ntergouvernementale Rohstoffregimes im Zwielicht — Lehren aus dem Zinn-Debakel Ludwig Gramlich Verfassung und Recht in Übersee / Law and Politics in Africa, Asia and Latin America Vol. 20, No. 4 (1987), pp. 486-514} 

Die EU verfolgt im Rahmen ihrer Rohstoffpolitik verschiedene Typen von Partnerschaften: Strategische Partnerschaften mit Drittstaaten, etwa mit Kanada, Kasachstan oder Namibia, zielen auf langfristige Kooperationsformate zur Sicherstellung nachhaltiger Versorgung, während Technologie- und Kapazitätsaufbauprogramme (z. B. durch die EU-Rohstoffallianz, ERMA)  dem Know-how-Transfer und der Förderung lokaler Wertschöpfungsketten dienen. Partnerschaften unter dem EU \glqq Global Gateway Programm\grqq spielen aus Rohstoffsicht noch eine untergeordnete Rolle.\footnote{Zwar hat die Kommission im Rahmen des Global Gateways mit der DRK Ende 2023 ein MoU unterzeichnet, jedoch sind seitdem keine nennenswerten Enwticklungen festzustellen.}

Aktuell 

COM(2012) 82 final 

Die Rolle der strategischen Rohstoffpartnerschaften und ihre Bedeutung für die Lieferketten kritischer Rohstoffe wurde zudem durch die Kommission auch im Kontext strategischer Projekte in der Vorstellung des Industrial Action Plan for the European automotive sector erneut unterstrichen.\footnote{COM(2025) 95 final, S. 13.}

%RohPolPress

Dennoch zeigt sich in der Praxis, dass diese Partnerschaften häufig wenig verbindlich, politisch überformt und operativ schwer umsetzbar sind. Die meisten Vereinbarungen basieren auf Memoranda of Understanding (MoUs), die keine rechtliche Bindungswirkung entfalten und lediglich Absichtserklärungen darstellen. Diese Unverbindlichkeit führt dazu, dass Rohstoffpartnerschaften oft als „alter Wein in neuen Schläuchen“ erscheinen: Sie greifen auf bekannte Instrumente der Entwicklungs- und Außenwirtschaftspolitik zurück, ohne die strukturellen Defizite der europäischen Rohstoffversorgung – etwa die fehlende Verarbeitungskapazität oder die mangelnde Investitionssicherheit – wirksam zu adressieren. Zudem fehlt es an kohärenter Governance: Die Partnerschaften sind weder in ein einheitliches beihilferechtliches Regime eingebettet, noch existieren klare Mechanismen zur Durchsetzung von Umwelt-, Sozial- und Governance-Standards. Auch die Einbindung der Industrie erfolgt bislang nur punktuell, was die praktische Relevanz für rohstoffintensive Sektoren wie die Automobilindustrie einschränkt. Aus juristisch-politischer Sicht lässt sich daher argumentieren, dass Rohstoffpartnerschaften nur dann ein wirksames Instrument der Rohstoffverwaltung darstellen, wenn sie durch verbindliche rechtliche Rahmenbedingungen, finanzielle Absicherung und institutionelle Koordination flankiert werden. Andernfalls drohen sie, hinter ihren politischen Ambitionen zurückzubleiben und die strukturelle Rohstoffabhängigkeit der EU lediglich rhetorisch zu überdecken.


\subsection{Charakterisierung der EU-Rohstoffgesetzgebung}

Kommission: Die Kommission verfolgt 
Sie wird hierbei insbesondere vom Europäischen Green Deal beeinflusst

Die rohstoffpolitischen Aktivitäten der Kommission beschränkten sich zunächst nur auf Mitteilungen in Form von Strategien; erst mit der Verabschiedung des CRMA und der Konfliktmineralien-Verordnung sind hier die ersten konkreten Rechtsakte in Erscheinung getreten, sodass eine gewisse Diskrepanz zwischen Forderung und Strategieempfehlungen auf der einen Seite und tatsächlicher Umsetzung auf der anderen zu erkennen ist.

Parlament: 

Darüber hinaus ist das Parlament als ein vollwertiger Akteur im Bereich der Rohstoffpolitik zu klassifizieren

Rat:



Durch die fehlenden Durchsetzungskompetenzen beschränkt sich der rohstoffverwaltungsrechtscharakter der Union fast ausschließlich auf die rechtsetzende Tätigkeit, die Umsetzung und Durchführung obliegt den Mitgliedsstaaten, ebenso die Kontrolle des letzteren durch die betroffenen Akteure. Dies deckt die Erkenntnis, dass sich die Unionsaktivität auf ein Minimum beschränkt und der überwiegende Teil der Rohstoffpolitik von den Mitgliedsstaaten geprägt ist, was die Erörterungen im Folgenden Abschnitt rechtfertigt.

Es ist jedoch auch festzuhalten, dass der EU-Rohstoffgesetzgebung mitunter nur ein eingeschränkter Erfolg bescheinigt werden kann, denn sowohl im Bereich der Konfliktmineralien-VO als auch im CRMA sind bisher nur limitierte Erfolge sichtbar.

Es existiert keine ausdrückliche Zuständigkeit, ebenso wenig wie eine eindeutige Rechtsgrundlange, die unmittelbar angewandt werden kann, sowie die Problematik der Koordination bzw. viel mehr übergreifenden Integration der verschiedenen Interessen und Aktivitäten der Mitgliedsstaaten.

Zur Einleitung einer solchen Grundlage und Zuständigkeit verbleibt also zunächst die Ausgestaltung einer kohärenten Rohstoffpolitik, die sich dann hin zu einer Rohstoffzuständigkeit mit einem EU-Rohstoffrechtsrahmen und einer effektiven Rohstoffverwaltung als Instrument der Problematik widmet.


\subsection{Interimsfazit}
Der vorangegangene Teil hat die Verbindung von mineralischen Rohstoffe im Sinne dieser Arbeit und der Europäischen Union verdeutlicht. In der historischen Betrachtung konnte festgestellt werden, dass die Union durchaus bereits vor der \glqq Zeitenwende\grqq im rohstoffrechtlichen Bereich aktiv war.
Die Frage nach der Zuständigkeit \textit{per se} ist abschließend zu bejahen -- jedoch ist hierbei festzustellen, dass die Union von ihrer Kompetenz nicht vollumfänglich Gebrauch macht und sich die weitere Ausgestaltung in den Rechtsrahmen der Mitgliedsstaaten bewegt.
Die Organe der Union haben bisher nahezu alle Formen der Ausübung der Zuständigkeit gewählt, hierbei sticht jedoch der CRMA als erste Verordnung abzielend auf mineralische Rohstoffe im Sinne dieser Arbeit hervor.

Inwieweit nun die limitierte Aktivität der Union im Rahmen der geteilten Zuständigkeit durch die Mitgliedsstaaten aufgefüllt wird, soll im Folgenden genauer betrachtet
werden.

Rohstoffrechtliche Aktivität in Bezug auf kritische Mineralien zählt zu den jüngeren Betätigungsfeldern mitgliedsstaatlicher Politik
%hier näher darauf eingehen, welche Staaten hier wie aktiv sind




\section{Rohstoffe und Nationalstaaten}



\subsection{Zum Verhältnis nationaler Regelungen und Unionsrecht}
Unter Berücksichtigung des Schutzbereiches der unionsrechtlichen Regelungen zu Rohstoffen ist in Bezug auf nationale Maßnahmen der Vollständigkeit halber zu betrachten, wie weit der Handlungsspielraum der Mitgliedsstaaten ausfällt.

\subsection{Bergbau in den Mitgliedsstaaten der Union}

Da die Sicherung der Versorgungssicherheit auf die gesamte Union erstreckt, ist auch die Erschließung von mineralischen Vorkommen in der gesamten EU zu beobachten.	
%https://www.europarl.europa.eu/RegData/etudes/STUD/2022/729156/IPOL_STU(2022)729156_EN.pdf

Jedoch ist mit einer Dauer von 10-15 Jahren zwischen ENtdeckung und Abbau zu rechnen.

%https://fennia.journal.fi/article/view/87223

Im Januar 2023 gab der schwedische LKAB-Konzern bekannt, dass nahe der schwedischen Stadt Kiruna ein Vorkommen an seltenen Erden entdeckt worden sei 

Dass dieses Vorkommen keineswegs neu ist, zeigt die Experteneinschätzung, dass das Vorkommen bereits seit dem Zweiten Weltkrieg längst bekannt sei und bei Rentabilität eine Erschließung schon längst erfolgt worden wäre.\autocite{VDi Nachrichten: Seltene Erden: Deutschland importiert lieber, als selbst zu fördern}


Jedoch 

Auch stößt der geforderte Abbau in den Mitgliedsstaaten nicht immer auf Unterstützung: Gibt es hier Klagen?

\section{Rohstoffe und Deutschland}

Deutschland zählt aufgrund seines Status als eines der führenden Industrieländer zu den Großverbrauchern von Mineralien im Allgemeinen und kann seinen Bedarf insbesondere im Bereich der Steine und Erden aus heimischen Quellen decken; da die Metallerzgewinnung in Deutschland seit 1992 quasi nicht mehr existiert, ist es jedoch insbesondere im Bereich der metallischen Rohstoffe und damit auch bei den kritischen sowie strategischen auf Einfuhren angewiesen. \autocites{Bericht zur Rohstoffsituation in Deutschland 2023, S. 10ff.}{	Commodity TopNews 73, S. 3} Deutschland ist in hohem Maße auf den Zugang zu Rohstoffen angewiesen, die als kritisch oder strategisch gelten; Auch zur Sicherstellung des Bedarfs an Batterierohstoffen ist der Import umfangreicher Lieferungen erforderlich. Lithium bezieht Deutschland dabei hauptsächlich aus Chile. Bei anderen Technologiemetallen fällt der absolute Importumfang zwar geringer aus, dennoch bestehen oftmals erhebliche Abhängigkeiten – sei es von einzelnen Staaten wie China oder von wenigen Unternehmen, die eine nahezu monopolartige Stellung innehaben.




So wurde zudem konstatiert, dass die Bundesregierung sich ihrer Aufgabe entziehe, \glqq  als starker eigenständiger Akteur für mehr Rohstoffsouveränität zu sorgen\grqq.\autocite{Wirtschaft fordert mehr Tempo bei Rohstofffonds}

Der Bergbau im deutschen Bereich erfolgt sowohl unter Beachtung der einschlägigen Bundesgesetzgebung und den jeweiligen Landesgesetzen.

Im deutschen Raum obliegen Maßnahmen zur Rohstoffsicherung den Staatlichen Geologischen Diensten Deutschlands (SGD). Hierbei unterhält jedes Bundesland einen eigenen SGD. Diese unterstehen entweder dem jeweiligen Umwelt- oder Wirtschaftsministerium

Akitvität zu kritischen Mineralien

Die jeweiligen Rohstoffvorkommen sind naturgemäß regional standortgebunden und ungleich verteilt; der Zugang wird dabei zusätzlich erschwerend durch überlagernde Flächennutzungen eingeschränkt.\autocite{Bericht zur Rohstoffsituation in Deutschland 2023, S. 17}

Die deutsche Produktion von kritischen Mineralien fällt äußerst gering aus und umfasst darüber hinaus lediglich Rohstoffe für die Siliziumherstellung.\autocite{Commodity TopNews 73, S. 4} Jedoch sind durchaus Potentiale für die Gewinnung von kritischen und strategischen Rohstoffen in Deutschland vorhanden, die zwar regional ungleich verteilt sind, aber zumindest einen Teil des Bedarfs decken könnten. Jedoch befinden sich diese fast ausschließlich in frühen Projektphasen,\autocite{Commodity TopNews 73, S. 10-11} sodass eine Einschätzung entsprechend schwerfällt.

Die BGR kritisiert, dass trotz der Verfolgung von nachhaltigen Strategien bei den Explorations- und Bergbauunternehmen im deutschen Raum Genehmigungen oftmals aufgrund einer mangelnden öffentlichen Akzeptanz für die Rohstoffgewinnung scheitern.\autocite{Commodity TopNews 73, S. 14}

\subsection{Grundgesetz}

\subsubsection{BBergG}
Das Bundesberggesetz regelt die Exploration und den Abbau aller sog. bergfreien Bodenschätze -- somit auch für die Rohstoffgewinnung von Mineralien auf See.[ausführlich]\autocite{Jenisch II}

Die Fragmentierung der Rohstoffgesetzgebung erschwert somit eine einheitliche rohstoffverwaltungsgerichtete Betrachtung.

Parallelen zwischen dem deutschen Bergrecht und dem unionalen CRMA sind zu erkennen

zu 48 BBergG

\subsubsection{Rohstofffonds}
Der von der Bundesregierung aufgelegte Rohstofffonds kann zu den rohstoffverwaltungsbezogenen Instrumenten gezählt werden.

Der Fond sollte deutsche Unternehmen bei der Erschließung eigener, nicht-chinesischer Herkunftsquellen mit einer Bundesbeteiligung über die KfW unterstützen, jedoch wurde auch über ein Jahr nach Initiierung noch kein Projekt finanziell unterstützt, trotz entsprechenden Interesses aus der Wirtschaft. Kritistiert wurde zudem, dass das Finanzierungsbudget teilweise zu hoch angesetzt sei; die Bundesregierung kündigte zudem an, \glqq verzichtbare Bürokratevorgaben [zu identifizieren] und [den] Fonds besser an die Bedürfnisse der Wirtschaft [anzupassen]\grqq.\autocite{Wirtschaft fordert mehr Tempo bei Rohstofffonds}

\subsection{Interimsfazit}
Die Rohstoffsicherung in Deutschland mit ihrer Fragmentierung ist ein verkleinertes Abbild der Versorgungssicherungssituation auf europäischer Ebene.

Auch die BGR kommt zum Ergebnis, dass aufgrund einer fehlenden systematischen nationalen Erfassung von Genehmigungen, Bewilligungen und des allgeminen Status von Explorations- und potentiellen Projekten für kritische Rohstoffe \glqq eine umfassende Beurteilung der heimischen Rohstoffpotenziale derzeit nur sehr eingeschränkt möglich\grqq sei.\autocite{Top Commodity News 73, S. 14}

\section{Die Entstehung von Rohstoffstrategien}

Dennoch entwickelten die meisten westlichen Regierungen Strategien für kritische Mineralien und entschieden sich dann, sie nicht zu finanzieren. Die Hersteller sprechen von Resilienz, doch manche halten nur einen Wochenvorrat an Seltenerdmagneten auf Lager.

\subsection{Auf nationaler Ebene}
Stand xxxx haben xxxx europäische Länder eine nationale Rohstoffstrategie entwickelt

\subsubsection{Die Rohstoffstrategie der Bundesregierung}
Zum gegenwärtigen Zeitpunkt gilt die Rohstoffstrategie der Bundesregierung von 2020 als Grundlage.

Erstmals wurde 2010 eine solche Strategie vorgestellt;

Die BGR stellte 2011 fest, dass trotz der Bemühungen um eine heimische Versorgung aus Recycling absehbar sei, dass \glqq zukünftig nur diejenigen Unternehmen von den großen Marktchancen in vielen Bereichen der 	\glq Grünen Technologien\grq profitieren können, die sich in der Rohstoffversorgung mit Seltenen Erden abgesichert haben\grqq.\autocite{Top Commodity News 36, S. 8}

\subsection{Auf europäischer Ebene}

Im Rahmen des Modells des EU-Verwaltungsrechts obliegt der Vollzug und die Umsetzung grundsätzlich den Mitgliedsstaaten


Mitunter werden der unionalen Rohstoffpolitik Mängel diagnostiziert: 

\section{Wechselwirkungen zwischen deutschem und europäischen Rohstoffrecht}
Im Rahmen des Modells des EU-Verwaltungsrechts obliegt der Vollzug und die Umsetzung grundsätzlich den Mitgliedsstaaten

\section{Betrachtung weiterer ausgewählter Rechtsakte}

Dies dient insbesondere dem Vergleich 

\subsection{Dodd-Frank-Act}
Bereits 2010 und damit deutlich vor Legislativinititaven der EU hatten die USA über den \textit{Dodd-Frank-Act} \footnote{Dodd-Frank Wall Street Reform and Consumer Protection Act, Art. 1502-1504 (\textit{sections})} Transparenzpflichten im Umgang mit Konfliktmineralien in Lieferketten eingeführt. Der Gesetzesakt ist nicht primär auf den Bereich des Rohstoffrechts ausgerichtet, sondern enthält eher nebenbei Regelungen zu Konfliktmnineralien, insbesondere diejenigen des Art. 1502.  Hierbei ist der Ansatz zu beachten, dass nicht der Bezug der entsprechenden Rohstoffe eingeschränkt wird, sondern ein Reputationsrisiko für Unternehmen bei Rohstoffbezug in Konfliktregionen durch die Veröffentlichungspflichten entsteht.\autocite[Rn. 415]{ruttloff_lieferkettensorgfaltspflichtengesetz_2022} Eine relevante Schnittmenge zwischen dem Dodd-Frank-Act und dem LkSG kann jedoch nicht festgestellt werden, denn Unternehmen die bereits den Regelungen zur Transparenz des Art. 1502 unterliegen, können vom LkSG höchstens auf indirektem Wege profitieren, da der Handel mit Konfliktmineralien und die damit verbundene Vermeidung der Finanzierung bewaffneter Konflikte nicht im Fokus des LkSG stehen.\autocite[Rn 423]{ruttloff_lieferkettensorgfaltspflichtengesetz_2022} Stattdessen ist der Anknüpfungspunkte in EU-Verordnung 2017/821 zu suchen, die in Erwägungsgründen explizit eine dem Dodd-Frank-Act-ähnliche Rechtsvorschrift erwähnt.\footnote{2017/821, Erwägungsgrund 9: "[...] forderte das Europäische Parlament die Union auf, mit dem US-amerikanischen Gesetz über Konfliktminerale, Artikel 1502 des Dodd-Frank-Gesetzes (...), vergleichbare Rechtsvorschriften zu erlassen.} 

\subsection{Kimberley-Prozess}
Das \textit{Kimberley Process Cetification} Schema stellt einen Prozess zur Unterbindung des Handels mit Diamanten dar, durch dessen Verkauf kriegerische Auseinandersetzungen und Konflikte finanziert werden (\glqq Blutdiamanten\grqq oder Konfliktdiamanten)


Der Prozess kann grundsätzlich als Vorbild für eine entsprechende \glqq Kopie\grqq für den Bereich der seltenen Erden bzw. kritischen/strategischen Rohstoffe dienen -- so z. B. insbesondere hinsichtlich des Konflikts in der Demokratischen Republik Kongo (DRK/DRC), was ebenfalls von der EU-Außenbeauftragten Kallas erwähnt wurde.\footnote{\glqq We need something like this for essential raw materials […] if a country is attacking another country, taking over the mines and selling these raw materials as if they were its own and can finance its war with these raw materials\grqq; European Newsroom: Blindspots, raw materials, aid visibility: Kallas on development work and the EU’s presence in Africa, 14.02.2025. %https://europeannewsroom.com/blindspots-raw-materials-aid-visibility-kallas-on-development-work-and-the-eus-presence-in-africa/} 
\end{document}