\documentclass[12pt,a4paper,oneside]{book} % 'oneside' für einseitigen Druck

% Kodierung, Sprache und Schrift
\usepackage[utf8]{inputenc} % Erlaubt die Verwendung von Umlauten
\usepackage[T1]{fontenc} % Bessere Schriftkodierung
\usepackage[ngerman]{babel} % Deutsche Lokalisierung

% Schriftarten
\usepackage{lmodern} % Modernere Schriftart, gut für Skalierbarkeit und Lesbarkeit

% Für Abbildungen
\usepackage{graphicx}
\graphicspath{{bilder/}} % Verzeichnis, in dem Bilder gespeichert sind

% Für Tabellen
\usepackage{booktabs}

% Für Links und PDF-Metadaten
\usepackage[hidelinks]{hyperref}
\hypersetup{
	pdftitle={Titel der Dissertation},
	pdfauthor={Autor},
	pdfsubject={Doktorarbeit in den Sozial- und Rechtswissenschaften},
	pdfkeywords={Schlüsselwörter},
}

\usepackage{array}
% Für Bibliographie - Anpassung für Geisteswissenschaften
%%\usepackage[style=authoryear-icomp,backend=biber]{biblatex}
%%\usepackage[backend=biber, style=authoryear-icomp]{biblatex}
%%\usepackage[backend=biber, style=verbose-trad1]{biblatex}
\usepackage[backend=biber, style=verbose-inote]{biblatex}
%%\usepackage{biblatex}
\addbibresource{literatur.bib} % Name der BibTeX-Datei
\DeclareNameAlias{author}{family-given} % Nachname des Autors zuerst

% Anpassung der Nummerierung mit Punkten
\renewcommand{\thechapter}{\arabic{chapter}.} % Kapitel: 1., 2., 3., ...
\renewcommand{\thesection}{\Alph{section}.} % Abschnitt: A., B., C., ...
\renewcommand{\thesubsection}{\Roman{subsection}.} % Unterabschnitt: I., II., III., ...
\renewcommand{\thesubsubsection}{\arabic{subsubsection}.} % Unterunterabschnitt: 1., 2., 3., ...
\renewcommand{\theparagraph}{\alph{paragraph}.} % Absatz: a., b., c., ...


% Für Fußnoten
%%\usepackage[bottom]{footmisc} % Fußnoten am Seitenende


% Anpassung der Kapitelüberschriften
%\usepackage{titlesec}
%\titleformat{\chapter}[hang]{\Huge\bfseries}{\thechapter\quad}{0pt}{\Huge\bfseries}

% Abstand der Fußnoten
\setlength{\footnotesep}{0.5cm}

% Tiefe der Nummerierung und des Inhaltsverzeichnisses
\setcounter{secnumdepth}{4} % Nummerierungstiefe einstellen
\setcounter{tocdepth}{4} % Inhaltsverzeichnistiefe einstellen

% Abstand zwischen Absätzen und kein Einzug
%\usepackage{parskip}
%\setlength{\parskip}{0.5em}
%\setlength{\parindent}{0pt}

% Für Abkürzungsverzeichnis
\usepackage[printonlyused]{acronym}

% Für Zitate und Theoreme (falls benötigt)
\usepackage{csquotes}

% Für Gesetzestexte, Zitate und andere strukturierte Texte
\usepackage{enumitem}

% Zeilenabstand auf 1.3
\usepackage{setspace}



% Beginn des Dokuments
\begin{document}
	% Hier beginnt der eigentliche Inhalt der Arbeit
	% ...
	
\chapter{Rohstoffpolitische Zuständigkeiten und Aktivität auf europäischer und nationaler Ebene}

Dieses Kapitel betrachtet im Sinne einer Rechtsanalyse Kompetenzen, Zuständigkeiten und rechtssetzende Aktivität sowohl auf der europäischen Ebene, als auch auf Ebene der Mitgliedsstaaten, hioerbei vorrangig Deutschland sowie zur Komplettierung ein skizzenhafter Überblick ausgewählter Fallbeispiele aus weiteren EU-Mitgliedsstaaten.	
	
	
	Insofern - aufgrund der mangelnden Beziehung Staat-Bürger im konkreten Fall der Rohstoffverwaltung ist also nahezu ausschließlich die Domäne des Wirtschaftverwaltungsrechts für diese Arbeit ausschlaggebend.
	
	\section{Rohstoffe und die EU}
	
	
	\subsection{Zuständigkeiten der EU im Bereich der Rohstoffe}
	
	Zunächst: Eine dezidierte Rohstoffkompetenz der Union existiert zum aktuellen Zeitpunkt nicht. Dies erfordert also einen Rohstoffbezug der Politiken und Kompetenzen der Union, um eine Handlung überhaupt zu ermöglichen. Im Folgenden werden daher solche Politiken betrachtet, die im Sinne einer Rohstoffpolitik, eines Rohstoffverwaltungsrechts und insbesondere für die Rohstoffversorgung im Rahmen der Definition der Rohstoffe dieser Arbeit als relevant durch ihren Rohstoffbezug angesehen werden. Im Rahmen der Versorgung und Verwaltung sind hier also besonders der Abbau, die Beschaffung und Einfuhr sowie die außenkompetenzbezogenen Bereiche relevant, andere Bereiche werden daher nur bei Notwendigkeit mit in die Berachtung einbezogen.
	
	Daraus ergeben sich hier Intersektionsfragen insbesondere im Bereich der horizontalen Kompetenzabgrenzuung, im zweiten Schritt solche der nach vertikaler Abgrenzung und schließlich der Ausgestaltung auf nationaler Ebene.
	
	Die Frage nach der Zuständigkeit und insbesondere der Handlungskompetenz der Union zur Rohstoffverwaltung erfordert eine genaue Betrachtung verschiedener Politik- und Kompetenzbereiche der EU.
	
	Bereits die Europäische Gemeinschaft für Kohle und Stahl (EGKS) als eine der Vorläuferorganisationen der heutigen EU, die durch den Vertrag von Paris 1951 gegründet wurde, kann als erste bedeutende europäische Institution zur Verwaltung von Rohstoffen im Allgemeinen betrachtet werden. Durch die Koordinierung der Kohle- und Stahlproduktion in ihren Mitgleidsstaaten wurde so eine erste Form einer Rohstoffverwaltung geschaffen, insbesondere auch aus strategischen Aspekten.%\autocite{}
	
	Eine wirkliche Aktivität der Europäischen Kommission hinsichtlich der Diverisifzierung der Rohstoffpolitk erfolgte dann erst 1994 mit der Veröffentlichung eines Weißbuches u. A. zur Rohstoffpoliitk, welches bereits dort die Bedeutung einer sicheren und nachhaltigen Rohstoffversorgung für die europäische Industrie betonte.\footnote{Europäische Kommission: Weißbuch - Wachstum, Wettbewerbsfähigkeit, Beschäftigung. Herausforderung der Gegenwart und Wege ins 21. Jahrhundert, COM(93)700}
	Das auch als \glqq Delors-Weißbuch\grqq bezeichnete  fordert eine stärkere Integration von Umwelt- und Wirtschaftsstrategien und hob die Notwendigkeit einer Diversifizerung der Rohstoffquellen vor.
	
	%Untersuchen inwieweit welche Rechtsbereiche hier schont erwähnt und flankiert wurden
	
	Das Weißbuch von 1994 kann somit als Grundstein für spätere Strategien der Kommission und Initiativen der EU betrachtet werden, denn es markierte die erste Verschriftlichung eine strategischen Neuordnung der EU\footnote{Historisch korrekt müsste in Ausführungen, die die Zeitleiste der EU betroffen, von den Vorläuferorganisationen EWG bzw. EG die Rede sein.} hin zu einer integrierteren Wirtschafts- und Industriepolitik, die auch die nachhaltige Verwaltung von Rohstoffen miteinschließt. Die sichere Rohstoffversorgung wird als Notwendigkeit der Wettbewersbfähigkeit der europäischen Industrie herausgestellt: Die EU müsse in einer globalisierten Wirtschaft eine aktive Rolle einnehmen, um den Zugang zu  Rohstoffen auf internationalen Märkten zu gewährleisten, denn aus strategischer Sicht würde die \glqq (...) äußerst große (...) Rohstoffabhängigkeit der Gemeinschaft\grqq von nichteruopäischen Märkten verringert, was entsprechende Einspareffekte zur Folge hätte, aber auch eine Kräftigung der Wettbewersbfähigkeit.\footnote{KOM(94) 700, S. 178.} Es ist jedoch hervorzuheben, dass diese Erkenntnisse im Weißbuch noch als \glqq sekundäre Vorteile\grqq von umweltpolitischen Aktivitäten eingestuft wurden. Eine wirkliche strategische Erkenntnis kann hier also nicht unterstellt werden.
	
	Diese historische Verbindung legt dennoch im Sinne einer solchen Auslegung nahe, dass die Verwaltung von Rohstoffen bereits zu Beginn sowohl politische als auch wirtschaftliche Aspekte umfasst, die heute sowohl in der Umwelt- als auch Energiepolitik der Union zu finden sind.
	
	Nichtsdestotrotz ist festzustellen, dass eine weitere normative Aktivität der Union trotz der vergleichsweise frühen Berücksichtigung rohstoffpolitischer Aspekte dann doch erst zu deutlich späterem Zeitpunkt erfolgte -- in besonderer Weise erst nach den Erkenntnissen im Laufe der Zeitenwende\footnote{Zur Begrifflichkeit der Zeitenwende sei hier erneut auf Kapitel X verwiesen.} und im Kontext dieser Arbeit besonders bei Veranschaulichung, dass Überlegungen und Forderungen zur Umstellung auf fossilfreie Antriebe (i.e. EV-Fahrzeuge) auch bereits seit geraumer Zeit zu regisrieren sind,\footnote{darauf eingehen} und besonders im Lichte der Energie- und Rohstoffintensität dieses Trends doch überrascht, da zudem hier ein gewisses Eingriffspotenzial mit variierender Relevanz für die Unionsschutzgüter vorliegt. 
	
	Im Rahmen des Prinzips der begrenzten Einzelermächtigung, verankert in Art. 5 EUV  und insbesondere Abs. II (Die Union handelt nur innerhalb der Grenzen der Zuständigkeiten, die ihr von den Mitgliedstaaten in den Verträgen zur Verwirklichung der darin festgelegten Ziele übertragen wurden), wird die Union nur in den Bereichen tätig, die ausdrücklich in den Verträgen festgelegt sind, und  die Souveränität der Mitgliedstaaten in allen anderen Bereichen zunächst gewahrt bleibt (Abs. III).
	
	Zumindest im vorliegnden Fall ist es fraglich, ob Art. 4, 194 AEUV einschlägig für den hier betrachteten Anwendungsfall sind, kann doch die Rohstoffversorgung mit kritischen Mineralien und Rohstoffen bei engerer Auslegung nicht zwangsläufig in den von den Artikeln umfasstden Bereich der Energie bzw. Energieversorgungssicherheit fallen.
	
	Interessant gestaltet sich die Auslegung von Art. 3 AEUV in Bezugnahme auf Roshtoffpartnerschaften, die in %\ref{label}
	genauer beleuchtet werden; die EU hat also hier die ausschließliche Zuständigkeit für die Zollunion und die Festlegung der Handelsregeln, sodass diese Kompetenzes der EU ermöglicht, Handelsabkommen mit Drittstaaten zu schließen, die auch die Versorgung mit Rohstoffen regeln können.
	
	Ferner ist im jeweiligen betroffenen Bereich zu prüfen, inwieweit die in Art. 288 AEUV beschriebenen Handlungsinstrumente den EU-Organen für die jeweilige Rechtssetzung zur Verfügung stehen.
	
	Im Bereich der Auslegung der einzelnen Unionspolitiken in ihrem Bezug zur Rohstoffvkompetenz der EU scheint daher eine teleologische Auslegung zielführend. Kompetenzfragen sind zunächst anhand der Abgrenzung von Spezialität und Subsidiarität zu beantworten.\autocite{Callies, Berliner Beiträge 137, S. 39f.}
	
	Auch wenn Rechtsprechung des EuGH im Bereich der Rohstoffpolitik bisher nicht erfolgt ist, kann auf Basis der ständigen Rechtsprechung des Gerichtshofes festgehalten werden, dass die Kompetenzabgrenzungskriterien anhand der \glqq Schwerpunkttheorie\grqq\footnote{Siehe exemplarisch EuGH-Rechtsprechung Kommission gegen Rat (Titaniumdioxid), C-300/89.} festgestellt werden: Im Falle von zwei infrage kommenden Kompetenzgrundlagen wird nach dieser Theorie verfahren, sodass hier der entsprechende inhaltliche objektive Regelungsschwerpunkt geprüft wird\autocite[Callies, Art. 175, Rn 22 sowie Kahl, Art. 95, Rn 73ff]{callies_euvaeuv_2022}. Jedoch tut sich gelegentlich der Vorwurf der Beliebigkeit oder Rechtfertigung dieser Spruchpraxis auf.\autocite{Nettesheim, Grabits/Hilf/Nettesheim, AEUV Art 194 Rn. 35} Hierbei kann aber die Kompetenz sowohl vertikal als auch horizontal vorliegen.\autocite[30f.]{callies_stellungnahme_2007} Steht jedoch fest, dass ein Rechtsakt mehrere miteinander verbundene Zielsetzungen oder Rechtsbereiche umfasst und eine Zweitrangigkeit oder Nebensächlichkeit einer der Bereiche nicht vorliegt, so können als Kompetenzgrundlage im Ausnahmefall mehrere Rechtsgrundlagen herangezogen werden.\footnote{C-178/03, Rn. 43; C-94/03, Rn. 36; I-9713, Rn. 23;  I-869, Rnr. 38} Dies sei aber keine optimale Lösung, denn das Problem des Konkurrenzverhältnis wird bestenfalls verschoben, sodass gilt dass ein Rechtsakt durch seine Maßnahmen hauptsächlich auf einen Politikbereich zurückzuführen ist und nur indirekte, geringe Effekte auf solch andere Bereiche der Union hat, ist dann auf ebenjenen Einzeltitel zu stützen, sodass die Doppelabstützung nicht in Beracht gezogen wird und anhand eines \glqq objekltiv zu ermittelnden Schwerpunkt[s]\grqq durch den eigentlichen Regelungsgehalt sowie die tatsächliche Sachnähe zum Bereich mit erkennbaren Zielen eine klare Kompetenzgrundlage ausgewiesen wird.\autocite{Callies, Berliner Beiträge 137, S. 40} Im Falle der Kollision verschiedener unvereinbarer  Rechtssetzungverfahren ist eine Doppelabstützung nicht anzuwenden, sodass der Kompetemzgrundlage mit der weiteren Beteiligung Vorrang eingeräumt wird.\autocite{Callies, Berliner Beiträge 137, S. 40}
	Es ist jedoch eine Verpflichtung zur Abstützung auf mehreren (und somit auch mehr als zwei) Rechtsgrundlagen geboten -- zumindest dann, wenn der Rechtsakte mehrere Ziele in sich vereint bzw. solche Bestandteile enthält, die als zusammengehörig gelten ohne aber hierbei eine Zweitrangigkeit einzuräumen, tritt dann die Verpflcihtung zur Gewährung der Ausnahme ein dass entsprechender Rechtsakt auf die \glqq verschiedenen einschlägigen Rechtsgrundlagen gestützt werden\grqq muss.\footnote{So der EuGH, C-178/03, Rn 43.} Im Rohstoffverwaltungsbereich ließe sich also bereits nun argumentieren, dass eine Feststellung eines Schwerpunkts nur schwierig gelänge und zudem die Ziel- als auch Inhaltsvorstellungen diverse Elemente verschiedener Politiken solchermaßen untrennbar miteinander verknüpft sind, dass ein Rückgriff auf mehr als zwei einschlägige Rechtsgrundlagen geboten ist. Ein Spezialitätscharakter ist bei keiner der Normen zu erkennen, sodass die Frage des Schwerpunkts stellt, wobei es einer genauen Prüfung bedarfen, inwieweit welche Ziele als gleichrangig zuz betrachten sind. Jedoch stellt der EuGH ebenfalls klar, dass es bei der Abweichung von der Schwerpunktfeststellung um eine Ausnahme handelt -- es kann also nicht erwartet werden, dass diese Ausnahme für eine Rohstoffpolitik zur dauerhaften Lösung wird, auch um Kohärenz des Ausnahmecharakters zu wahren und zudem die Spezialitäts- bzw. Subsidairätsanfälligkeit der Rohstoffpolitik zu reduzieren bzw. zu vermeiden.
	
	Die Notwendigkeit der detaillierten Betrachtung ergibt sich schon allein aus der Tatsache, dass eine eindeutige Zuordnung bzw. eine Schwerpunktfeststellung auf Anhieb nicht gelingt.
	
	\subsubsection{Handelspolitik}
	
	
	Analog zur internationalen Dimension des Energietitels aufgrund der Notwendigkeit der internationalen Kooperation im Bereich des Klimaschutzes und der steigenden Abhängigkeit der Union von Energieimporten\autocite{Gundel, Theobald/Kühling, Europäisches Energierecht, V. Rn. 86} ergibt sich die internationale Dimension aus denselben Aspekten und insbesondere aus der zweitgenannten Tatsache nahezu automatisch für den unionalen Rohstofftitel.
	
	Eine Stützung einer Energie-Außenpolitik könne zumindest nicht auf Umwelt- oder Binnenmarktkompetenz erfolgen, sodass der Energietitel einschlägig erscheint, wenn auch nicht explizit.\autocite{Gundel, Theobald/Kühling, Europäisches Energierecht, V. Rn. 87}
	
	Seit 2020 kann eine Orientierung der Unionshandelspolitik im Kontext von geopolitischen Entwicklungen und Techgnologiewandel hin zu der Verfolgung einer \glqq offenen strategischen Autonomie\grqq beobachtet werden, auch zur Verbesserung der wirtschaftlichen Sicherheit und zur Erzielung einer hohen Ausprägung von Offenheit und Dynamik, besonders durch die Nutzung von Instrumenten wie Antidumping-/Antisubventionsverfahren, Ein- und Ausfuhrkontrollen, Lieferkettenregulierung sowie eine resilienz- und wertebasierte Handelspolitik.\autocite{Paschke, RdTW 2024, 206; Schäffer/Hach, ZRP 2023, 207f} Die Maximen dieser offenen strategischen Autonomie bestehen aus einem Resilienzaufbau der EU vor dem Hintergrund der (Geo-)Politisierung des Handels durch die Schaffung von leistungsfähigen Wertschöpfungsketten, einer Identifizierung von Abhängigkeiten und demntsprechendes De-Risking, während die Werte und Ziele der Union sowie eine regelbasierte globale Ordnung weiter vertreten werden.\autocite{Schäffer/hach, ZRP 2023, 207}  Es wird somit ein sog. \textit{managed trade} verfolgt unter einer \glqq skeptischen Offenheit\grqq, also unter politischen und strategischen Gescihtspunkten gestalteter Handel zur Sicherstellung wirtschaftlicher Autonomie, ohne aber die primärrechtliche Grundlage der Union hierbei zu verändern, denn die Verträge scheinen die hierfür notwendige Flexibilität bereitzustellen.\autocite[so zumindest]{Paschke, RdTW 2024, 206, 216; Schäffer, EuZW 2023, 695, 700} 

	Im Rahmen der handelspolitischen Schutzinstrumente sticht im Rahmen der Rohstoffversorgung das Instrument des Antidumpings hervor. 
	Wie bei der Aktivität im Rahmen eines Antisubventionsverfahrens gegen chinesische EV-Fahrzeuge sichtbar wurde, besteht durchaus die Möglichkeit weiterer Aktivität im Rahmen des sog. \glqq Staatsdumpings\grqq, bei dem ausgesprochen oft künstlich niedrige Preise für Rohstoffe den Auslöser darstellen, insbesondere in Kombination mit stark eingeschränktem Zugang zum Ausfuhrlandsmarkt sowie bei Vorliegen von unvollständigem Wettbewerb und Marktsegregation.\autocite{Hoffmesiter, Hrenzler/Herrmann/Niestedt, EU Außenwirtschaftrecht, VII 70 Rn14ff} Auch wenn die gemeinsame Handelspolitik der Union in den Verträgen nicht definiert ist und durch Art. 207 I AEUV lediglich eine Andeutung zu regelnder Bereiche und die dafür erforderlichen Instrumente ansprechen, so sind die Organe der Union bei der Gestaltung handelspolitischer Maßnahmen nicht zur Anwendung protektionistischer Maßnahmen bedingt durch den Freihandelscharakter des unionalen Handels befugt -- vielmehr kann über Art. 3 V EUV i.V.m. Art. 21 2 EUV eine rechtsverbindliche Grundlage Maßnahmen zur Schaffung von Gerechtigkeit auf Basis einer regelbasierten Handelspolitik gefunden werden.\autocite{Paschke, RdTW 2024, 216.} Die Organe der Union sind also befugt, im Rahmen des Vollzugs von Zielen des Unionsrechts entsprechende Maßnahmen zur Sicherstellung eines fairen und dennoch freien Handels zu treffen und entsprechende handelspolitische Instrumente einzusetzen, sollte dies im Interesse der Union sein.\autocite{Paschke, RdTW 2024, 206.; Callies/Ruffert Art 206 AEUV 8f; Grabitz Art. 206 AEUV Rn. 12} Der EuGH betonte zudem mehrfach die weite Gestaltungsfreiheit der Organe bei der Einführung handelspolitischer Schutzmaßnahmen im Rahmen der gemeinsamen Handelspolitik,\footnote{Rs. 245/81 (Edeka Zentrale)} auch in Fällen von Antidumpingmaßnahmen\footnote{Rs. C-358/89), wobei der Ermessensspielraum der Organe bei der Bewertung ökonomischer Faktoren im Handelsbereich im Vordergrund stand} und noch präziser bei der genauen Ausgestaltung von Zöllen und der Bestimmung von \glqq Dumping\grqq.\footnote{Rs. C-16/90 (Nölle)}. Diese Entscheidungen verdeutlichen die \textit{de facto} weite Handlungsfreiheit der Organe in handelspolitischen Fragen, solange sie sich im Rahmen der Verfahrensvorschriften und Grundprinzipien der EU bewegen. Insbesondere im Lichte des gegenwärtigen Ausfalls des WTO-Berufungsgremiums (Dispute Settlement Body, DSB) ist daher auch die Reform solcher bestehender Instrumente auf Unionsebene\footnote{Verordnung (EU) 654/2014 bzw. Verordnung 2021/167; Verordnung 2015/1843} hinsichtlich Handelsmaßnahmen von Interesse.\autocite{Schäffer, EuZW 2023, 695, 698}
		
	Demnach besitzt die Union einen entsprechenden handelspolitischen Spielraum, um auf die entsprechenden Herausforderungen einzugehen, denn die Leitlinien sind im Primärrechtsspielraum zu verorten\autocite{Paschke, RdTW 2024, 206, 216} -- eine \glqq Zeitenwende\grqq liege hier aber nicht vor, denn die Umorientierung betreffe lediglich die Programmatik, und trotz der Abwendung von tradiotionellem Freihandel sei die Umorientierung \glqq keine die Grundfesten erschütternde Veränderung\grqq der EU-Handelspolitik.\autocite[So argumentieren]{Paschke, RdTW 2024, 206, 216; Müller-Ibold/Herrmann EuZW 2022, 1029} Dem ist nicht uneingeschränkt zuzustimmen.  Der Übergang zur strategischen Autonomie stellt nicht nur eine programmatische Änderung dar, sondern beeinflusst grundlegende Werte der EU-Handelspolitik. Laut dem Gutachten der European Parliamentary Research Service (2022) bekräftigt die EU zunehmend handelspolitische Souveränität gegenüber globalen Lieferkettenrisiken und geostrategischen Spannungen. Diese systematische Neuorientierung kann als Paradigmenwechsel gewertet werden, da sie eine Abkehr von einer rein liberalen Freihandelspolitik hin zu einer sicherheits- und resilientzorientierten Strukturpolitik darstellt. die EU durch diese strategische Neuausrichtung das Prinzip des freien Marktzugangs zugunsten einer selektiveren Marktöffnung aufweicht, was das europäische Binnenmarktrecht sowie die GATT-Prinzipien der WTO indirekt beeinflusst. Die damit verbundenen protektionistischen Elemente unterscheiden sich erheblich von früheren Programmatiken der EU, die sich stark am Prinzip des globalen Wettbewerbs orientierten (vgl. Matsushita, Schoenbaum, Mavroidis, "The World Trade Organization: Law, Practice, and Policy", Oxford University Press, 2020) Mit dem Konzept der "strategischen Autonomie" und "Derisking" verschiebt sich der Fokus von bloßem Freihandel hin zu handelspolitischen Instrumenten, die Sicherheits- und Nachhaltigkeitsaspekte priorisieren. Diese Entwicklung zeigt sich in der zunehmenden Regulierung von Lieferketten und der Implementierung von Mechanismen wie dem Anti-Coercion Instrument und dem Foreign Subsidies Regulation. Durch die Einführung solcher Instrumente verändert die EU ihre rechtliche Haltung im Außenhandelsrecht wesentlich und gestaltet neue Handlungsspielräume (vgl. European Commission, "Trade Policy Review - An Open, Sustainable and Assertive Trade Policy", 2021).
	
	%Die Orientierung an strategischer Autonomie und Derisking ist nicht nur eine Anpassung innerhalb des bestehenden Systems, sondern stellt eine Neudefinition dar, die tief in die Grundprinzipien des europäischen und internationalen Handelsrechts eingreift. Die EU weicht von ihrem bisherigen Kernziel des Freihandelsprinzips ab und etabliert gezielt schützende, selektive Marktzugänge. Der Bruch mit dem bisherigen Gleichgewicht zwischen Offenheit und Wettbewerbsneutralität – Prinzipien, die sich etwa im Vertrag über die Arbeitsweise der Europäischen Union (AEUV), insbesondere in den Artikeln 206 und 207 AEUV, ausdrücken – ist tiefgreifend. Diese Artikel stellen Offenheit und das Wettbewerbsprinzip als Basis der EU-Handelspolitik heraus und definieren deren „essentielle Funktionen“ als stark an Freihandelsgrundsätzen orientiert (vgl. Craig/de Búrca, EU Law: Text, Cases, and Materials, Oxford University Press, 2020). Der Paradigmenwechsel manifestiert sich in neuen legislativen und regulatorischen Instrumenten wie dem Anti-Coercion Instrument und der Foreign Subsidies Regulation. Diese Instrumente markieren eine Rechtsverschiebung, die nicht nur programmatisch, sondern auch materiell-rechtlich bedeutend ist. Sie dienen dazu, externe Einflussnahmen aktiv zu steuern und zu beschränken, was einen deutlichen Bruch mit dem früheren Fokus auf den Schutz eines fairen Wettbewerbs darstellt. Ein solcher Schritt entfernt sich von den Grundsätzen des GATT Art. I und Art. XI, die Freihandel und Gleichbehandlung als zentrale Säulen der WTO festschreiben. Der Europäische Gerichtshof (EuGH) hat mehrfach betont, dass die EU bei der Gestaltung ihrer Handelspolitik grundsätzlich im Einklang mit den WTO-Regeln handeln muss (siehe Gutachten 1/94 und Gutachten 2/15 des EuGH), was die Einführung selektiver, potenziell diskriminierender Maßnahmen als Ausdruck eines Paradigmenwechsels kennzeichnet. Das Konzept der „strategischen Autonomie“ und der damit verbundene Ansatz des „Deriskings“ stellen im Lichte des Art. 21 EUV eine Neuakzentuierung des europäischen Außenhandelrechts dar. Während Art. 21 EUV ursprünglich die Förderung der wirtschaftlichen Offenheit und Stabilität zum Ziel hatte, wird das „Derisking“ als Strategie der Risikovermeidung und Autonomiesicherung instrumentalisiert. Die Europäische Kommission verweist in ihrer Mitteilung zur „Handelspolitischen Überprüfung“ von 2021 darauf, dass dies als Kernziel der EU zu verstehen sei. Dies bedeutet eine substanzielle Änderung der Rechtsdogmatik im Sinne einer stärker sicherheitsorientierten Handelsregelung und steht in einem klaren Spannungsverhältnis zu den bisherigen Dogmen der Handelsfreiheit und Marktoffenheit (vgl. European Commission, „Trade Policy Review - An Open, Sustainable and Assertive Trade Policy“, 2021).
	
	%Es ließe sich zwar argumentieren, dass die strategische Autonomie lediglich ein neues „politisches Narrativ“ ohne tatsächliche rechtliche Neuerungen ist. Diese Annahme hält jedoch einer tiefergehenden Analyse nicht stand. Die jüngste Orientierung hin zu „Derisking“ und „strategischer Autonomie“ verschiebt den Fokus der EU von der bisherigen, auf Offenheit ausgerichteten Handelspolitik hin zu einer selektiven und sicherheitszentrierten Handelsstrategie. Diese Zielverschiebung verändert die normative Grundlage des Außenhandelsrechts der Union und ist nicht allein programmatisch, sondern impliziert eine rechtliche Umstrukturierung, die sich in der Einführung neuer Schutzmechanismen manifestiert, etwa durch das Anti-Coercion Instrument und die Foreign Subsidies Regulation. Beide Regelungen zielen darauf ab, durch Einschränkungen und selektive Maßnahmen ausländische Einflussnahmen zu kontrollieren und bedürfen einer deutlichen Abkehr von Art. 206 und 207 AEUV, die „größtmögliche Liberalisierung“ als Ziel des gemeinsamen Handelspolitikrahmens definieren (vgl. Craig/de Búrca, EU Law: Text, Cases, and Materials, Oxford University Press, 2020). Der normative Bruch ist daher unbestreitbar und geht über eine bloße Neujustierung der Programmatik hinaus. Diese Sichtweise unterschätzt jedoch den strukturellen Wandel, den die neuen Instrumente bewirken. Die Foreign Subsidies Regulation und das Anti-Coercion Instrument gehen weit über bloße Abwehrmechanismen hinaus und schaffen ein neues, ausdifferenziertes Regime der Handelssteuerung, das an eine „Systemautonomie“ der EU anknüpft und somit das fundamentale Prinzip der Marktoffenheit untergräbt. Diese Regelungen setzen selektive, an Bedingungen geknüpfte Marktöffnungen durch und divergieren damit erheblich von den WTO-Grundsätzen, insbesondere von den Regeln zur Inländerbehandlung und dem Meistbegünstigungsprinzip nach Art. I und III GATT. Im Kontext der WTO-Verpflichtungen hat der EuGH in Gutachten 1/94 und Gutachten 2/15 explizit bekräftigt, dass die EU bei der Gestaltung ihrer Handelspolitik mit den WTO-Grundsätzen in Einklang stehen muss, was hier zu einer problematischen Überschreitung führt. Das neue Instrumentarium deutet somit auf eine tiefgreifende rechtliche Neubewertung hin, die den Charakter eines Paradigmenwechsels trägt. Ein weiterer Einwand könnte sein, dass das Prinzip der „strategischen Autonomie“ lediglich eine Anpassung an veränderte geopolitische Kontexte ist und den traditionellen Zielen der EU-Handelspolitik wie wirtschaftlicher Offenheit nicht widerspricht. Doch gerade diese Argumentation verkennt die normative Tiefenstruktur des Begriffs. Das „Derisking“ bedeutet im Ergebnis eine Zielverschiebung der EU-Außenwirtschaftspolitik hin zu einer sicherheitsorientierten und souveränitätszentrierten Autonomiepolitik, die über das bislang gesetzte Verständnis wirtschaftlicher Integration hinausgeht. Laut Art. 21 EUV hat die EU durch ihre Handelspolitik im Kern die wirtschaftliche und soziale Stabilität zu fördern. Die Übernahme sicherheitsorientierter Autonomie ist demgegenüber Ausdruck eines neuen Autonomieverständnisses, das auf wirtschaftliche Unabhängigkeit und strukturelle Resilienz zielt, anstatt die Offenheit und Kooperationsbereitschaft gegenüber internationalen Handelspartnern zu fördern (vgl. European Commission, „Trade Policy Review - An Open, Sustainable and Assertive Trade Policy“, 2021). Dieser Wandel ist materieller Natur und somit als Paradigmenwechsel einzuordnen, weil er die ursprünglichen Grundsätze des EU-Außenwirtschaftsrechts fundamental neu definiert. Insgesamt wird deutlich, dass die gegenwärtige EU-Handelspolitik mehr als eine bloße programmatische Adaption darstellt. Die neuen Instrumente, die substantielle Änderung der handelspolitischen Zielsetzung sowie die rechtlichen Spannungsverhältnisse zu den Grundprinzipien des GATT und des Binnenmarktes zeugen von einem echten Paradigmenwechsel. Dies lässt sich durch die normative Divergenz der strategischen Autonomie zur bisherigen Offenheitsstrategie belegen, die sowohl in der Zieldimension als auch im instrumentellen Vollzug eine Neuausrichtung markiert und die Grundpfeiler der bisherigen EU-Handelspolitik in wesentlichen Punkten infrage stellt.
	
	So ist es bereits zu einer Überversorgung des Lithium-Weltmarktes durch China gekommen.\autocite{https://www.reuters.com/markets/commodities/china-is-oversupplying-lithium-eliminate-rivals-us-official-says-2024-10-08/}
	Seit 2016 mit Verlängerung seit 2023 bestehen bereits Antidumpingmaßnahmen der EU gegen Wolfram aus China,\footnote{ (EU) 2023/1618, (EU) 2016/1036} was auf die grundsätzliche Angemessenheit dieser Maßnahme im Rohstoffsektor hinweist.
	
	Die Verknüpfung der Neuorientierung der Handelspolitik mit weiteren Zielen birgt iein erhebliches Risiko für Zielkonflikte durch die gebotene Gleichzeitigkeit der Verfolgung bzw. Erfüllung.\autocite{Schäffer/Hach, ZRP 2023, 207, 208}

	
	COM2021 66 final
	COM2022 470
	JOIN2023 20
	
	\subsubsection{Umweltpolitik}
	Die Union kann sich aus Art. 192 AEUV auf eine eigene Kompetenz zur Adressierung von umweltbezogene Dimensionen stützen.
	
	Ähnlich wie im Bereich der Energiekompetenz der Union bzw. zu den Fragestellungen der Auslegeung der Energiekompetenz der Union ergibt sich die generelle Problematik, inwieweit 
	
	Insbesondere der CRMA verweist nebst Art. 114 im Vorschlag hinsichtlich der Kohärenz mit bestehenden Vorschriften in diesem Bereich auf den europäischen \textit{Green Deal} und das \textit{Europäische Klimagesetz}\footnote{Vorschlag des CRMA, COM(2023) 160 final, S. 3}, was auf einen eindeutigen Bezug zur Umweltpolitik der Union hinweist.\footnote{Die \glqq Gegenprobe\grqq ergibt hier auch: Der \textit{Green Deal} verweist zwar nicht explizit auf Art. 192 AEUV, kann aber kraft seiner Natur nicht als Rechtsakt für sich verstanden werden, was auch die Existenz einer \glqq Mitteilung"\grqq als nicht verbindlichen Rechtsakt unterstreicht und vielmehr der Information und Orientierung dient im Sinne der nicht verbindlichen Rechtsakte aus Art. 288 S. 7; das \textit{Klimagesetz} hingegen verweist explizit auf Art. 192 I AEUV als Grundlage (Verordnung (EU) 2021/1119).}
	
	Interessant gestaltet sich auch die Querschnittsklausel aus Art. 11 AEUV
	
	Es stellt sich zudem die Frage, ob die gestiegende Bedeutung eines EU-Rohstofftitels (insbesondere in Bezug auf die kritischen Rohstoffe) auf vertiefte Klimaschutz[zur Entwicklung dessen]\autocite{Gundel in Tehobald/Kühling, IV, Energieumweltrecht, Rn. 74ff.} oder aber gänzlich andere Bemühungen zurückzuführen ist - das tatsächliche Motiv erfordert also eine weitere Untersuchung.
	
	\subsubsection{Energiepolitik}
	Die Energiepolitik der Europäischen Union, wie sie in Artikel 194 AEUV festgelegt ist, zielt darauf ab, eine sichere, nachhaltige und wettbewerbsfähige Energieversorgung zu gewährleisten. Wie bereits erwähnt, richtet sich die Energiepolitk der EU auf die Sicherstellung der Versorgungssicherheit insbesondere in Anbetracht des Funktionierens der Energieinfratsruktur, der Nutzung von erneuerbaren Energien und Versorgungsnotlagenvorsorge.\autocite[387]{frau_rohstoffe_2025} Genauer sind die konkreten Ziele in Abs. I über drei sog. \glqq Leitprinzipien\grqq definiert, wobei den Zielen keine Rangfolge o. Ä. zugewiesen ist und die jeweilige Gewichtung den EU-Organen respektive den Mitgliedsstaaten obliegt.\autocite[Gundel § M Rn. 26 27]{dauses_handbuch_2024} 
	Der Begriff der Energie ist als sehr weit zu verstehen und umfasst daher alle weiteren Aspekte, somit auch ihre Verwendung und Verbrauch.\autocite[Hamer Art. 194 Rn 8]{von_der_groeben_europaisches_2024} 
	Es wird somit zutreffenderweise festgestellt, dass Energiepolitik auch Roshtoffpolitk sei, jedoch in erster Linie die Energieerzeugung umfasst und eben nicht die die Rohstoffversorgung. Während diese Maßnahmen unweigerlich auch die Verwaltung bestimmter kritischer Rohstoffe betreffen, die für die Energieerzeugung und -speicherung benötigt werden, liegt der Fokus der Energiepolitik eindeutig auf der Energieproduktion und -versorgung.
	Insofern ergibt sich Frage nach der Abgrenzung hinsichtlich der Zugehörigkeit der kritischen Rohstoffe, die bei der Herstellung von Automobilen und insbesondere in der Produktion von Elektrofahrzeugen herangezogen werden, denn diese Rohstoffe fallen nicht in den primären Anwendungsbereich der Energiepolitik. Art. 194 AEUV bezieht sich explizit auf Maßnahmen zur Sicherstellung der Energieversorgung und zur Förderung von Energieeffizienz und erneuerbaren Energien. Die Anwendung dieses Artikels ist daher auf Aktivitäten beschränkt, die direkt mit der Erzeugung, Speicherung und Verteilung von Energie verbunden sind. Während die Sicherstellung der Versorgung mit Rohstoffen, die für die Energieproduktion benötigt werden, unter diese Politik fallen kann, schließt dies die umfassende Verwaltung und Versorgung mit nichtenergetischen Rohstoffen, die für andere Industriezweige, wie die Automobilwirtschaft, entscheidend sind, aus.
	
	Inwieweit trägt also die Produktion eines Elektrofahrzeugs zur Umsetzung der Energiepolitik der Union bei? Auch in Hinblick auf weitere verwaltungsrechtliche Aspekte wie Fördermaßnahmen?
	Die Produktion von EVs, die auf nichtenergetische Rohstoffe angewiesen sind, könnte als Teil der Unions-Energiepolitik betrachtet werden. Hierbei stellt sich jedoch die Frage, inwieweit die Produktion und Nutzung von EVs zur Umsetzung der Energiepolitik der Union beiträgt und ob dies durch Artikel 194 AEUV gedeckt ist. Elektrofahrzeuge tragen erheblich zur Reduktion von Treibhausgasemissionen bei, insbesondere wenn der für ihren Betrieb erforderliche Strom aus erneuerbaren Energiequellen stammt. Dies unterstützt das Ziel von Artikel 194 Abs. 1 lit. c AEUV, die Umweltverträglichkeit der Energieversorgung zu fördern und die Klimaziele der EU zu erreichen. Eine enge Auslegung von Artikel 194 AEUV könnte die Möglichkeit der EU, eine ganzheitliche und integrierte Energie- und Klimapolitik zu verfolgen, einschränken, da die Förderung von EVs eng mit den Zielen der Energieeffizienz und Umweltverträglichkeit verbunden ist.
	Die Produktion und Nutzung von Elektrofahrzeugen kann als wesentlicher Beitrag zur Umsetzung der Energiepolitik der EU gemäß Artikel 194 AEUV betrachtet werden, insbesondere durch die Reduktion von Emissionen und die Förderung erneuerbarer Energien. Gleichzeitig ist jedoch eine klare Abgrenzung der Kompetenzen notwendig, um sicherzustellen, dass spezifische Maßnahmen zur Rohstoffversorgung der Automobilindustrie nicht über die Ziele und Befugnisse der Energiepolitik hinausgehen.
	
	Am ehesten relevant scheinen die Ziele der ENergieversorgungssicherheit (Art. 194 I b. AEUV) und der Energieeffizienz (Art. 194 I c.) der Energiepolitik. Ist ersteres darauf bedacht, Lieferunterbechungen zu vermeiden und die Versorgungssicherheit zu gewährleisten\autocite[Nettesheim, Art. 194 Rn 16]{grabitz_recht_2024} und erfordert zudem eine vorausschauende Sichtweise\autocite[631]{umbach_europaische_2005}, ist die Energieeffizienz und die Entwicklung neuer und erneuerbaren Energien vielmehr auf den Nachhaltigkeitsaspekt ausgerichtet.\autocite[Nettesheim, Art. 194 Rn 17]{grabitz_recht_2024} Hierbei ist besonders von Relevanz, das als geeignete Maßnahmen zur Energieeffizienzförderung all jene anzusehen sind, die den Wirkungsgrad des Einsatzes von Energie erhöhen.\autocite[Nettesheim, Art. 194 Rn 16]{grabitz_recht_2024}
	
	Neben den explizit erwähnten EU-Energiepolitik-Zielen sind bei der Umsetzung dieser nach Art. 194 I AEUV auch allgemeine Maßgaben zu berücksichtigen. Zum einen sind die Ziele \glqq tatbestandlich auf die Verwirklichung und das Funktionieren des Binnenmarktes beschränkt\grqq \autocite{Bings in Streinz, Art. 194 Rn. 32f.}, sodass keine externen Kompetenzen interpretiert werden können\autocite{Bings in Streinz, Art. 194 Rn. 33} und dementsprechend die Paradigmen hinsichtlich der Subsidiarität und Verhältnismäßigkeit heranzuziehen sind. Ferner sind die Umweltpolitik und die mit ihr verbundenen Ziele, zumindest lässt sich das trotz des Fehlens der expliziten Erwähnung unmittelbar aus der den Zielen vorangestellten zu berücksichtendem Erhalt und Verbesserung der Umwelt aus Abs. I ableiten.\autocite{Bings in Streinz, Art. 194 Rn. 33}. Dies bestätigt auch die in Abs. I enthaltende Formulierung \glqq Unbeschadet der Anwendung anderer Bestimmungen\grqq, sodass hier keine Bestimmmung einer \glqq formellen Subsidiarität\grqq erkannt werden könnte.\autocite{Nettesheim, Grabitz/Hilf/nettesheim AEUV Art. 194, Rn 35}
	
	In Bezug auf die oben aufgeworfene Frage, ob der Einsatz von mineralischen Rohstoffen für EV-Batterien lässt sich argumentieren, dass diese durchaus den Wirkungsgrad der Energie erhöhen. 
	
	SO weisen Lithium-Ionen-Batterien, in denen wie eingangs erläutert Lithium, Kobalt, Nickel oder Mangan verwendet werden, einen deutlich erhöhten Wirkungsgrad auf, wenigstens bei Betrachtung der Energieverluste beim Entladen gegenüber der Nergieverluste beim Verbrennen von Kraftstoffen (Wärmeverluste). So weist also das Nutzen der gespeicherten Energie bei einen sehr hohen Wirkungsgrad auf (<99 \%) und somit damit gegenüber ICEs deutlich effizienter.%\autocite{CITATIONNEEDED} 
	Dies hat zur Folge, dass also die zur Ladung des Fahrzeuges genutzte elektrische Energie (die ja schließlich auch generiert werden muss) vom Fahrzeug effizienter genutzt wird. Dies bestätigt erneut die Tatsache, dass solche Maßnahmen erfasst werden, die \glqq darauf abzielen, dass der zur Erzielung eines bestimmten Nutzens erforderliche Energieeinsatz gesenkt wird\grqq\autocite{Nettesheim in Grabitz/Hilf, AEUV Art. 194, Rn. 17}
	Der Vollständigkeit halber ist aber zu erwähnen, dass es im oben beschriebenden Fall darauf ankommt, wie der Strom bzw. die Kraftstoffe hergestellt wurden.
	Es ist jedoch zu erwähnen, dass im Bereich des Rohstoffverwaltungsrecht und insbesondere im Bereich der Rohstoffpolitik der konkrete Anwendungsfall der EV-Batterien nur einen Teilbereich darstellt und die Prüfung anderer rohstofflichen Anwendungsfälle, besonders beim Einsatz von kritischen und strategischen Mineralien, die energiepolitische Überprüfung nicht standhält (jedoch andere Politiken als Auffangpolitik herangezogen werden können).
	
	Der integrationspolitischen Hintergund des Art. 194 AEUV zum Lückenschluss der Energiepolitik der Union, zu dem die Problemstellungen des Zugangs, der Verteilung und Nutzung von Energie, generelle energiepolitische Hrrausforderungen und der Importabhängigkeit,\autocite[Nettesheim, Art. 194 Rn 10]{grabitz_recht_2024} lässt sich analog auf den Hintegrund der Rohstoffpolitik üerbtragen, denn auch (Energie-)Konkurrenz zu anderen Nachfragen in anderen Märkten, Unsicherheit in den Lieferketten und dementsprechende Volatilität sowie die starke nationale Organisation von Versorgungsmärkten.\autocite[Nettesheim, Art. 194 Rn 10]{grabitz_recht_2024} Dies bestätigt auch die Erkenntnis, dass die Sicherung der Energieversorgung sowohl eine nach innen und nach außen gerichtete gesamteuropäische Dimension aufweist, woraus auch eine entsprechende Rohstoffverwaltung herausarfumentiert werden kann, sodass zumindest aus rechtsbildungsprozessualer Sicht die Energiepolitik weiterhin der kompetenzgebenden Prüfung standhält.
	
	Eine weitere \glqq Kompetenzschöpfung\grqq für die Rohstoffverwaltung im Sinne der unionalen Energiepolitk ist daher aus der Perspektive der Absicherung 
	
	Die Energiepolitik kann zudem als eine Art Blaupause für die Rohstoffpolitik gesehen werden: Durch die Herausbildung und Verankerung der EU-Energiepolitik im Laufe ihrer Entwicklung hin zu einem gestärkten Selbstbewusstsein der Union in energiepolitischen Fragen, durch die strategische Bedeutung die analog zur Energiepolitik eine solche für Rohstoffe aufweisen würde und schließlich das \glqq Machtpotential\grqq eines vereinten Auftretens würden hier zu einer Europäisierung beitragen.\autocite[Nettesheim, Art. 194 Rn 44]{grabitz_recht_2024}
	
	\subsubsection{Industriepolitik}
	Im vorgehenden Abschnitt wurde festgestellt, dass die Energiepolitk trotz einiger Anknüpfungspunkte nicht vollumfänglich als Zuständigkeitsbereich herangezogen werden darf, sodass sich die Frage nach der Anwendung anderer Ermächtigungsgrundlagen auftut.
	Art 173.
	
	\subsubsection{Binnenmarktpolitik}
	Art. 16, 26
	
	Insbesondere ergibt sich hinsichtlich der energiebezogenen Beziehung der Binnenmarktkompetenz zu den Art. 192, 194 AEUV die Fragestellung, inwieweit Art. 114 die Rechtsgrundlagen der Umweltpolitik ebenfalls stützt, nicht zuletzt durch die Tatsache dass in Art. 192 II AEUV konkret auf Art. 114 referenziert wird, aber auch der Umweltschutz in Art. III explizit erwähnt wird.\autocite[siehe insbesondere hierzu Gundel §m Rn 28f]{dauses_handbuch_2024} Jedoch indizieren sowohl die EuGH-Rechtsprechung als auch die EU-Rechtssetzungspraxis auf eine bevorzugte Anwendung des Art. 194 AEUV für umweltpolitische und insbesondere energiespezifische Belange, jedoch mit der Einschränkung, dass \glqq Rechtsakte mit umweltpolitischer Zielsetzung, die den Energiesektor ebenso wie andere Branchen erfassen, werden zutreffend weiter auf Grundlage von Art. 192 AEUV erlassen \grqq\autocite[Gundel §m Rn 29]{dauses_handbuch_2024}, sodass die Frage nach einer eindeutigen Rohstoffverwaltungsrechtlichen Kompetenzgrundlage weiterhin nicht eindeutig beantwortet werden kann respektive nur in der Form, als dass die Kompetenz aus einer Melange an Politikkompetenzen der Union besteht.
	
	In der Praxis lässt sich bisher nur der CRMA (s. u.) als Beispiel heranziehen, hier erfolgt die Erlassung des Rechtsakts \grqq insbesondere\glqq gestützt auf Art. 114 AEUV, da eine direkte Verbindung zwischen dem Rohstoffzugang für die Wirtschaft und dem Funktionieren des Binnenmarktes hergestellt wird.\footnote{2024/1252}
	
	Bei der Betrachtung der Binnenmarktpolitik sollte zudem bedacht werden, dass sich Rechtsunsicherheiten aufgrund des Ziels der Verwirklichung des Binnenmarktes in liberaler Weise in Bezug zur Verwirklichung anderer Politiken der Union ergeben könnten, denn es nicht vollumfänglich deutlich, inwieweit der Unionsgesetzgeber hier einen Vorrang gewähren muss.\autocite[§23 Rn 10]{classen_europarecht_2021} Insbesondere umweltpolitische Belange sind hier als ein solch komterkarierendes Beispiel zu nennen.
	
	\subsubsection{Zu den Grundfreiheiten}
	
	\subsubsection{Außenpolitik}
	Es ist nicht von der Hand zu weisen dass die Union bzw. das Unionsgebiet im Bereich der kritischen bzw. strategischen Rohstoffe auf keine eigenen bzw. nur wenige eigene Quellen zurückgreifen kann. Für eine sichere Versorgung ist daher ein auswärtiges Handeln der Union unerlässlich.
	
	Art. 21 II f EUV
	
	
	\subsubsection{Wirtschaftspolitik}
	Art. 120-126
	
	
	Die Wirtschaftspolitik der Union ist im ersten Schritt nur ansatzweise systematisch festgelegt, stattdessen erfolgt eine Regelung sektorspezifisch, denn die allgemeine Wirtschaftspolitik ist nicht EU-rechtlich adressiert worden.\autocite{Nettesheim, § 18 Wirtschaftsverfassung und Wirtschaftspolitik, in: Oppermann/Classen/Nettesheim, Europarecht, Rn. 13 } Dies bedeutet, dass der \textit{allgemeinen} Wirtschaftspolitik der Union so auch kein rohstoffspezifischer Bezug unterstellt werden kann.\autocite{frau_rohstoffe_2025} Daher ist also eine genauere Betrachtung der einzelnen Sektoren der Wirtschaftspolitik erforderlich, wobei auch hier erneut der Bezug auf die Rohstoffversorgung gelegt wird.
	
	
	%kommt es zu kompetenzproblemem?
	
	Art. 122 AEUV
	Art. 122 AEUV dient dazu, der Union und ihren Mitgliedstaaten in außergewöhnlichen wirtschaftlichen Situationen und bei schwerwiegenden Schwierigkeiten in der Versorgung mit bestimmten Erzeugnissen Handlungsmöglichkeiten zu eröffnen. Der Artikel hat eine klare Notfallfunktion und zielt darauf ab, die wirtschaftliche Stabilität und Sicherheit der EU-Mitgliedstaaten zu gewährleisten. Der Begriff der „schwerwiegenden Schwierigkeiten in der Versorgung“ kann auf die Versorgung mit kritischen Mineralien angewendet werden, insbesondere wenn diese Mineralien für die Batterieproduktion unerlässlich sind. Angesichts der zentralen Rolle, die Batterien für die Elektromobilität spielen, könnten Versorgungsengpässe bei kritischen Mineralien schwerwiegende wirtschaftliche und technologische Auswirkungen haben. Die zunehmende Abhängigkeit der EU von Importen kritischer Mineralien und die geopolitischen Risiken im Zusammenhang mit diesen Rohstoffen können als außergewöhnliche wirtschaftliche Gegebenheiten betrachtet werden. Diese Abhängigkeit macht die EU anfällig für Störungen in der Lieferkette, die durch politische Instabilität, Handelskonflikte oder andere außergewöhnliche Ereignisse verursacht werden können.  In einem solchen Szenario könnte die EU auf Grundlage von Art. 122 Abs. 1 AEUV Maßnahmen ergreifen, um die Versorgung zu sichern, z.B. durch: Förderung alternativer Lieferquellen: Die EU könnte Maßnahmen zur Diversifizierung der Lieferquellen für kritische Mineralien fördern.
	Strategische Reserven: Die Einrichtung strategischer Reserven für kritische Mineralien könnte beschlossen werden, um kurzfristige Versorgungsengpässe zu überbrücken. Finanzielle Unterstützung: Unternehmen, die in der Rohstoffgewinnung und -verarbeitung tätig sind, könnten finanzielle Unterstützung erhalten, um ihre Produktionskapazitäten zu erhöhen.
	
	Lediglich Artikel 122 2 AEUV bildet eine echte Rechtsgrundlage für Sofortmaßnahmen. Im Gegensatz dazu stellt Artikel 122 I AEUV keine Rechtsgrundlage für Krisensituationen dar und weist daher einen breiteren Anwendungsbereich auf. Die sehr weit gefasste Befugnis, die Art. 122 I AEUV dem Rat einräumt, unterliegt zwei grundsätzlichen Bedingungen: Zum Einen sollte der Rückgriff auf Art. 122 I AEUV nicht dazu führen, dass der Grundsatz der Zuständigkeit der Mitgliedsstaaten für die Wirtschaftspolitik aus Art. 2 III, 5 I AEUV ausgehöhlt oder gar reversiert wird, zum anderen dass der Rückgriff auf diese Norm unbeschadet anderer Rechtsgrundlagen erfolgt, die in den Verträgen vorgesehen sind, ohne aber diesen anderen Bestimmungen einen Vorrang vor Art. 122 I AEUV einzuräumen.\autocite{chamon_anwendung_2023}
	
	Art. 122 AEUV bietet eine rechtliche Grundlage, auf der die EU Maßnahmen zur Bewältigung schwerwiegender Schwierigkeiten in der Versorgung mit kritischen Mineralien für die Batterieproduktion ergreifen kann. Angesichts der zentralen Bedeutung dieser Mineralien für die Industrie und die strategischen Ziele der EU im Bereich der Elektromobilität und der erneuerbaren Energien ist eine Anwendung dieses Artikels gerechtfertigt.
	
	Es scheint also sinnvoll, ähnlich der Querschnittsklausel der Art 11, 122 AEUV, eine solche Klausel auch in eine Rohstoffpolitik aufzunehmen -- wobei zu beachten ist, dass sich insbesondere in Bezug auf andere Klauseln ergibt, dass diese keinerlei Richtungsempfehlung für die Nutzung der geeigneten Kompetenznorm geben, sondern vielmehr Gebote aussprechen, ihnen aber keine eigene Aussage zur Kompetenzregelung zugesprochen wird,\autocite[39]{callies_umweltrecht_2022}
	
	
	\subsubsection{Wettbewerbsrecht}
	
	\subsubsection{Verwaltungszusammenarbeit}
	Art. 197 AEUV
	
	Art. 114 AEUV
	
	\subsubsection{Sonstige Politiken}
	Art. 208-214 AEUV, 216-219, 122
	
	Im Geiste der der Solidarität zwischen den Mitgliedsstaaten, Art. 194 I AEUV, war in Art. III-256 EVV nicht enthalten und findet sich allgemein in Art. 3 III 3 EUV wieder
	
	Hingegen könnte eine Widersprüchlichkeit zwischen einer Solidaritäsklausel (auch im rohstoffpolitischen Bereich) und dem Wettbewerbsprinzip im Sinne des Binnenmarkts festgestellt werden, denn die in erster Linie eigennützige wirtschaftliche Betätigung wird der mutuellen Solidarität gegenübergestellt.\autocite{Ehricke/Hackländer: ZEuS 2008, S. 579, 595f.} Jedoch ist hier der Argumentation Callies' zu folgen, dass diese Ansicht abzulehnen sei, da hier die Vermengung von staatlichen Akteuren (nämlich den Mitgliedsstaaten als Adressaten der Solidaritätspflicht) und den eiggenützig am Wettbewerb teilnehmenden Unternehmen stattfindet, wobei für letztere die Regelungsherkunft (ob nun von nationaler oder europäischer Ebene) keine Rolle spiele, da sich ohnehin an diese Regeln gehalten werden müsse; aus der Solidaritätsklausel kann also keinerleiu Wettbewerbshindernis begründet werden -- die Solidarität richtet sich auf die Mitgliedsstaaten, nicht auf die Unternehmen.
	
	Ein Aktivwerden im Rahmen des Art. 352 AEUV (\glqq Flexibilitätsklausel\grqq) scheint allein aufgrund der Vielzahl an Kompetenzgrundlagen nicht angebracht.
	
	Fragwürdig ist zudem zunächst, inwieweit die Entwicklungspolitik der Union in diesem Bereich einen Rohstoffbezug aufweist. Nach Art. 208 AEUV verfolgt die EU das Ziel, die nachhaltige Entwicklung in Entwicklungsländern zu fördern. Die Formulierung „nachhaltige Entwicklung“ impliziert sowohl ökologische als auch ökonomische und soziale Aspekte. Verwaltungen in den Mitgliedstaaten sowie auf EU-Ebene spielen eine entscheidende Rolle bei der Umsetzung dieser Ziele. Der Art. 208 Abs. 2 AEUV sieht vor, dass die EU und die Mitgliedstaaten die Kohärenz ihrer Politiken sicherstellen müssen, was eine koordinierte Verwaltung voraussetzt, auch in Bezug auf Sekundärrecht durch Schaffung konkreter Verwaltungsaufgaben und -strukturen.\footnote{Siehe exemplarisch Verordnung (EU) 233/2014.} Der Art. 208 AEUV nennt spezifisch keine Rohstoffe, doch wird im Kontext der nachhaltigen Entwicklung implizit auch der verantwortungsvolle Umgang mit natürlichen Ressourcen, einschließlich Rohstoffen, angesprochen. Die Nutzung und der Handel mit Rohstoffen sind integrale Bestandteile der wirtschaftlichen Entwicklung vieler Entwicklungsländer.
	
	Ein deutlicher Bezug zur Rohstoffverwaltung ergibt sich aus der Konflitmineralienverordnung, sodass hier durch die Union über die Entwicklungspolitik Einfluss auf den Rohstoffsektor in Drittländern genommen wird.
	
	Die Entwicklungspolitik der EU nach Art. 208ff. AEUV hat einen klaren Verwaltungsbezug, da die Umsetzung der Ziele eine koordinierte Verwaltung auf EU- und nationaler Ebene erfordert. Ein Rohstoffbezug ist implizit gegeben, da nachhaltige Entwicklung ohne verantwortungsvollen Umgang mit Rohstoffen nicht denkbar ist. Schließlich gibt es einen Bezug zu einem Rohstoffverwaltungsrecht, der sich aus spezifischen Regelungen im Sekundärrecht der EU ergibt, welche die Verwaltung und Kontrolle von Rohstoffen betreffen. Diese Auslegung wird durch die teleologische Methode gestützt, da sie die übergeordneten Ziele der EU-Entwicklungspolitik in den Vordergrund stellt.
	
	\subsection{Fazit zu den Unionskompetenzen}
	Wie im vorherigen Teil umfassend dargestellt, lässt sich die Frage nach einer \textit{eindeutigen} unionsrechtlichen Kompetenz im Bereich der Rohstoffverwaltung auf EU-Ebene nicht vollumfassend und abschließend zuordnen, im Umkehrschluss bedeuet dies dass die sachlichen Anwendungsbereiche der einzelnen Politiken nicht trennscharf voneinander hinsichtlich ihrer rohstoffpolitiscihen Kompetenzverleihung getrennt sind und unterschiedlich stark ausgeprägte Überlappungen der Kompetenzfelder erscheinen. Es tritt zunächst ein Gefüge an verschiedenen Politikbereichen der Union auf, die als Kompetenzgrundlage infrage kommen - mit der Besonderheit, dass diese sowohl für sich genommen als auch als zusammengeführtes Konstrukt der einzelnen Bereiche zumindest in den meisten und weit gefassten Fällen angewandt werden können. Die Schaffung von Synergien erscheint schon allein aus rechtspolitischer Sicht durch die Menge an Instrumenten in einem ähnlichen geopolitischen Bezugsrahmen mit überlappendem Regelungsinhalt oder teleologischen Gemeinsamkeiten empfehlenswert, zudem müssen die diversen und teils extensiven Kompetenztitel eine Balance der Zielkonflikte zwischen dem Gewinn von politischer Autonomie und förderlichen Wohlstandseffekten und dem Verlust dieser Autonomie sowie negativen Effekten auf den Binnenmarkt gefunden werden.\autocite{Schäffer, EuZW 2023, 695, 700}
	
	Ähnlich zur \glqq Ziel- und Maßnahmenverschränktheit\grqq \footcite[siehe zu diesem Begriff]{Callies, Berliner Online Beiträge zum Europarecht, Nr 52} lässt sich in der bisher nicht explizit realisierten Rohstoffpolitik eine \textit{Kompetenzverschränkung} erkennen, aus der aber keine Zielverschränkheit ableiten lässt. Das bedeutet also konkret, dass die Rohstoffverwaltung zwar mehrere Kompetenzen in unterschiedlicher Weise in sich bündelt, hingegen aber nur ein Hauptziel verfolgt, namentlich das der Sicherstellung der Rohstoffversorgung -- es ergeben sich durchaus sekundäre \textit{Zieleffekte}, es existiert jedoch kein Anspruch einer Rohstoffverwaltungspolitik hauptsächlich zur Vollendung des Binnenmarkts beizutragen oder zu umweltpolitischen Belangen. Die Problematik einer kohärenten Verfolgung der Ziele ergibt sich hier also gar nicht, was die Ausgestaltung vereinfacht, jedoch aber wiederum höhere Ansprüche an die kohärente Anwendung der Kompetenzgrundlagen stellt, so lange noch keine eigenständige solche vorliegt -- sollten also zur Verfolgung des Ziels immer wieder andere Grundlagen herangezogen, wird also die Widerspruchsfreiheit verkompliziiert, die durch mangelnde Kompetenzklarheit weiter verschräft werden kann, obwohl wie festgestellt Schnittmengen mit diesen Politiken festgestellt werden kann. Letztendlich muss hier also eine uniforme Handhabe der Union einerseits ermöglicht, aber vor Allem sichergestellt werden -- gebunden an die Einschränkung, dass nur ein Ziel verfolgt wird, denn eine \textit{doppelte} Verschränkung sowohl von Kompetenzen als auch Maßnahmen und Zielen würde die Effektivtät und Effizienz des Handels doch sehr beeinträchtigen.
	
	Insofern kommt der Klärung einerseits der Zuordnungs- und andererseits der Komptenzfrage eine hohe Bedeutung zu, nicht zwangsläufig aufgrund von Konflikt- bzw. Kollisionsmöglichkeiten, lassen sich die einzelnen Politiken doch fast ausnahmslos miteinander vereinigen, jedoch vielmehr aus ungeklärten Fragen hinsichtlich des reinen prozedualen Verfahrens (im Rechtsetzungsprozess) oder aus rohstoffverwsltungsrechtlicher Sicht. Vermindert wird diese Relevanz lediglich durch die Tatsache, dass für fast alle Rechtsbereiche ohnehin das ordentliche Gesetzgebungsverfahren vorgesehen ist und somit die Wahl der Kompetenzgrundlage selbst bei Anwendung eines Schwerpunktsverfahrens für die Abgrenzung aus Praxissicht daher weniger relevant wird. 
	
	Desweiteren ist offen, inwieweit eine Rohstoffpolitik der Union von einer Verfahrensautonomie der Mitgliedsstaaten einerseits und einem Ermessensspielraum der Organe der Union andererseits profitieren könnte, bzw. wie sich diese Interessen vereinigen lassen, ist es doch unter dem anzunehmenden Regelungsbereich nicht ausgeschlossen, dass beide Konzepte entsprechende Relevanz entfalten, denn: Wo der Ermessensspielraum der EU-Organe stark ist, können Mitgliedstaaten ihre Verfahrensautonomie in Bereichen, die nicht harmonisiert sind, ausüben. Hier spielt jedoch das mehrfach angesprochene Loyalitätskonzept eine weitere Rolle, müssen nationale Verfahren bekanntermaßen so auszugestalten, dass sie die effektive Umsetzung unionsrechtlicher Maßnahmen nicht behindern, wodurch der Spielraum der Mitgliedstaaten indirekt durch den Ermessensspielraum der EU-Organe beeinflusst wird.
	
	Grundsätzlich gehen die Maßnahmen der Union auch hier im Sinne des Art. 5 IV EUV \glqq inhaltich wie formal nciht über das zur Erreichung der Ziele der Verträge erforderliche Maß hinaus\grqq, sodass auch hier die Verhältnismäßigkeitsprüfung unerlässlich scheint -- spätestens dann, wenn getroffene Maßnahmen zur Erreichung der Ziele mindestens ersichtlich nicht geeignet ist.\autocite{Gundel in Dause/Ludwigs, M., Rn 45 iVm u. mwN v. Danwitz EWS 2003 393ff.}
	
	Ein \textit{lex specialis} im Bereich der Rohstoffpolitik oder gar Rohstoffverwaltung, liegt demnach nicht vor, denn keiner der Titel kann als ein solches gewertet werden. Lediglich einzelne Titel könnten als Abgrenzung zum Art. 114 AEUV als dem \textit{lex specialis} näherstehend bezeichnet werden, um hier einen Vorrang zu begründen,\autocite{Calliies, Berliner Beiträge 52, S. 18ff., mwN Hamer in von der Groeben/schwarze/Hatje 2015, Art. 194, Rn. 26} eine Abgrenzung der einzelnen Titel voneinander (mithin nicht eindeutig)\autocite{Gundel EWS 2011, 25, 29; Kahl EuR 2009, 601, 608} erfolgt nicht.  Hinsichtlich etwaiger Kollisionen ist die EuGH-Rechtsprechung als einschlägig zu betrachten.
	
	Nichtsdestotrotz ist festzuhalten, dass die rohstoffverwaltungsrechtliche Kompetenz in Einzelfällen eine gesonderte, individuelle Prüfung erfordert.
	
	Dass die Union von ihrer Kompetenz trotz teil undeindeutiger Ermächtigungsgrundlage bereits entsprechenden Gebrauch macht, zeigen die nachfolgend betrachteten sekundärrechtlichen Akte, sodass also von der eingangs erwähnten Kompetenz im Rahmen des Art. 2 AEUV Gebrauch gemacht wird, denn diese liegen ja zutreffenderweise vor. Jedoch ist weiterhin zu beachten und zu erwähnen, dass die schiere rechtssetzende Aktivtät der Union im rohstoffverwaltungsrechtlichen Spektrum \textit{per se} keine Kompetenz begründet und die entsprechende Rechtsgrundlage in den Verträgen sowie die Beachtung des Subsidiaritätsprinzip weiter eingehalten werden müssen.\footnote{Beispielhaft im Bereich der Angleichung von Rechts- und Verwaltungsvorschriften der Mitgliedsstaaten sei das Urteil des EuGH zur sog. \textit{Tabakrichtlinie} erwähnt (Rs. C-376/98, Deutschland ./. Parlament und Rat); hier waren die Bestimmungen bzw. die Auslegung der Verträge als Rechtsgrundlage für den Erlass der Richtlinie unzureichend, und der EuGH stellte zudem fest, dass die EU-Organe keine Kompetenzen ausüben dürfen, um Ziele zu verfolgen, die außerhalb der im Vertrag festgelegten Zuständigkeiten liegen, was unterstreicht, dass die rechtssetzende Aktivität streng an die im Vertrag vorgesehenen Ziele und Kompetenzen gebunden ist.} Die bloße rechtssetzende Aktivität von EU-Organen in einem Politikbereich kann nicht als Beweis für eine entsprechende Kompetenz und Ermächtigungsgrundlage dienen. Die EU handelt ausschließlich im Rahmen der ihr durch die Verträge übertragenen Kompetenzen, die klar und objektiv definiert sein müssen. Subsidiaritäts- und Verhältnismäßigkeitsprinzipien stellen zusätzliche Anforderungen an die Rechtsetzung. Der EuGH hat wiederholt betont, dass jede Maßnahme der EU auf einer festen Rechtsgrundlage basieren und die Prinzipien der Subsidiarität und Verhältnismäßigkeit beachten muss. Rechtssetzende Maßnahmen, die diese Prinzipien missachten, werden vom EuGH für nichtig erklärt. Die Rechtsprechung verdeutlicht somit, dass die Kompetenz der EU nicht durch schiere rechtssetzende Aktivität begründet werden kann, sondern eine klare und rechtlich fundierte Ermächtigungsgrundlage erforderlich ist.
	
	Ferner ist mit Vorausschau auf den dritten Teil dieser Arbeit festzuhalten, dass eine explizite und primärrechtlich verankerte Kompetenzkonsolidierung nicht vorliegt, sodass mitunter zur Absicherung des kompetenzfolgenden Handelns eine Art legislativer \glqq Kompetenzumweg\grqq über die oben betrachteten Politiken genommen werden muss. Dies stellt in erster Linie mitunter keine unverhältnismäßige Belastung dar, jedoch sollte hierbei der programmatische Ansatz und insbesondere die \glqq Leuchtturmwirkung\grqq auf europäischer und internationaler Ebene und zum Management etwaiger industriepolitischer Erwartungen bedacht werden.
	
	Es ist dennoch denkbar, dass im Sinne eine Kompetenzerstellung durch die Union eine neue Unionspolitk geschaffen werden kann, die zwar bereits durch andere Kompetenznormen wie oben dargestellt zumindest teilweise abgedeckt wird und somit ein primärrechtliches Mandat für schon Existierendes geschaffen wird. Es ist jedoch zu beachten, dass nur durch Errichtung einer Kompetenznorm nicht zwangsläufig unmittelbare Tätigkeit folgen muss, sondern insbesondere längerfristige Wirkungen mit teils unklaren Entwicklungsperspektiven.\autocite[siehe hierzu Nettesheim Art. 194 En 40]{grabitz_recht_2024}
	
	Es ist daher allein aus dem Aspekt der Schaffung einer Rohstoffpolitik mit ganzheitlichem Aspekt sinnvoll, diesen Ansatz zu verfolgen, um wie mit dem Rohstoffverwaltungsrecht entsprechend für den Gesamtzusammenhang relevante, für sich alleinstehende Regelungsansätze einer gemeinsamen Ordnung zu unterwerfen
	
	Ferner könnte sich die Union durch eine primärrechtlich ausgewiesene
	
	Unter Anwendung der Schwerpunkttheorie stellt sich also die Frage, welche Kompetenznorm denn nun ausschlaggebend für eine Rohstoffpolitik und ein Rohstoffverwaltungsrecht wären. Grundlegend ist zu erkennen, dass die Schwerpunkttheorie des EuGH die Wahl der Rechtsgrundlage von der Hauptzielsetzung des Rohstoffverwaltungsrechts abhängig machen würde. Es müsste also zunächst festgelegt werden, ob ein solcher Rechtsbereich das Hauptziel der Versorgungssicherheit für Energie, Nachhaltigkeit und Umweltschutz, Wettbewerbsfähigkeit der Industrie oder aber die Bewältigung von Krisen hat. Es lässt sich nur vermuten, dass insbesondere Art. 173 AEUV, sekundär Art. 194 AEUV als Schwerpunktkompetenznorm hervortritt. Abschließend kann diese Frage jedoch in erster Linie nur von den Organen selbst, und im Zweifelsfall vom EuGH eindeutig beantwortet werden. Durch die Konstellation von bis zu sechs verschiedenen Kompetenzgrundlagen ist also kein Fall einer mitunter bereits gesehenen Doppelgrundlage gegeben, sondern eine durchaus komplexere Konstellation. 
	
	Ferner sind auch bei Anerkennung jeglicher Kompetenz die allgemeinen und spezifischen nationalen Regelungsvorbehalte der einzelnen Mitgliedsstaaten zu beachten, die diese Kompetenz begrenzen; schließlich lässt sich bereits jetzt erkennen, dass die rohstoffverwaltungsrechtliche Praxis stark national geprägt ist und sich die kompetenzausübende Funktion der Union zumindest aktuell noch auf ein Minimum beschränkt. Es muss zudem nach der Rechtskraft einzelner Akte unterschieden werden bzw. hinsichtlich ihrer Bindungswirkung und der entsprechenden Umsetzung auf mitgliedsstaatlicher Seite.
	
	Zusammengefasst ist die Existenz der Unionsrohstoffkompetenz zunächst eindeutig zu bejahen, jedoch auf mehrere Politiken zu stützen die -- je nach Anwendungs- und insbesondere Rohstofffall -- in unterschiedlichem Grade kompetenzverleihend sind. In den meisten Bereichen ist die Kompetenz eine geteilte, lediglich %in welchen bereichen nicht
	Die Sicherstellung der Rohstoffversorgung der Inudstrie ist darüber hinaus nicht nur rein kompetenztheoretischer, primärrechtlich verankerter Teil der Rohstoffpolitik, sondern ein bereits tatsächlich umgesetzer Teil der Organpolitik der EU, wie insbesondere im Folgenden noch weiter verdeutlicht wird.
	
	\subsection{Zum Subsidiaritätsprinzip}
	Art. 5 AEUV III schreibt vor, dass die EU nur tätig werden soll, wenn die Ziele der in Betracht gezogenen Maßnahmen auf Unionseben besser zu erreichen sind als auf mitgleidsstaatlicher Ebene. Inwieweit dies der Fall im Bereich der Rohstoffpolitik und im Rohstoffverwaltungsrecht auf unionaler Ebene ist, ist im Folgenden zu prüfen.
	

	
	
	\subsubsection{Zur primärrechtlichen Verankerung einer unionalen Rohstoffpolitik}
	Es ist anzunehmen, dass die Ausgestaltung einer primärrechtlich verankerten Rohstoffpolitik der EU als von erheblicher Bedeutung für die zukünftige Gestaltung und Effektivität in einem von Ressourcenkonkurrenz geprägtem Wirtschaftsumfeld angesehen werden kann. Es ist hervorzuheben, dass die Rohstoffpolitik in der Union beileibe auch einen deutlich lückenhafteren bzw. unvollkommenen Binnenmarktbereich bildet als andere.  Wie obig erläutert würde eine primärechtliche Verankerung eine klare Kompetenzzuweisung für die Organe der EU schaffen, sodass die Union in den Grenzen des Art. 5 AEUV tätig werden könnte -- hervorzuheben ist, dass durch den parallen Charakter dennoch die Mitwirkung der Mitgliedsstaaten nicht ausgeschlossen wird. Die unionale Rohstoffstrategie wäre demnach auch kompetenzrechtlich auf sicherem Fundament verankert, was zudem zukünftigen etwaigen Kompetenzstreitigkeiten (s. Tabakurteil) vorbeugt. Auch die Außenwirkung auf rohstoffliche Vertragspartner, aber auch Konkurrenten sollte bei der Abwägung in Beracht gezogen werden
	
		
	\subsection{Entwicklung}
	
	In Anbetracht des Weißbuches von 1994 ergibt sich die Fragestellung, inwieweit der Eruopäische Green Deal und der Aktionsplan für kritische Rohstoffe die strategischen Prioritäten der EU hinsichtlich der Rohstoffverwaltung in den 2010er-Jahren verändert hat.
	
	Arbeitspapier SEC(2007)771
	
	Bereits 2008 hat die Kommission in der Mitteilung "Die Rohstoffinitative -- Sicherung der Versorgung Europas mit den für Wachstum und Beschäftigung notwendigen Gütern"\footnote{KOM(2008) 699} erkannt, dass der verlässliche und unbeeinträchtigte Zugang zu Rohstoffen als essentieller Faktor der EU-Wettbewerbsfähigkeit eingstuft werden kann; die Mitteilung kann dementsprechend auch als erste strategische Auseinandersetzung der EU mit Rohstoffverwaltung im weitesten Sinne verstanden werden. Schon hier identifizierte die Kommission auch den Automobilsektor und die Abhängigkeit von "nichtenergetischen Rohstoffen" als betroffen, und sah Möglichkeiten zur Abhilfe "durch Steigerung der Ressourceneffizienz und durch Recycling von Altstoffen". Diese Aussage ist zwar im Grundsatz auch weiterhin aktuell, jedoch müssen insbesondere die veränderten Rahmenbedingungen und Herausforderungen wie in \ref{Kapitel 1} dargestellt hier miteinbezogen werden -- insbesondere die Steigerung der Ressourceneffizienz ist hier nur bedingt anwendbar, da sich die Abhängigkeit von Rohstoffen der Batterieproduktion weiter erhöht hat und schlichtweg Vorkommen auf Seiten der EU nicht oder nur unzureichend vorhanden sind. Seit 2008 haben sich die globalen wirtschaftlichen und geopolitischen Rahmenbedingungen drastisch verändert. Die zunehmenden geopolitischen Spannungen und Handelskonflikte, insbesondere im Hinblick auf China, das ein bedeutender Akteur in der globalen Rohstoffversorgung ist, haben die Abhängigkeit von kritischen Mineralien und Metallen verschärft. 
	
	Neben dem Weißbuch von 1994 kann also die Rohstoffinittaive von 2008 als ein weiterer, aber durchaus bedeutenderer Meilenstein in der EU-Rohstoffpolitik betrachtet werden.
	
	Diese Entwicklungen machen deutlich, dass die Strategien von 2008, so wichtig sie auch bleiben, nicht ausreichen, um den Herausforderungen der heutigen Zeit gerecht zu werden. Neben der Fortführung und Intensivierung der Ressourceneffizienz und des Recyclings sind also zusätzliche Maßnahmen erforderlich, die im folgenden weiter beleuchtet werden.
	
	\subsection{Europäisches Gesetz zu kritischen Rohstoffen}\label{EU-Verordnung}
	% hier ausführliche textanalyse
	
	%Betrachtung COM(2020 474 final)
	%EU Grundsätze für nachhaltuge Rohstoffe
	
	Seit dem 23. Mai 2024 ist die Verordnung 2024/1252 \glqq zur Schaffung eines Rahmes zur Gewährleistung einer sicheren und nachhaltigen Versorgung mit kritischen Rohstoffen\grqq inkraftgetreten, nachdem die Europäische Kommission im März 2023 einen entsprechenden Vorschlag unterbreitete. Dies ist insbesondere herauszustellen, da im Vergleich zu anderen Rechtsakten der EU innerhalb des ordentlichen Gesetzgebungsverfahrens (s. u.) die kurze Dauer von etwas mehr als einem Jahr auf die erkannte Dringlichkeit des Aktes hinweist, denn die zügige Gesetzgebung unterstreicht die Priorität, welche die EU der Sicherung kritischer Rohstoffe für die industrielle Resilienz und technologische Souveränität zumindest in diesem konkreten Fall beimisst. Nichtsdestotrotz folgt diese Geschwindigkeit des Verfahrens dem Trend der Verkürzung der Dauer von Rechtsakten, die nach der ersten Lesung abgeschlossen werden.\footnote{So betrug die durchschnittliche Dauer im ordentlichen Gesetzgebungsverfahrens für Rechtsakte, die nach der ersten Lesung abgeschlossen wurden, im Zeitraum der 8. Wahlperiode (2014-2019) 18 Monate, während der 9. Wahlperiode hingegen nur 13 Monate; siehe hierzu Europäisches Parlament: Facts and Figures, Briefing, Mai 2023, BRI(2023)747102, S. 13.} 
	
	Beim CRMA tritt die enge Verbindung zum veränderten Fokus der unionalen Handelspolitik hervor -- jedoch ist zu betonen, dass nicht allein auf die Handelsdimension bei der Sicherung der Versorgung abgestellt wird, sondern vielmehr auf mögliche Wettbewerbsverzerrungen wie sie durch für Marktteilnehmer divergierende Rechtsvorschirften und Zugangsbedingungen bedingt werden könnten, aber auch durch die tatsächliche Versorgungsrisikoüberwachung, ungleiche Begünstigungen nationaler Maßnahmen oder durch Hürden im Bereich des grenzüberschreitenden Warenverkehrs. Neu im Vergleich zu den eingangs beschriebenen Rechtsakten ist hierbei jedoch die Erkenntnis, dass die Kommission die Erforderlichkeit einer \textit{verbindlichen Steuerung} erkennt.\autocite{Schäffer/Hach, ZRP 2023, 207, 208}
	
	
	Die handelspolitische Dimension weißt hier einen Doppelcharakter auf, die einerseits auf Drittländer und andererseits auf den Binnenmarkt gerichtet ist.\autocite{Paschke, Rdtw 2024, 206, 211f}
	
	
	Es wird also deutlich, dass sich die kompetenzgebende Verknüpfung der Politiken in der unionalen Rohstoffpolitik eindeutig in den Ausführungen des CRMAs widerspiegeln und somit auch nochmals die Erforderlichkeit einer interdisziplinären Betrachtung sowohl innerhalb der rechtlichen Sichtweise aber auch unter Einbeziehung weiterer Disziplinen unerlässlich erscheint. 
	
	%Vergleich mit anderen Legislativprozessen
	
	Dieser \glqq Critical Raw Materials Act\grqq (CRMA) ist Teil des \textit(EU Green Deals) und wurde 2022 von der damaligen EU-Kommissionspräsidentin von der Leyen angekündigt \footnote{Ursula von der Leyen, "`State of the Union"' Rede, 14. September 2022.}, in Bezug zur informellen "`Versailles Decleration"' des Europäischen Rates von 2022.\footnote{Die Deklaration entstand vor dem Hintergrund des russischen Angriffskrieges auf die Ukraine im Februar 2022; in Kapitel III `"Building a more robust economic base'" wird unter dem Stichwort \textit{Critical raw materials} eine Sicherstellung der EU-Versorgung durch strategische Partnerschaften, Bevorratung und Förderung einer Kreislaufwirtschaft und Ressoruceneffizienz gefordert (Europäischer Rat, Informal meeting of the Heads of State or Government, Versailles Declaration, 11. März 2022, S. 7).} Die Kommissions-Generaldirektion Binnenmarkt, Industrie, Unternehmertum und KMU (DG GROW) ist betraut mit der Umsetzung des CRMA.
	
	Im Allgemeinen unterscheidet die Verordnung zwischen \textit(strategischen) Rohstoffen einerseits und \textit(kritischen) Rohstoffen andererseits
	
	Generell kann eine geographische Beschränkung auf einzelne Länder als kritisch angesehen werden, solange die Rohstoffe auch aus anderen Regionen von Unternehmen bezogen werden können, mit entsprechendem Folgen für die lokale wirtschaftliche Stabilität.\autocite[s. hierzu]{ruettinger_doddfrank_2015}
	
	\subsubsection{Der Vorschlag der Europäischen Kommission}
	
	Zu betonen ist hierbei der Fokus der Verordnung auf \glqq nichtenergetisch[e], nichtlandwirtschaftlich[e] Rohstoff[e]\grqq mit entsprechender Bedeutung für die EU-Wirtschaft und Vorliegen eines Vorsorgungsrisikos. Der Entwurf orientiert sich zudem explizit an der offenen strategischen Autonomie\footnote{siehe hierzu Abschnitt Handelspolitik} der Union.\footnote{ErwGr 1}
	
	Als Rechtsgrundlage des Entwurfs wurde Art. 114 AEUV gewählt
	Die Wahl der Rechtsgrundlage deutet also darauf hin, da die über das Vehikel der Vervollständigung des Binnenmarktes 
	
	Im Bereich der Rohstoffverwaltung liefern die Ergebnisse der Ex-Post-Bewertung sowie die Konsultation der Interessenträger im Vorschlag der Verordnung Erkenntnisse: So kritisierten Unternehmen und Betreibe Verfahrens- und Verwaltungskosten sowie die Dauer von Genehmigungsverfahren im Allgemeinen.\footnote{COM(2023) 160 final, S. 10.}
	
	Entsprechend berücksichtigt wird hierbei, dass durch den CRMA keine oder nur begrenzte weitere Verwaltungskosten für Unternehmen entstehen (so beispielsweise durch Berichtspflichten), wobei durch die Kommission ein Kostenausgleich durch die Profitierung von effizienteren Verwaltungsverfahren angenommen wird.\footnote{COM(2023) 160 final, S. 14.} Dieser lässt sich \textit{ex ante} nicht verifizieren, ist jedoch auch Bestandteil der in dieser Arbeit durchgeführten Befragung. %hier dann verweis einfügen
	
	Ferner schlägt die Kommission vor, dass \glqq große Unternehmen, die [...] strategische Rohstoffe verwenden, ihre Lieferketten prüfen und [...] regelmäßige Stresstests ihrer Lieferketten strategischer Rohstoffe durchführen, um sicherzustellen, dass sie alle verschiedenen Szenarien berücksichtigen, die sich im Falle einer Unterbrechung auf ihre Versorgung auswirken könnten\grqq \footnote{COM(2023) 160 final, S. 14.} Hierbei ergibt sich die Frage nach dem Zuständigkeits- und Verantwortungsbereich: Inwieweit können, müssen oder sollten Unternehmen zu solchen Auflagen verpflichtet werden, und inwieweit sollten diese Aufgaben von staatlicher Seite übernommen werden?
	% Abstimmungsverhalten
	
	Ein weiterer zentraler Bestandteil des Entwurfs ist das \glqq strategische Projekt\grqq. %Art. 5ff.
	
	Zwar kann die Verordnung als ein erster Beitrag zur Herausbildung eines europäischen Rohstoffverwaltungsrecht gesehen werden, erkennt aber dass keine aktive Harmonisierung der einzelnen mitgliedsstaatlichen Vorschriften beabsichtigt ist.\footnote{COM(2023) 160 final, S. 13.} Die Mitgliedstaaten der EU behalten somit weitgehend ihre Souveränität -- hierbei birgt sich aber das Risiko nach einer weiteren Fragmentierung der Rechtsvorschriften, unter Umständen mit entsprechenden Folgen für den EU-Binnenmarkt.
	
	%Methodik zur Auswahl strategischer Rohstoffe
	%Vergleich der Liste strategischer Rohstoffe
	
	Das Konzept einer Liste kritischer Rohstoffe ist nicht neu, lag eine solche bereits 2011 vor mit insgesamt 14 Mineralien, die mittlerweile auf 34 angewachsen ist.
	
	\begin{tabular}{|>{\raggedright}p{5cm}|c|c|}
		\hline
		\textbf{Mineral} & \textbf{Kritisch} & \textbf{Strategisch} \\
		\hline
		Antimon & x & \\
		\hline
		Arsen & x & \\
		\hline
		Bauxit & x & \\
		\hline
		Baryt & x & \\
		\hline
		Beryllium & x & \\
		\hline
		Bismut & x & x \\
		\hline
		Bor – metallurgische Qualität & x & x \\
		\hline
		Kobalt & x & x \\
		\hline
		Kokskohle & x & \\
		\hline
		Kupfer & x & x \\
		\hline
		Feldspat & x & \\
		\hline
		Flussspat & x & \\
		\hline
		Gallium & x & x \\
		\hline
		Germanium & x & x \\
		\hline
		Hafnium & x & \\
		\hline
		Helium & x & \\
		\hline
		Schwere seltene Erden & x & \\
		\hline
		Leichte seltene Erden & x & \\
		\hline
		Lithium – Batteriequalität & x & x \\
		\hline
		Magnesium & x & \\
		\hline
		Magnesiummetall & & x \\
		\hline
		Mangan – Batteriequalität & & x \\
		\hline
		Natürlicher Grafit – Batteriequalität & x & x \\
		\hline
		Nickel – Batteriequalität & x & x \\
		\hline
		Niob & x & \\
		\hline
		Phosphorit & x & \\
		\hline
		Phosphor & x & \\
		\hline
		Metalle der Platingruppe & x & x \\
		\hline
		Scandium & x & \\
		\hline
		Siliciummetall & x & x \\
		\hline
		Strontium & x & \\
		\hline
		Tantal & x & \\
		\hline
		Titanmetall & x & x \\
		\hline
		Wolfram & x & x \\
		\hline
		Vanadium & x & \\
		\hline
	\end{tabular}
	
	\subsubsection{Stellungnahme des Europäischen Wirtschafts- und Sozialaussschuss}
	
	Die Bedeutung einer entsprechenden Infrastruktur für eine Rohstoffverwaltung erkennt auch der Ausschuss, indem er den "Aufbau von Verwaltungskapazitäten in den öffentlichen Verwaltungen der EU-Mitgliedsstaaten" fordert % C 349/142,3.
	und 
	
	\subsubsection{Lesung im Europäischen Parlament}
	
	
	\subsubsection{Betrachtung des Verordnungstextes}
	Im Rahmen des Gesetzgebungsverfahren erfuhr der Vorschlag der Kommission entsprechende Veränderungen.
	
	%Rechtsanalyse
	
	\subsubsection{Fazit zum CRMA}
	Durch die Verabschiedung wurde die Kooperation im Bereich der Rohstoffverwaltung auf europäischer Ebene gestärkt.
	
	Der CRMA sei daher zu begrüßen, auch wenn man keine kurzfristigen Effekte kurz nach den Inkrafttreten erwarten konnte würde der CRMA dennoch mittel- und langfristig einen ausschlaggebenden Beitrag zur offenen strategischen Autonomie und zum Binnenmarkt liefern, trotz der enthaltenden Zielkonflikte,\autocite{Schäffer/Hach, ZRP 2023, 210f.}
	
	%Zielkonflikte
	
	\subsection{Konfliktmineralien}
	Erneut existiert keine einheitliche Definition zu den sog. Konfliktmineralien - also solche, 
	
	Meist werden die Elemente Zinn, Tantal, Wolfram und Gold als Konfliktmineralien eingestuft, 
	
	Die Regulierung von Konfliktmineralien und ihre Integration in das Wirtschaftsverwaltungsrecht stellt 
	
	\subsubsection{Die Konfliktmineralien-VO}
	Die Verordnung (EU) 2017/821, auch bekannt als "Konfliktmineralien-Verordnung" (VO) bindet Unionseinführer (UE) bestimmter sog. "Konfliktmineralien", genauer Wolfram, Tantal, Zinn ("3T" nach den englischen Bezeichnungen Tungsten und Tin) und Gold ("3TG") an Sorgfaltspflichten, die sich auf die entsprechenden Lieferketten auswirken, und kann als das am weitesten entwickelte Regelungsregime auf euiropäischer Ebene eingestuft werden\autocite{Kalls, ZfPW 2024, 181, 199}. Diese Sorgfaltspflichten sollen sicherstellen, dass UE mit ihrem Handel der genannten Metalle und deren Erze, also die als Konfliktmineralien verstandenen Rohstoffe, nicht zur Finanzierung bewaffneter Konflikte oder zu Menschenrechtsverletzungen beitragen und somit die "Verknüpfung zwischen Konflikten und illegalem Mineralabbau durchbrochen wird"\footnote{2017/821}. Die Verordnung gilt seit dem 1. Januar 2021 und erstreckt sich im Anwenungsbereich auf Importeure innerhalb der Europäischen Union, wie in Art. 1 Konfliktmineralien-VO geregelt. Unionseinführer i. S. d. Verordnung sind natürliche oder juristische Personen, die die vom Anwendungsbereich umfassten Minerale, Metalle und deren Erze ab in denen in Anhang 1 festgelegten Mengenschwellen in das Zollgebiet der Union einführen.\footnote{Art. 2 lit. l), VO 2017/821.} 
	
	Der entsprechende Entwurf der Verordnung wurde von der Kommission bereits 2014 an Rat und Parlament übermittelt, hingegen stimmte der Rat erst 2017 ohne Gegenstimmen dem Vorschlag zu.\footnote{ST 7937 2017 INIT} Das Parlament nahm insgesamt 60 Änderungen zum Kommissionsvorschlag an,\footnote{8645/15} und erkannte bereits hier die Problematik wie "die richtige Balance zwischen operationeller Flexibilität bei der Umsetzung (...) und angemessener Einbeziehung des Gesetzgebers bei wichtigen praktischen Fragen (...) gefunden werden kann."\footnote{A8-0141/2015} Dies lässt sich als Erkenntnis festhalten: eine flexible Umsetzung von Rechtsnormen notwendig, um auf die dynamischen und oft volatilen Bedingungen im Rohstoffsektor angemessen reagieren zu können. Rohstoffmärkte sind von globalen Entwicklungen, geopolitischen Risiken und technologischen Innovationen geprägt, die schnelle Anpassungen erfordern. Eine starre rechtliche Struktur würde das Risiko bergen, dass Unternehmen nicht in der Lage sind, auf solche Veränderungen zeitnah zu reagieren, was ihre Wettbewerbsfähigkeit und letztlich die Versorgungssicherheit gefährden könnte. Gleichzeitig ist es jedoch unerlässlich, dass der Gesetzgeber bei der Ausgestaltung und Anpassung des Rohstoffverwaltungsrechts eine aktive Rolle spielt. Dies gewährleistet, dass wirtschaftliche Interessen nicht auf Kosten von Rechtsstaatlichkeit, Umweltstandards oder ethischen Verpflichtungen verfolgt werden. Insbesondere im Kontext der Konfliktmineralien-Verordnung zeigt sich die Notwendigkeit, legislative Vorgaben so zu gestalten, dass sie praktikabel sind, ohne die Kontrolle über kritische Aspekte wie Menschenrechtsverletzungen oder Umweltzerstörung zu verlieren. Im Rohstoffverwaltungsrecht ist diese Balance daher nicht nur eine Frage der rechtlichen Präzision, sondern auch ein wesentlicher Faktor für die wirtschaftliche Stabilität und ethische Integrität der gesamten Rohstoffwirtschaft.
	Ein weiterer Punkt, der insbesondere den multilateralen Aspekt umfasst, wird in der Stellungnahme des Entwicklungsausschusses\footnote{A8-0141/2015, S. 53} angeführt: Eine Verodnung wie die Konfliktmineralien-VO können nicht als ein "in sich geschlossendes handelspolitisches Instrument angesehen werden", sondern müsse aus einer weiter gefassten Perspektive auch international betrachtet werden.
	
	Die jeweiligen Konflikt- und Risikogebiete nach Art. 2 lit. f werden, einer jährlichen Aktualisierung unterliegend, auf der \textit{Conflict-Affected and High Risk Areas}-Liste (CAHRAS) der EU bekanntgegeben.\footnote{Die Liste ist abrufbar unter https://www.cahraslist.net/cahras. Stand 2024 umfasst die Liste die Staaten Afghanistan, Äthiopien, Burkina Faso, Burundi, DR Kongo, Eritrea, Indien, Jemen, Kamerun, Kolumbien, Libanon, Libyen, Mali, Mosambik, Myanmar, Niger, Nigeria, Pakistan, die Philippinen, Russland, Simbabwe, Somalia, Sudan, Südsudan, Ukraine, Venezuela, sowie die Zentralafrikanische Republik.} Durch den Verweis auf die CAHRAS-Liste wird ermöglicht, den geographischen Anwendungsbereich entsprechend anpassen zu können.
	
	Hinsichtlich der Sorgfaltspflichten orientiert sich die Konfliktmineralien-VO eng an den, \textit{per se} rfechtlich unverbindlichen, OECD-Leitsätzen \autocite{OECDleitfaden2019} für die Erfüllung dieser, insbesondere durch die Bestrebungen der Kommission, diese Leitsätze stärker zu unterstützen wie auch die Anwendung der Leitsätze durch Unternehmen und die "Erfüllung der Sorgfaltspflicht", auch in nicht-OECD Staaten.\footnote{s. Präambel 9, VO 2017/821.} Daraus lässt sich schließen, dass eine Erfüllung der Vorgaben und Ziele der Konfliktmineralien-VO hinreichend durch eine Orientierung an den und eine Erfüllung der OECD-Leitsätzen erreicht werden kann, was letztendlich die Erfüllung die Sorgfaltspflicht betrifft. Letztlich werden also die Lieferketten der EU-Einführer gemäß einer Due-Dilligence-Prüfung betrachtet.\autocite[Rn. 390]{ruttloff_lieferkettensorgfaltspflichtengesetz_2022}
	%wie gestaltet sich das im Umsetzungsrechtsakt
	
	%genaue Analyse Konflikt-VO, Art 9
	
	
	Zum 7. Mai 2020 trat das entsprechende deutsche Durchführungsgesetz (Mineralische-Rohstoffe-Sorgfaltspflichten-Gesetz, MinRohSorgG) zur Verordnung 2017/821 in Kraft \footnote{Gesetz vom 29.04.2020 - BGBl. I 2020, Nr. 21 vom 06.05.2020, S. 864.}, es gab zuvor keine entsprechenden Rechtsakte, sodass die Thematik vergleichsweise spät in den Bereich des Rohstoffrechts aufgenommen wurde. Die Implementierung der EU-Verordnung in den deutschen Rechtsrahmen zeigt, dass eine Verstärkung des Rohstoffwirtschaftsverwaltungsrechts erforderlich ist, um die Sorgfaltspflichten der Unternehmen wirksam durchzusetzen.
	
	Hierbei agiert die Bundesanstalt für Geowissenschaften und Rohstoffe (BGR) als benannte Nationale Behörde und hat mit der deutschen "Kontrollstelle EU-Sorgfaltspflichten in Rohstofflieferketten" (DEKSOR) eine entsprechende Kontrollstelle geschaffen, die mit der Anwendung der Konfliktmineralien-VO betraut ist und Unionseinführer auf Einhaltung der Regelungen kontrolliert. 
	
	Insofern hat die Union hier durch Vorgabe der Schaffung der Verpflcihtung der Mitgliedsstaaten zur Einrichtung einer nationalen Vollzugsbehörde die Richtlinien für die nationalen Verwaltungsstrukturen vorgeformt und intensiviert bzw. der mitgliedsstaatliche Vollzug einerseits mit Maßgaben konfrontiert, aber auch dementsprechend mit Insrumenten zum tatsächlichen Vollzug ausgestattet.
	
	Die Relevanz des Rechtsaktes, zumindest aus deutscher Perspektive, kann gemischt betrachtet werden. Für den Berichtszeitraum 2022 \autocite{deutsche\_kontrollstelle\_eu-sorgfaltspflichten\_in\_rohstofflieferketten\_jahresbericht\_2023} (sowie teilweise 2023) wurden insgesamt 2.402 Unionseinführer erfasst, wovon aber lediglich 150 überhaupt die erforderliche Mengenschwelle überschritten (6\%) und davon 15 Unternehmen von nachträglichen Kontrollen betroffen waren. Unionseinführer können hierbei auch Privatpersonen sein, worauf ca. 2.000 der 2.402 zurückgehen. Ferner entfiel mehr als die Hälfte der Unionseinführer auf Zinnprodukte, bei Überschreitung der Mengenschwelle jedoch der überwiegende Teil auf Wolfram entfällt. Der Bericht beschreibt ein weiteres Problem, was sich insbesondere aus der zollrechtlichen Definition eines "Ursprungslandes" ergibt, also dem Land, in dem der letzte Verarbeitungsschritt stattfand – somit können Konfliktmineralien durchaus aus nicht vom geographischen Regelungsbereich der Verordnung umfassten Drittländern erfolgen, sodass die Nachverfolgbarkeit erschwert wird. Die Rückverfolgbarkeit über die gesamte Lieferkette ist daher nur deutlich eingeschränkt möglich.
	Zudem verfügen die DEKSOR und BGR nur über eine stark eingeschränkte Kompetenz. So ist zwangsläufig eine Bewertung der Kommission erforderlich, die die nationalen Behörden zum Verhängen von Strafen bei Nichteinhaltung befugt (Art. 17 III Konfliktmineralien-VO). Auch der DEKSOR-Bericht nennt das Nichtvorhandensein einer Befugnis bei Unklarheiten über die Überschreitung der Mengenschwelle Auskunft zu erhalten als nachteilig, setzt eine "Auskunftspflicht [$\ldots$] voraus, dass ein Unionsführer tatsächlich die Mengenschwelle überschritten hat". Dies schränkt die Durchsetzungskraft dieses rohstoffverwaltungsrechtlichen Akts ein; ferner kann daraus eine mangelnde Umsetzung der erwähnten Sorgfaltspflicht abgeleitet werden, was die DEKSOR in ihrem Bericht bestätigt, denn die jeweiligen Offenlegungspflichten der Unionseinführer über den Mengenschwellen werden auch bisher nur von einer Minderheit in vollem Umfang erfüllt \autocite[25]{deutsche\_kontrollstelle\_eu-sorgfaltspflichten\_in\_rohstofflieferketten\_jahresbericht\_2023}[]. Bei nachträglichen Kontrollen waren zudem Nachweise oft nicht ausreichend oder wurden nur zeitverzögert bereitgestellt, was insbesondere dem § 6 MinRohSorgG zuwiderläuft bzw. erkennen lässt, dass lediglich den Auskunftspflichten aus Art. 6 Konfliktmineralien-VO nachgekommen wird, nicht aber den Risikomanagementpflichten (Art. 5 Konfliktmineralien-VO), den Pflichten in Bezug auf das Managementsystem (Art. 4) und den generellen Offenlegungspflichten nach Art. 7.
	
	DEKSOR: "Im Rahmen der nachträglichen Kontrollen ist zudem aufgefallen,
	dass es für die Unternehmen eine Herausforderung ist, die erlangten und auf aktuellem Stand gehaltenen Informationen bezüglich der Lieferkette nach Art. 7 Abs. 2 VO von vorgelagerten Hütten und Raffinerien zu erhalten."
	
	Gemäß Art. 11 der Konfliktmineralien-VO i. V. m. § 3 III MinRohSorgG ist die DEKSOR zur Durchführung "geeigneter nachträglicher Kontrollen" befugt, sodass entweder auf Grundlage eines "risikobasierte[n] Ansatz[es]" oder bei Vorliegen entsprechender Hinweise auf Verstöße (auch durch Dritte) nachträgliche Kontrollen durchgeführt werden können, in denen insbesondere ein Einhaltung der Sorgfaltspflichten und Prüfpflichten nachgegangen wird.
	
	Es wird von der DEKSOR zudem berichtet, dass sich hierbei auf ein mögliches Risiko für die Geschäftstätigkeit der involvierten Unternehmen bei Offenlegung der Lieferketten im Sinne der Wahrung von Geschäftsgeheimnissen und anderen wettbewerbsrelevanten Informationen berufen wird, was als ein Widerspruch zum Transparenzziel der Konfliktmineralien-VO gewertet werden kann, wobei auch die OECD-Leitsätze eine Offenlegung zumindest gegenüber staatlichen Stellen als verpflichtend darstellen.\autocite[vgl.][37-38]{deutsche\_kontrollstelle\_eu-sorgfaltspflichten\_in\_rohstofflieferketten\_jahresbericht\_2023} Insbesondere in Verbindung mit der Frage der Rohstoffversorgungssicherheit scheint es sinnvoll, dass sich die Kommission bei ihrer Überprüfung im Dreijahresrhythmus nach Art. 17 II Konfliktmineralien-VO 
	%hier noch genauer die Artikel anschauen hinsichtlich Wortlaut 
	
	Es wird zudem festgestellt, dass die entsprechenden Verstoßverfahren den von der DEKSOR erwarteten Umfang und Komplexität überschritten, was in verzögerten Verfahrensdauern resultiert. 
	
	Die beschränkten Befugnisse der nationalen Behörden weisen auf eine Schwäche im aktuellen Rohstoffverwaltungsrecht hin. Es zeigt sich die Notwendigkeit, die Durchsetzungsmechanismen zu stärken, damit die Behörden effektiver handeln und die Einhaltung der Sorgfaltspflichten sicherstellen können. Wenn nur eine Minderheit der Unternehmen die Pflichten einhält, besteht die Gefahr, dass die Wirksamkeit der Verordnung und der zugrunde liegenden Rechtsnorm untergraben wird. Dies könnte das Vertrauen in das Rohstoffverwaltungsrecht schädigen und zu einer Fragmentierung der Rechtsdurchsetzung führen. Eine Einführung von Maßnahmen zur Compliancesteigerung, z. B. Anreizsysteme, sind denkbar -- zumindest muss aber der Fokus auf wirksamen Durchsetzungsmechanismen liegen, nicht nur im Bereich der Konfliktmineralien-VO.
	
	Es lässt sich außerdem argumentieren, dass das MinRohSorgG möglicherweise nur eine ungenügende Eingriffsgrundlage bei einer unzureichenden Erfüllung der unternehmerischen Auskunfts- und Mitwirkungspflichten bietet.  Dies wird deutlich, wenn man die Sanktionsmechanismen, Kontrollmöglichkeiten und Durchsetzungskraft dieses Gesetzes betrachtet. Als Bußgeld ist lediglich ein Zwangsgeld in Höhe von "bis zu 50.000 Euro im Verwaltungszwangsverfahren vorgesehen (§ 9).
	
	%Verwaltungszwangverfahren: Eingriffshürde betrachten
	
	Strafrechtliche Sanktionen wie im benachbarten Umweltstrafrecht sind nicht vorgesehen. m Vergleich zum MinRohSorgG bietet das deutsche Lieferkettensorgfaltspflichtengesetz, das seit 2023 in Kraft ist, deutlich schärfere Sanktionen. Es sieht nicht nur höhere Bußgelder vor, sondern ermöglicht auch den Ausschluss von Unternehmen aus öffentlichen Vergabeverfahren. 
	
	Schließlich stellt die DEKSOR fest, dass Versuche von Unternehmen wahrgenommen werden, die erforderlichen Nachweise zum Beleg der Erfüllung der Sorgfaltspflichtsvorschriften auf andere Bereiche auszulagern, die nicht 
	
	%Umgehungstatbestände
	
	
	
	Diese Verzahnung bzw. Verbindung von Rechtsakten zeigt, dass dieses Vorgehen insbesondere im rohstoffverwaltungsrechtlichen Bereich sinnvoll erscheint. Die Übernahme internationaler Standards in das europäische Rechtssystem stärkt die Integration globaler Best Practices in die nationale Gesetzgebung und fördert die Kohärenz im Rohstoffverwaltungsrecht. Im Beispiel der Konfliktmineralien-VO erleichtert die Orientierung an den OECD-Leitsätzen sowohl zunächst die Anforderungen der EU-Verordnung, aber auch potentieller weiterer (internationaler), nicht zwangsläufig staatlicher Regelungen nachzukommen, sodass hier mit der Schaffung einer sektor- und rechtsübergreifenden Grundlage begonnen wird. Langfristig entstehen hier Vorteile sowohl auf Unternehmens- und Verwaltungsseite: Durch die Nutzung anerkannter und immer weiter verbreiteter Richtlinien und Leitsätzen wird der Compliance- und Verwaltungsaufwand auf beiderlei Seiten auf ein erforderliches Minimum reduziert, mit den entsprechenden positiven Effekten. Ferner wird widersprüchlichen Regelungen oder Schlupflöchern vorgebeugt. Den zunächst rechtlich nicht verbindlichen Leitsätzen kommt durch den Verweis in der Konfliktmineralien-VO schließlich trotz ihres Soft-Law-Status durch ihre normative Einbettung eine mittelbare Rechtswirkung zukommt: Die Konfliktmineralien-VO hebt die Leitsätze aus ihrem ursprünglichen Soft Law-Kontext heraus und verleiht ihnen innerhalb des Anwendungsbereichs (und nur hier) der Verordnung eine rechtliche Relevanz -- "von 'Soft Law' zu 'Hard Law'"\autocite[42]{teicke_gesetzliche_2018}. Ferner kann inn Fällen, in denen die Verordnung unspezifisch ist oder Interpretationsspielraum lässt, die OECD-Leitsätze zur Konkretisierung herangezogen werden. Auf der anderen Seite lässt sich hingegen argumentieren, dass die als Soft Law konzipierten Leitsätze durch den schlichten Verweis in Art. 4 lit. b) Konfliktmineralien-VO nicht formell durch ein Rechtsetzungsakt in "harte" Rechtsnormen überführt wurden, denn es kann argumentiert werden, dass der Verweis als ein schlichter Hinweis auf eine bewährte bzw. empfohlene Praxis verstanden werden kann, ohne aber eine rechtliche Bindungswirking (mittelbar oder unmittelbar) zu entfalten -- "Soft Law bleibt Soft Law". Zudem: Die Verordnung weist keine zwingende Normenkonkretisierung auf, das heißt die Verordnung selbst gibt den Adressaten einen gewissen Spielraum, \textit{wie} die Sorgfaltspflichten genau umzusetzen sind, sodass der Verweis schlicht als eine Empfehlung oder Vorschlag interpretiert werden kann, der keine strikte rechtliche Bindung (sofern diese Begriffskonstellation überhaupt als existierend gewertet werden kann) entfaltet und ggf. durch UE alternative, ebenfalls geeignete Maßnahmen zur Einhaltung der Sorgfaltspflichten ergriffen werden könnten (Orientierungshilfen-Argument). Denn schließlich bleibt auch unklar, inwieweit ein Verstoß gegen die OECD-Leitsätze, ohne gleichzeitig gegen die formellen Vorschriften der Verordnung zu verstoßen, tatsächlich zu rechtlichen Konsequenzen führen würde. Da die Leitsätze nicht selbst Teil des bindenden Rechts sind, könnte argumentiert werden, dass ihre Nichtbeachtung, solange die formellen Pflichten der Verordnung erfüllt werden, keine Sanktionen nach sich ziehen sollte. Dies schwächt das Argument für eine mittelbare Rechtswirkung ab. Diese Auslegungsfrage lässt sich auch in der gegenwärtigen Literaturlage wiederfinden: So argumentiert Grado\autocite{grado_eu_2018}, dass die Umsetzung der OECD-Leitsätze in der EU-Verordnung als ein Schritt in Richtung einer stärkeren Verbindlichkeit von Soft Law betrachtet werden kann, aber dass dies nicht gleichbedeutend mit einer vollständigen Rechtswirkung ist. Die Verordnung hat zwar das Potenzial, die Einhaltung der Leitsätze zu fördern, jedoch bleibt sie in ihrer Fähigkeit, diese Leitsätze in hartes Recht zu verwandeln, begrenzt. Im Kontext möglicher Konflikte zum deutschen ABG-Recht wird ebenfalls darauf hingewiesen, dass ein pauschaler Verweis auf die OECD-Leitsätze aus Transparenzgründen nicht zu empfehlen ist und zudem "wenig harmonisch" gestaltet worden sei.\autocite[42]{teicke_gesetzliche_2018}
	Die Frage der unklaren mittelbaren Rechtswirkung hängt daher auch stark von der Auslegung und Anwendung im Rahmen des Rohstoffverwaltungsrechts ab. Weiterhin ist also noch nicht ersichtlich, inwieweit die Konfliktmineralien-VO diese Fragen klären und zukünftigen Entwicklungen beispielhaft dienen kann, insbesondere hinsichtlich der Soft-Law-Bindungsfrage.
	
	%im Legislativprozess der VO checken, wie ein Verweis auf die Leitsätze argumentiert war
	
	Dieses Vorgehen der Verzahnung kann also als ein beispielhaftes Leitbild für die weitere Entwicklung eines interdisziplinären und anwendungsbetontem Rohstoff- und Rohstoffverwaltungsrecht verstanden werden. Weitere Beispiele für solche Leitsätze lassen sich finden: %!
	
	Hingegen existiert, wie oben ausführlich dargestellt, im Bereich der Verzahnung von zunächst rechtlich nicht bindenden mit sekundärrechtlichen Akten die "Soft-Law-Problematik" der mittelbaren Anwendbarkeit.
	
	Ferner lässt die VO ausreichenden Gestaltungsspielraum für die die jeweiligen nationalen Umsetzungsmaßnahmen und insbesondere die Einführung von Regeln bei Verstößen gegen die Verordnung.\footnote{\textit{Siehe hierzu} Erwägungsgrund 20, VO 2017/821 sowie Art. 16.} So liegt beispielsweise die Höchststrafe 
	
	Die Durchführungsbefugnisse hingegen "sollten der Kommission übertragen werden".\footnote{Erwägungsgrund 21, VO 2017/821.} 
	
	%Analysieren: https://dip.bundestag.de/vorgang/.../255348
	
	Die Verordnung somit zielt darauf ab, Transparenz und Verantwortlichkeit in den Lieferketten der Unternehmen zu erhöhen, indem sie sie verpflichtet, Informationen über die Herkunft und den Handel der von ihnen verwendeten Mineralien offenzulegen. Diese Verpflichtungen stehen in direktem Zusammenhang mit den möglichen Grundsätzen des Rohstoffwirtschaftsverwaltungsrechts, und das RWvR kann erneut als rechtlicher Rahmen zur Umsetzung sowie Einhaltung der Regelungen agieren. Interessant gestaltet sich hierbei nicht nur die Erwähnungsgrund 13 genannte Transparenzschaffung zur Vertrauensbildung gegenüber den Wirtschaftsbeteiligten, sondern auch eine Öffentlichmachung des rohstoffverwaltungsrechtlichen Aspektes -- es ließe sich argumentieren, dass durch die öffentliche Salienz hinsichtlich der Verfahren und Strategien sowie die eigentliche Erfüllung auch hier eine Art "Demokratisierung" des Rohstoffrechts stattfindet, auch aus Unternehmenssicht durch die eindeutige Verwaltungs-Unternehmens-Beziehung in dieser Angelegenheit und die zunächst eindeutige Regelung des Bereiches.
	
	Die Regulierung von Konfliktmineralien ist ein entscheidender Bestandteil eines umfassenden Rohstoffwirtschaftsverwaltungsrechts, das sowohl die wirtschaftlichen Interessen als auch die ethischen Verpflichtungen Deutschlands und der EU berücksichtigt. Die bisher bestehenden gesetzlichen Regelungen, insbesondere die Verordnung  2017/821 und das Lieferkettensorgfaltspflichtengesetz, bieten eine Grundlage, die jedoch durch weitere rechtliche Entwicklungen und eine verbesserte Durchsetzung ergänzt werden muss. Ein integrierter Ansatz im Rahmen eines nationalen Rohstoffwirtschaftsverwaltungsrechts kann dazu beitragen, diese Herausforderungen zu bewältigen und gleichzeitig die Rohstoffversorgung weiterhin sicherzustellen und negative Effekte bei Unternehmen, aber auch wie schon beschrieben in den konfliktmineralienexportierenden Staaten zu vermeiden.
	
	Nichtsdestotrotz kann festgehalten werden, dass es sich bei der Konfliktmineralien-VO um einen sehr kleinen und limitierten Regelungsbereich handelt, was sich insbesondere in der Zahl der erfassten Fälle widerspiegelt und daher nun eine geringere Relevanz für die Wirtschaft angenommen werden kann. Darüber hinaus ist die Konfliktmineralien-VO eindeutig dem internationalen Teil des Rohstoffverwaltungsrechts zuzurechnen und fällt ebenso in den Bereich des Wirtschaftsvölkerrechts, nicht zuletzt durch die eindeutigen Menschenrechtsbezogenen Aspekte im Rechtsakt.
	
	\paragraph{Nearshoring und Konfliktmineralien}
	
	Es liegt nahe, dass Unternehmen im Bereich der Konfliktmineralien eine Verlagerung von Importen, nichtg zwangsläufig in Form des Nearshorings, in Betracht ziehen können, um die Auskunftspflichten im Rahmen der Konfliktmineralien-VO zu reduzieren oder gar gänzlich zu vermeiden und so den unternehmerischen Dokumentationsaufwand einzustellen. Im DEKSOR-Report wird beschrieben, dass aus unternehmerischer Sicht die Aufwendungen für die Erfüllung der Sorgfaltspflichten nicht im Verhältnis zum erforderlichen Umsetzungsaufwand stehen würden, sodass entsprechende Verlagerungen in Betracht gezogen werden um die Tatbestandsmerkmale der Konfliktmineralien-VO bzw. des MinRohSorgG zu verlassen -- insbesondere durch ein Nearshoring des Rohstoffbezugs hin zu einem direkten Import aus EU-Mitgliedsstaaten. Auch die DEKSOR erkennt hier richtigerweise, dass hier ein Umgehungstatbestand infrage kommen könnte; dieser sei zwar nicht illegal, könne aber nicht als Sorgfaltsmaßnahme des Unternehmens selbst betrachtet werden, sondern als eine Verlagerung dieser Sorgfaltspflicht auf den dann infrage kommenden Unionseinführer in das Zollgebiet der Union.\autocite[vgl.][43]{deutsche\_kontrollstelle\_eu-sorgfaltspflichten\_in\_rohstofflieferketten\_jahresbericht\_2023}
	
	%Bewertung von Umgehungstatbeständen
	Berufen auf Vertrauen des Unionseinführers/Verhütter/Rohstoffproduzent
	
	Auch die DEKSOR stellte hierbei fest, dass das Sorgfaltsvertrauen der (nachgelagerten) Unternehmen trotz Zertifizierungen nicht immer begründet sei, denn "eine Verlagerung der Sorgfaltspflichten auf andere Unternehmen verschiebt nur das Problem anstatt im Verbund auf transparente Lieferketten hinzuwirken" \autocite[43]{deutsche\_kontrollstelle\_eu-sorgfaltspflichten\_in\_rohstofflieferketten\_jahresbericht\_2023}
	
	Nearshoring als Umgehungsstrategie
	
	Juristisch gesehen kann die Verlagerung von Beschaffungsprozessen in stabilere, näher gelegene Regionen als eine direkte Reaktion auf die Anforderungen der Konfliktmineralien-Verordnung gesehen werden. Diese Verlagerung ist im Wesentlichen eine Form des Nearshoring, da sie den geografischen Schwerpunkt der Beschaffung näher an die Heimatmärkte bringt
	
	Nur eine indirekte Verbindung zum Nearshoring weisen sekundäre Effekte der Konfliktmineralien-VO auf Unternehmen auf -- Compliance in Lieferketten ist im engeren Verständnis kein Bestandteil eines Nearshorings, kann aber durchaus als ein solches interpretiert werden, da letztendlich durch die Anwendung von Sorgfaltspflichten wie im Rahmen der Konfliktmineralien-VO Unternehmen dazu angehalten sind, ihre Lieferketten so zu gestalten, dass sie die mitunter strengen Anforderungen der VO erfüllen und somit über Nearshoring auf eine Lösung zurückgreifen, um Compliance-Kosten zu senken, die Kontrolle über die Lieferkette zu verbessern und Risiken zu minimieren, die mit weit entfernten, instabilen Lieferquellen verbunden sind. So wird Nearshoring zu einer strategischen Entscheidung, die durch rechtliche Vorgaben motiviert ist und die betriebliche Resilienz und Nachhaltigkeit fördert. Dies bestätigt auch die Beobachtung, dass die sog. \textit{Corporate Social Responsibility} (CSR), in der unternehmerischen Praxis immer weiter an Bedeutung gewinnt und sogar nur Empfehlungen im Sinne eines Soft Laws übernommen werden, die sich teilweise bis auf Lieferanten erstrecken.\autocite[39]{teicke_gesetzliche_2018}
	
	Hier lassen sich aus rohstoffverwaltungsrechtlicher Perspektive weitere Vorteile zur Erfolgsmaximierung und Umsetzungsvereinfachung erkennen, denn Soft Law kann eine bedeutende und sinnvolle Ergänzung sowie Erweiterung des Rohstoffverwaltungsrechts darstellen, indem es flexiblere, anpassungsfähigere und dennoch wirksame Rahmenbedingungen schafft, die sowohl den regulatorischen Anforderungen als auch den praktischen Bedürfnissen der Wirtschaft und der staatlichen Verwaltung gerecht werden. „Hard Law“-Regelungen bieten zwar Rechtssicherheit, sind jedoch oft starr und weniger anpassungsfähig an die dynamischen Entwicklungen in globalen Lieferketten und Rohstoffmärkten. Soft Law, das typischerweise in Form von unverbindlichen Leitlinien, Empfehlungen, Verhaltenskodizes oder Branchenstandards existiert, bietet hier eine notwendige Flexibilität. Es ermöglicht eine schnellere Anpassung an technologische Innovationen, Marktveränderungen und neue ethische Anforderungen. Soft Law kann somit Lücken im bestehenden Rechtssystem füllen und eine dynamische Anpassung ermöglichen, die für eine effektive und zeitnahe Regulierung im Rohstoffsektor unerlässlich ist. Für staatliche Behörden bietet Soft Law den Vorteil, regulatorische Ziele durch kooperative Ansätze zu erreichen, anstatt ausschließlich auf Zwangsmaßnahmen zurückzugreifen. Soft Law kann als Instrument zur Förderung der Selbstregulierung innerhalb von Industrien dienen, wobei staatliche Kontrolle durch Beratung, Überwachung und Förderung von Best Practices ergänzt wird. Ein solcher kooperativer Ansatz reduziert den administrativen Aufwand und ermöglicht eine effizientere Allokation staatlicher Ressourcen. Zudem kann Soft Law als „Laboratorium“ für neue Regulierungsansätze fungieren, die, sobald sie sich in der Praxis bewährt haben, in verbindliche Rechtsnormen überführt werden können. Dies fördert eine schrittweise und praxisnahe Weiterentwicklung des Rohstoffverwaltungsrechts. Darüber hinaus kann CSR als ein international geprägtes Feld wahrgenommen werden, sodass Soft Law hier eine Art Brücke zwischen diversen nationalen und internationalen Rechtsordnungen schlagen kann.
	
	
	\subsection{ACI VO}
	Die Verordnung 
	
	\section{Rohstoffe und Deutschland}
	
	\subsection{BBergG}
	
	\subsection{Interimsfazit}
	
	\section{Die Entstehung von Rohstoffstrategien}
	
	\subsection{Auf nationaler Ebene}
	Stand xxxx haben xxxx europäische Länder eine nationale Rohstoffstrategie entwickelt
	
	\subsubsection{Die Rohstoffstrategie der Bundesregierung}
	
	\subsection{Auf europäischer Ebene}
	
	Im Rahmen des Modells des EU-Verwaltungsrechts obliegt der Vollzug und die Umsetzung grundsätzlich den Mitgliedsstaaten
	
	\section{Wechselwirkungen zwischen deutschem und europäischen Rohstoffrecht}
		Im Rahmen des Modells des EU-Verwaltungsrechts obliegt der Vollzug und die Umsetzung grundsätzlich den Mitgliedsstaaten
	
	\section{Rohstoffe und Menschenrechte}
	
	Rohstoffkontrollregime
	
	\section{Rohstoffe und Umweltschutz}
	
	Rohstoffe und Klimaschutzrecht
	
	%menschenrechte rohstoff verwaltung
	
	\section{Rohstoffe in der Entwicklungspolitik}
	
	\section{Betrachtung weiterer ausgewählter Rechtsakte}
	
	Dies dient insbesondere dem Vergleich 
	
	Bereits 2010 und damit deutlich vor Legislativinititaven der EU hatten die USA über den \textit{Dodd-Frank-Act} \footnote{Dodd-Frank Wall Street Reform and Consumer Protection Act, Art. 1502-1504 (\textit{sections})} Transparenzpflichten im Umgang mit Konfliktmineralien in Lieferketten eingeführt. Der Gesetzesakt ist nicht primär auf den Bereich des Rohstoffrechts ausgerichtet, sondern enthält eher nebenbei Regelungen zu Konfliktmnineralien, insbesondere diejenigen des Art. 1502.  Hierbei ist der Ansatz zu beachten, dass nicht der Bezug der entsprechenden Rohstoffe eingeschränkt wird, sondern ein Reputationsrisiko für Unternehmen bei Rohstoffbezug in Konfliktregionen durch die Veröffentlichungspflichten entsteht.\autocite[Rn. 415]{ruttloff_lieferkettensorgfaltspflichtengesetz_2022} Eine relevante Schnittmenge zwischen dem Dodd-Frank-Act und dem LkSG kann jedoch nicht festgestellt werden, denn Unternehmen die bereits den Regelungen zur Transparenz des Art. 1502 unterliegen, können vom LkSG höchstens auf indirektem Wege profitieren, da der Handel mit Konfliktmineralien und die damit verbundene Vermeidung der Finanzierung bewaffneter Konflikte nicht im Fokus des LkSG stehen.\autocite[Rn 423]{ruttloff_lieferkettensorgfaltspflichtengesetz_2022} Stattdessen ist der Anknüpfungspunkte in EU-Verordnung 2017/821 zu suchen, die in Erwägungsgründen explizit eine dem Dodd-Frank-Act-ähnliche Rechtsvorschrift erwähnt.\footnote{2017/821, Erwägungsgrund 9: "[...] forderte das Europäische Parlament die Union auf, mit dem US-amerikanischen Gesetz über Konfliktminerale, Artikel 1502 des Dodd-Frank-Gesetzes (...), vergleichbare Rechtsvorschriften zu erlassen.} 
	
	
	
	\section{Nichtbeachtung der Rohstoffthematik}
	Der Mangel an spezifischen gesetzlichen Vorgaben zur Rohstoffverwaltung lässt sich durch mehrere Faktoren erklären, deren plausible Relevanz im Folgenden weiter untersucht und evaluiert werden soll. Aus rein historischer Sicht standen in Deutschland und der EU andere rechtliche und wirtschaftliche Prioritäten im Vordergrund - insbesondere in einer Zeit, in der elektrisch angetriebende Fahrzeuge und dementsprechend EV-Batterien und ihre Herkunft ein Schattendasein pflegte, und die langfristige Sicherung von Rohstoffquellen wurde oft als weniger dringlich angesehen, insbesondere in Zeiten wirtschaftlicher Prosperität oder stabiler internationaler Lieferketten – also insbesondere vor disruptiven Ereignissen wie der Covid-19-Krise oder dem russischen Angriffskrieg gegen die Ukraine und den entsprechenden Folgen.
	Ferner kann vermutet werden, dass die Nichtexistenz eines spezifischen Rechtsbereiches zur Rohstoffverwaltung in sich selbst begründet werden kann, sodass das Nichtvorhandensein rechtfertigt, hier auch nichtg tätig zu werden. Wie im Bereich des Bergrechts zu sehen, ist der Rechtsrahmen zudem als eher fragmentiert im Vergelich zu anderen Rechtsbereichen einzustufen, denn auch das Rohstoffverwaltungsrecht findet Anküpfungspunkte dort, aber auch im Umwelt- und Handelsrecht oder dem internationalen Wirtschafts(völker-)recht. Diese Fragmentierung führt dazu, dass es keine einheitlichen, spezifischen gesetzlichen Vorgaben gibt, die alle Aspekte der Rohstoffverwaltung umfassend regeln. Stattdessen existieren zahlreiche Einzelregelungen, die unterschiedliche Bereiche und Aspekte der Rohstoffnutzung abdecken, ohne jedoch eine kohärente Gesamtstrategie zu bilden.
	Ferner: Langfristige, mitunter politisch begründete oder geprägte Rohstoffstrategien und nachhaltige Rohstoffverwaltung stehen häufig im Widerspruch zu eher kurzfristigen bzw. schneller umsetzbaren Zielen und insbesondere auch schneller wahrnehmbaren Erfolgen, auch da jede Art von rohstoffverwaltungsrechtlichen Akten einen entsprechenden Eingriff mitsichbringt. 
	Es drängt sich zudem die Vermutung auf, dass das öffentliche Bewusstsein für die Bedeutung einer Rohstoffverwaltung, insbesondere in Politik und Wirtschaft, eher gering war und im Rahmen der Zeitenwende einen entsprechenden Bedeutungsgewinn erfahren hat. Ferner kann der Bereich des Rohtsoffverwaltungsrecht durchaus als ein technisches und spezialisiertes Thema wahrgenommen werden, sodass die Bedeutungskraft hinter anderen Themen zurücktritt - Stichwort Salienz.
	
	Die Rohstoffverwaltung ist ein globales Thema, das internationale Kooperation und Abstimmung erfordert. Unterschiedliche nationale Interessen und Prioritäten erschweren die Entwicklung einheitlicher internationaler Regelungen und Standards. Zudem sind viele rohstoffreiche Länder außerhalb Europas angesiedelt, was die Einflussmöglichkeiten der deutschen und europäischen Gesetzgeber einschränkt und die Notwendigkeit internationaler Verhandlungen und Vereinbarungen erhöht.
	
	Ein umfassendes Verständnis dieser Hintergründe ist essentiell, um die bestehenden Regelungen zu evaluieren und mögliche Wege für die Entwicklung kohärenter und effektiver rechtlicher Rahmenbedingungen zu identifizieren. Eine weitergehende wissenschaftliche Auseinandersetzung mit diesen Themen ist daher dringend notwendig, um die Herausforderungen der globalen Rohstoffversorgung nachhaltig und rechtlich fundiert zu bewältigen.
	
	% Rohstofffluch Frau
	
	
	\section{Herausforderungen für das Rohstoffverwaltungsrecht auf EU-Ebene}
	%Tabelle Rohstoffinitativen EU-Mitgliedsländer
	Es muss erkannt werden, dass nicht zwangsläufig alle Mitgliedsstaaten Schemata zum Umgang mit kritischen Rohstoffen in Lieferketten ausformuliert haben und so auch Unternehmen über mögliche Risiken im Falle von (zeitweilig) gestörten oder versiegten Lieferketten aufzuklären beziehunsgweise entsprechende Verantwortung für solche zu übernehmen, nciht zuletzt aufgrund unterschiedlicher Relevanz und Betroffenheit einzelner Mitgliedsstaaten und daraus resultierendem, divergierendem Risikobewusstsein und entsprechender Vorsorge.
	
	\subsection{Verfahrensautonomie der EU-Mitgliedsstaaten}
	Grundsätzlich obliegt der Vollzug einzelner Titel den Mitgliedsstaaten, mit der Einschränkung durch entsprechende Vorgaben der Union, sodass die Verfahrensautonomie der Mitgliedsstaaten gegenüber der zunehmenden Europäisierung der nationalen (Verwaltungs-)Rechtssysteme als eine \glqq ungeklärte Kardinalsfrage\grqq \autocite{Ludiwgs, NVwZ 2018, 1417} betrachtet werden kann
	
	
	\section{Die Politisierung des Rohstoffrechts}
	
	Es ist mithin bekannt, dass ein Großteil der globalen mineralischen Rohstoffvorräte und insbesondere solche der seltenen Erden in diversen Staaten, darunter zahlreichen Entwicklungsländern, lagern, 
	
	Somit kam es hier insbesondere in den letzten Jahre, ausgehend von ?, zu einer zunehmenden Politisierung im Bereich der Rohstoffe, die sich unweigerlich auch auf das Rohstoffrecht ausdehnt. Gründe hierfür sind einerseits die Verbindung des Rechtsbereiches mit internationalem Recht, welches zwangsläufig einer Politisierung unterliegt, andererseits aber auch die Erkenntnis der Politik, dass ein Handeln erforderlich sei, wie in den bereits ausführlich dargestellten Rohstoffstrategien deutlich wird.
	
	Blickt man in die primärrechtlichen Vorgaben der Union, insbesondere im Rahmen der Handelskompetenz, so fällt auf dass die Politisierung dieses und anderer Rechtsbereiche durchaus vorgesehen ist: So wird die Union gem. Art. 205 AEUV bei ihrem Handeln auf internationaler Ebene von den Zielen und Bestimmungen aus Titel V Kap. 1 EUV geleitet und richtet es zudem daran aus. In Art. 21 EUV wird ein ähnliches Leitmotiv aufgegriffen, insbesondere unter Abs. II -- die Politisierung lässt sich hier problemlos aus den Zielen herauslesen.
	%HIER BENÖTIGT ES NOCH ENTSPRECHENDER LITERATUR ZU 205 AEUV, 21 EUV DIE DAS BESTÄTIGEN
	%DAUSE/LUDWIGS O. CALLIES HABEN BESTIMMT WAS
	
	
	% auf räumliche Verteilung der Rostoffe eingehen
	
	Es wird ersichtlich, dass die Verteilung von seltenen Erden höchst ungleich einzustufen ist, im Kontrast zu anderen mineralischen Rohstoffen. Deutlich wird auch, dass insbesondere als Entwicklungsländer bzw. "least developed countries" (LDC)\footnote{Eine allgemeine Definition für ein solches Entwicklungsland existiert nicht. Für den Gebrauch in dieser Arbeit wird daher die englische Abkürzung LDC im Sinne der least bzw. less developed countries genutzt mit dem Umfang der "DAC-Liste der Entwicklungsländer und -gebiete" des Bundesministeriums für wirtschaftliche Zusammenarbeit und Entwicklung.} einzustufende Staaten über solche Rohstoffvorkommen verfügen, viele Industrieländer\footnote{Bzw. "developed countries"  folgend der Definition des International Monetary Fund (IMF)} hingegen nicht. 
	
	\section{Fazit}
	Obgleich es notwendig ist, die Wertschöpfungskette kritischer Rohstoffe in der Union zu stärken, um die Versorgungssicherheit zu verbessern, bleiben die Lieferketten für diese Rohstoffe weltweit und unterliegen externen Einflüssen. Inwieweit diese externen Einflüsse, insbesondere aus einer wirtschaftspolitischen Sichtweise, relevant für das Rohstoffverwaltungsrecht sind, soll im folgenden Kapitel weiter betrachtet werden.
	
	Insbesondere die Rohstoffverwaltung ist als ein Handlungsgebiet von nationalen Lösungen zu klassifizierenm geprägt durch nationale Alleingänge und keiner feststellbaren Tendenz zu einer Zentralisierung afu europäischer Ebene und erst recht keinen sekunderrechtlichen Strukturierung. Der Hauptgrund der mangelhaften bzw. nicht erfolgten Ausgestaltung der unionalen Rohstoffpolitik dürfte also darin liegen, dass die Mitgliedsstaaten sich auf nationale Alleingänge konzentrieren und dadruch eine eigenständige Versrogungspolitik betreiben und zudem nur die jeweiligen nationalen Unternehmen ansprechen.
	
	Die Gestaltung eines stabilen Rechtsrahmens zur Sicherstellung der Rohstoffversrogung der Wirtschaft, die die Ziele der Union unmittelbar beeinflusst, ist also die wesentliche Aufgabe der europäischen Rohstoffpolitik.
	
\end{document}