\documentclass[12pt,a4paper,oneside]{book} % 'oneside' für einseitigen Druck

% Kodierung, Sprache und Schrift
\usepackage[utf8]{inputenc} % Erlaubt die Verwendung von Umlauten
\usepackage[T1]{fontenc} % Bessere Schriftkodierung
\usepackage[ngerman]{babel} % Deutsche Lokalisierung

% Schriftarten
\usepackage{lmodern} % Modernere Schriftart, gut für Skalierbarkeit und Lesbarkeit

% Für Abbildungen
\usepackage{graphicx}
\graphicspath{{bilder/}} % Verzeichnis, in dem Bilder gespeichert sind

% Für Tabellen
\usepackage{booktabs}

% Für Links und PDF-Metadaten
\usepackage[hidelinks]{hyperref}
\hypersetup{
	pdftitle={Titel der Dissertation},
	pdfauthor={Autor},
	pdfsubject={Doktorarbeit in den Sozial- und Rechtswissenschaften},
	pdfkeywords={Schlüsselwörter},
}

\usepackage{array}
% Für Bibliographie - Anpassung für Geisteswissenschaften
%%\usepackage[style=authoryear-icomp,backend=biber]{biblatex}
%%\usepackage[backend=biber, style=authoryear-icomp]{biblatex}
%%\usepackage[backend=biber, style=verbose-trad1]{biblatex}
\usepackage[backend=biber, style=verbose-inote]{biblatex}
%%\usepackage{biblatex}
\addbibresource{literatur.bib} % Name der BibTeX-Datei
\DeclareNameAlias{author}{family-given} % Nachname des Autors zuerst

% Anpassung der Nummerierung mit Punkten
\renewcommand{\thechapter}{\arabic{chapter}.} % Kapitel: 1., 2., 3., ...
\renewcommand{\thesection}{\Alph{section}.} % Abschnitt: A., B., C., ...
\renewcommand{\thesubsection}{\Roman{subsection}.} % Unterabschnitt: I., II., III., ...
\renewcommand{\thesubsubsection}{\arabic{subsubsection}.} % Unterunterabschnitt: 1., 2., 3., ...
\renewcommand{\theparagraph}{\alph{paragraph}.} % Absatz: a., b., c., ...


% Für Fußnoten
%%\usepackage[bottom]{footmisc} % Fußnoten am Seitenende


% Anpassung der Kapitelüberschriften
%\usepackage{titlesec}
%\titleformat{\chapter}[hang]{\Huge\bfseries}{\thechapter\quad}{0pt}{\Huge\bfseries}

% Abstand der Fußnoten
\setlength{\footnotesep}{0.5cm}

% Tiefe der Nummerierung und des Inhaltsverzeichnisses
\setcounter{secnumdepth}{4} % Nummerierungstiefe einstellen
\setcounter{tocdepth}{4} % Inhaltsverzeichnistiefe einstellen

% Abstand zwischen Absätzen und kein Einzug
%\usepackage{parskip}
%\setlength{\parskip}{0.5em}
%\setlength{\parindent}{0pt}

% Für Abkürzungsverzeichnis
\usepackage[printonlyused]{acronym}

% Für Zitate und Theoreme (falls benötigt)
\usepackage{csquotes}

% Für Gesetzestexte, Zitate und andere strukturierte Texte
\usepackage{enumitem}

% Zeilenabstand auf 1.3
\usepackage{setspace}

% Beginn des Dokuments
\begin{document}
	

\section{De Lege Ferenda: Vorschlag für ein nationales Rohstoff(wirtschafts)verwaltungsrecht im Sinne einer EU-Rohstoffpolitik}


Der Union bietet sich hier die Möglichkeit, durch eine Stärkung der regulatorischen Wirkung der EU, wie in der OSA festgestellt, einen Wettbewersbvorteil über die Fähigkeit der Mitbeeinflussung der Entwicklung von Vorschriften und Normen mit globaler Signifikanz zu erzielen.\footnote{COM(2021) 66 S. 18.}

Der Entwicklung eines Rohstoffrechts geht die Annahme voraus, dass unter den aktuellen Gesichtspunkten nicht mehr von einer Prämisse der Annahme der vorhersehbaren und allmählichen und daher stabilen Veränderungen für gesetzgeberische Aktivität vorliegt\autocite{Craig et al, Balancing stability and flexibility in adaptive governance: an analysis of tools available in U.S. environmental law}, sondern die Energiewende vielmehr wesentliche Unsicherheiten nicht nur im (geo-)politischen und wirtschaftlichen Rahmen bedingt bei gleichzeitig vorherrschender Ertwatung, dass entsprechende Dynamiken verstärkt werden.\autocite{bibid}

Da Bergbauprojekte eine entsprechend bekannte lange Vorlaufzeit haben, bietet sich insbesondere die Kreislaufwirtschaft als erster Ansatzpunkt eines neu entwickelten Rohstoffrechts an.

Bereits in der Vergangenheit wurde kritisiert, dass die Einbeziehung betroffener Stakeholder es bei Entwicklung (rohstoff-)lieferkettenbezogener Rechtsakte mangelhaft ausgefallen sei.

\subsection{Inhaltlicher Regelungsbereich}



Aufgrund des bisher fehlenden umfassenden Rechtsrahmens ist es bis zur Schaffung eines solchen Rechts erforderlich, dass nationale und europäische Gesetzgebungen und Regulierungen entsprechend mit Hinblick im Sinne des \textit{Lex ferenda} umgesetzt und ausgelegt werden. 

Ferner besteht die Problematik vielfach nicht definierter Rechtsbegriffe, die Unternehmen vor entsprechende Herausforderungen stellt, in weiterer Weise auch das Potential des Zusammenlegens und Verbinden verschiedener Berichtspflichten.\autocite{Schäffer, EuZW 2023, 695, 700}

Es stellt sich zudem die Frage der systematischen Zuordnung: Wie bereits von Frau \autocite{frau2023} dargelegt, bieten verschiedene Rechtsbereiche rohstoffliche Anknüpfungspunkte. Eine Zuordnung zum BBergG fällt aber schon allein aus Gründen der Sachnähe heraus, handelt es sich ja gerade nicht um den Abbau betreffende Regelungen, obwohl hierbei der Vorteil eines

Wie bereits erläutert setzt also ein solches RWvR "hinter" der ersten Phase der Rohstoffwirtschaft an, also hinter Exploration und Abbau beginnend bei Verarbeitung bis hin zur Wiederverwertung und, sofern erforderlich, au

Ferner \textit{kann} ein 

Es muss zudem zunächst erörtert werden, auf welcher Ebene

In der Annahme, dass die EU zum \textit{aktuellen} Zeitpunkt keine Kompetenz in der Hinsicht aufweist, dass die Verwaltungsvorschriften für die Mitgliedsstaaten im Bereich der Rohstoffe erlassen könnte, bietet sich eine nationale Betrachtung an. Dies fügt sich auch der Beobachtung, dass insbesondere im Rohstoffbereich mitgliedsstaatliche Gegebenheiten deutlich unterschiedlich ausfallen und somit eine nationale Betrachtung hier allein aus Gründen der Subsidiarität logisch erscheint.

Eine laufende Beobachtung von rohstoffpolitischen Parametern und Faktoren sowie Abhängigkeitspotentialen scheint empfehlenswert, nicht zuletzt um die entsprechenden Auswirkungen auf die Binnenmarktvervollständigung entsprechend steuern zu können -- gleich einem \glqq Büro für Widerstandsfähigkeit\grqq, welches tatsächlich in der Wirtschaftszwang-VO von einigen Wirtschaftsverbänden und Behörden gefordert wurde, letztendlich aber von der Kommission als ungeeignet eingestuft und verworfen wurde.\footnote{COM(2021) 66 final, S. 5f.}

Das Verständnis des Rohstoffverwaltungsrechts umfasst also zum jetzigen Zeitpunkt größtenteils die Verwaltung von entsprechenden Berichtspflichten, Handelsauflagen bzw. der Verwaltung von handelspolitischen Instrumenten und die entsprechende Ausrichtung dieser Verwsltung. Hierbei ist auch festzuhalten, dass die Verwaltung hierbei größtenteils durch die Mitgliedsstaaten erfolgt, bestehen zum aktuellen Zeitpunkt zumindest keine speziellen oder allgemeinen Einrichtung auf gesamteuropäischer Ebene, durch die die Union selbst die Verwaltung übernehmen könnte. Stattdessen wird über die Rechtsakte die Verwaltung in die Verwaltungsautonomie der Mitgliedsstaate entlassen.

% evaluieren ob Entsorgung relevant wäre für RVWR, im Sinne von staatlich geregelter Entsorgung
% EInrichtung eines verpflichtenden Fonds zum finanziellen Ausgleich bei Krisen durch Rohstoffabhängigkeit?
% generell Frage klären, inwieweit der Bereich staatlich zu kontrollieren ist und inwieweit Freiheit durch Unternehmen
% Vergleich Regelungsintensität BBergG und Co und eines möglichen RVWR

Der Zweck eines solchen Gesetzes kann sich aber durchaus an § 1 BBergG orientieren: Hierbei würde die Sicherung der Rohstoffversorgung eben nicht durch "das Aufsuchen, Gewinnen und Aufbereiten von Bodenschätzen (...)" erzielt werden, sondern durch freien Handel, Verarbeitung, Verbindung und Verwertung von Rohstoffen; ferner die Gewährleistung der Betriebe und Beschäftigten der rohstoffabhängigen Industrie zu gewährleisten (Abs. 2). Eine Versorge gegen Gefahren

Durchaus können aber mittelbare Gefahren, die sich aus Aktivität im Rohstoffbereich ergeben, unter einem Abs. 3 berücksichtigt werden. Denkbar sind, je nach Regelungsumfang und -gehalt, Gefahren im Sinne des LkSG, Gefahren für Unternehmen durch Abhängigkeiten und handelspolitischer Risiken, aber auch Gefahren aus staatlicher Sicht.

Insbesondere sollte durch einen § 1 festgestellt werden, dass der Anwendungsbereich sich auf alle Phasen erstreckt, beginnend mit dem Ende des Anwenungsbereiches des BBergG und endend mit dem Beginn eines Verwertungsgesetzes.

Während das BBergG sich auf mineralische Rohstoffe konzentriert, sollte das Rohstoffwirtschaftsverwaltungsrecht eine größere Auswahl von Rohstoffen abdecken, einschließlich erneuerbarer und kritischer Rohstoffe, und unter Umständen den Begriff des Rohstoffs in der Hinsicht erweitern, dass auch verarbeite bzw. ganzheitliche Produkte als gleichgestellter Rohstoff anzusehen sind. Nehme man exemplarisch eine Verringerung oder Einstellung von Exporten bestimmter Mineralien für den Fahrzeugbatteriebau an, so würde statt einem Mineral im engeren Sinne stattdesssen auch die Fahrzeugbatterie als Produkt mehrerer Rohstoff und Verarbeitungsschritte als Rohstoff gewertet werden. Die Definition umfasst somit ein breiteres Verstädnnis von "kritischen Rohstoffen", um dem Bedarf der Wirtschaft gerecht zu werden. Entsprechende Begriffsbestimmungen, insbesondere auch zur Abgrenzung der einzelnen Phasen, können auch hier analog dem § 4 BBergG getroffen werden. Insbesondere die Abgrenzung zur Aufbereitung nach § 4 III BBergG zum Begriff der Verarbeitung ist hierbei entscheidend.

"Verarbeiten im Sinne dieses Gesetzes ist das mechanische, physikalische, chemische, thermische oder biologische Behandeln von Rohstoffen, einschließlich, aber nicht beschränkt auf, die Prozesse der Reinigung, Trennung, Anreicherung, Umwandlung und Formgebung, die notwendig sind, um Rohstoffe in Zwischen- oder Endprodukte zu überführen."

Als industrielles Beispiel umfasst hier das Verarbeiten von Rohstoffen zu einer EV-Batterie die Anreicherung und Umwandlung (z. B. zu hochreinem Lithium), die eigentliche Elektrodenproduktion und Zellfertigung sowie die Zusammenstellung der einzelnen Zellen zu Modulen und Packs.

Ein RWvR sollte zudem spezifische und eindeutige Regelungen zur Klassifizierung strategischer Rohstoffe einführen - ähnlich der ausführlich dargestellten Klassifizierung zu kritischen und strategischen Rohstoffen in der Verordnung der EU zur Rohstoffversorgung. Auch Regelungen zur Schaffung und Verwaltung strategischer Reserven sind denkbar  % analog zu?

Die Festlegung von Regeln zur Umweltverträglichkeit und Nachhaltigkeit betont die Notwendigkeit, ökologische und soziale Aspekte in der Rohstoffwirtschaft zu berücksichtigen.
Die Maßnahmen zur Förderung der Kreislaufwirtschaft und der Wiederverwertung von Rohstoffen werden als Schlüsselstrategien zur nachhaltigen Rohstoffnutzung betrachtet.

Zentralisierung oder Koordinierung der Zuständigkeiten: Klare Definition der Rollen und Verantwortlichkeiten verschiedener Behörden, um Redundanzen und Ineffizienzen zu vermeiden.

Digitalisierung und Transparenz: Einführung digitaler Verwaltungsprozesse und Transparenzvorschriften zur Erhöhung der Effizienz und Nachvollziehbarkeit.
Die Einführung transparenter und effizienter Verwaltungsprozesse durch Digitalisierung und klare Zuständigkeitsregelungen soll die Umsetzung und Durchsetzung des Gesetzes erleichtern.

Vorteile der Schaffung eines Rohstoffverwaktungsrechts liegen besonders in der automatischen Legitimitätsstärkung von Aktivitäten in der rohstoffpolitischen Dimension, nicht zuletzt auch durch die Visualisierung und Verankerung eines rohstoffprogrammatischen Anspruches, der insbesondere auch aus politscher Sicht (s. Teil II) von Relevanz sein dürfte.

\subsection{Anwendungsbereich}

Ein allgemeines Rohstoffrecht der Union würde also nicht nur in ihren Mitgliedsstaaten Anwendung finden, sondern grundsätzlich auch in den Staaten des EWR\footnote{Anhang XX, Art. 74 EWR-Abkommen}


Auch eine Anwendung in Beitrittsstaaten der Union kommt dann infrage, ist eine Anpassung des nationalen Rechts an das unionale Rcht vorgesehen, also auch an das Rohstoffrecht -- was insbesondere im Fall Serbien eine durchaus untersuchenswerte Konstellation darstellt.

\subsection{Beziehung zur Rohstoffpolitik der Union}
Wie in \ref{kapitel3} dargestellt, kann eine eigens geschaffene Rohstoffpolitik der Union 

\subsection{Schaffung einer EU-Rohstoffagentur}

Die Mitwirkung der Mitgliedsstaaten darf hier keinesfalls außer Acht gelassen werden: Zum einen ist der weiterhin weite Handlungsspielraum zu bedenken, zum anderen sind durchaus Fälle bekannt, in denen Vorschläge zu Einrichtung einer unionalen Agentur am Veto der Mitgliedsstaaten scheiterten.\autocite[im konkreten Fall ging es um die Einführung einer EU-Agentur zur Kontrolle der Umsetzung des EU-Abfallrechts]{Dauses/Ludwigs, O. Umweltpolitik Rn. 226; Europäische Kommission: Study on the feasibility of the establishment of a Waste Implementation Agency}

Als Vorbild könnte hierbei die seit 1990 bestehende Europäische Umweltagentur (EAU) dienen.
%Dauses/Ludwigs O. Umweltrecht Rn. 231 zur EAU

\subsection{Preisbildung}
Wie bereits mehrfach festgestellt, reflektieren die Marktpreise für Mineralien nicht das enthaltende Risiko und erschweren zudem europäischen und anderen alternativen Lieferanten den Markteinstieg.

Zur Sicherstellung wettbewerbsfähiger Rahmenbedingungen bedarf es eines europäischen Preismechanismus, der sich an marktwirtschaftlichen Prinzipien orientiert. Eine mögliche Maßnahme wäre die Etablierung einer eigenen Handelsplattform auf europäischer Ebene, die funktional mit der London Metal Exchange vergleichbar wäre.

\subsection{Fazit}


%====================================

Internationale Abkommen: Regelungen zur Umsetzung und Einhaltung internationaler Abkommen und Standards.
Harmonisierung mit EU-Recht: Sicherstellung der Kohärenz und Kompatibilität mit europäischen Rechtsvorschriften und Strategien.
Ein Verweis bzw. ein "Streamlining" mit Verordnung 1252 ist daher allein aus Kohärenzgründen unausweichlich.

Beispielhafte Struktur eines Rohstoffwirtschaftsverwaltungsrechts
Allgemeine Bestimmungen

Ziele und Anwendungsbereich
Definitionen
Rohstoffpolitik und -strategie

Nationale Rohstoffstrategie
Liste strategischer Rohstoffe
Genehmigungsverfahren

Antragstellung und Genehmigung
Umweltverträglichkeitsprüfungen
Nachhaltigkeit und Umweltschutz

Umwelt- und Nachhaltigkeitskriterien
Maßnahmen zur Kreislaufwirtschaft
Zuständigkeiten und Verwaltungsprozesse

Zuständige Behörden
Koordinationsmechanismen
Forschung und Innovation

Förderprogramme und Forschungsinitiativen
Kooperationen
Internationale Zusammenarbeit

Umsetzung internationaler Abkommen
Harmonisierung mit EU-Recht
Rechtsdurchsetzung und Sanktionen

Kontrollmechanismen
Sanktionen bei Verstößen

\end{document}