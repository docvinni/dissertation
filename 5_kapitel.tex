\section{De Lege Ferenda: Vorschlag für ein Rohstoff(wirtschafts)verwaltungsrecht}

Aufgrund des bisher fehlenden umfassenden Rechtsrahmens ist es bis zur Schaffung eines solchen Rechts erforderlich, dass nationale und europäische Gesetzgebungen und Regulierungen entsprechend mit Hinblick im Sinne des \textit{Lex ferenda} umgesetzt und ausgelegt werden. 

Es stellt sich zudem die Frage der systematischen Zuordnung: Wie bereits von Frau \autocite{frau2023} dargelegt, bieten verschiedene Rechtsbereiche rohstoffliche Anknüpfungspunkte. Eine Zuordnung zum BBergG fällt aber schon allein aus Gründen der Sachnähe heraus, handelt es sich ja gerade nicht um den Abbau betreffende Regelungen, obwohl hierbei der Vorteil eines

Wie bereits erläutert setzt also ein solches RWvR "hinter" der ersten Phase der Rohstoffwirtschaft an, also hinter Exploration und Abbau beginnend bei Verarbeitung bis hin zur Wiederverwertung und, sofern erforderlich, au

% evaluieren ob Entsorgung relevant wäre für RVWR, im Sinne von staatlich geregelter Entsorgung
% EInrichtung eines verpflichtenden Fonds zum finanziellen Ausgleich bei Krisen durch Rohstoffabhängigkeit?
% generell Frage klären, inwieweit der Bereich staatlich zu kontrollieren ist und inwieweit Freiheit durch Unternehmen
% Vergleich Regelungsintensität BBergG und Co und eines möglichen RVWR

Der Zweck eines solchen Gesetzes kann sich aber durchaus an § 1 BBergG orientieren: Hierbei würde die Sicherung der Rohstoffversorgung eben nicht durch "das Aufsuchen, Gewinnen und Aufbereiten von Bodenschätzen (...)" erzielt werden, sondern durch freien Handel, Verarbeitung, Verbindung und Verwertung von Rohstoffen; ferner die Gewährleistung der Betriebe und Beschäftigten der rohstoffabhängigen Industrie zu gewährleisten (Abs. 2). Eine Versorge gegen Gefahren

Durchaus können aber mittelbare Gefahren, die sich aus Aktivität im Rohstoffbereich ergeben, unter einem Abs. 3 berücksichtigt werden. Denkbar sind, je nach Regelungsumfang und -gehalt, Gefahren im Sinne des LkSG, Gefahren für Unternehmen durch Abhängigkeiten und handelspolitischer Risiken, aber auch Gefahren aus staatlicher Sicht.

Insbesondere sollte durch einen § 1 festgestellt werden, dass der Anwendungsbereich sich auf alle Phasen erstreckt, beginnend mit dem Ende des Anwenungsbereiches des BBergG und endend mit dem Beginn eines Verwertungsgesetzes.

Während das BBergG sich auf mineralische Rohstoffe konzentriert, sollte das Rohstoffwirtschaftsverwaltungsrecht eine größere Auswahl von Rohstoffen abdecken, einschließlich erneuerbarer und kritischer Rohstoffe, und unter Umständen den Begriff des Rohstoffs in der Hinsicht erweitern, dass auch verarbeite bzw. ganzheitliche Produkte als gleichgestellter Rohstoff anzusehen sind. Nehme man exemplarisch eine Verringerung oder Einstellung von Exporten bestimmter Mineralien für den Fahrzeugbatteriebau an, so würde statt einem Mineral im engeren Sinne stattdesssen auch die Fahrzeugbatterie als Produkt mehrerer Rohstoff und Verarbeitungsschritte als Rohstoff gewertet werden. Die Definition umfasst somit ein breiteres Verstädnnis von "kritischen Rohstoffen", um dem Bedarf der Wirtschaft gerecht zu werden. Entsprechende Begriffsbestimmungen, insbesondere auch zur Abgrenzung der einzelnen Phasen, können auch hier analog dem § 4 BBergG getroffen werden. Insbesondere die Abgrenzung zur Aufbereitung nach § 4 III BBergG zum Begriff der Verarbeitung ist hierbei entscheidend.

"Verarbeiten im Sinne dieses Gesetzes ist das mechanische, physikalische, chemische, thermische oder biologische Behandeln von Rohstoffen, einschließlich, aber nicht beschränkt auf, die Prozesse der Reinigung, Trennung, Anreicherung, Umwandlung und Formgebung, die notwendig sind, um Rohstoffe in Zwischen- oder Endprodukte zu überführen."

Als industrielles Beispiel umfasst hier das Verarbeiten von Rohstoffen zu einer EV-Batterie die Anreicherung und Umwandlung (z. B. zu hochreinem Lithium), die eigentliche Elektrodenproduktion und Zellfertigung sowie die Zusammenstellung der einzelnen Zellen zu Modulen und Packs.

Ein RWvR sollte zudem spezifische und eindeutige Regelungen zur Klassifizierung strategischer Rohstoffe einführen - ähnlich der ausführlich dargestellten Klassifizierung zu kritischen und strategischen Rohstoffen in der Verordnung der EU zur Rohstoffversorgung. Auch Regelungen zur Schaffung und Verwaltung strategischer Reserven sind denkbar  % analog zu?

Die Festlegung von Regeln zur Umweltverträglichkeit und Nachhaltigkeit betont die Notwendigkeit, ökologische und soziale Aspekte in der Rohstoffwirtschaft zu berücksichtigen.
Die Maßnahmen zur Förderung der Kreislaufwirtschaft und der Wiederverwertung von Rohstoffen werden als Schlüsselstrategien zur nachhaltigen Rohstoffnutzung betrachtet.

Zentralisierung oder Koordinierung der Zuständigkeiten: Klare Definition der Rollen und Verantwortlichkeiten verschiedener Behörden, um Redundanzen und Ineffizienzen zu vermeiden.

Digitalisierung und Transparenz: Einführung digitaler Verwaltungsprozesse und Transparenzvorschriften zur Erhöhung der Effizienz und Nachvollziehbarkeit.
Die Einführung transparenter und effizienter Verwaltungsprozesse durch Digitalisierung und klare Zuständigkeitsregelungen soll die Umsetzung und Durchsetzung des Gesetzes erleichtern.

Internationale Abkommen: Regelungen zur Umsetzung und Einhaltung internationaler Abkommen und Standards.
Harmonisierung mit EU-Recht: Sicherstellung der Kohärenz und Kompatibilität mit europäischen Rechtsvorschriften und Strategien.
Ein Verweis bzw. ein "Streamlining" mit Verordnung 1252 ist daher allein aus Kohärenzgründen unausweichlich.

Beispielhafte Struktur eines Rohstoffwirtschaftsverwaltungsrechts
Allgemeine Bestimmungen

Ziele und Anwendungsbereich
Definitionen
Rohstoffpolitik und -strategie

Nationale Rohstoffstrategie
Liste strategischer Rohstoffe
Genehmigungsverfahren

Antragstellung und Genehmigung
Umweltverträglichkeitsprüfungen
Nachhaltigkeit und Umweltschutz

Umwelt- und Nachhaltigkeitskriterien
Maßnahmen zur Kreislaufwirtschaft
Zuständigkeiten und Verwaltungsprozesse

Zuständige Behörden
Koordinationsmechanismen
Forschung und Innovation

Förderprogramme und Forschungsinitiativen
Kooperationen
Internationale Zusammenarbeit

Umsetzung internationaler Abkommen
Harmonisierung mit EU-Recht
Rechtsdurchsetzung und Sanktionen

Kontrollmechanismen
Sanktionen bei Verstößen

\end{document}